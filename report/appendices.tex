\begin{appendices}

\section {Typing Rules}

\subsection {System F}
\label{appendix:sysFrules}
\[\begin{array}{c@{\qquad}c}
    \mbox{\begin {prooftree}
      \hypo{x:\sigma \in \Gamma}
      \infer1[var]{\Gamma \vdash x:\sigma}
    \end {prooftree}}
    &
    \mbox{\begin{prooftree}
      \infer0[const]{\Gamma \vdash c:T}
    \end{prooftree}}
    \\\\
    \mbox{\begin{prooftree}
      \hypo{\Gamma, x:\sigma \vdash e : \tau}
      \infer1[abs]{\Gamma \vdash (\lambda x_\sigma \cdot e) : \sigma
        \to \tau}
    \end{prooftree}}
    &
    \mbox{\begin{prooftree}
      \hypo{\Gamma \vdash e_1 : \sigma \to \tau}
      \hypo{\Gamma \vdash e_2 : \sigma}
      \infer2[app]{\Gamma \vdash (e_1 e_2): \tau}
    \end{prooftree}}
    \\
\end{array} \]

Thus far the typing rules are the same as in the simply typed lambda
calculus, to complete his system Reynolds extends it with two more
rules to introduce parametric polymorphism:

\[\begin{array}{c@{\qquad}c}  
    \mbox{\begin{prooftree}
      \hypo{\Gamma \vdash M : \sigma}
      \infer1[$\Delta$-abs]{\Gamma \vdash (\Lambda \alpha \cdot M) : \Delta
      \alpha \cdot \sigma}
    \end{prooftree}}
    &
    \mbox{\begin{prooftree}
      \hypo{\Gamma \vdash M : \Delta \alpha \cdot \sigma}
      \infer1[$\Delta$-app]{\Gamma \vdash (M \tau) : \sigma[\tau / \alpha]}
    \end{prooftree}}
    \\\\
  \end{array} \]

Note that in these rules, $\alpha$ is a type variable.

\subsection {Hindley-Milner}
\label{appendix:HMrules}

In Hindley-Milner, we have very similar rules for typing lambda
abstraction, application, free type variables and and type constants:

\[\begin{array}{c@{\qquad}c}  
    \mbox{\begin{prooftree}
      \hypo{x : \sigma \in \Gamma}
      \infer1[var]{\Gamma \vdash x : \sigma}
    \end{prooftree}}
    &
    \mbox{\begin{prooftree}
      \infer0[const]{\Gamma \vdash c : T}
    \end{prooftree}}
    \\\\
    \mbox{\begin{prooftree}
      \hypo{\Gamma , x : \sigma \vdash e : \tau }
      \infer1[abs]{\Gamma \vdash (\lambda x \cdot e) : \sigma \to \tau}
    \end{prooftree}}
    &
    \mbox{\begin{prooftree}
      \hypo{\Gamma \vdash e_1 : \sigma \to \tau}
      \hypo{\Gamma \vdash e_2 : \sigma}
      \infer2[app]{\Gamma \vdash (e_1 e_2) : \tau}
    \end{prooftree}}
    \\\\      
\end{array} \]

These rules are then extended to accomodate the 'let' language
construct:

  \[\begin{array}{c}  
    \mbox{\begin{prooftree}
      \hypo{\Gamma \vdash e_1 : \sigma}
      \hypo{\Gamma, x : \sigma \vdash e_2 : \tau }
      \infer2[let]{\Gamma \vdash (let\; x = e_1\; in\; e_2) : \tau}
    \end{prooftree}}
    \\\\
  \end{array} \]

Before we detail the last two rules detailing instantiation and
generification we first outline the meaning of a judgement $\sigma
\sqsubseteq \sigma'$. Intuitively this means that $\sigma$ is some
subtype of $\sigma'$ - we can create it by some substitution of the
quantified variables in $\sigma'$. More precisely it is defined as
follows:

\[\begin{array}{c@{\qquad}c}  
    \mbox{\begin{prooftree}
      \hypo{\tau' = \{\alpha_i \mapsto \tau_i\}\tau}
      \hypo{\beta_i \notin free(\forall \alpha_1 \ldots \forall
        \alpha_n \cdot \tau)}
      \infer2[spec]{\forall \alpha_1 \ldots \forall \alpha_n \cdot
        \tau \sqsubseteq \forall \beta_1 \ldots \forall
        \beta_m \cdot \tau' }
    \end{prooftree}}
    \\\\
\end{array} \]  
  
Lastly we have rules to instantiate a type scheme, or generify a
type (i.e. to make a type more specific and narrow, or less specific
and general):

\[\begin{array}{c@{\qquad}c}  
    \mbox{\begin{prooftree}
      \hypo{\Gamma \vdash e : \sigma}
      \hypo{\sigma' \sqsubseteq \sigma}
      \infer2[inst]{\Gamma \vdash e : \sigma'}
    \end{prooftree}}
    &
    \mbox{\begin{prooftree}
      \hypo{\Gamma \vdash e : \sigma}
      \hypo{\alpha \notin free(\Gamma)}
      \infer2[gen]{\Gamma \vdash e : \forall \alpha \sigma}
    \end{prooftree}}
    \\\\
\end{array} \]

The latter three rules are the ones that capture the idea of
Hindley-Milner polymorphism.

\clearpage
\section{Example Specifications}
\label{appendix:examplespecifications}

This appendix shows example specification files for various
languages.

\subsection{Simply Typed Lambda Calculus with Product Types}
\label{STLCspec}

\begin{verbatim}
type: α
  value: a

type: β
  value: b

type: A -> B
  if:
    type A
    type B
  eliminated-by: E
    if:
      (A) <- E
    resulting-in-type: B
  value: \ X. -> M
    if:
      X : (A) |- (B) <- M
    reduces-to: M/[, E:A]
  expanded-by: \ Y. -> Y
  
type: A x B
  if:
    type A
    type B
  eliminated-by: fst
    resulting-in-type: A
  eliminated-by: snd
    resulting-in-type: B
  value: L and R
    if:
      (A) <- L
      (B) <- R
    reduces-to: L
    reduces-to: R
  expanded-by: fst , snd
\end{verbatim}

\clearpage
\subsection{System-F-like language }
\label{SystemFspec}

Our variation is slighly more general than the original encoding as
opposed to only polymorphic functions we give the means of
representing arbitrary polymorphic types.

\begin{verbatim}
type: α
  value: a

type: β
  value: b

type: A -> B
  if:
    type A
    type B
  eliminated-by: E
    if:
      (A) <- E
    resulting-in-type: B
  value: \ X. -> M
    if:
      X : (A) |- (B) <- M
    reduces-to: M/[, E:A]
  expanded-by: \ Y. -> Y

type: ∀ T. => M
  if:
    T : (set) |- type M
  eliminated-by: TY
    if:
      type TY
    resulting-in-type: M/[, TY:set]
  value: δ T. PTY
    if:
      T : (set) |- (M/[, .T]) <- PTY
    reduces-to: PTY/[, TY:set]
\end{verbatim}

\clearpage
\subsection{Lambda Calculus Variation with Dependent Types}
\label{DTLCspec}

\begin{verbatim}
type: α
  value: a1
  value: a2

type: isa TY TM
  if:
    type TY
    (TY) <- TM
  value: is V
    if:
      (TY) <- V
      (TM) = V

type: A X. -> B
  if:
    type A
    X : (A) |- type B
  eliminated-by: E
    if:
      (A) <- E
    resulting-in-type: B/[, E:A]
  value: \ X. -> M
    if:
      X : (A) |- (B/[, .X]) <- M
    reduces-to: M/[, E:A]
  expanded-by: \ Y. -> Y
\end{verbatim}
\clearpage
\section{Parser Combinators}
\label{appendix-parsercombinators}

\begin{verbatim}
fail : Parser A
fail str = nothing

-- guarantees a parser, if it succeeds, consumes some input
safe : Parser A → Parser A
safe p s = do
             (a , leftover) ← p s
             if length leftover <ᵇ length s then just (a , leftover) else nothing
           where open maybemonad

peak : Parser Char
peak str with toList str
... | []      = nothing
... | c ∷ rest = just ((c , fromList (c ∷ rest)))

all : Parser String
all = λ s → just (s , "")

try : Parser A → A → Parser A
try p a str = p str <∣> just (a , str)

consumes : Parser A → Parser (ℕ × A)
consumes p = do
               bfr ← (λ s → just (length s , s))
               a ← p
               afr ← (λ s → just (length s , s))
               return (∣ bfr - afr ∣ , a)
             where open parsermonad

biggest-of_and_ : Parser A → Parser A → Parser A
biggest-of_and_ p1 p2 str with p1 str | p2 str
... | nothing | nothing = nothing
... | nothing | just p = just p
... | just p | nothing = just p
... | just (a1 , rst1) | just (a2 , rst2) = just (if length rst1 <ᵇ length rst2 then (a1 , rst1) else (a2 , rst2))

_or_ : Parser A → Parser B → Parser (A ⊎ B)
(pa or pb) str = (inj₁ <$> pa) str <∣> (inj₂ <$> pb) str
  where
    open parsermonad

either_or_ : Parser A → Parser A → Parser A
(either pa or pb) str = maybe′ just (pb str) (pa str)

ifp_then_else_ : Parser A → Parser B → Parser B → Parser B
(ifp p then pthen else pelse) str with p str
... | nothing = pelse str
... | just x  = pthen str

nout : Parser ⊤
nout = return tt
  where open parsermonad

optional : Parser A → Parser (A ⊎ ⊤)
optional = _or nout

complete : Parser A → Parser A
complete p = do
               a ← p
               rest ← all
               if rest ==ˢ "" then return a else fail
             where open parsermonad

takeIfc : (Char → Bool) → Parserp Char
takeIfc p []          = nothing
takeIfc p (c ∷ chars) = if p c then just (c , chars) else nothing

takeIf : (Char → Bool) → Parser Char
takeIf p = →[ takeIfc p ]→

-- This terminates as we ensure to make p "safe" before we
-- use it, forcing the parser to fail if it does not consume
-- any input
{-# TERMINATING #-}
_*[_,_] : Parser A → (A → B → B) → B → Parser B
p *[ f , b ] = do
                 inj₁ a ← optional (safe p)
                   where inj₂ _ → return b
                 b ← p *[ f , b ] 
                 return (f a b)
  where open parsermonad


_⁺[_,_] : Parser A → (A → B → B) → B → Parser B
p ⁺[ f , b ] = pure f ⊛ p ⊛ (p *[ f , b ])
  where open parsermonad

anyof : List (Parser A) → Parser A
anyof []       = fail
anyof (p ∷ ps) = either p or (anyof ps)

biggest-consumer : List (Parser A) → Parser A
biggest-consumer [] = fail
biggest-consumer (p ∷ ps) = biggest-of p and biggest-consumer ps

all-of : List (Parser A) → Parser (List A)
all-of [] str = just ([] , str)
all-of ps str = just (foldr (λ p las → maybe′ (λ (a , _) → a ∷ las) las (p str) ) [] ps , "")

{-# TERMINATING #-}
how-many? : Parser A → Parser (Σ[ n ∈ ℕ ] Vec A n)
how-many? p = ifp p then (do
                            a ←  safe p
                            (n , as) ← how-many? p
                            return (ℕ.suc n , a ∷ as))
              else return (0 , [])
  where open parsermonad

max_how-many? : ℕ → Parser A → Parser (Σ[ n ∈ ℕ ] Vec A n)
max zero how-many? _    = return (0 , [])
  where open parsermonad
max (suc n) how-many? p = ifp p then (do
                            a ←  safe p
                            (n , as) ← max n how-many? p
                            return (ℕ.suc n , a ∷ as))
              else return (0 , [])
  where open parsermonad

exactly : (n : ℕ) → Parser A → Parser (Vec A n)
exactly zero _  = return []
  where open parsermonad
exactly (suc n) p = do
                      a ← p
                      as ← exactly n p
                      return (a ∷ as)
-- text/number parsing

whitespace : Parser ⊤
whitespace str = just (tt , trim← str)

ws+nl : Parser ⊤
ws+nl str = just (tt , trim←p str)

ws+nl! : Parser ⊤
ws+nl! = do
            c ← takeIf Data.Char.isSpace
            ws+nl
           where open parsermonad

ws-tolerant : Parser A → Parser A
ws-tolerant p = do
                  whitespace
                  r ← p
                  whitespace
                  return r
  where open parsermonad

wsnl-tolerant : Parser A → Parser A
wsnl-tolerant p = do
                    ws+nl
                    r ← p
                    ws+nl
                    return r
  where open parsermonad

wsnl-tolerant! : Parser A → Parser A
wsnl-tolerant! p = do
                    ws+nl!
                    r ← p
                    ws+nl!
                    return r
  where open parsermonad

literalc : Char → Parserp Char
literalc c [] = nothing
literalc c (x ∷ rest)
  = if c == x then just (c , rest)
              else nothing

literal : Char → Parser Char
literal c = →[ literalc c ]→

newline = literal "\n"

nextCharc : Parserp Char
nextCharc []          = nothing
nextCharc (c ∷ chars) = just (c , chars)

nextChar = →[ nextCharc ]→

literalAsString = (fromChar <$>_) ∘′ literal
  where open parsermonad

string : String → Parser String
string str = foldr
                 (λ c p → (pure _++_) ⊛ (literalAsString c) ⊛ p)
                 (return "") (toList str)
  where open parsermonad

  
stringof : (Char → Bool) → Parser String
stringof p = takeIf p *[ _++_ ∘ fromChar , "" ]

until : (Char → Bool) → Parser String
until p = stringof (not ∘ p)

nonempty : Parser String → Parser String
nonempty p = do
               r ← p
               if r ==ˢ "" then (λ _ → nothing) else return r
  where open parsermonad

nat : Parser ℕ
nat = do
        d ← nonempty (stringof isDigit)
        return (toNat d)
      where open parsermonad


bracketed : Parser A → Parser A
bracketed p = do
                literal "("
                a ← wsnl-tolerant p
                literal ")"
                return a
   where open parsermonad

potentially-bracketed : Parser A → Parser A
potentially-bracketed p = either bracketed p or p

curlybracketed : Parser A → Parser A
curlybracketed p = do
                literal "{"
                a ← wsnl-tolerant p
                literal "}"
                return a
  where open parsermonad

squarebracketed : Parser A → Parser A
squarebracketed p = do
                literal "["
                a ← wsnl-tolerant p
                literal "]"
                return a
  where open parsermonad
  open Parsers
\end{verbatim}
\end{appendices}
