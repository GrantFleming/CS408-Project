\begin{abstract}
  Static analysis is an important part of all modern compilers.
  In this project, we explore how much of the code used to
  implement  type-checkers can be commonly abstracted across a
  family of type systems.

  We will aim to create a generic type checker that is
  able to type-check code so long as the type system of the language
  is described to it. 
  
  We begin by performing a general type theory literature review:
  taking the reader on a whirlwind tour of the simply typed lambda calculus
  \cite{church1940}, System F \cite{reynolds1974} and Hindley-Milner
  type theory \cite{hindley1969,milner1978,milner1982}
  before finally exploring Intuitionistic type theory
  \cite{martinlof1980} (sometimes known as Martin L\"{o}f type
  theory). We will then review some select literature relevant to
  the implementation of various type-checker components and some
  relevant meta-theory.

  Armed with our new found knowledge we will develop a domain-specific
  language (DSL) for describing type systems, paying special attention
  to not collect redundant information from the user.

  We will put our DSL to good use by designing and
  implementing a type-checker-generator in Agda. This will take the
  form of a piece of software that receives, as its input, a
  specification file written in our previously developed DSL and a
  source code file in the language it describes. The software will then 
  type-check the source code according to the type-system described in
  the specification. 

  We will conclude by evaluating our DSL, software, and design and
  implementation process. Finally, we speculate as to how we might
  improve our software and conclude by outlining possible areas for
  future work.
\end{abstract}
