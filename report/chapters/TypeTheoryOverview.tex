\chapter{An Overview of Type Theory}

        \section{Background, Lambda Calculus and System F}

          Type systems, as with so many concepts in computer science,
          are a concept borrowed from the study of logic. Their initial
          purpose was to resolve certain inconsistencies in an
          underlying logic. 
      
          Although originally proposed by Bertrand Russell in 1902 to
          resolve a paradox he himself had discovered in a formalization of
          Gottlob Frege's naive set theory in 1901 \cite{Russell1901}, much
          introductory material on type theory from a computer science
          perspective begins with Alonzo Church and his Simply Typed
          Lambda Calculus \cite{church1940}.
          
          Like Russell, Church was searching for a way to make his
          previously defined system of logic, the lambda calculus,
          consistent. In 1940 he published his seminal paper that outlined a
          type system and added extra criteria to which a term must
          conform in order to be considered well-formed - it must be typeable.
      
          A concrete, if informal, summation of Church's work here is that
          in this new system, if a lambda abstraction is applied to a term,
          the type of the term must be the same type as the binder in the
          abstraction. As a consequence, previously well formed terms which
          caused inconsistencies in the logic, such as
          $ (\lambda x.xx)(\lambda x.xx)  $
          were no longer well formed under the new system.
      
          As the field of computer science developed in sophistication in
          the late 1960s and 1970s, prominent logicians and computer
          scientists started to develop the notion of parametric
          polymorphism. It was noted that some functions had common
          behaviour over differing types, and that the behaviour of these
          functions did not depend on the types themselves. This lead to
          redundant definitions, such as having to define a function with the
          same behaviour multiple times, once for each type you wish to
          operate over. Under type systems akin to the simply typed lambda
          calculus, the \emph{id} function given by the term $\lambda
          x_{\mathbb{N}}.x $ is only ever applicable to natural numbers. It
          is clear that there was immense benefit in being able to define
          functions such as \emph{id} once and have them operate over any type.
      
          One such solution to this problem came to be known as System F
          or polymorphic lambda calculus. This system was independently
          discovered by Jean-Yves Girard and John Reynolds in 1972 and 1974
          respectively \cite{Girard1972,reynolds1974}.
      
          In lambda calculus, a single binder $\lambda$ is used to bind variables
          that range over values. In the simply typed lambda calculus, a
          lambda term of the form $\lambda x_{\alpha}.M_{\beta}$ (where
          $x_{\alpha}$ binds variables $x$ over a type $\alpha$ and
          $M_{\beta}$ is some term of type $\beta $) has type
          $\alpha\to\beta$ by modern notation. System F introduces a new
          binder $\Lambda$ that is used to bind variables that range over
          types. In this system, terms of the form $\Lambda t.M$ denotes a
          function that takes as its first argument some type and returns a
          term with all references to t replaced by that type. This function
          has type $\Delta t.M^{t}$ where $M^t$ is is the type of $M$.
      
          Under this system, we may write the \emph{id} function as $\Lambda t
          . \lambda x_t . x$ and thus have it operate over any type t,
          so long as we supply it.
      
          One disadvantage of this system is that terms become verbose
          because of their explicit references to type
          information. Another is that explicit quantification over
          types can quickly become unwieldy when writing more complex
          types for polymorphic functions.

          \section{Hindley-Milner}
            An alternative approach to parametric polymorphism was
            developed in the late 60s with the Hindley-Milner type
            system which grew to become the basis of many functional
            languages such as ML and Haskell.
                        
            Hindley introduced a system of \textit{type-schemes}
            \cite{hindley1969} (types that may contain quantified type
            variables) and defined a type as being a type-scheme that
            contains no such variables. Milner refered to these
            concepts as polytypes and monotypes respectively
            \cite{milner1978}. 
            
            This has the consequence that a given term may have any number of
            type-schemes, some more general than others, some more specific,
            allowing us to commit to stating more or less about our
            values as the situation requires. Hindley then presents
            the idea of a \textit{principle type scheme} (p.t.s) where
            all possible type-schemes for a given term are instances
            of the p.t.s. (that is, can be constructed by consistent
            substitution for quantified variables in the p.t.s.).
            
            He then proves that any term for which you can deduce a type scheme,
            has a p.t.s., that it is always possible to work out what the p.t.s. is
            up to \textit{trivial instances} and introduces an
            algorithm to accomplish this type inference - Algorithm W.

            As a consequence of this system there is no need for us to
            annotate the definitions in our code with type
            information since we can always infer the p.t.s (most
            general type-scheme) of our terms. This can result
            in cleaner looking code in some circumstances however
            there is an argument that providing type information
            explicitly in the syntax of the language serves as
            important documentation.
            
            The type inference algorithm that is detailed in the Hindley-Milner
            system is also instrumental in the way types are checked.
            
            In a system like System F, whenever a polymorphic
            function is applied, it must first be provided the type
            parameter before being provided
            the value parameter so that the value may be
            type-checked. With Hindley-Milner, as a direct consequence
            of its type inference features, this is not
            necessary. Instead the
            most general type of the function can be inferred, and it
            is then possible to check that the type of the supplied
            value parameter is some instance of the input type of the function.
            
            The classic example of such systems is the creation of lists in the
            standard $cons/[]$ way. In a language with a type system akin to
            System F, this may look
            something like:
            
            \begin{minted}{haskell}
              (cons Number 1 (cons Number 9 (cons Number 6 [])))  
            \end{minted}
            
            whereas if we can infer the most general type of $cons$ and
            subsequently verify that the supplied input is some instance of this
            type as in Hindley-Milner, the same expression might be written as:
            
            \begin{minted}{haskell}
              (cons 1 (cons 9 (cons 6 [])))  
            \end{minted}
            
            We can see this difference reflected in the typing rules of the
            systems by way of the following example on \emph{id}:
            
            \[\begin{array}{c@{\qquad}|@{\qquad}c}
                  \mbox{Hindley-Milner}
                  &
                  \mbox{System F}
                  \\
                  id : \forall \alpha \cdot \alpha \to \alpha
                  &
                  id : \Delta \alpha \cdot \alpha \to \alpha
                  \\                  id = \lambda x \cdot x
                  &
                  id = \Lambda t . \lambda x_t \cdot x
            \end{array} \]
            
            The type in the Hindley-Milner system can be proven by:
            
            
            \[\begin{array}{c}
            \mbox{\begin{prooftree}
                    \hypo{x : \alpha \in x : \alpha}   
                  \infer1[var]{x : \alpha \vdash x : \alpha}
                \infer1[abs]{\vdash \lambda x \cdot x : \alpha \to \alpha}
                \hypo{\alpha \notin free(\epsilon)}
               \infer2[gen]{\vdash \lambda x \cdot x : \forall \alpha \cdot
               \alpha \to \alpha}
            \end{prooftree}}
            \end{array} \]
            
            
            The equivalent type in System F can be proven by:
            
            \[\begin{array}{c}
            \mbox{\begin{prooftree}
                  \hypo{x : \alpha \in x : \alpha}
                \infer1[var]{x : \alpha \vdash x : \alpha}
               \infer1[abs]{\vdash \lambda x_t \cdot x : \alpha \to \alpha}
               \infer1[$\Delta$-abs]{\vdash \Lambda t . \lambda x_t \cdot
                 x : \Delta \alpha \cdot \alpha \to \alpha}
            \end{prooftree}}
            \end{array} \]
            
            In system F, this deduces the only type for $\Lambda t . \lambda x_t
            \cdot x$ (up to the renaming of bound variables) however note that
            in Hindley-Milner we may also deduce other type (schemes)
            for $\lambda x \cdot x$: 
            
            \[\begin{array}{c}
            \mbox{\begin{prooftree}
                        \hypo{x : \alpha \to \beta \in x : \alpha \to \beta}
                    \infer1[var]{x : \alpha \to \beta \vdash x : \alpha \to \beta}   
                    \infer1[abs]{\vdash \lambda x \cdot x : (\alpha \to \beta) \to
                    (\alpha \to \beta)}
                    \hypo{\beta \notin free(\epsilon)}        
                \infer2[gen]{\vdash \lambda x \cdot x : \forall \beta \cdot
                  (\alpha \to \beta) \to (\alpha \to \beta)}
                \hypo{\alpha \notin free(\epsilon)}
               \infer2[gen]{\vdash \lambda x \cdot x : \forall \alpha . \forall
                 \beta \cdot (\alpha \to \beta) \to (\alpha \to \beta)}
            \end{prooftree}}
            \end{array} \]
                                   
            If you wanted a term with an equivalent type in system F you would
            need to create a new term $\Lambda t_1 . \Lambda t_2 . \lambda x_{t_1
              \to t_2} \cdot x$
            
            A full list of the typing rules are available in
            appendices \ref{appendix:sysFrules} and
            \ref{appendix:HMrules}. 
                    
          \section{Intuitionistic Type Theory}
        
          Intuitionistic type theory was devised by Per Martin-Löf
          \cite{martinlof1980} originally in the 1970s, although
          several variations have been devised since.

          Martin-Löf was motivated, in part, by what we now call the
          Curry-Howard isomorphism: the idea that statements in logic can be
          thought of as types, and a proof of a logical statement can
          be thought of as a program (consequently simplification of
          proofs correspond to the execution of a program)
          \cite{wadler2015}. He noted that despite this isomorphism,
          computer scientists often developed their own
          languages and type systems instead of using existing
          logics and hence his theory of Intuitionistic types was
          designed to have practical uses as both a logic and
          programming language.

          While the Curry-Howard isomorphism clearly had an existing
          foothold, relating function types to implication for
          instance, Martin-Löf was the first to extend this to
          predicate logic by having his types depend on values,
          using dependent functions and dependent pairs to encode
          universal and existential quantification respectively.

          It was this introduction of `dependent types` that made this
          theory different from previous type theories and allowing
          the development of dependently typed languages with flexible
          and powerful type systems such as Agda \cite{norell}.

          Although Martin-Löf describes many types in this system,
          there are that a few that are of particular importance which we will
          briefly describe here: 

          \begin{itemize}
          \item Types of the form $( \Pi x \in A ) B(x)$ describe
            dependent functions, where the type $B(x)$ of the co-domain
            depends on the \emph{value} $x$ of the domain.
          \item Types of the form $( \Sigma x \in A ) B(x)$ describe
            dependent pairs, where the type $B(x)$ of the second
            element of the pair depends on the \emph{value} $x$ of the
            first element.
          \item Type of the form $A + B$ represent the disjoint union
            of two types $A$ and $B$.
          \item Types of the form $\mathbb{N}_n$ describe types
            with finite numbers of values. For instance we could take
            $\mathbb{N}_1$ to be unit, $\mathbb{N}_2$ to be booleans
            etc.
          \item Types of the form $I (A , a , b)$ encode propositional
            equality where values of this type represent proofs
            that $a \equiv b \in A$.
          \item W-types are a little more complex. They have the form
            $( W x \in A ) B(x)$ and represent \emph{wellorderings} (a
            somewhat more intuitive explanation is that they represent
            well-founded trees). This type allows us to define types
            with complex structures that are inductive in nature. In fact,
            some of the types that Martin-Löf includes in his
            type-theory, such as that of Lists or $\mathbb{N}$ can be
            defined as W-types.
          \item Universes are introduces as a type of types, allowing
            the transfinite types needed to talk about a "sets of
            sets", Martin-Löf recognised the need for such types when
            reasoning about certain areas of maths such as category
            theory.
          \end{itemize}

          One of the consequence of this, more powerful, type system,
          is that unlike Hindley-Milner, we can no longer infer the
          most general type of any expression.

          Take, for example, a value of some dependent pair $(2 , [a :: b
            :: nil])$ perhaps the type of the second element is
          dependent on the first, and the $B$ in the type $(
          \Sigma x \in A ) B(x)$ corresponds to some function $\lambda
          n \to Vec \: X \: n$, and yet maybe it does not depend on the first
          argument at all and $B$ simply corresponds to $\lambda n \to
          Vec \: X \: 2$ - there is no way for us to definitively
          deduce $B$ from the value alone. \hl{The essense of this
            part is that functions and substitution are, in general,
            not uniquely invertable - rewrite}
