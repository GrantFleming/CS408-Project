\chapter{Relevant Related Work}

In the previous chapter, we explored material designed to give a
general background of type-theory. Here we will explore some more
modern works which will help us in the construction of a
type-checker-generator.

\section{Contexts and Type Inference}

Gundry, McBride and McKinna present research detailing an
novel way of implementing unification and
type-inference \cite{TypeInferenceInContext}. An area of
particular interest is in their approach to the
\emph{context} - traditionally used to track the types of
term variables.

In this work, the authors propose a system of explicitly
tracking type-variables in the context, even before they are
bound to a type-scheme. As well as allocating types
to these variables, the unification algorithm is also
able to pull these type-variables leftward to scope them
appropriately and resolve dependencies, while it tries to
solve unification constraints.

This leads to an interesting consequence in their approach to
generalization in let expressions. Now that type variables
can exist in the context, we are able to instantiate a type
scheme by introducing a new type variable to the context,
and removing the necessary $\forall$ quantifier in the
scheme. Consequently, to generalize, we can remove these
type-variables from the context and introduce and
appropriate $\forall$ quantifier. The authors explicily
place a third element into the context, a marker that delimits
generalization to an appropriate scope.

\section{Syntactic Universes}

More recently, this project has began an exploration into the concept of
universe constructions. These may be used in building a syntactic universe
where we define various mechanisms for accomplishing
static analysis independent of the underlying syntax of the
language.

The concept of a syntactic universe was first introduced by
the author's supervisor, Conor McBride, and further general
material on universe constructions was obtained when covered
as a topic in an Advanced Functional Programming class, taught
by Fredrik Nordvall Forsberg.

This idea is explored in a more in-depth context when we
explore work detailing how we might build a universe of
syntaxes that are both scope and type safe and abstractly
define common elements of their semantics such as
substitution \cite{DBLP:journals/corr/abs-2001-11001}.

This is an area that is clearly relevant to this authors
work, although - at the time of writing - progress has still
to be made by the author in fully understanding the paper.

\section{Related Work Still To Be Explored}

There are several other papers to read in the following
areas of interest:

\begin{itemize}
\item An implementation of a dependently typed lambda
  calculus \cite{ATutorialImplementationOfDTLC}
\item Formalization of bi-directional dependent type systems \cite{TypesWhoSayNi}
\end{itemize}

Although there are other less critical papers that the
author may also tackle if time allows.
