\chapter{Introducing The Type Systems }
	content