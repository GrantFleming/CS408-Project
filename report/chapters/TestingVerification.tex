\chapter{Testing and Verification}

Throughout the process of building the software, we have tried to
ensure we leverage Agda's expressive type system to maintain
invariants that allow us to forgo large swathes of testing that would
otherwise be required. There are certainly benefits to be had here
however there are limitations to this method and it is often not
practical to rely on capturing such invariants in all cases.

Firstly we note that we are explicit about scope throughout the
code. We formalise the notion of something being \emph{Scoped} early
on and make liberal use of this type throughout. Variables are defined
to be well-scoped by construction allowing 
us to ensure that the scoped entites that use such variables (like
terms and expressions) are guaranteed to be well scoped and as such we
manage to eliminate a huge testing space where various functions would
have otherwise had to deal with potential "out of scope"
errors. Similarly we can use scope to mandate that we are given a
context sufficient for use with a particular scoped term and more
generally ensure that we are explicit about the scope that we expect,
that the scope that we choose to return.

Another area where we make careful use of types is that of patterns
and environments. The way that an environment is
constructed relative to some pattern that indexes it ensures that it
is well formed according to the pattern in question. This also gives
us the power to request such a well formed environment, simultaniously
excluding the need to test for behaviour on ill-formed environments
while having the effect of guaranteeing a sensible output where we
might otherwise have had to employ a Maybe result. Schematic variables
get similar treatment here allowing us to provide a function to look
up the term referred to by these variable in a manner that cannot
fail. 

The benefits of the careful use of types here has far reaching
consequences in our software. By way of example let us look at the
type of our \emph{toTerm} function which builds a term from an expression.

\begin{verbatim}
toTerm : (γ ⊗ p) -Env → Expr p d γ’ → Term d (γ + γ’)
\end{verbatim}

First note how we are able to ensure that we get the correct kind of
environment for the expression by sharing the same pattern \emph{p} in
the environment and expression types. We ensure we get the correct
kind of term, either a construction or a computation by sharing the
same directionality \emph{d} between the expression type and term
type. Finally we manage the scope explicitly, ensuring that for an
expression in some scope \emph{γ'} and some pattern (in the empty
scope although you cannot see that explicitly in the type here) which
has been opened in \emph{γ} results in a term scoped in (γ + γ'). We
certainly get a lot of verification "bang-for-the-buck" and make sure
to capitalize on this throughout.

Another example where could expose our type-led verification is in
that of premise chains described in section \ref{section-premises}. We
give a type \emph{Placeless} which is indexed by a pattern and is able
to encode the information in the pattern if and only if the pattern
contains no places. We then ensure that the only way to end a chain of
premise is to give some proof of the placelessness of what is left to
trust. This eliminates a vast amount of potential validation code and
allows us to enforce this aspect of the well-formedness of premise
chains rather than having to test how the software might deal
with the failures resulting from ill-formed premises.

\begin{verbatim}
data _Placeless {γ : Scope} : Pattern γ → Set where
  ’     : (s : String) → ’s Placeless
  _∙_   : {l r : Pattern γ} → (l Placeless) → (r Placeless) → (l ∙ r) Placeless
  bind  : {t : Pattern (suc γ)} → (t Placeless) → (bind t) Placeless
  ⊥     : ⊥ Placeless

data Prems (p₀ q₀ : Pattern γ) : (p₂ : Pattern γ) → Set where
  ε     :  (q₀ Placeless) → Prems p₀ q₀ p₀
  _⇉_  :  Prem p₀ q₀ γ pᵍ q₁’ →
          Prems (p₀ ∙ pᵍ) q₁’ p₂’ →
          Prems p₀ q₀ p₂’
\end{verbatim}

We have given some example of where we have used the type system to
verify important properties of our code however this is a technique we
used extensively when constructing the software and we consider it our
primary method of ensuring a good level of correctness.

\section{Where verification is insufficient}

What we certainly need to test is aspects of the system that depend on
user supplied instantiations of the concepts and rules we have
defined. Especially where the results depend on several parts of the
system working together. Using the type system to verify software is
more suited to checking the software at a lower level of verifying
small component parts. The invariants required when we start to
combine these parts are often large, complex and overly difficult to
capture in a type and so these aspects of the system are tested in a
more traditional manner.

When writing this software, we include a suite of tests that cover
areas such as beta reduction, normalisation, eta expansion, type
checking and various aspects of parsing, an area where we
focused less on type-led verification. A larger example of the testing
conducted is available in appendix \ref{appendix-tests}.
