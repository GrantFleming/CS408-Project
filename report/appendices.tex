\begin{appendices}

\section {Typing rules}

\subsection {System F}
\label{appendix:sysFrules}
\[\begin{array}{c@{\qquad}c}
    \mbox{\begin {prooftree}
      \hypo{x:\sigma \in \Gamma}
      \infer1[var]{\Gamma \vdash x:\sigma}
    \end {prooftree}}
    &
    \mbox{\begin{prooftree}
      \infer0[const]{\Gamma \vdash c:T}
    \end{prooftree}}
    \\\\
    \mbox{\begin{prooftree}
      \hypo{\Gamma, x:\sigma \vdash e : \tau}
      \infer1[abs]{\Gamma \vdash (\lambda x_\sigma \cdot e) : \sigma
        \to \tau}
    \end{prooftree}}
    &
    \mbox{\begin{prooftree}
      \hypo{\Gamma \vdash e_1 : \sigma \to \tau}
      \hypo{\Gamma \vdash e_2 : \sigma}
      \infer2[app]{\Gamma \vdash (e_1 e_2): \tau}
    \end{prooftree}}
    \\
\end{array} \]

Thus far the typing rules are the same as in the simply typed lambda
calculus, to complete his system Reynolds extends it with two more
rules to introduce parametric polymorphism:

\[\begin{array}{c@{\qquad}c}  
    \mbox{\begin{prooftree}
      \hypo{\Gamma \vdash M : \sigma}
      \infer1[$\Delta$-abs]{\Gamma \vdash (\Lambda \alpha \cdot M) : \Delta
      \alpha \cdot \sigma}
    \end{prooftree}}
    &
    \mbox{\begin{prooftree}
      \hypo{\Gamma \vdash M : \Delta \alpha \cdot \sigma}
      \infer1[$\Delta$-app]{\Gamma \vdash (M \tau) : \sigma[\tau / \alpha]}
    \end{prooftree}}
    \\\\
  \end{array} \]

Note that in these rules, $\alpha$ is a type variable.

\subsection {Hindley-Milner}
\label{appendix:HMrules}

In Hindley-Milner, we have very similar rules for typing lambda
abstraction, application, free type variables and and type constants:

\[\begin{array}{c@{\qquad}c}  
    \mbox{\begin{prooftree}
      \hypo{x : \sigma \in \Gamma}
      \infer1[var]{\Gamma \vdash x : \sigma}
    \end{prooftree}}
    &
    \mbox{\begin{prooftree}
      \infer0[const]{\Gamma \vdash c : T}
    \end{prooftree}}
    \\\\
    \mbox{\begin{prooftree}
      \hypo{\Gamma , x : \sigma \vdash e : \tau }
      \infer1[abs]{\Gamma \vdash (\lambda x \cdot e) : \sigma \to \tau}
    \end{prooftree}}
    &
    \mbox{\begin{prooftree}
      \hypo{\Gamma \vdash e_1 : \sigma \to \tau}
      \hypo{\Gamma \vdash e_2 : \sigma}
      \infer2[app]{\Gamma \vdash (e_1 e_2) : \tau}
    \end{prooftree}}
    \\\\      
\end{array} \]

These rules are then extended to accomodate the 'let' language
construct:

  \[\begin{array}{c}  
    \mbox{\begin{prooftree}
      \hypo{\Gamma \vdash e_1 : \sigma}
      \hypo{\Gamma, x : \sigma \vdash e_2 : \tau }
      \infer2[let]{\Gamma \vdash (let\; x = e_1\; in\; e_2) : \tau}
    \end{prooftree}}
    \\\\
  \end{array} \]

Before we detail the last two rules detailing instantiation and
generification we first outline the meaning of a judgement $\sigma
\sqsubseteq \sigma'$. Intuitively this means that $\sigma$ is some
subtype of $\sigma'$ - we can create it by some substitution of the
quantified variables in $\sigma'$. More precisely it is defined as
follows:

\[\begin{array}{c@{\qquad}c}  
    \mbox{\begin{prooftree}
      \hypo{\tau' = \{\alpha_i \mapsto \tau_i\}\tau}
      \hypo{\beta_i \notin free(\forall \alpha_1 \ldots \forall
        \alpha_n \cdot \tau)}
      \infer2[spec]{\forall \alpha_1 \ldots \forall \alpha_n \cdot
        \tau \sqsubseteq \forall \beta_1 \ldots \forall
        \beta_m \cdot \tau' }
    \end{prooftree}}
    \\\\
\end{array} \]  
  
Lastly we have rules to instantiate a type scheme, or generify a
type (i.e. to make a type more specific and narrow, or less specific
and general):

\[\begin{array}{c@{\qquad}c}  
    \mbox{\begin{prooftree}
      \hypo{\Gamma \vdash e : \sigma}
      \hypo{\sigma' \sqsubseteq \sigma}
      \infer2[inst]{\Gamma \vdash e : \sigma'}
    \end{prooftree}}
    &
    \mbox{\begin{prooftree}
      \hypo{\Gamma \vdash e : \sigma}
      \hypo{\alpha \notin free(\Gamma)}
      \infer2[gen]{\Gamma \vdash e : \forall \alpha \sigma}
    \end{prooftree}}
    \\\\
\end{array} \]

The latter three rules are the ones that capture the idea of
Hindley-Milner polymorphism.

\clearpage
\section{Example specifications}
\label{appendix:examplespecifications}

This appendix shows example specification files for various
languages.

\subsection{Simply typed lambda calculus with product types}
\label{STLCspec}

\begin{verbatim}
type: α
  value: a

type: β
  value: b

type: A -> B
  if:
    type A
    type B
  eliminated-by: E
    if:
      (A) <- E
    resulting-in-type: B
  value: \ X. -> M
    if:
      X : (A) |- (B) <- M
    reduces-to: M/[, E:A]
  expanded-by: \ Y. -> Y
  
type: A x B
  if:
    type A
    type B
  eliminated-by: fst
    resulting-in-type: A
  eliminated-by: snd
    resulting-in-type: B
  value: L and R
    if:
      (A) <- L
      (B) <- R
    reduces-to: L
    reduces-to: R
  expanded-by: fst , snd
\end{verbatim}

\clearpage
\subsection{System-F-like language}
\label{SystemFspec}

Our variation is slighly more general than the original encoding as
opposed to only polymorphic functions we give the means of
representing arbitrary polymorphic types.

\begin{verbatim}
type: α
  value: a

type: β
  value: b

type: A -> B
  if:
    type A
    type B
  eliminated-by: E
    if:
      (A) <- E
    resulting-in-type: B
  value: \ X. -> M
    if:
      X : (A) |- (B) <- M
    reduces-to: M/[, E:A]
  expanded-by: \ Y. -> Y

type: ∀ T. => M
  if:
    T : (set) |- type M
  eliminated-by: TY
    if:
      type TY
    resulting-in-type: M/[, TY:set]
  value: δ T. PTY
    if:
      T : (set) |- (M/[, .T]) <- PTY
    reduces-to: PTY/[, TY:set]
\end{verbatim}

\clearpage
\subsection{Lambda calculus variation with dependent types}
\label{DTLCspec}

\begin{verbatim}
type: α
  value: a1
  value: a2

type: isa TY TM
  if:
    type TY
    (TY) <- TM
  value: is V
    if:
      (TY) <- V
      (TM) = V

type: A X. -> B
  if:
    type A
    X : (A) |- type B
  eliminated-by: E
    if:
      (A) <- E
    resulting-in-type: B/[, E:A]
  value: \ X. -> M
    if:
      X : (A) |- (B/[, .X]) <- M
    reduces-to: M/[, E:A]
  expanded-by: \ Y. -> Y
\end{verbatim}
\clearpage
\section{Parser combinators}
\label{appendix-parsercombinators}

\begin{verbatim}
fail : Parser A
fail str = nothing

-- guarantees a parser, if it succeeds, consumes some input
safe : Parser A → Parser A
safe p s = do
             (a , leftover) ← p s
             if length leftover <ᵇ length s then just (a , leftover) else nothing
           where open maybemonad

peak : Parser Char
peak str with toList str
... | []      = nothing
... | c ∷ rest = just ((c , fromList (c ∷ rest)))

all : Parser String
all = λ s → just (s , "")

try : Parser A → A → Parser A
try p a str = p str <∣> just (a , str)

consumes : Parser A → Parser (ℕ × A)
consumes p = do
               bfr ← (λ s → just (length s , s))
               a ← p
               afr ← (λ s → just (length s , s))
               return (∣ bfr - afr ∣ , a)
             where open parsermonad

biggest-of_and_ : Parser A → Parser A → Parser A
biggest-of_and_ p1 p2 str with p1 str | p2 str
... | nothing | nothing = nothing
... | nothing | just p = just p
... | just p | nothing = just p
... | just (a1 , rst1) | just (a2 , rst2) = just (if length rst1 <ᵇ length rst2 then (a1 , rst1) else (a2 , rst2))

_or_ : Parser A → Parser B → Parser (A ⊎ B)
(pa or pb) str = (inj₁ <$> pa) str <∣> (inj₂ <$> pb) str
  where
    open parsermonad

either_or_ : Parser A → Parser A → Parser A
(either pa or pb) str = maybe′ just (pb str) (pa str)

ifp_then_else_ : Parser A → Parser B → Parser B → Parser B
(ifp p then pthen else pelse) str with p str
... | nothing = pelse str
... | just x  = pthen str

nout : Parser ⊤
nout = return tt
  where open parsermonad

optional : Parser A → Parser (A ⊎ ⊤)
optional = _or nout

complete : Parser A → Parser A
complete p = do
               a ← p
               rest ← all
               if rest ==ˢ "" then return a else fail
             where open parsermonad

takeIfc : (Char → Bool) → Parserp Char
takeIfc p []          = nothing
takeIfc p (c ∷ chars) = if p c then just (c , chars) else nothing

takeIf : (Char → Bool) → Parser Char
takeIf p = →[ takeIfc p ]→

-- This terminates as we ensure to make p "safe" before we
-- use it, forcing the parser to fail if it does not consume
-- any input
{-# TERMINATING #-}
_*[_,_] : Parser A → (A → B → B) → B → Parser B
p *[ f , b ] = do
                 inj₁ a ← optional (safe p)
                   where inj₂ _ → return b
                 b ← p *[ f , b ] 
                 return (f a b)
  where open parsermonad


_⁺[_,_] : Parser A → (A → B → B) → B → Parser B
p ⁺[ f , b ] = pure f ⊛ p ⊛ (p *[ f , b ])
  where open parsermonad

anyof : List (Parser A) → Parser A
anyof []       = fail
anyof (p ∷ ps) = either p or (anyof ps)

biggest-consumer : List (Parser A) → Parser A
biggest-consumer [] = fail
biggest-consumer (p ∷ ps) = biggest-of p and biggest-consumer ps

all-of : List (Parser A) → Parser (List A)
all-of [] str = just ([] , str)
all-of ps str = just (foldr (λ p las → maybe′ (λ (a , _) → a ∷ las) las (p str) ) [] ps , "")

{-# TERMINATING #-}
how-many? : Parser A → Parser (Σ[ n ∈ ℕ ] Vec A n)
how-many? p = ifp p then (do
                            a ←  safe p
                            (n , as) ← how-many? p
                            return (ℕ.suc n , a ∷ as))
              else return (0 , [])
  where open parsermonad

max_how-many? : ℕ → Parser A → Parser (Σ[ n ∈ ℕ ] Vec A n)
max zero how-many? _    = return (0 , [])
  where open parsermonad
max (suc n) how-many? p = ifp p then (do
                            a ←  safe p
                            (n , as) ← max n how-many? p
                            return (ℕ.suc n , a ∷ as))
              else return (0 , [])
  where open parsermonad

exactly : (n : ℕ) → Parser A → Parser (Vec A n)
exactly zero _  = return []
  where open parsermonad
exactly (suc n) p = do
                      a ← p
                      as ← exactly n p
                      return (a ∷ as)
-- text/number parsing

whitespace : Parser ⊤
whitespace str = just (tt , trim← str)

ws+nl : Parser ⊤
ws+nl str = just (tt , trim←p str)

ws+nl! : Parser ⊤
ws+nl! = do
            c ← takeIf Data.Char.isSpace
            ws+nl
           where open parsermonad

ws-tolerant : Parser A → Parser A
ws-tolerant p = do
                  whitespace
                  r ← p
                  whitespace
                  return r
  where open parsermonad

wsnl-tolerant : Parser A → Parser A
wsnl-tolerant p = do
                    ws+nl
                    r ← p
                    ws+nl
                    return r
  where open parsermonad

wsnl-tolerant! : Parser A → Parser A
wsnl-tolerant! p = do
                    ws+nl!
                    r ← p
                    ws+nl!
                    return r
  where open parsermonad

literalc : Char → Parserp Char
literalc c [] = nothing
literalc c (x ∷ rest)
  = if c == x then just (c , rest)
              else nothing

literal : Char → Parser Char
literal c = →[ literalc c ]→

newline = literal "\n"

nextCharc : Parserp Char
nextCharc []          = nothing
nextCharc (c ∷ chars) = just (c , chars)

nextChar = →[ nextCharc ]→

literalAsString = (fromChar <$>_) ∘′ literal
  where open parsermonad

string : String → Parser String
string str = foldr
                 (λ c p → (pure _++_) ⊛ (literalAsString c) ⊛ p)
                 (return "") (toList str)
  where open parsermonad

  
stringof : (Char → Bool) → Parser String
stringof p = takeIf p *[ _++_ ∘ fromChar , "" ]

until : (Char → Bool) → Parser String
until p = stringof (not ∘ p)

nonempty : Parser String → Parser String
nonempty p = do
               r ← p
               if r ==ˢ "" then (λ _ → nothing) else return r
  where open parsermonad

nat : Parser ℕ
nat = do
        d ← nonempty (stringof isDigit)
        return (toNat d)
      where open parsermonad


bracketed : Parser A → Parser A
bracketed p = do
                literal "("
                a ← wsnl-tolerant p
                literal ")"
                return a
   where open parsermonad

potentially-bracketed : Parser A → Parser A
potentially-bracketed p = either bracketed p or p

curlybracketed : Parser A → Parser A
curlybracketed p = do
                literal "{"
                a ← wsnl-tolerant p
                literal "}"
                return a
  where open parsermonad

squarebracketed : Parser A → Parser A
squarebracketed p = do
                literal "["
                a ← wsnl-tolerant p
                literal "]"
                return a
  where open parsermonad
  open Parsers
\end{verbatim}

\section{A selection of testing code}
\label{appendix-tests}

\subsection{Description of STLC}
\hide{
\begin{code}%
\>[0]\AgdaKeyword{module}\AgdaSpace{}%
\AgdaModule{Test.Specs.STLC}\AgdaSpace{}%
\AgdaKeyword{where}\<%
\end{code}
}

\begin{code}%
\>[0]\AgdaKeyword{open}\AgdaSpace{}%
\AgdaKeyword{import}\AgdaSpace{}%
\AgdaModule{CoreLanguage}\<%
\\
\>[0]\AgdaKeyword{open}\AgdaSpace{}%
\AgdaKeyword{import}\AgdaSpace{}%
\AgdaModule{Data.Nat}\AgdaSpace{}%
\AgdaKeyword{using}\AgdaSpace{}%
\AgdaSymbol{(}\AgdaInductiveConstructor{suc}\AgdaSymbol{)}\<%
\\
\>[0]\AgdaKeyword{open}\AgdaSpace{}%
\AgdaKeyword{import}\AgdaSpace{}%
\AgdaModule{Pattern}\AgdaSpace{}%
\AgdaKeyword{using}\AgdaSpace{}%
\AgdaSymbol{(}\AgdaDatatype{Pattern}\AgdaSymbol{;}\AgdaSpace{}%
\AgdaInductiveConstructor{`}\AgdaSymbol{;}\AgdaSpace{}%
\AgdaInductiveConstructor{place}\AgdaSymbol{;}\AgdaSpace{}%
\AgdaInductiveConstructor{bind}\AgdaSymbol{;}\AgdaSpace{}%
\AgdaOperator{\AgdaInductiveConstructor{\AgdaUnderscore{}∙\AgdaUnderscore{}}}\AgdaSymbol{;}%
\>[58]\AgdaInductiveConstructor{⋆}\AgdaSymbol{;}\AgdaSpace{}%
\AgdaOperator{\AgdaInductiveConstructor{\AgdaUnderscore{}∙}}\AgdaSymbol{;}\AgdaSpace{}%
\AgdaOperator{\AgdaInductiveConstructor{∙\AgdaUnderscore{}}}\AgdaSymbol{;}\AgdaSpace{}%
\AgdaDatatype{svar}\AgdaSymbol{)}\<%
\\
\>[0]\AgdaKeyword{open}\AgdaSpace{}%
\AgdaKeyword{import}\AgdaSpace{}%
\AgdaModule{Expression}\AgdaSpace{}%
\AgdaKeyword{using}\AgdaSpace{}%
\AgdaSymbol{(}\AgdaOperator{\AgdaInductiveConstructor{\AgdaUnderscore{}/\AgdaUnderscore{}}}\AgdaSymbol{;}\AgdaSpace{}%
\AgdaInductiveConstructor{`}\AgdaSymbol{;}\AgdaSpace{}%
\AgdaOperator{\AgdaInductiveConstructor{\AgdaUnderscore{}∙\AgdaUnderscore{}}}\AgdaSymbol{;}\AgdaSpace{}%
\AgdaOperator{\AgdaInductiveConstructor{\AgdaUnderscore{}∷\AgdaUnderscore{}}}\AgdaSymbol{)}\<%
\\
\>[0]\AgdaKeyword{open}\AgdaSpace{}%
\AgdaKeyword{import}\AgdaSpace{}%
\AgdaModule{Rules}\AgdaSpace{}%
\AgdaKeyword{using}%
\>[29I]\AgdaSymbol{(}\AgdaRecord{ElimRule}\AgdaSymbol{;}\AgdaSpace{}%
\AgdaRecord{TypeRule}\AgdaSymbol{;}\AgdaSpace{}%
\AgdaRecord{UnivRule}\AgdaSymbol{;}\AgdaSpace{}%
\AgdaRecord{∋rule}\AgdaSymbol{;}\AgdaSpace{}%
\AgdaInductiveConstructor{ε}\AgdaSymbol{;}\AgdaSpace{}%
\AgdaOperator{\AgdaFunction{\AgdaUnderscore{}placeless}}\AgdaSymbol{;}\AgdaSpace{}%
\AgdaInductiveConstructor{type}\AgdaSymbol{;}\<%
\\
\>[29I][@{}l@{\AgdaIndent{0}}]%
\>[25]\AgdaOperator{\AgdaInductiveConstructor{\AgdaUnderscore{}⇉\AgdaUnderscore{}}}\AgdaSymbol{;}\AgdaSpace{}%
\AgdaOperator{\AgdaInductiveConstructor{\AgdaUnderscore{}⊢'\AgdaUnderscore{}}}\AgdaSymbol{;}\AgdaSpace{}%
\AgdaOperator{\AgdaInductiveConstructor{\AgdaUnderscore{}∋'\AgdaUnderscore{}[\AgdaUnderscore{}]}}\AgdaSymbol{;}\AgdaSpace{}%
\AgdaInductiveConstructor{`}\AgdaSymbol{)}\<%
\\
\>[0]\AgdaKeyword{open}\AgdaSpace{}%
\AgdaKeyword{import}\AgdaSpace{}%
\AgdaModule{Thinning}\AgdaSpace{}%
\AgdaKeyword{using}\AgdaSpace{}%
\AgdaSymbol{(}\AgdaFunction{Ø}\AgdaSymbol{;}\AgdaSpace{}%
\AgdaOperator{\AgdaInductiveConstructor{\AgdaUnderscore{}O}}\AgdaSymbol{;}\AgdaSpace{}%
\AgdaFunction{ι}\AgdaSymbol{)}\<%
\\
\>[0]\AgdaKeyword{open}\AgdaSpace{}%
\AgdaKeyword{import}\AgdaSpace{}%
\AgdaModule{BwdVec}\AgdaSpace{}%
\AgdaKeyword{using}\AgdaSpace{}%
\AgdaSymbol{(}\AgdaInductiveConstructor{ε}\AgdaSymbol{)}\<%
\\
\>[0]\AgdaKeyword{open}\AgdaSpace{}%
\AgdaKeyword{import}\AgdaSpace{}%
\AgdaModule{Data.Product}\AgdaSpace{}%
\AgdaKeyword{using}\AgdaSpace{}%
\AgdaSymbol{(}\AgdaOperator{\AgdaInductiveConstructor{\AgdaUnderscore{},\AgdaUnderscore{}}}\AgdaSymbol{)}\<%
\\
\>[0]\AgdaKeyword{open}\AgdaSpace{}%
\AgdaKeyword{import}\AgdaSpace{}%
\AgdaModule{TypeChecker}\AgdaSpace{}%
\AgdaKeyword{using}\AgdaSpace{}%
\AgdaSymbol{(}\AgdaDatatype{RuleSet}\AgdaSymbol{;}\AgdaSpace{}%
\AgdaInductiveConstructor{rs}\AgdaSymbol{)}\<%
\\
\>[0]\AgdaKeyword{open}\AgdaSpace{}%
\AgdaKeyword{import}\AgdaSpace{}%
\AgdaModule{Semantics}\AgdaSpace{}%
\AgdaKeyword{renaming}\AgdaSpace{}%
\AgdaSymbol{(}\AgdaRecord{β-rule}\AgdaSpace{}%
\AgdaSymbol{to}\AgdaSpace{}%
\AgdaRecord{β-Rule}\AgdaSymbol{)}\<%
\\
\>[0]\AgdaKeyword{open}\AgdaSpace{}%
\AgdaKeyword{import}\AgdaSpace{}%
\AgdaModule{EtaRule}\AgdaSpace{}%
\AgdaKeyword{using}\AgdaSpace{}%
\AgdaSymbol{(}\AgdaRecord{η-Rule}\AgdaSymbol{)}\<%
\\
\>[0]\AgdaKeyword{open}\AgdaSpace{}%
\AgdaKeyword{import}\AgdaSpace{}%
\AgdaModule{BwdVec}\<%
\\
\>[0]\AgdaKeyword{open}\AgdaSpace{}%
\AgdaModule{β-Rule}\<%
\\
\>[0]\AgdaKeyword{open}\AgdaSpace{}%
\AgdaModule{ElimRule}\<%
\\
\>[0]\AgdaKeyword{open}\AgdaSpace{}%
\AgdaModule{TypeRule}\<%
\\
\>[0]\AgdaKeyword{open}\AgdaSpace{}%
\AgdaModule{UnivRule}\<%
\\
\>[0]\AgdaKeyword{open}\AgdaSpace{}%
\AgdaModule{∋rule}\<%
\\
\>[0]\AgdaKeyword{open}\AgdaSpace{}%
\AgdaModule{η-Rule}\<%
\end{code}

We begin by introducing some combinators to construct our language terms
as creating these terms directly in our internal language can be tedious.

\begin{code}%
\>[0]\AgdaKeyword{module}\AgdaSpace{}%
\AgdaModule{combinators}\AgdaSpace{}%
\AgdaKeyword{where}\<%
\\
\>[0][@{}l@{\AgdaIndent{0}}]%
\>[2]\AgdaFunction{α}\AgdaSpace{}%
\AgdaSymbol{:}\AgdaSpace{}%
\AgdaSymbol{∀\{}\AgdaBound{γ}\AgdaSymbol{\}}\AgdaSpace{}%
\AgdaSymbol{→}\AgdaSpace{}%
\AgdaFunction{Term}\AgdaSpace{}%
\AgdaInductiveConstructor{const}\AgdaSpace{}%
\AgdaBound{γ}\<%
\\
%
\>[2]\AgdaFunction{α}\AgdaSpace{}%
\AgdaSymbol{=}\AgdaSpace{}%
\AgdaInductiveConstructor{`}\AgdaSpace{}%
\AgdaString{"α"}\<%
\\
\>[0]\<%
\\
%
\>[2]\AgdaFunction{a}\AgdaSpace{}%
\AgdaSymbol{:}\AgdaSpace{}%
\AgdaSymbol{∀\{}\AgdaBound{γ}\AgdaSymbol{\}}\AgdaSpace{}%
\AgdaSymbol{→}\AgdaSpace{}%
\AgdaFunction{Term}\AgdaSpace{}%
\AgdaInductiveConstructor{const}\AgdaSpace{}%
\AgdaBound{γ}\<%
\\
%
\>[2]\AgdaFunction{a}\AgdaSpace{}%
\AgdaSymbol{=}\AgdaSpace{}%
\AgdaInductiveConstructor{`}\AgdaSpace{}%
\AgdaString{"a"}\<%
\\
\>[0]\<%
\\
%
\>[2]\AgdaFunction{β}\AgdaSpace{}%
\AgdaSymbol{:}\AgdaSpace{}%
\AgdaSymbol{∀\{}\AgdaBound{γ}\AgdaSymbol{\}}\AgdaSpace{}%
\AgdaSymbol{→}\AgdaSpace{}%
\AgdaFunction{Term}\AgdaSpace{}%
\AgdaInductiveConstructor{const}\AgdaSpace{}%
\AgdaBound{γ}\<%
\\
%
\>[2]\AgdaFunction{β}\AgdaSpace{}%
\AgdaSymbol{=}\AgdaSpace{}%
\AgdaInductiveConstructor{`}\AgdaSpace{}%
\AgdaString{"β"}\<%
\\
\>[0]\<%
\\
%
\>[2]\AgdaFunction{b}\AgdaSpace{}%
\AgdaSymbol{:}\AgdaSpace{}%
\AgdaSymbol{∀\{}\AgdaBound{γ}\AgdaSymbol{\}}\AgdaSpace{}%
\AgdaSymbol{→}\AgdaSpace{}%
\AgdaFunction{Term}\AgdaSpace{}%
\AgdaInductiveConstructor{const}\AgdaSpace{}%
\AgdaBound{γ}\<%
\\
%
\>[2]\AgdaFunction{b}\AgdaSpace{}%
\AgdaSymbol{=}\AgdaSpace{}%
\AgdaInductiveConstructor{`}\AgdaSpace{}%
\AgdaString{"b"}\<%
\\
\>[0]\<%
\\
%
\>[2]\AgdaOperator{\AgdaFunction{\AgdaUnderscore{}⇨\AgdaUnderscore{}}}\AgdaSpace{}%
\AgdaSymbol{:}\AgdaSpace{}%
\AgdaSymbol{∀\{}\AgdaBound{γ}\AgdaSymbol{\}}\AgdaSpace{}%
\AgdaSymbol{→}\AgdaSpace{}%
\AgdaDatatype{Const}\AgdaSpace{}%
\AgdaBound{γ}\AgdaSpace{}%
\AgdaSymbol{→}\AgdaSpace{}%
\AgdaDatatype{Const}\AgdaSpace{}%
\AgdaBound{γ}\AgdaSpace{}%
\AgdaSymbol{→}\AgdaSpace{}%
\AgdaFunction{Term}\AgdaSpace{}%
\AgdaInductiveConstructor{const}\AgdaSpace{}%
\AgdaBound{γ}\<%
\\
%
\>[2]\AgdaBound{x}\AgdaSpace{}%
\AgdaOperator{\AgdaFunction{⇨}}\AgdaSpace{}%
\AgdaBound{y}\AgdaSpace{}%
\AgdaSymbol{=}\AgdaSpace{}%
\AgdaBound{x}\AgdaSpace{}%
\AgdaOperator{\AgdaInductiveConstructor{∙}}\AgdaSpace{}%
\AgdaSymbol{((}\AgdaInductiveConstructor{`}\AgdaSpace{}%
\AgdaString{"→"}\AgdaSymbol{)}\AgdaSpace{}%
\AgdaOperator{\AgdaInductiveConstructor{∙}}\AgdaSpace{}%
\AgdaBound{y}\AgdaSymbol{)}\<%
\\
%
\>[2]\AgdaKeyword{infixr}\AgdaSpace{}%
\AgdaNumber{20}\AgdaSpace{}%
\AgdaOperator{\AgdaFunction{\AgdaUnderscore{}⇨\AgdaUnderscore{}}}\<%
\\
\>[0]\<%
\\
%
\>[2]\AgdaFunction{lam}\AgdaSpace{}%
\AgdaSymbol{:}\AgdaSpace{}%
\AgdaSymbol{∀}\AgdaSpace{}%
\AgdaSymbol{\{}\AgdaBound{γ}\AgdaSymbol{\}}\AgdaSpace{}%
\AgdaSymbol{→}\AgdaSpace{}%
\AgdaFunction{Term}\AgdaSpace{}%
\AgdaInductiveConstructor{const}\AgdaSpace{}%
\AgdaSymbol{(}\AgdaInductiveConstructor{suc}\AgdaSpace{}%
\AgdaBound{γ}\AgdaSymbol{)}\AgdaSpace{}%
\AgdaSymbol{→}\AgdaSpace{}%
\AgdaFunction{Term}\AgdaSpace{}%
\AgdaInductiveConstructor{const}\AgdaSpace{}%
\AgdaBound{γ}\<%
\\
%
\>[2]\AgdaFunction{lam}\AgdaSpace{}%
\AgdaBound{t}\AgdaSpace{}%
\AgdaSymbol{=}\AgdaSpace{}%
\AgdaInductiveConstructor{`}\AgdaSpace{}%
\AgdaString{"λ"}\AgdaSpace{}%
\AgdaOperator{\AgdaInductiveConstructor{∙}}\AgdaSpace{}%
\AgdaInductiveConstructor{bind}\AgdaSpace{}%
\AgdaBound{t}\<%
\\
\>[0]\<%
\\
%
\>[2]\AgdaFunction{\textasciitilde{}}\AgdaSpace{}%
\AgdaSymbol{:}\AgdaSpace{}%
\AgdaSymbol{∀}\AgdaSpace{}%
\AgdaSymbol{\{}\AgdaBound{γ}\AgdaSymbol{\}}\AgdaSpace{}%
\AgdaSymbol{→}\AgdaSpace{}%
\AgdaDatatype{Var}\AgdaSpace{}%
\AgdaBound{γ}\AgdaSpace{}%
\AgdaSymbol{→}\AgdaSpace{}%
\AgdaFunction{Term}\AgdaSpace{}%
\AgdaInductiveConstructor{const}\AgdaSpace{}%
\AgdaBound{γ}\<%
\\
%
\>[2]\AgdaFunction{\textasciitilde{}}\AgdaSpace{}%
\AgdaBound{vr}\AgdaSpace{}%
\AgdaSymbol{=}\AgdaSpace{}%
\AgdaInductiveConstructor{thunk}\AgdaSpace{}%
\AgdaSymbol{(}\AgdaInductiveConstructor{var}\AgdaSpace{}%
\AgdaBound{vr}\AgdaSymbol{)}\<%
\\
\>[0]\<%
\\
%
\>[2]\AgdaFunction{app}\AgdaSpace{}%
\AgdaSymbol{:}\AgdaSpace{}%
\AgdaSymbol{∀}\AgdaSpace{}%
\AgdaSymbol{\{}\AgdaBound{γ}\AgdaSymbol{\}}\AgdaSpace{}%
\AgdaSymbol{→}\AgdaSpace{}%
\AgdaDatatype{Compu}\AgdaSpace{}%
\AgdaBound{γ}\AgdaSpace{}%
\AgdaSymbol{→}\AgdaSpace{}%
\AgdaDatatype{Const}\AgdaSpace{}%
\AgdaBound{γ}\AgdaSpace{}%
\AgdaSymbol{→}\AgdaSpace{}%
\AgdaFunction{Term}\AgdaSpace{}%
\AgdaInductiveConstructor{compu}\AgdaSpace{}%
\AgdaBound{γ}\<%
\\
%
\>[2]\AgdaFunction{app}\AgdaSpace{}%
\AgdaBound{e}\AgdaSpace{}%
\AgdaBound{s}\AgdaSpace{}%
\AgdaSymbol{=}\AgdaSpace{}%
\AgdaInductiveConstructor{elim}\AgdaSpace{}%
\AgdaBound{e}\AgdaSpace{}%
\AgdaBound{s}\<%
\end{code}

\begin{code}%
\>[0]\AgdaComment{-- Can we model STLC?}\<%
\\
%
\\[\AgdaEmptyExtraSkip]%
\>[0]\AgdaComment{-- we have a universe}\<%
\\
\>[0]\AgdaFunction{U}\AgdaSpace{}%
\AgdaSymbol{:}\AgdaSpace{}%
\AgdaDatatype{Pattern}\AgdaSpace{}%
\AgdaNumber{0}\<%
\\
\>[0]\AgdaFunction{U}\AgdaSpace{}%
\AgdaSymbol{=}\AgdaSpace{}%
\AgdaInductiveConstructor{`}\AgdaSpace{}%
\AgdaString{"U"}\<%
\\
%
\\[\AgdaEmptyExtraSkip]%
\>[0]\AgdaFunction{U-type}\AgdaSpace{}%
\AgdaSymbol{:}\AgdaSpace{}%
\AgdaRecord{TypeRule}\<%
\\
\>[0]\AgdaField{subject}%
\>[9]\AgdaFunction{U-type}\AgdaSpace{}%
\AgdaSymbol{=}\AgdaSpace{}%
\AgdaFunction{U}\<%
\\
\>[0]\AgdaField{premises}\AgdaSpace{}%
\AgdaFunction{U-type}\AgdaSpace{}%
\AgdaSymbol{=}\AgdaSpace{}%
\AgdaInductiveConstructor{`}\AgdaSpace{}%
\AgdaString{"⊤"}\AgdaSpace{}%
\AgdaOperator{\AgdaInductiveConstructor{,}}\AgdaSpace{}%
\AgdaSymbol{(}\AgdaInductiveConstructor{ε}\AgdaSpace{}%
\AgdaSymbol{(}\AgdaFunction{U}\AgdaSpace{}%
\AgdaOperator{\AgdaFunction{placeless}}\AgdaSymbol{))}\<%
\\
%
\\[\AgdaEmptyExtraSkip]%
\>[0]\AgdaFunction{U-univ}\AgdaSpace{}%
\AgdaSymbol{:}\AgdaSpace{}%
\AgdaRecord{UnivRule}\<%
\\
\>[0]\AgdaField{input}%
\>[9]\AgdaFunction{U-univ}\AgdaSpace{}%
\AgdaSymbol{=}\AgdaSpace{}%
\AgdaFunction{U}\<%
\\
\>[0]\AgdaField{premises}\AgdaSpace{}%
\AgdaFunction{U-univ}\AgdaSpace{}%
\AgdaSymbol{=}\AgdaSpace{}%
\AgdaField{input}\AgdaSpace{}%
\AgdaFunction{U-univ}\AgdaSpace{}%
\AgdaOperator{\AgdaInductiveConstructor{,}}\AgdaSpace{}%
\AgdaSymbol{(}\AgdaInductiveConstructor{ε}\AgdaSpace{}%
\AgdaSymbol{(}\AgdaInductiveConstructor{`}\AgdaSpace{}%
\AgdaString{"⊤"}\AgdaSymbol{))}\<%
\\
%
\\[\AgdaEmptyExtraSkip]%
\>[0]\AgdaComment{-- a base type α in the universe}\<%
\\
%
\\[\AgdaEmptyExtraSkip]%
\>[0]\AgdaFunction{α}\AgdaSpace{}%
\AgdaSymbol{:}\AgdaSpace{}%
\AgdaDatatype{Pattern}\AgdaSpace{}%
\AgdaNumber{0}\<%
\\
\>[0]\AgdaFunction{α}\AgdaSpace{}%
\AgdaSymbol{=}\AgdaSpace{}%
\AgdaInductiveConstructor{`}\AgdaSpace{}%
\AgdaString{"α"}\<%
\\
%
\\[\AgdaEmptyExtraSkip]%
\>[0]\AgdaFunction{α-rule}\AgdaSpace{}%
\AgdaSymbol{:}\AgdaSpace{}%
\AgdaRecord{TypeRule}\<%
\\
\>[0]\AgdaField{subject}%
\>[9]\AgdaFunction{α-rule}\AgdaSpace{}%
\AgdaSymbol{=}\AgdaSpace{}%
\AgdaFunction{α}\<%
\\
\>[0]\AgdaField{premises}\AgdaSpace{}%
\AgdaFunction{α-rule}\AgdaSpace{}%
\AgdaSymbol{=}\AgdaSpace{}%
\AgdaSymbol{(}\AgdaInductiveConstructor{`}\AgdaSpace{}%
\AgdaString{"⊤"}\AgdaSymbol{)}\AgdaSpace{}%
\AgdaOperator{\AgdaInductiveConstructor{,}}\AgdaSpace{}%
\AgdaSymbol{(}\AgdaInductiveConstructor{ε}\AgdaSpace{}%
\AgdaSymbol{(}\AgdaFunction{α}\AgdaSpace{}%
\AgdaOperator{\AgdaFunction{placeless}}\AgdaSymbol{))}\<%
\\
%
\\[\AgdaEmptyExtraSkip]%
\>[0]\AgdaFunction{α-inuniv}\AgdaSpace{}%
\AgdaSymbol{:}\AgdaSpace{}%
\AgdaRecord{∋rule}\<%
\\
\>[0]\AgdaField{subject}%
\>[9]\AgdaFunction{α-inuniv}\AgdaSpace{}%
\AgdaSymbol{=}\AgdaSpace{}%
\AgdaFunction{α}\<%
\\
\>[0]\AgdaField{input}%
\>[9]\AgdaFunction{α-inuniv}\AgdaSpace{}%
\AgdaSymbol{=}\AgdaSpace{}%
\AgdaFunction{U}\<%
\\
\>[0]\AgdaField{premises}\AgdaSpace{}%
\AgdaFunction{α-inuniv}\AgdaSpace{}%
\AgdaSymbol{=}\AgdaSpace{}%
\AgdaSymbol{(}\AgdaInductiveConstructor{`}\AgdaSpace{}%
\AgdaString{"U"}\AgdaSymbol{)}\AgdaSpace{}%
\AgdaOperator{\AgdaInductiveConstructor{,}}\AgdaSpace{}%
\AgdaSymbol{(}\AgdaInductiveConstructor{ε}\AgdaSpace{}%
\AgdaSymbol{(}\AgdaFunction{α}\AgdaSpace{}%
\AgdaOperator{\AgdaFunction{placeless}}\AgdaSymbol{))}\<%
\\
%
\\[\AgdaEmptyExtraSkip]%
\>[0]\AgdaComment{-- which has a value "a"}\<%
\\
\>[0]\AgdaFunction{a}\AgdaSpace{}%
\AgdaSymbol{:}\AgdaSpace{}%
\AgdaDatatype{Pattern}\AgdaSpace{}%
\AgdaNumber{0}\<%
\\
\>[0]\AgdaFunction{a}\AgdaSpace{}%
\AgdaSymbol{=}\AgdaSpace{}%
\AgdaInductiveConstructor{`}\AgdaSpace{}%
\AgdaString{"a"}\<%
\\
%
\\[\AgdaEmptyExtraSkip]%
\>[0]\AgdaFunction{a-rule}\AgdaSpace{}%
\AgdaSymbol{:}\AgdaSpace{}%
\AgdaRecord{∋rule}\<%
\\
\>[0]\AgdaField{subject}%
\>[9]\AgdaFunction{a-rule}\AgdaSpace{}%
\AgdaSymbol{=}\AgdaSpace{}%
\AgdaFunction{a}\<%
\\
\>[0]\AgdaField{input}%
\>[9]\AgdaFunction{a-rule}\AgdaSpace{}%
\AgdaSymbol{=}\AgdaSpace{}%
\AgdaFunction{α}\<%
\\
\>[0]\AgdaField{premises}\AgdaSpace{}%
\AgdaFunction{a-rule}\AgdaSpace{}%
\AgdaSymbol{=}\AgdaSpace{}%
\AgdaSymbol{(}\AgdaInductiveConstructor{`}\AgdaSpace{}%
\AgdaString{"α"}\AgdaSymbol{)}\AgdaSpace{}%
\AgdaOperator{\AgdaInductiveConstructor{,}}\AgdaSpace{}%
\AgdaSymbol{(}\AgdaInductiveConstructor{ε}\AgdaSpace{}%
\AgdaSymbol{(}\AgdaFunction{a}\AgdaSpace{}%
\AgdaOperator{\AgdaFunction{placeless}}\AgdaSymbol{))}\<%
\\
%
\\[\AgdaEmptyExtraSkip]%
%
\\[\AgdaEmptyExtraSkip]%
\>[0]\AgdaFunction{β}\AgdaSpace{}%
\AgdaSymbol{:}\AgdaSpace{}%
\AgdaDatatype{Pattern}\AgdaSpace{}%
\AgdaNumber{0}\<%
\\
\>[0]\AgdaFunction{β}\AgdaSpace{}%
\AgdaSymbol{=}\AgdaSpace{}%
\AgdaInductiveConstructor{`}\AgdaSpace{}%
\AgdaString{"β"}\<%
\\
%
\\[\AgdaEmptyExtraSkip]%
\>[0]\AgdaFunction{β-rule}\AgdaSpace{}%
\AgdaSymbol{:}\AgdaSpace{}%
\AgdaRecord{TypeRule}\<%
\\
\>[0]\AgdaField{subject}%
\>[9]\AgdaFunction{β-rule}\AgdaSpace{}%
\AgdaSymbol{=}\AgdaSpace{}%
\AgdaFunction{β}\<%
\\
\>[0]\AgdaField{premises}\AgdaSpace{}%
\AgdaFunction{β-rule}\AgdaSpace{}%
\AgdaSymbol{=}\AgdaSpace{}%
\AgdaSymbol{(}\AgdaInductiveConstructor{`}\AgdaSpace{}%
\AgdaString{"⊤"}\AgdaSymbol{)}\AgdaSpace{}%
\AgdaOperator{\AgdaInductiveConstructor{,}}\AgdaSpace{}%
\AgdaSymbol{(}\AgdaInductiveConstructor{ε}\AgdaSpace{}%
\AgdaSymbol{(}\AgdaFunction{β}\AgdaSpace{}%
\AgdaOperator{\AgdaFunction{placeless}}\AgdaSymbol{))}\<%
\\
%
\\[\AgdaEmptyExtraSkip]%
\>[0]\AgdaFunction{β-inuniv}\AgdaSpace{}%
\AgdaSymbol{:}\AgdaSpace{}%
\AgdaRecord{∋rule}\<%
\\
\>[0]\AgdaField{subject}%
\>[9]\AgdaFunction{β-inuniv}\AgdaSpace{}%
\AgdaSymbol{=}\AgdaSpace{}%
\AgdaFunction{β}\<%
\\
\>[0]\AgdaField{input}%
\>[9]\AgdaFunction{β-inuniv}\AgdaSpace{}%
\AgdaSymbol{=}\AgdaSpace{}%
\AgdaFunction{U}\<%
\\
\>[0]\AgdaField{premises}\AgdaSpace{}%
\AgdaFunction{β-inuniv}\AgdaSpace{}%
\AgdaSymbol{=}\AgdaSpace{}%
\AgdaSymbol{(}\AgdaInductiveConstructor{`}\AgdaSpace{}%
\AgdaString{"U"}\AgdaSymbol{)}\AgdaSpace{}%
\AgdaOperator{\AgdaInductiveConstructor{,}}\AgdaSpace{}%
\AgdaSymbol{(}\AgdaInductiveConstructor{ε}\AgdaSpace{}%
\AgdaSymbol{(}\AgdaFunction{β}\AgdaSpace{}%
\AgdaOperator{\AgdaFunction{placeless}}\AgdaSymbol{))}\<%
\\
%
\\[\AgdaEmptyExtraSkip]%
%
\\[\AgdaEmptyExtraSkip]%
\>[0]\AgdaComment{-- and a value "b"}\<%
\\
\>[0]\AgdaFunction{b}\AgdaSpace{}%
\AgdaSymbol{:}\AgdaSpace{}%
\AgdaDatatype{Pattern}\AgdaSpace{}%
\AgdaNumber{0}\<%
\\
\>[0]\AgdaFunction{b}\AgdaSpace{}%
\AgdaSymbol{=}\AgdaSpace{}%
\AgdaInductiveConstructor{`}\AgdaSpace{}%
\AgdaString{"b"}\<%
\\
%
\\[\AgdaEmptyExtraSkip]%
\>[0]\AgdaFunction{b-rule}\AgdaSpace{}%
\AgdaSymbol{:}\AgdaSpace{}%
\AgdaRecord{∋rule}\<%
\\
\>[0]\AgdaField{subject}%
\>[9]\AgdaFunction{b-rule}\AgdaSpace{}%
\AgdaSymbol{=}\AgdaSpace{}%
\AgdaFunction{b}\<%
\\
\>[0]\AgdaField{input}%
\>[9]\AgdaFunction{b-rule}\AgdaSpace{}%
\AgdaSymbol{=}\AgdaSpace{}%
\AgdaFunction{β}\<%
\\
\>[0]\AgdaField{premises}\AgdaSpace{}%
\AgdaFunction{b-rule}\AgdaSpace{}%
\AgdaSymbol{=}\AgdaSpace{}%
\AgdaSymbol{(}\AgdaInductiveConstructor{`}\AgdaSpace{}%
\AgdaString{"β"}\AgdaSymbol{)}\AgdaSpace{}%
\AgdaOperator{\AgdaInductiveConstructor{,}}\AgdaSpace{}%
\AgdaSymbol{(}\AgdaInductiveConstructor{ε}\AgdaSpace{}%
\AgdaSymbol{(}\AgdaFunction{b}\AgdaSpace{}%
\AgdaOperator{\AgdaFunction{placeless}}\AgdaSymbol{))}\<%
\\
%
\\[\AgdaEmptyExtraSkip]%
\>[0]\AgdaComment{-- REMEMBER TO ADD RULE TO BOTTOM!!!}\<%
\\
%
\\[\AgdaEmptyExtraSkip]%
%
\\[\AgdaEmptyExtraSkip]%
\>[0]\AgdaComment{-- and a function type \AgdaUnderscore{}⇛\AgdaUnderscore{} in the universe}\<%
\\
\>[0]\AgdaFunction{⇛}\AgdaSpace{}%
\AgdaSymbol{:}\AgdaSpace{}%
\AgdaDatatype{Pattern}\AgdaSpace{}%
\AgdaNumber{0}\<%
\\
\>[0]\AgdaFunction{⇛}\AgdaSpace{}%
\AgdaSymbol{=}\AgdaSpace{}%
\AgdaInductiveConstructor{place}\AgdaSpace{}%
\AgdaFunction{ι}\AgdaSpace{}%
\AgdaOperator{\AgdaInductiveConstructor{∙}}\AgdaSpace{}%
\AgdaInductiveConstructor{`}\AgdaSpace{}%
\AgdaString{"→"}\AgdaSpace{}%
\AgdaOperator{\AgdaInductiveConstructor{∙}}\AgdaSpace{}%
\AgdaInductiveConstructor{place}\AgdaSpace{}%
\AgdaFunction{ι}\<%
\\
%
\\[\AgdaEmptyExtraSkip]%
\>[0]\AgdaFunction{⇛-rule}\AgdaSpace{}%
\AgdaSymbol{:}\AgdaSpace{}%
\AgdaRecord{TypeRule}\<%
\\
\>[0]\AgdaField{subject}%
\>[9]\AgdaFunction{⇛-rule}\AgdaSpace{}%
\AgdaSymbol{=}\AgdaSpace{}%
\AgdaFunction{⇛}\<%
\\
\>[0]\AgdaField{premises}\AgdaSpace{}%
\AgdaFunction{⇛-rule}\AgdaSpace{}%
\AgdaSymbol{=}\AgdaSpace{}%
\AgdaSymbol{((}\AgdaInductiveConstructor{`}\AgdaSpace{}%
\AgdaString{"⊤"}\AgdaSpace{}%
\AgdaOperator{\AgdaInductiveConstructor{∙}}\AgdaSpace{}%
\AgdaInductiveConstructor{place}\AgdaSpace{}%
\AgdaFunction{ι}\AgdaSymbol{)}\AgdaSpace{}%
\AgdaOperator{\AgdaInductiveConstructor{∙}}\AgdaSpace{}%
\AgdaInductiveConstructor{place}\AgdaSpace{}%
\AgdaFunction{ι}\AgdaSymbol{)}\AgdaSpace{}%
\AgdaOperator{\AgdaInductiveConstructor{,}}%
\>[351I]\AgdaSymbol{((}\AgdaInductiveConstructor{type}\AgdaSpace{}%
\AgdaSymbol{(}\AgdaInductiveConstructor{⋆}\AgdaSpace{}%
\AgdaOperator{\AgdaInductiveConstructor{∙}}\AgdaSymbol{)}\AgdaSpace{}%
\AgdaFunction{ι}\AgdaSpace{}%
\AgdaOperator{\AgdaInductiveConstructor{⇉}}\<%
\\
\>[351I][@{}l@{\AgdaIndent{0}}]%
\>[52]\AgdaInductiveConstructor{type}\AgdaSpace{}%
\AgdaSymbol{(}\AgdaOperator{\AgdaInductiveConstructor{∙}}\AgdaSpace{}%
\AgdaOperator{\AgdaInductiveConstructor{∙}}\AgdaSpace{}%
\AgdaInductiveConstructor{⋆}\AgdaSymbol{)}\AgdaSpace{}%
\AgdaFunction{ι}\AgdaSpace{}%
\AgdaOperator{\AgdaInductiveConstructor{⇉}}\<%
\\
%
\>[52]\AgdaInductiveConstructor{ε}\AgdaSpace{}%
\AgdaSymbol{(}\AgdaFunction{⇛}\AgdaSpace{}%
\AgdaOperator{\AgdaFunction{placeless}}\AgdaSymbol{)))}\<%
\\
%
\\[\AgdaEmptyExtraSkip]%
\>[0]\AgdaFunction{⇛-inuniv}\AgdaSpace{}%
\AgdaSymbol{:}\AgdaSpace{}%
\AgdaRecord{∋rule}\<%
\\
\>[0]\AgdaField{subject}%
\>[9]\AgdaFunction{⇛-inuniv}\AgdaSpace{}%
\AgdaSymbol{=}\AgdaSpace{}%
\AgdaFunction{⇛}\<%
\\
\>[0]\AgdaField{input}%
\>[9]\AgdaFunction{⇛-inuniv}\AgdaSpace{}%
\AgdaSymbol{=}\AgdaSpace{}%
\AgdaFunction{U}\<%
\\
\>[0]\AgdaField{premises}\AgdaSpace{}%
\AgdaFunction{⇛-inuniv}\AgdaSpace{}%
\AgdaSymbol{=}\AgdaSpace{}%
\AgdaSymbol{(((}\AgdaFunction{U}\AgdaSpace{}%
\AgdaOperator{\AgdaInductiveConstructor{∙}}\AgdaSpace{}%
\AgdaInductiveConstructor{place}\AgdaSpace{}%
\AgdaFunction{ι}\AgdaSymbol{)}\AgdaSpace{}%
\AgdaOperator{\AgdaInductiveConstructor{∙}}\AgdaSpace{}%
\AgdaInductiveConstructor{place}\AgdaSpace{}%
\AgdaFunction{ι}\AgdaSymbol{))}\AgdaSpace{}%
\AgdaOperator{\AgdaInductiveConstructor{,}}%
\>[379I]\AgdaSymbol{((}\AgdaInductiveConstructor{type}\AgdaSpace{}%
\AgdaSymbol{(}\AgdaInductiveConstructor{⋆}\AgdaSpace{}%
\AgdaOperator{\AgdaInductiveConstructor{∙}}\AgdaSymbol{)}\AgdaSpace{}%
\AgdaFunction{ι}%
\>[66]\AgdaOperator{\AgdaInductiveConstructor{⇉}}\<%
\\
\>[379I][@{}l@{\AgdaIndent{0}}]%
\>[52]\AgdaInductiveConstructor{type}\AgdaSpace{}%
\AgdaSymbol{(}\AgdaOperator{\AgdaInductiveConstructor{∙}}\AgdaSpace{}%
\AgdaOperator{\AgdaInductiveConstructor{∙}}\AgdaSpace{}%
\AgdaInductiveConstructor{⋆}\AgdaSymbol{)}\AgdaSpace{}%
\AgdaFunction{ι}\AgdaSpace{}%
\AgdaOperator{\AgdaInductiveConstructor{⇉}}\<%
\\
%
\>[52]\AgdaInductiveConstructor{ε}\AgdaSpace{}%
\AgdaSymbol{(}\AgdaFunction{⇛}\AgdaSpace{}%
\AgdaOperator{\AgdaFunction{placeless}}\AgdaSymbol{)))}\<%
\\
%
\\[\AgdaEmptyExtraSkip]%
\>[0]\AgdaComment{-- which has lambda terms as it's values}\<%
\\
\>[0]\AgdaFunction{lam}\AgdaSpace{}%
\AgdaSymbol{:}\AgdaSpace{}%
\AgdaDatatype{Pattern}\AgdaSpace{}%
\AgdaNumber{0}\<%
\\
\>[0]\AgdaFunction{lam}\AgdaSpace{}%
\AgdaSymbol{=}\AgdaSpace{}%
\AgdaInductiveConstructor{`}\AgdaSpace{}%
\AgdaString{"λ"}\AgdaSpace{}%
\AgdaOperator{\AgdaInductiveConstructor{∙}}\AgdaSpace{}%
\AgdaInductiveConstructor{bind}\AgdaSpace{}%
\AgdaSymbol{(}\AgdaInductiveConstructor{place}\AgdaSpace{}%
\AgdaFunction{ι}\AgdaSymbol{)}\<%
\\
%
\\[\AgdaEmptyExtraSkip]%
\>[0]\AgdaComment{-- we check the type of abstractions}\<%
\\
\>[0]\AgdaFunction{lam-rule}\AgdaSpace{}%
\AgdaSymbol{:}\AgdaSpace{}%
\AgdaRecord{∋rule}\<%
\\
\>[0]\AgdaField{subject}%
\>[9]\AgdaFunction{lam-rule}\AgdaSpace{}%
\AgdaSymbol{=}\AgdaSpace{}%
\AgdaFunction{lam}\<%
\\
\>[0]\AgdaField{input}%
\>[9]\AgdaFunction{lam-rule}\AgdaSpace{}%
\AgdaSymbol{=}\AgdaSpace{}%
\AgdaFunction{⇛}\<%
\\
\>[0]\AgdaField{premises}\AgdaSpace{}%
\AgdaFunction{lam-rule}\AgdaSpace{}%
\AgdaSymbol{=}\AgdaSpace{}%
\AgdaField{input}\AgdaSpace{}%
\AgdaFunction{lam-rule}\AgdaSpace{}%
\AgdaOperator{\AgdaInductiveConstructor{∙}}\AgdaSpace{}%
\AgdaInductiveConstructor{bind}\AgdaSpace{}%
\AgdaSymbol{(}\AgdaInductiveConstructor{place}\AgdaSpace{}%
\AgdaFunction{ι}\AgdaSymbol{)}\AgdaSpace{}%
\AgdaOperator{\AgdaInductiveConstructor{,}}%
\>[415I]\AgdaSymbol{(((}\AgdaInductiveConstructor{⋆}\AgdaSpace{}%
\AgdaOperator{\AgdaInductiveConstructor{∙}}\AgdaSymbol{)}\AgdaSpace{}%
\AgdaOperator{\AgdaInductiveConstructor{/}}\AgdaSpace{}%
\AgdaInductiveConstructor{ε}\AgdaSymbol{)}\AgdaSpace{}%
\AgdaOperator{\AgdaInductiveConstructor{⊢'}}\AgdaSpace{}%
\AgdaSymbol{(((}\AgdaOperator{\AgdaInductiveConstructor{∙}}\AgdaSpace{}%
\AgdaOperator{\AgdaInductiveConstructor{∙}}\AgdaSpace{}%
\AgdaInductiveConstructor{⋆}\AgdaSymbol{)}\AgdaSpace{}%
\AgdaOperator{\AgdaInductiveConstructor{/}}\AgdaSpace{}%
\AgdaInductiveConstructor{ε}\AgdaSymbol{)}\AgdaSpace{}%
\AgdaOperator{\AgdaInductiveConstructor{∋'}}\AgdaSpace{}%
\AgdaOperator{\AgdaInductiveConstructor{∙}}\AgdaSpace{}%
\AgdaInductiveConstructor{bind}\AgdaSpace{}%
\AgdaInductiveConstructor{⋆}\AgdaSpace{}%
\AgdaOperator{\AgdaInductiveConstructor{[}}\AgdaSpace{}%
\AgdaFunction{ι}\AgdaSpace{}%
\AgdaOperator{\AgdaInductiveConstructor{]}}\AgdaSymbol{))}\<%
\\
\>[.][@{}l@{}]\<[415I]%
\>[54]\AgdaOperator{\AgdaInductiveConstructor{⇉}}\AgdaSpace{}%
\AgdaInductiveConstructor{ε}\AgdaSpace{}%
\AgdaSymbol{((}\AgdaInductiveConstructor{`}\AgdaSpace{}%
\AgdaString{"λ"}\AgdaSpace{}%
\AgdaOperator{\AgdaInductiveConstructor{∙}}\AgdaSpace{}%
\AgdaInductiveConstructor{bind}\AgdaSpace{}%
\AgdaSymbol{(}\AgdaInductiveConstructor{`}\AgdaSpace{}%
\AgdaString{"⊤"}\AgdaSymbol{))}\AgdaSpace{}%
\AgdaOperator{\AgdaFunction{placeless}}\AgdaSymbol{)}\<%
\\
%
\\[\AgdaEmptyExtraSkip]%
\>[0]\AgdaComment{-- and we can type lam elimination}\<%
\\
\>[0]\AgdaFunction{app-rule}\AgdaSpace{}%
\AgdaSymbol{:}\AgdaSpace{}%
\AgdaRecord{ElimRule}\<%
\\
\>[0]\AgdaField{targetPat}%
\>[11]\AgdaFunction{app-rule}\AgdaSpace{}%
\AgdaSymbol{=}\AgdaSpace{}%
\AgdaFunction{⇛}\<%
\\
\>[0]\AgdaField{eliminator}\AgdaSpace{}%
\AgdaFunction{app-rule}\AgdaSpace{}%
\AgdaSymbol{=}\AgdaSpace{}%
\AgdaInductiveConstructor{place}\AgdaSpace{}%
\AgdaFunction{ι}\<%
\\
\>[0]\AgdaField{premises}%
\>[11]\AgdaFunction{app-rule}\AgdaSpace{}%
\AgdaSymbol{=}%
\>[449I]\AgdaField{targetPat}\AgdaSpace{}%
\AgdaFunction{app-rule}\AgdaSpace{}%
\AgdaOperator{\AgdaInductiveConstructor{∙}}\AgdaSpace{}%
\AgdaInductiveConstructor{place}\AgdaSpace{}%
\AgdaFunction{ι}\AgdaSpace{}%
\AgdaOperator{\AgdaInductiveConstructor{,}}\<%
\\
\>[.][@{}l@{}]\<[449I]%
\>[22]\AgdaSymbol{(((}\AgdaInductiveConstructor{⋆}\AgdaSpace{}%
\AgdaOperator{\AgdaInductiveConstructor{∙}}\AgdaSymbol{)}\AgdaSpace{}%
\AgdaOperator{\AgdaInductiveConstructor{/}}\AgdaSpace{}%
\AgdaInductiveConstructor{ε}\AgdaSymbol{)}\AgdaSpace{}%
\AgdaOperator{\AgdaInductiveConstructor{∋'}}\AgdaSpace{}%
\AgdaInductiveConstructor{⋆}\AgdaSpace{}%
\AgdaOperator{\AgdaInductiveConstructor{[}}\AgdaSpace{}%
\AgdaFunction{ι}\AgdaSpace{}%
\AgdaOperator{\AgdaInductiveConstructor{]}}\AgdaSymbol{)}\AgdaSpace{}%
\AgdaOperator{\AgdaInductiveConstructor{⇉}}\<%
\\
%
\>[22]\AgdaInductiveConstructor{ε}\AgdaSpace{}%
\AgdaSymbol{((}\AgdaInductiveConstructor{`}\AgdaSpace{}%
\AgdaString{"⊤"}\AgdaSymbol{)}\AgdaSpace{}%
\AgdaOperator{\AgdaFunction{placeless}}\AgdaSymbol{)}\<%
\\
\>[0]\AgdaField{output}%
\>[11]\AgdaFunction{app-rule}\AgdaSpace{}%
\AgdaSymbol{=}\AgdaSpace{}%
\AgdaSymbol{(((}\AgdaOperator{\AgdaInductiveConstructor{∙}}\AgdaSpace{}%
\AgdaOperator{\AgdaInductiveConstructor{∙}}\AgdaSpace{}%
\AgdaInductiveConstructor{⋆}\AgdaSymbol{)}\AgdaSpace{}%
\AgdaOperator{\AgdaInductiveConstructor{∙}}\AgdaSymbol{)}\AgdaSpace{}%
\AgdaOperator{\AgdaInductiveConstructor{/}}\AgdaSpace{}%
\AgdaInductiveConstructor{ε}\AgdaSymbol{)}\<%
\\
%
\\[\AgdaEmptyExtraSkip]%
\>[0]\AgdaComment{-- β rules}\<%
\\
%
\\[\AgdaEmptyExtraSkip]%
\>[0]\AgdaFunction{app-βrule}\AgdaSpace{}%
\AgdaSymbol{:}\AgdaSpace{}%
\AgdaRecord{β-Rule}\<%
\\
\>[0]\AgdaField{target}%
\>[12]\AgdaFunction{app-βrule}%
\>[23]\AgdaSymbol{=}%
\>[26]\AgdaInductiveConstructor{`}\AgdaSpace{}%
\AgdaString{"λ"}\AgdaSpace{}%
\AgdaOperator{\AgdaInductiveConstructor{∙}}\AgdaSpace{}%
\AgdaInductiveConstructor{bind}\AgdaSpace{}%
\AgdaSymbol{(}\AgdaInductiveConstructor{place}\AgdaSpace{}%
\AgdaFunction{ι}\AgdaSymbol{)}\<%
\\
\>[0]\AgdaField{erule}%
\>[12]\AgdaFunction{app-βrule}%
\>[23]\AgdaSymbol{=}%
\>[26]\AgdaFunction{app-rule}\<%
\\
\>[0]\AgdaField{redTerm}%
\>[12]\AgdaFunction{app-βrule}%
\>[23]\AgdaSymbol{=}%
\>[26]\AgdaSymbol{((}\AgdaOperator{\AgdaInductiveConstructor{∙}}\AgdaSpace{}%
\AgdaInductiveConstructor{bind}\AgdaSpace{}%
\AgdaInductiveConstructor{⋆}\AgdaSymbol{)}\AgdaSpace{}%
\AgdaOperator{\AgdaInductiveConstructor{∙}}\AgdaSymbol{)}\AgdaSpace{}%
\AgdaOperator{\AgdaInductiveConstructor{/}}\AgdaSpace{}%
\AgdaSymbol{(}\AgdaInductiveConstructor{ε}\AgdaSpace{}%
\AgdaOperator{\AgdaInductiveConstructor{-,}}\AgdaSpace{}%
\AgdaSymbol{(((}\AgdaOperator{\AgdaInductiveConstructor{∙}}\AgdaSpace{}%
\AgdaSymbol{(}\AgdaOperator{\AgdaInductiveConstructor{∙}}\AgdaSpace{}%
\AgdaInductiveConstructor{⋆}\AgdaSymbol{))}\AgdaSpace{}%
\AgdaOperator{\AgdaInductiveConstructor{/}}\AgdaSpace{}%
\AgdaInductiveConstructor{ε}\AgdaSymbol{)}\AgdaSpace{}%
\AgdaOperator{\AgdaInductiveConstructor{∷}}\AgdaSpace{}%
\AgdaSymbol{((}\AgdaOperator{\AgdaInductiveConstructor{∙}}\AgdaSpace{}%
\AgdaSymbol{((}\AgdaInductiveConstructor{⋆}\AgdaSpace{}%
\AgdaOperator{\AgdaInductiveConstructor{∙}}\AgdaSymbol{)}\AgdaSpace{}%
\AgdaOperator{\AgdaInductiveConstructor{∙}}\AgdaSymbol{))}\AgdaSpace{}%
\AgdaOperator{\AgdaInductiveConstructor{/}}\AgdaSpace{}%
\AgdaInductiveConstructor{ε}\AgdaSymbol{)))}\<%
\\
%
\\[\AgdaEmptyExtraSkip]%
%
\\[\AgdaEmptyExtraSkip]%
\>[0]\AgdaComment{-- η rules}\<%
\\
%
\\[\AgdaEmptyExtraSkip]%
\>[0]\AgdaFunction{lam-ηrule}\AgdaSpace{}%
\AgdaSymbol{:}\AgdaSpace{}%
\AgdaRecord{η-Rule}\<%
\\
\>[0]\AgdaField{checkRule}\AgdaSpace{}%
\AgdaFunction{lam-ηrule}\AgdaSpace{}%
\AgdaSymbol{=}\AgdaSpace{}%
\AgdaFunction{lam-rule}\<%
\\
\>[0]\AgdaField{eliminators}\AgdaSpace{}%
\AgdaFunction{lam-ηrule}\AgdaSpace{}%
\AgdaSymbol{=}\AgdaSpace{}%
\AgdaInductiveConstructor{`}\AgdaSpace{}%
\AgdaOperator{\AgdaInductiveConstructor{∙}}\AgdaSpace{}%
\AgdaSymbol{(}\AgdaInductiveConstructor{bind}\AgdaSpace{}%
\AgdaSymbol{(}\AgdaInductiveConstructor{Pattern.thing}\AgdaSpace{}%
\AgdaSymbol{(}\AgdaInductiveConstructor{thunk}\AgdaSpace{}%
\AgdaSymbol{(}\AgdaInductiveConstructor{var}\AgdaSpace{}%
\AgdaInductiveConstructor{ze}\AgdaSymbol{))))}\<%
\\
%
\\[\AgdaEmptyExtraSkip]%
\>[0]\AgdaComment{-- first lets get all our rules together:}\<%
\\
%
\\[\AgdaEmptyExtraSkip]%
\>[0]\AgdaKeyword{open}\AgdaSpace{}%
\AgdaKeyword{import}\AgdaSpace{}%
\AgdaModule{Data.List}\AgdaSpace{}%
\AgdaKeyword{using}\AgdaSpace{}%
\AgdaSymbol{(}\AgdaDatatype{List}\AgdaSymbol{;}\AgdaSpace{}%
\AgdaInductiveConstructor{[]}\AgdaSymbol{;}\AgdaSpace{}%
\AgdaOperator{\AgdaInductiveConstructor{\AgdaUnderscore{}∷\AgdaUnderscore{}}}\AgdaSymbol{)}\<%
\\
%
\\[\AgdaEmptyExtraSkip]%
\>[0]\AgdaFunction{typerules}\AgdaSpace{}%
\AgdaSymbol{:}\AgdaSpace{}%
\AgdaDatatype{List}\AgdaSpace{}%
\AgdaRecord{TypeRule}\<%
\\
\>[0]\AgdaFunction{typerules}\AgdaSpace{}%
\AgdaSymbol{=}\AgdaSpace{}%
\AgdaFunction{U-type}\AgdaSpace{}%
\AgdaOperator{\AgdaInductiveConstructor{∷}}\AgdaSpace{}%
\AgdaFunction{α-rule}\AgdaSpace{}%
\AgdaOperator{\AgdaInductiveConstructor{∷}}\AgdaSpace{}%
\AgdaFunction{⇛-rule}\AgdaSpace{}%
\AgdaOperator{\AgdaInductiveConstructor{∷}}\AgdaSpace{}%
\AgdaFunction{β-rule}\AgdaSpace{}%
\AgdaOperator{\AgdaInductiveConstructor{∷}}\AgdaSpace{}%
\AgdaInductiveConstructor{[]}\<%
\\
%
\\[\AgdaEmptyExtraSkip]%
\>[0]\AgdaFunction{univrules}\AgdaSpace{}%
\AgdaSymbol{:}\AgdaSpace{}%
\AgdaDatatype{List}\AgdaSpace{}%
\AgdaRecord{UnivRule}\<%
\\
\>[0]\AgdaFunction{univrules}\AgdaSpace{}%
\AgdaSymbol{=}\AgdaSpace{}%
\AgdaFunction{U-univ}%
\>[20]\AgdaOperator{\AgdaInductiveConstructor{∷}}\AgdaSpace{}%
\AgdaInductiveConstructor{[]}\<%
\\
%
\\[\AgdaEmptyExtraSkip]%
\>[0]\AgdaFunction{∋rules}\AgdaSpace{}%
\AgdaSymbol{:}\AgdaSpace{}%
\AgdaDatatype{List}\AgdaSpace{}%
\AgdaRecord{∋rule}\<%
\\
\>[0]\AgdaFunction{∋rules}\AgdaSpace{}%
\AgdaSymbol{=}\AgdaSpace{}%
\AgdaFunction{lam-rule}\AgdaSpace{}%
\AgdaOperator{\AgdaInductiveConstructor{∷}}\AgdaSpace{}%
\AgdaFunction{α-inuniv}\AgdaSpace{}%
\AgdaOperator{\AgdaInductiveConstructor{∷}}\AgdaSpace{}%
\AgdaFunction{a-rule}\AgdaSpace{}%
\AgdaOperator{\AgdaInductiveConstructor{∷}}\AgdaSpace{}%
\AgdaFunction{⇛-inuniv}\AgdaSpace{}%
\AgdaOperator{\AgdaInductiveConstructor{∷}}\AgdaSpace{}%
\AgdaFunction{b-rule}\AgdaSpace{}%
\AgdaOperator{\AgdaInductiveConstructor{∷}}\AgdaSpace{}%
\AgdaInductiveConstructor{[]}\<%
\\
%
\\[\AgdaEmptyExtraSkip]%
\>[0]\AgdaFunction{elimrules}\AgdaSpace{}%
\AgdaSymbol{:}\AgdaSpace{}%
\AgdaDatatype{List}\AgdaSpace{}%
\AgdaRecord{ElimRule}\<%
\\
\>[0]\AgdaFunction{elimrules}\AgdaSpace{}%
\AgdaSymbol{=}\AgdaSpace{}%
\AgdaFunction{app-rule}\AgdaSpace{}%
\AgdaOperator{\AgdaInductiveConstructor{∷}}\AgdaSpace{}%
\AgdaInductiveConstructor{[]}\<%
\\
%
\\[\AgdaEmptyExtraSkip]%
\>[0]\AgdaFunction{betarules}\AgdaSpace{}%
\AgdaSymbol{:}\AgdaSpace{}%
\AgdaDatatype{List}\AgdaSpace{}%
\AgdaRecord{β-Rule}\<%
\\
\>[0]\AgdaFunction{betarules}\AgdaSpace{}%
\AgdaSymbol{=}\AgdaSpace{}%
\AgdaFunction{app-βrule}\AgdaSpace{}%
\AgdaOperator{\AgdaInductiveConstructor{∷}}\AgdaSpace{}%
\AgdaInductiveConstructor{[]}\<%
\\
%
\\[\AgdaEmptyExtraSkip]%
\>[0]\AgdaFunction{etarules}\AgdaSpace{}%
\AgdaSymbol{:}\AgdaSpace{}%
\AgdaDatatype{List}\AgdaSpace{}%
\AgdaRecord{η-Rule}\<%
\\
\>[0]\AgdaFunction{etarules}\AgdaSpace{}%
\AgdaSymbol{=}\AgdaSpace{}%
\AgdaFunction{lam-ηrule}\AgdaSpace{}%
\AgdaOperator{\AgdaInductiveConstructor{∷}}\AgdaSpace{}%
\AgdaInductiveConstructor{[]}\<%
\\
%
\\[\AgdaEmptyExtraSkip]%
\>[0]\AgdaFunction{rules}\AgdaSpace{}%
\AgdaSymbol{:}\AgdaSpace{}%
\AgdaDatatype{RuleSet}\<%
\\
\>[0]\AgdaFunction{rules}\AgdaSpace{}%
\AgdaSymbol{=}%
\>[9]\AgdaInductiveConstructor{rs}\AgdaSpace{}%
\AgdaFunction{typerules}\AgdaSpace{}%
\AgdaFunction{univrules}\AgdaSpace{}%
\AgdaFunction{∋rules}\AgdaSpace{}%
\AgdaFunction{elimrules}\AgdaSpace{}%
\AgdaFunction{betarules}\AgdaSpace{}%
\AgdaFunction{etarules}\<%
\end{code}


\subsection{Beta reduction and normalisation}
\hide{
\begin{code}%
\>[0]\AgdaKeyword{module}\AgdaSpace{}%
\AgdaModule{Test.BetaTest}\AgdaSpace{}%
\AgdaKeyword{where}\<%
\\
%
\\[\AgdaEmptyExtraSkip]%
\>[0]\AgdaKeyword{open}\AgdaSpace{}%
\AgdaKeyword{import}\AgdaSpace{}%
\AgdaModule{CoreLanguage}\<%
\\
\>[0]\AgdaKeyword{open}\AgdaSpace{}%
\AgdaKeyword{import}\AgdaSpace{}%
\AgdaModule{Semantics}\<%
\\
\>[0]\AgdaKeyword{open}\AgdaSpace{}%
\AgdaKeyword{import}\AgdaSpace{}%
\AgdaModule{Relation.Binary.PropositionalEquality}\AgdaSpace{}%
\AgdaKeyword{using}\AgdaSpace{}%
\AgdaSymbol{(}\AgdaOperator{\AgdaDatatype{\AgdaUnderscore{}≡\AgdaUnderscore{}}}\AgdaSymbol{;}\AgdaSpace{}%
\AgdaInductiveConstructor{refl}\AgdaSymbol{)}\<%
\\
\>[0]\AgdaKeyword{open}\AgdaSpace{}%
\AgdaKeyword{import}\AgdaSpace{}%
\AgdaModule{Failable}\AgdaSpace{}%
\AgdaKeyword{hiding}\AgdaSpace{}%
\AgdaSymbol{(}\AgdaOperator{\AgdaFunction{\AgdaUnderscore{}>>=\AgdaUnderscore{}}}\AgdaSymbol{)}\<%
\\
\>[0]\AgdaKeyword{open}\AgdaSpace{}%
\AgdaKeyword{import}\AgdaSpace{}%
\AgdaModule{Test.Specs.STLC}\AgdaSpace{}%
\AgdaKeyword{using}\AgdaSpace{}%
\AgdaSymbol{(}\AgdaFunction{rules}\AgdaSymbol{;}\AgdaSpace{}%
\AgdaFunction{betarules}\AgdaSymbol{;}\AgdaSpace{}%
\AgdaFunction{etarules}\AgdaSymbol{)}\<%
\\
\>[0]\AgdaKeyword{open}\AgdaSpace{}%
\AgdaModule{Test.Specs.STLC.combinators}\<%
\\
%
\\[\AgdaEmptyExtraSkip]%
\>[0]\AgdaKeyword{private}\<%
\\
\>[0][@{}l@{\AgdaIndent{0}}]%
\>[2]\AgdaKeyword{variable}\<%
\\
\>[2][@{}l@{\AgdaIndent{0}}]%
\>[4]\AgdaGeneralizable{γ}\AgdaSpace{}%
\AgdaSymbol{:}\AgdaSpace{}%
\AgdaFunction{Scope}\<%
\\
%
\\[\AgdaEmptyExtraSkip]%
\>[0]\AgdaKeyword{open}\AgdaSpace{}%
\AgdaModule{β-rule}\<%
\end{code}
}
\begin{code}%
\>[0]\AgdaComment{------------------------------------------------------}\<%
\\
\>[0]\AgdaComment{-- β-reduction tests}\<%
\\
\>[0]\AgdaComment{------------------------------------------------------}\<%
\\
%
\\[\AgdaEmptyExtraSkip]%
\>[0]\AgdaKeyword{module}\AgdaSpace{}%
\AgdaModule{βredtests}\AgdaSpace{}%
\AgdaKeyword{where}\<%
\\
%
\\[\AgdaEmptyExtraSkip]%
\>[0][@{}l@{\AgdaIndent{0}}]%
\>[2]\AgdaKeyword{open}\AgdaSpace{}%
\AgdaKeyword{import}\AgdaSpace{}%
\AgdaModule{TypeChecker}\AgdaSpace{}%
\AgdaKeyword{using}\AgdaSpace{}%
\AgdaSymbol{(}\AgdaFunction{check-premise-chain}\AgdaSymbol{)}\<%
\\
%
\>[2]\AgdaKeyword{open}\AgdaSpace{}%
\AgdaKeyword{import}\AgdaSpace{}%
\AgdaModule{BwdVec}\AgdaSpace{}%
\AgdaKeyword{using}\AgdaSpace{}%
\AgdaSymbol{(}\AgdaInductiveConstructor{ε}\AgdaSymbol{)}\<%
\\
%
\>[2]\AgdaFunction{PC}\AgdaSpace{}%
\AgdaSymbol{=}\AgdaSpace{}%
\AgdaFunction{check-premise-chain}\AgdaSpace{}%
\AgdaFunction{rules}\AgdaSpace{}%
\AgdaInductiveConstructor{ε}\<%
\\
%
\\[\AgdaEmptyExtraSkip]%
%
\>[2]\AgdaFunction{test1}\AgdaSpace{}%
\AgdaSymbol{:}\AgdaSpace{}%
\AgdaDatatype{Failable}\AgdaSpace{}%
\AgdaSymbol{(}\AgdaDatatype{Compu}\AgdaSpace{}%
\AgdaNumber{0}\AgdaSymbol{)}\<%
\\
%
\>[2]\AgdaFunction{test1}\AgdaSpace{}%
\AgdaSymbol{=}\AgdaSpace{}%
\AgdaFunction{reduce}\AgdaSpace{}%
\AgdaFunction{betarules}\AgdaSpace{}%
\AgdaFunction{PC}\AgdaSpace{}%
\AgdaSymbol{(}\AgdaFunction{lam}\AgdaSpace{}%
\AgdaSymbol{(}\AgdaFunction{\textasciitilde{}}\AgdaSpace{}%
\AgdaInductiveConstructor{ze}\AgdaSymbol{))}\AgdaSpace{}%
\AgdaSymbol{(}\AgdaFunction{α}\AgdaSpace{}%
\AgdaOperator{\AgdaFunction{⇨}}\AgdaSpace{}%
\AgdaFunction{α}\AgdaSymbol{)}\AgdaSpace{}%
\AgdaFunction{a}\<%
\\
\>[0]\<%
\\
%
\>[2]\AgdaFunction{\AgdaUnderscore{}}\AgdaSpace{}%
\AgdaSymbol{:}\AgdaSpace{}%
\AgdaFunction{test1}\AgdaSpace{}%
\AgdaOperator{\AgdaDatatype{≡}}\AgdaSpace{}%
\AgdaInductiveConstructor{succeed}\AgdaSpace{}%
\AgdaSymbol{(}\AgdaFunction{a}\AgdaSpace{}%
\AgdaOperator{\AgdaInductiveConstructor{∷}}\AgdaSpace{}%
\AgdaFunction{α}\AgdaSymbol{)}\<%
\\
%
\>[2]\AgdaSymbol{\AgdaUnderscore{}}\AgdaSpace{}%
\AgdaSymbol{=}\AgdaSpace{}%
\AgdaInductiveConstructor{refl}\<%
\\
\>[0]\<%
\\
%
\>[2]\AgdaComment{-- function as input}\<%
\\
%
\>[2]\AgdaFunction{test2}\AgdaSpace{}%
\AgdaSymbol{:}\AgdaSpace{}%
\AgdaDatatype{Failable}\AgdaSpace{}%
\AgdaSymbol{(}\AgdaDatatype{Compu}\AgdaSpace{}%
\AgdaNumber{0}\AgdaSymbol{)}\<%
\\
%
\>[2]\AgdaFunction{test2}\AgdaSpace{}%
\AgdaSymbol{=}\AgdaSpace{}%
\AgdaFunction{reduce}\AgdaSpace{}%
\AgdaFunction{betarules}\AgdaSpace{}%
\AgdaFunction{PC}\AgdaSpace{}%
\AgdaSymbol{(}\AgdaFunction{lam}\AgdaSpace{}%
\AgdaSymbol{(}\AgdaFunction{\textasciitilde{}}\AgdaSpace{}%
\AgdaInductiveConstructor{ze}\AgdaSymbol{))}\AgdaSpace{}%
\AgdaSymbol{((}\AgdaFunction{α}\AgdaSpace{}%
\AgdaOperator{\AgdaFunction{⇨}}\AgdaSpace{}%
\AgdaFunction{α}\AgdaSymbol{)}\AgdaSpace{}%
\AgdaOperator{\AgdaFunction{⇨}}\AgdaSpace{}%
\AgdaSymbol{(}\AgdaFunction{α}\AgdaSpace{}%
\AgdaOperator{\AgdaFunction{⇨}}\AgdaSpace{}%
\AgdaFunction{α}\AgdaSymbol{))}\AgdaSpace{}%
\AgdaSymbol{(}\AgdaFunction{lam}\AgdaSpace{}%
\AgdaSymbol{(}\AgdaFunction{\textasciitilde{}}\AgdaSpace{}%
\AgdaInductiveConstructor{ze}\AgdaSymbol{))}\<%
\\
\>[0]\<%
\\
%
\>[2]\AgdaFunction{\AgdaUnderscore{}}\AgdaSpace{}%
\AgdaSymbol{:}\AgdaSpace{}%
\AgdaFunction{test2}\AgdaSpace{}%
\AgdaOperator{\AgdaDatatype{≡}}\AgdaSpace{}%
\AgdaInductiveConstructor{succeed}\AgdaSpace{}%
\AgdaSymbol{(}\AgdaFunction{lam}\AgdaSpace{}%
\AgdaSymbol{(}\AgdaFunction{\textasciitilde{}}\AgdaSpace{}%
\AgdaInductiveConstructor{ze}\AgdaSymbol{)}\AgdaSpace{}%
\AgdaOperator{\AgdaInductiveConstructor{∷}}\AgdaSpace{}%
\AgdaSymbol{(}\AgdaFunction{α}\AgdaSpace{}%
\AgdaOperator{\AgdaFunction{⇨}}\AgdaSpace{}%
\AgdaFunction{α}\AgdaSymbol{))}\<%
\\
%
\>[2]\AgdaSymbol{\AgdaUnderscore{}}\AgdaSpace{}%
\AgdaSymbol{=}\AgdaSpace{}%
\AgdaInductiveConstructor{refl}\<%
\\
\>[0]\<%
\\
%
\>[2]\AgdaComment{-- take in a function, an argument and apply them}\<%
\\
%
\>[2]\AgdaFunction{func}\AgdaSpace{}%
\AgdaSymbol{:}\AgdaSpace{}%
\AgdaDatatype{Const}\AgdaSpace{}%
\AgdaGeneralizable{γ}\<%
\\
%
\>[2]\AgdaFunction{func}\AgdaSpace{}%
\AgdaSymbol{=}\AgdaSpace{}%
\AgdaFunction{lam}\AgdaSpace{}%
\AgdaSymbol{(}\AgdaFunction{lam}\AgdaSpace{}%
\AgdaSymbol{(}\AgdaInductiveConstructor{thunk}\AgdaSpace{}%
\AgdaSymbol{(}\AgdaFunction{app}\AgdaSpace{}%
\AgdaSymbol{(}\AgdaInductiveConstructor{var}\AgdaSpace{}%
\AgdaSymbol{(}\AgdaInductiveConstructor{su}\AgdaSpace{}%
\AgdaInductiveConstructor{ze}\AgdaSymbol{))}\AgdaSpace{}%
\AgdaSymbol{(}\AgdaFunction{\textasciitilde{}}\AgdaSpace{}%
\AgdaInductiveConstructor{ze}\AgdaSymbol{))))}\<%
\\
\>[0]\<%
\\
%
\>[2]\AgdaFunction{ftype}\AgdaSpace{}%
\AgdaSymbol{:}\AgdaSpace{}%
\AgdaDatatype{Const}\AgdaSpace{}%
\AgdaGeneralizable{γ}\<%
\\
%
\>[2]\AgdaFunction{ftype}\AgdaSpace{}%
\AgdaSymbol{=}\AgdaSpace{}%
\AgdaSymbol{(}\AgdaFunction{α}\AgdaSpace{}%
\AgdaOperator{\AgdaFunction{⇨}}\AgdaSpace{}%
\AgdaFunction{α}\AgdaSymbol{)}\AgdaSpace{}%
\AgdaOperator{\AgdaFunction{⇨}}\AgdaSpace{}%
\AgdaFunction{α}\AgdaSpace{}%
\AgdaOperator{\AgdaFunction{⇨}}\AgdaSpace{}%
\AgdaFunction{α}\<%
\\
\>[0]\<%
\\
%
\>[2]\AgdaFunction{arg1}%
\>[8]\AgdaSymbol{:}\AgdaSpace{}%
\AgdaDatatype{Const}\AgdaSpace{}%
\AgdaGeneralizable{γ}\<%
\\
%
\>[2]\AgdaFunction{arg1}\AgdaSpace{}%
\AgdaSymbol{=}\AgdaSpace{}%
\AgdaFunction{lam}\AgdaSpace{}%
\AgdaSymbol{(}\AgdaFunction{\textasciitilde{}}\AgdaSpace{}%
\AgdaInductiveConstructor{ze}\AgdaSymbol{)}\<%
\\
\>[0]\<%
\\
%
\>[2]\AgdaFunction{reducable-term}\AgdaSpace{}%
\AgdaSymbol{:}\AgdaSpace{}%
\AgdaDatatype{Compu}\AgdaSpace{}%
\AgdaNumber{1}\<%
\\
%
\>[2]\AgdaFunction{reducable-term}\AgdaSpace{}%
\AgdaSymbol{=}\AgdaSpace{}%
\AgdaInductiveConstructor{elim}\AgdaSpace{}%
\AgdaSymbol{(}\AgdaInductiveConstructor{elim}\AgdaSpace{}%
\AgdaSymbol{(}\AgdaFunction{func}\AgdaSpace{}%
\AgdaOperator{\AgdaInductiveConstructor{∷}}\AgdaSpace{}%
\AgdaFunction{ftype}\AgdaSymbol{)}\AgdaSpace{}%
\AgdaFunction{arg1}\AgdaSymbol{)}\AgdaSpace{}%
\AgdaSymbol{(}\AgdaInductiveConstructor{thunk}\AgdaSpace{}%
\AgdaSymbol{(}\AgdaInductiveConstructor{var}\AgdaSpace{}%
\AgdaInductiveConstructor{ze}\AgdaSymbol{))}\<%
\\
\>[0]\<%
\\
%
\>[2]\AgdaFunction{test3}\AgdaSpace{}%
\AgdaSymbol{:}\AgdaSpace{}%
\AgdaDatatype{Failable}\AgdaSpace{}%
\AgdaSymbol{(}\AgdaDatatype{Compu}\AgdaSpace{}%
\AgdaNumber{0}\AgdaSymbol{)}\<%
\\
%
\>[2]\AgdaFunction{test3}\AgdaSpace{}%
\AgdaSymbol{=}\AgdaSpace{}%
\AgdaFunction{reduce}\AgdaSpace{}%
\AgdaFunction{betarules}\AgdaSpace{}%
\AgdaFunction{PC}\AgdaSpace{}%
\AgdaFunction{func}\AgdaSpace{}%
\AgdaFunction{ftype}\AgdaSpace{}%
\AgdaFunction{arg1}\<%
\\
\>[0]\<%
\\
%
\>[2]\AgdaFunction{\AgdaUnderscore{}}\AgdaSpace{}%
\AgdaSymbol{:}\AgdaSpace{}%
\AgdaFunction{test3}\AgdaSpace{}%
\AgdaOperator{\AgdaDatatype{≡}}\AgdaSpace{}%
\AgdaInductiveConstructor{succeed}\AgdaSpace{}%
\AgdaSymbol{(}\AgdaFunction{lam}\AgdaSpace{}%
\AgdaSymbol{(}\AgdaInductiveConstructor{thunk}\AgdaSpace{}%
\AgdaSymbol{(}\AgdaFunction{app}\AgdaSpace{}%
\AgdaSymbol{(}\AgdaFunction{arg1}\AgdaSpace{}%
\AgdaOperator{\AgdaInductiveConstructor{∷}}\AgdaSpace{}%
\AgdaSymbol{(}\AgdaFunction{α}\AgdaSpace{}%
\AgdaOperator{\AgdaFunction{⇨}}\AgdaSpace{}%
\AgdaFunction{α}\AgdaSymbol{))}\AgdaSpace{}%
\AgdaSymbol{(}\AgdaFunction{\textasciitilde{}}\AgdaSpace{}%
\AgdaInductiveConstructor{ze}\AgdaSymbol{)))}\AgdaSpace{}%
\AgdaOperator{\AgdaInductiveConstructor{∷}}\AgdaSpace{}%
\AgdaSymbol{(}\AgdaFunction{α}\AgdaSpace{}%
\AgdaOperator{\AgdaFunction{⇨}}\AgdaSpace{}%
\AgdaFunction{α}\AgdaSymbol{))}\<%
\\
%
\>[2]\AgdaSymbol{\AgdaUnderscore{}}\AgdaSpace{}%
\AgdaSymbol{=}\AgdaSpace{}%
\AgdaInductiveConstructor{refl}\<%
\\
%
\\[\AgdaEmptyExtraSkip]%
\>[0]\AgdaComment{-------------------------------------------------}\<%
\\
\>[0]\AgdaComment{-- Normalization by evaluation tests}\<%
\\
\>[0]\AgdaComment{-------------------------------------------------}\<%
\\
%
\\[\AgdaEmptyExtraSkip]%
\>[0]\AgdaKeyword{module}\AgdaSpace{}%
\AgdaModule{normbyeval}\AgdaSpace{}%
\AgdaKeyword{where}\<%
\\
%
\\[\AgdaEmptyExtraSkip]%
\>[0][@{}l@{\AgdaIndent{0}}]%
\>[2]\AgdaKeyword{open}\AgdaSpace{}%
\AgdaKeyword{import}\AgdaSpace{}%
\AgdaModule{TypeChecker}\AgdaSpace{}%
\AgdaKeyword{using}\AgdaSpace{}%
\AgdaSymbol{(}\AgdaFunction{infer}\AgdaSymbol{;}\AgdaSpace{}%
\AgdaFunction{check-premise-chain}\AgdaSymbol{)}\<%
\\
%
\>[2]\AgdaKeyword{open}\AgdaSpace{}%
\AgdaKeyword{import}\AgdaSpace{}%
\AgdaModule{Data.Product}\AgdaSpace{}%
\AgdaKeyword{using}\AgdaSpace{}%
\AgdaSymbol{(}\AgdaOperator{\AgdaInductiveConstructor{\AgdaUnderscore{},\AgdaUnderscore{}}}\AgdaSymbol{)}\<%
\\
%
\>[2]\AgdaKeyword{open}\AgdaSpace{}%
\AgdaKeyword{import}\AgdaSpace{}%
\AgdaModule{BwdVec}\<%
\\
%
\>[2]\AgdaFunction{PC}\AgdaSpace{}%
\AgdaSymbol{=}\AgdaSpace{}%
\AgdaFunction{check-premise-chain}\AgdaSpace{}%
\AgdaFunction{rules}\<%
\\
%
\\[\AgdaEmptyExtraSkip]%
%
\>[2]\AgdaComment{-- take in a function, an argument and apply them}\<%
\\
%
\>[2]\AgdaFunction{func}\AgdaSpace{}%
\AgdaSymbol{:}\AgdaSpace{}%
\AgdaDatatype{Const}\AgdaSpace{}%
\AgdaGeneralizable{γ}\<%
\\
%
\>[2]\AgdaFunction{func}\AgdaSpace{}%
\AgdaSymbol{=}\AgdaSpace{}%
\AgdaFunction{lam}\AgdaSpace{}%
\AgdaSymbol{(}\AgdaFunction{lam}\AgdaSpace{}%
\AgdaSymbol{(}\AgdaInductiveConstructor{thunk}\AgdaSpace{}%
\AgdaSymbol{(}\AgdaFunction{app}\AgdaSpace{}%
\AgdaSymbol{(}\AgdaInductiveConstructor{var}\AgdaSpace{}%
\AgdaSymbol{(}\AgdaInductiveConstructor{su}\AgdaSpace{}%
\AgdaInductiveConstructor{ze}\AgdaSymbol{))}\AgdaSpace{}%
\AgdaSymbol{(}\AgdaFunction{\textasciitilde{}}\AgdaSpace{}%
\AgdaInductiveConstructor{ze}\AgdaSymbol{))))}\<%
\\
\>[0]\<%
\\
%
\>[2]\AgdaFunction{ftype}\AgdaSpace{}%
\AgdaSymbol{:}\AgdaSpace{}%
\AgdaDatatype{Const}\AgdaSpace{}%
\AgdaGeneralizable{γ}\<%
\\
%
\>[2]\AgdaFunction{ftype}\AgdaSpace{}%
\AgdaSymbol{=}\AgdaSpace{}%
\AgdaSymbol{(}\AgdaFunction{α}\AgdaSpace{}%
\AgdaOperator{\AgdaFunction{⇨}}\AgdaSpace{}%
\AgdaFunction{α}\AgdaSymbol{)}\AgdaSpace{}%
\AgdaOperator{\AgdaFunction{⇨}}\AgdaSpace{}%
\AgdaFunction{α}\AgdaSpace{}%
\AgdaOperator{\AgdaFunction{⇨}}\AgdaSpace{}%
\AgdaFunction{α}\<%
\\
\>[0]\<%
\\
%
\>[2]\AgdaFunction{arg1}%
\>[8]\AgdaSymbol{:}\AgdaSpace{}%
\AgdaDatatype{Const}\AgdaSpace{}%
\AgdaGeneralizable{γ}\<%
\\
%
\>[2]\AgdaFunction{arg1}\AgdaSpace{}%
\AgdaSymbol{=}\AgdaSpace{}%
\AgdaFunction{lam}\AgdaSpace{}%
\AgdaSymbol{(}\AgdaFunction{\textasciitilde{}}\AgdaSpace{}%
\AgdaInductiveConstructor{ze}\AgdaSymbol{)}\<%
\\
\>[0]\<%
\\
%
\>[2]\AgdaFunction{reducable-term}\AgdaSpace{}%
\AgdaSymbol{:}\AgdaSpace{}%
\AgdaDatatype{Compu}\AgdaSpace{}%
\AgdaNumber{1}\<%
\\
%
\>[2]\AgdaFunction{reducable-term}\AgdaSpace{}%
\AgdaSymbol{=}\AgdaSpace{}%
\AgdaInductiveConstructor{elim}\AgdaSpace{}%
\AgdaSymbol{(}\AgdaInductiveConstructor{elim}\AgdaSpace{}%
\AgdaSymbol{(}\AgdaFunction{func}\AgdaSpace{}%
\AgdaOperator{\AgdaInductiveConstructor{∷}}\AgdaSpace{}%
\AgdaFunction{ftype}\AgdaSymbol{)}\AgdaSpace{}%
\AgdaFunction{arg1}\AgdaSymbol{)}\AgdaSpace{}%
\AgdaSymbol{(}\AgdaInductiveConstructor{thunk}\AgdaSpace{}%
\AgdaSymbol{(}\AgdaInductiveConstructor{var}\AgdaSpace{}%
\AgdaInductiveConstructor{ze}\AgdaSymbol{))}\<%
\\
%
\\[\AgdaEmptyExtraSkip]%
%
\\[\AgdaEmptyExtraSkip]%
%
\>[2]\AgdaComment{-- should perform a single reduction}\<%
\\
%
\\[\AgdaEmptyExtraSkip]%
%
\>[2]\AgdaFunction{test1}\AgdaSpace{}%
\AgdaSymbol{:}\AgdaSpace{}%
\AgdaDatatype{Const}\AgdaSpace{}%
\AgdaNumber{0}\<%
\\
%
\>[2]\AgdaFunction{test1}\AgdaSpace{}%
\AgdaSymbol{=}\AgdaSpace{}%
\AgdaFunction{normalize}\AgdaSpace{}%
\AgdaFunction{etarules}\AgdaSpace{}%
\AgdaFunction{betarules}\AgdaSpace{}%
\AgdaSymbol{((}\AgdaFunction{infer}\AgdaSpace{}%
\AgdaFunction{rules}\AgdaSymbol{)}\AgdaSpace{}%
\AgdaOperator{\AgdaInductiveConstructor{,}}\AgdaSpace{}%
\AgdaFunction{PC}\AgdaSymbol{)}\AgdaSpace{}%
\AgdaInductiveConstructor{ε}\AgdaSpace{}%
\AgdaFunction{α}\AgdaSpace{}%
\AgdaSymbol{(}\AgdaInductiveConstructor{thunk}\AgdaSpace{}%
\AgdaSymbol{(}\AgdaFunction{app}\AgdaSpace{}%
\AgdaSymbol{(}\AgdaFunction{lam}\AgdaSpace{}%
\AgdaSymbol{(}\AgdaFunction{\textasciitilde{}}\AgdaSpace{}%
\AgdaInductiveConstructor{ze}\AgdaSymbol{)}\AgdaSpace{}%
\AgdaOperator{\AgdaInductiveConstructor{∷}}\AgdaSpace{}%
\AgdaSymbol{(}\AgdaFunction{α}\AgdaSpace{}%
\AgdaOperator{\AgdaFunction{⇨}}\AgdaSpace{}%
\AgdaFunction{α}\AgdaSymbol{))}\AgdaSpace{}%
\AgdaFunction{a}\AgdaSymbol{))}\<%
\\
%
\\[\AgdaEmptyExtraSkip]%
%
\>[2]\AgdaFunction{\AgdaUnderscore{}}\AgdaSpace{}%
\AgdaSymbol{:}\AgdaSpace{}%
\AgdaFunction{test1}\AgdaSpace{}%
\AgdaOperator{\AgdaDatatype{≡}}\AgdaSpace{}%
\AgdaFunction{a}\<%
\\
%
\>[2]\AgdaSymbol{\AgdaUnderscore{}}\AgdaSpace{}%
\AgdaSymbol{=}\AgdaSpace{}%
\AgdaInductiveConstructor{refl}\<%
\\
%
\\[\AgdaEmptyExtraSkip]%
%
\>[2]\AgdaComment{-- should reduce multiple nested eliminations}\<%
\\
%
\\[\AgdaEmptyExtraSkip]%
%
\>[2]\AgdaFunction{test2}\AgdaSpace{}%
\AgdaSymbol{:}\AgdaSpace{}%
\AgdaDatatype{Const}\AgdaSpace{}%
\AgdaNumber{1}\<%
\\
%
\>[2]\AgdaFunction{test2}\AgdaSpace{}%
\AgdaSymbol{=}%
\>[263I]\AgdaFunction{normalize}\AgdaSpace{}%
\AgdaFunction{etarules}\AgdaSpace{}%
\AgdaFunction{betarules}\<%
\\
\>[.][@{}l@{}]\<[263I]%
\>[10]\AgdaSymbol{((}\AgdaFunction{infer}\AgdaSpace{}%
\AgdaFunction{rules}\AgdaSymbol{)}\AgdaSpace{}%
\AgdaOperator{\AgdaInductiveConstructor{,}}\AgdaSpace{}%
\AgdaFunction{PC}\AgdaSymbol{)}\AgdaSpace{}%
\AgdaSymbol{(}\AgdaInductiveConstructor{ε}\AgdaSpace{}%
\AgdaOperator{\AgdaInductiveConstructor{-,}}\AgdaSpace{}%
\AgdaFunction{α}\AgdaSymbol{)}\AgdaSpace{}%
\AgdaFunction{α}\AgdaSpace{}%
\AgdaFunction{reducable-term}\<%
\\
\>[0]\<%
\\
%
\>[2]\AgdaFunction{\AgdaUnderscore{}}\AgdaSpace{}%
\AgdaSymbol{:}\AgdaSpace{}%
\AgdaFunction{test2}\AgdaSpace{}%
\AgdaOperator{\AgdaDatatype{≡}}\AgdaSpace{}%
\AgdaInductiveConstructor{thunk}\AgdaSpace{}%
\AgdaSymbol{(}\AgdaInductiveConstructor{var}\AgdaSpace{}%
\AgdaInductiveConstructor{ze}\AgdaSymbol{)}\<%
\\
%
\>[2]\AgdaSymbol{\AgdaUnderscore{}}\AgdaSpace{}%
\AgdaSymbol{=}\AgdaSpace{}%
\AgdaInductiveConstructor{refl}\<%
\\
%
\\[\AgdaEmptyExtraSkip]%
%
\>[2]\AgdaComment{-- should eta-long variable}\<%
\\
%
\\[\AgdaEmptyExtraSkip]%
%
\>[2]\AgdaFunction{test3}\AgdaSpace{}%
\AgdaSymbol{:}\AgdaSpace{}%
\AgdaDatatype{Const}\AgdaSpace{}%
\AgdaNumber{1}\<%
\\
%
\>[2]\AgdaFunction{test3}\AgdaSpace{}%
\AgdaSymbol{=}%
\>[286I]\AgdaFunction{normalize}\AgdaSpace{}%
\AgdaFunction{etarules}\AgdaSpace{}%
\AgdaFunction{betarules}\<%
\\
\>[.][@{}l@{}]\<[286I]%
\>[10]\AgdaSymbol{((}\AgdaFunction{infer}\AgdaSpace{}%
\AgdaFunction{rules}\AgdaSymbol{)}\AgdaSpace{}%
\AgdaOperator{\AgdaInductiveConstructor{,}}\AgdaSpace{}%
\AgdaFunction{PC}\AgdaSymbol{)}\AgdaSpace{}%
\AgdaSymbol{(}\AgdaInductiveConstructor{ε}\AgdaSpace{}%
\AgdaOperator{\AgdaInductiveConstructor{-,}}\AgdaSpace{}%
\AgdaSymbol{(}\AgdaFunction{α}\AgdaSpace{}%
\AgdaOperator{\AgdaFunction{⇨}}\AgdaSpace{}%
\AgdaFunction{α}\AgdaSymbol{))}\AgdaSpace{}%
\AgdaSymbol{(}\AgdaFunction{α}\AgdaSpace{}%
\AgdaOperator{\AgdaFunction{⇨}}\AgdaSpace{}%
\AgdaFunction{α}\AgdaSymbol{)}\AgdaSpace{}%
\AgdaSymbol{(}\AgdaInductiveConstructor{var}\AgdaSpace{}%
\AgdaInductiveConstructor{ze}\AgdaSymbol{)}\<%
\\
%
\\[\AgdaEmptyExtraSkip]%
%
\>[2]\AgdaFunction{\AgdaUnderscore{}}\AgdaSpace{}%
\AgdaSymbol{:}\AgdaSpace{}%
\AgdaFunction{test3}\AgdaSpace{}%
\AgdaOperator{\AgdaDatatype{≡}}\AgdaSpace{}%
\AgdaFunction{lam}\AgdaSpace{}%
\AgdaSymbol{(}\AgdaInductiveConstructor{thunk}\AgdaSpace{}%
\AgdaSymbol{(}\AgdaFunction{app}\AgdaSpace{}%
\AgdaSymbol{(}\AgdaInductiveConstructor{var}\AgdaSpace{}%
\AgdaSymbol{(}\AgdaInductiveConstructor{su}\AgdaSpace{}%
\AgdaInductiveConstructor{ze}\AgdaSymbol{))}\AgdaSpace{}%
\AgdaSymbol{(}\AgdaFunction{\textasciitilde{}}\AgdaSpace{}%
\AgdaInductiveConstructor{ze}\AgdaSymbol{)))}\<%
\\
%
\>[2]\AgdaSymbol{\AgdaUnderscore{}}\AgdaSpace{}%
\AgdaSymbol{=}\AgdaSpace{}%
\AgdaInductiveConstructor{refl}\<%
\\
%
\\[\AgdaEmptyExtraSkip]%
%
\>[2]\AgdaComment{-- should eta-long stuck eliminations with function type}\<%
\\
%
\\[\AgdaEmptyExtraSkip]%
%
\>[2]\AgdaFunction{test4}\AgdaSpace{}%
\AgdaSymbol{:}\AgdaSpace{}%
\AgdaDatatype{Const}\AgdaSpace{}%
\AgdaNumber{1}\<%
\\
%
\>[2]\AgdaFunction{test4}\AgdaSpace{}%
\AgdaSymbol{=}%
\>[319I]\AgdaFunction{normalize}\AgdaSpace{}%
\AgdaFunction{etarules}\AgdaSpace{}%
\AgdaFunction{betarules}\AgdaSpace{}%
\AgdaSymbol{((}\AgdaFunction{infer}\AgdaSpace{}%
\AgdaFunction{rules}\AgdaSymbol{)}\AgdaSpace{}%
\AgdaOperator{\AgdaInductiveConstructor{,}}\AgdaSpace{}%
\AgdaFunction{PC}\AgdaSymbol{)}\<%
\\
\>[.][@{}l@{}]\<[319I]%
\>[10]\AgdaSymbol{(}\AgdaInductiveConstructor{ε}\AgdaSpace{}%
\AgdaOperator{\AgdaInductiveConstructor{-,}}\AgdaSpace{}%
\AgdaSymbol{(}\AgdaFunction{α}\AgdaSpace{}%
\AgdaOperator{\AgdaFunction{⇨}}\AgdaSpace{}%
\AgdaSymbol{(}\AgdaFunction{α}\AgdaSpace{}%
\AgdaOperator{\AgdaFunction{⇨}}\AgdaSpace{}%
\AgdaFunction{α}\AgdaSymbol{)))}\AgdaSpace{}%
\AgdaSymbol{(}\AgdaFunction{α}\AgdaSpace{}%
\AgdaOperator{\AgdaFunction{⇨}}\AgdaSpace{}%
\AgdaFunction{α}\AgdaSymbol{)}\AgdaSpace{}%
\AgdaSymbol{(}\AgdaFunction{app}\AgdaSpace{}%
\AgdaSymbol{(}\AgdaInductiveConstructor{var}\AgdaSpace{}%
\AgdaInductiveConstructor{ze}\AgdaSymbol{)}\AgdaSpace{}%
\AgdaFunction{a}\AgdaSymbol{)}\<%
\\
%
\\[\AgdaEmptyExtraSkip]%
%
\>[2]\AgdaFunction{\AgdaUnderscore{}}\AgdaSpace{}%
\AgdaSymbol{:}\AgdaSpace{}%
\AgdaFunction{test4}\AgdaSpace{}%
\AgdaOperator{\AgdaDatatype{≡}}\AgdaSpace{}%
\AgdaFunction{lam}\AgdaSpace{}%
\AgdaSymbol{(}\AgdaInductiveConstructor{thunk}\AgdaSpace{}%
\AgdaSymbol{(}\AgdaFunction{app}\AgdaSpace{}%
\AgdaSymbol{(}\AgdaFunction{app}\AgdaSpace{}%
\AgdaSymbol{(}\AgdaInductiveConstructor{var}\AgdaSpace{}%
\AgdaSymbol{(}\AgdaInductiveConstructor{su}\AgdaSpace{}%
\AgdaInductiveConstructor{ze}\AgdaSymbol{))}\AgdaSpace{}%
\AgdaFunction{a}\AgdaSymbol{)}\AgdaSpace{}%
\AgdaSymbol{(}\AgdaFunction{\textasciitilde{}}\AgdaSpace{}%
\AgdaInductiveConstructor{ze}\AgdaSymbol{)))}\<%
\\
%
\>[2]\AgdaSymbol{\AgdaUnderscore{}}%
\>[352I]\AgdaSymbol{=}\AgdaSpace{}%
\AgdaInductiveConstructor{refl}\<%
\\
%
\\[\AgdaEmptyExtraSkip]%
\>[.][@{}l@{}]\<[352I]%
\>[4]\AgdaComment{-- should normalize under a binder}\<%
\\
%
\\[\AgdaEmptyExtraSkip]%
%
\>[2]\AgdaFunction{test5}\AgdaSpace{}%
\AgdaSymbol{:}\AgdaSpace{}%
\AgdaDatatype{Const}\AgdaSpace{}%
\AgdaNumber{0}\<%
\\
%
\>[2]\AgdaFunction{test5}\AgdaSpace{}%
\AgdaSymbol{=}%
\>[358I]\AgdaFunction{normalize}\AgdaSpace{}%
\AgdaFunction{etarules}\AgdaSpace{}%
\AgdaFunction{betarules}\<%
\\
\>[.][@{}l@{}]\<[358I]%
\>[10]\AgdaSymbol{((}\AgdaFunction{infer}\AgdaSpace{}%
\AgdaFunction{rules}\AgdaSymbol{)}\AgdaSpace{}%
\AgdaOperator{\AgdaInductiveConstructor{,}}\AgdaSpace{}%
\AgdaFunction{PC}\AgdaSymbol{)}\AgdaSpace{}%
\AgdaInductiveConstructor{ε}\AgdaSpace{}%
\AgdaFunction{α}\AgdaSpace{}%
\AgdaSymbol{(}\AgdaFunction{lam}\AgdaSpace{}%
\AgdaSymbol{(}\AgdaInductiveConstructor{thunk}\AgdaSpace{}%
\AgdaSymbol{(}\AgdaFunction{app}\AgdaSpace{}%
\AgdaSymbol{(}\AgdaFunction{lam}\AgdaSpace{}%
\AgdaSymbol{(}\AgdaFunction{\textasciitilde{}}\AgdaSpace{}%
\AgdaInductiveConstructor{ze}\AgdaSymbol{)}\AgdaSpace{}%
\AgdaOperator{\AgdaInductiveConstructor{∷}}\AgdaSpace{}%
\AgdaSymbol{(}\AgdaFunction{α}\AgdaSpace{}%
\AgdaOperator{\AgdaFunction{⇨}}\AgdaSpace{}%
\AgdaFunction{α}\AgdaSymbol{))}\AgdaSpace{}%
\AgdaFunction{a}\AgdaSymbol{)))}\<%
\\
%
\\[\AgdaEmptyExtraSkip]%
%
\>[2]\AgdaFunction{\AgdaUnderscore{}}\AgdaSpace{}%
\AgdaSymbol{:}\AgdaSpace{}%
\AgdaFunction{test5}\AgdaSpace{}%
\AgdaOperator{\AgdaDatatype{≡}}\AgdaSpace{}%
\AgdaFunction{lam}\AgdaSpace{}%
\AgdaFunction{a}\<%
\\
%
\>[2]\AgdaSymbol{\AgdaUnderscore{}}\AgdaSpace{}%
\AgdaSymbol{=}\AgdaSpace{}%
\AgdaInductiveConstructor{refl}\<%
\\
%
\\[\AgdaEmptyExtraSkip]%
%
\>[2]\AgdaComment{-- should normalize the eliminator, even if the target is neutral}\<%
\\
%
\\[\AgdaEmptyExtraSkip]%
%
\>[2]\AgdaFunction{test6}\AgdaSpace{}%
\AgdaSymbol{:}\AgdaSpace{}%
\AgdaDatatype{Const}\AgdaSpace{}%
\AgdaNumber{1}\<%
\\
%
\>[2]\AgdaFunction{test6}\AgdaSpace{}%
\AgdaSymbol{=}%
\>[388I]\AgdaFunction{normalize}\AgdaSpace{}%
\AgdaFunction{etarules}\AgdaSpace{}%
\AgdaFunction{betarules}\<%
\\
\>[.][@{}l@{}]\<[388I]%
\>[10]\AgdaSymbol{((}\AgdaFunction{infer}\AgdaSpace{}%
\AgdaFunction{rules}\AgdaSymbol{)}\AgdaSpace{}%
\AgdaOperator{\AgdaInductiveConstructor{,}}\AgdaSpace{}%
\AgdaFunction{PC}\AgdaSymbol{)}\AgdaSpace{}%
\AgdaSymbol{(}\AgdaInductiveConstructor{ε}\AgdaSpace{}%
\AgdaOperator{\AgdaInductiveConstructor{-,}}\AgdaSpace{}%
\AgdaSymbol{(}\AgdaFunction{α}\AgdaSpace{}%
\AgdaOperator{\AgdaFunction{⇨}}\AgdaSpace{}%
\AgdaFunction{α}\AgdaSymbol{))}\AgdaSpace{}%
\AgdaFunction{α}\AgdaSpace{}%
\AgdaSymbol{(}\AgdaFunction{app}\AgdaSpace{}%
\AgdaSymbol{(}\AgdaInductiveConstructor{var}\AgdaSpace{}%
\AgdaInductiveConstructor{ze}\AgdaSymbol{)}\<%
\\
%
\>[10]\AgdaSymbol{(}\AgdaInductiveConstructor{thunk}\AgdaSpace{}%
\AgdaSymbol{(}\AgdaFunction{app}\AgdaSpace{}%
\AgdaSymbol{(}\AgdaFunction{lam}\AgdaSpace{}%
\AgdaSymbol{(}\AgdaFunction{\textasciitilde{}}\AgdaSpace{}%
\AgdaInductiveConstructor{ze}\AgdaSymbol{)}\AgdaSpace{}%
\AgdaOperator{\AgdaInductiveConstructor{∷}}\AgdaSpace{}%
\AgdaSymbol{(}\AgdaFunction{α}\AgdaSpace{}%
\AgdaOperator{\AgdaFunction{⇨}}\AgdaSpace{}%
\AgdaFunction{α}\AgdaSymbol{))}\AgdaSpace{}%
\AgdaFunction{a}\AgdaSymbol{)))}\<%
\\
%
\\[\AgdaEmptyExtraSkip]%
%
\>[2]\AgdaFunction{\AgdaUnderscore{}}\AgdaSpace{}%
\AgdaSymbol{:}\AgdaSpace{}%
\AgdaFunction{test6}\AgdaSpace{}%
\AgdaOperator{\AgdaDatatype{≡}}\AgdaSpace{}%
\AgdaInductiveConstructor{thunk}\AgdaSpace{}%
\AgdaSymbol{(}\AgdaFunction{app}\AgdaSpace{}%
\AgdaSymbol{(}\AgdaInductiveConstructor{var}\AgdaSpace{}%
\AgdaInductiveConstructor{ze}\AgdaSymbol{)}\AgdaSpace{}%
\AgdaFunction{a}\AgdaSymbol{)}\<%
\\
%
\>[2]\AgdaSymbol{\AgdaUnderscore{}}\AgdaSpace{}%
\AgdaSymbol{=}\AgdaSpace{}%
\AgdaInductiveConstructor{refl}\<%
\\
%
\\[\AgdaEmptyExtraSkip]%
%
\>[2]\AgdaComment{-- should normalize the target even if it results in a neutral term}\<%
\\
%
\\[\AgdaEmptyExtraSkip]%
%
\>[2]\AgdaFunction{test7}\AgdaSpace{}%
\AgdaSymbol{:}\AgdaSpace{}%
\AgdaDatatype{Const}\AgdaSpace{}%
\AgdaNumber{1}\<%
\\
%
\>[2]\AgdaFunction{test7}\AgdaSpace{}%
\AgdaSymbol{=}%
\>[426I]\AgdaFunction{normalize}\AgdaSpace{}%
\AgdaFunction{etarules}\AgdaSpace{}%
\AgdaFunction{betarules}\<%
\\
\>[.][@{}l@{}]\<[426I]%
\>[10]\AgdaSymbol{((}\AgdaFunction{infer}\AgdaSpace{}%
\AgdaFunction{rules}\AgdaSymbol{)}\AgdaSpace{}%
\AgdaOperator{\AgdaInductiveConstructor{,}}\AgdaSpace{}%
\AgdaFunction{PC}\AgdaSymbol{)}\AgdaSpace{}%
\AgdaSymbol{(}\AgdaInductiveConstructor{ε}\AgdaSpace{}%
\AgdaOperator{\AgdaInductiveConstructor{-,}}\AgdaSpace{}%
\AgdaFunction{α}\AgdaSymbol{)}\AgdaSpace{}%
\AgdaFunction{α}\<%
\\
%
\>[10]\AgdaSymbol{(}\AgdaFunction{app}\AgdaSpace{}%
\AgdaSymbol{(}\AgdaFunction{app}\AgdaSpace{}%
\AgdaSymbol{(}\AgdaFunction{lam}\AgdaSpace{}%
\AgdaSymbol{(}\AgdaFunction{\textasciitilde{}}\AgdaSpace{}%
\AgdaSymbol{(}\AgdaInductiveConstructor{su}\AgdaSpace{}%
\AgdaInductiveConstructor{ze}\AgdaSymbol{))}\AgdaSpace{}%
\AgdaOperator{\AgdaInductiveConstructor{∷}}\AgdaSpace{}%
\AgdaSymbol{(}\AgdaFunction{α}\AgdaSpace{}%
\AgdaOperator{\AgdaFunction{⇨}}\AgdaSpace{}%
\AgdaFunction{α}\AgdaSymbol{))}\AgdaSpace{}%
\AgdaFunction{a}\AgdaSymbol{)}\AgdaSpace{}%
\AgdaFunction{a}\AgdaSymbol{)}\<%
\\
%
\\[\AgdaEmptyExtraSkip]%
%
\>[2]\AgdaFunction{\AgdaUnderscore{}}\AgdaSpace{}%
\AgdaSymbol{:}\AgdaSpace{}%
\AgdaFunction{test7}\AgdaSpace{}%
\AgdaOperator{\AgdaDatatype{≡}}\AgdaSpace{}%
\AgdaInductiveConstructor{thunk}\AgdaSpace{}%
\AgdaSymbol{(}\AgdaFunction{app}\AgdaSpace{}%
\AgdaSymbol{(}\AgdaInductiveConstructor{var}\AgdaSpace{}%
\AgdaInductiveConstructor{ze}\AgdaSymbol{)}\AgdaSpace{}%
\AgdaFunction{a}\AgdaSymbol{)}\<%
\\
%
\>[2]\AgdaSymbol{\AgdaUnderscore{}}\AgdaSpace{}%
\AgdaSymbol{=}\AgdaSpace{}%
\AgdaInductiveConstructor{refl}\<%
\\
%
\\[\AgdaEmptyExtraSkip]%
%
\>[2]\AgdaComment{-- should normalize even if the elimination target body was initially stuck}\<%
\\
%
\>[2]\AgdaFunction{test8}\AgdaSpace{}%
\AgdaSymbol{:}\AgdaSpace{}%
\AgdaDatatype{Const}\AgdaSpace{}%
\AgdaNumber{0}\<%
\\
%
\>[2]\AgdaFunction{test8}\AgdaSpace{}%
\AgdaSymbol{=}%
\>[461I]\AgdaFunction{normalize}\AgdaSpace{}%
\AgdaFunction{etarules}\AgdaSpace{}%
\AgdaFunction{betarules}\<%
\\
\>[.][@{}l@{}]\<[461I]%
\>[10]\AgdaSymbol{((}\AgdaFunction{infer}\AgdaSpace{}%
\AgdaFunction{rules}\AgdaSymbol{)}\AgdaSpace{}%
\AgdaOperator{\AgdaInductiveConstructor{,}}\AgdaSpace{}%
\AgdaFunction{PC}\AgdaSymbol{)}\AgdaSpace{}%
\AgdaInductiveConstructor{ε}\AgdaSpace{}%
\AgdaFunction{α}\<%
\\
%
\>[10]\AgdaSymbol{(}\AgdaFunction{app}\AgdaSpace{}%
\AgdaSymbol{(}\AgdaFunction{lam}\AgdaSpace{}%
\AgdaSymbol{(}\AgdaInductiveConstructor{thunk}\AgdaSpace{}%
\AgdaSymbol{(}\AgdaFunction{app}\AgdaSpace{}%
\AgdaSymbol{(}\AgdaInductiveConstructor{var}\AgdaSpace{}%
\AgdaInductiveConstructor{ze}\AgdaSymbol{)}\AgdaSpace{}%
\AgdaFunction{a}\AgdaSymbol{))}\AgdaSpace{}%
\AgdaOperator{\AgdaInductiveConstructor{∷}}\AgdaSpace{}%
\AgdaSymbol{((}\AgdaFunction{α}\AgdaSpace{}%
\AgdaOperator{\AgdaFunction{⇨}}\AgdaSpace{}%
\AgdaFunction{α}\AgdaSymbol{)}\AgdaSpace{}%
\AgdaOperator{\AgdaFunction{⇨}}\AgdaSpace{}%
\AgdaFunction{α}\AgdaSymbol{))}\<%
\\
%
\>[10]\AgdaSymbol{(}\AgdaFunction{lam}\AgdaSpace{}%
\AgdaSymbol{(}\AgdaFunction{\textasciitilde{}}\AgdaSpace{}%
\AgdaInductiveConstructor{ze}\AgdaSymbol{)))}\<%
\\
%
\\[\AgdaEmptyExtraSkip]%
%
\>[2]\AgdaFunction{\AgdaUnderscore{}}\AgdaSpace{}%
\AgdaSymbol{:}\AgdaSpace{}%
\AgdaFunction{test8}\AgdaSpace{}%
\AgdaOperator{\AgdaDatatype{≡}}\AgdaSpace{}%
\AgdaFunction{a}\<%
\\
%
\>[2]\AgdaSymbol{\AgdaUnderscore{}}\AgdaSpace{}%
\AgdaSymbol{=}\AgdaSpace{}%
\AgdaInductiveConstructor{refl}\<%
\\
%
\\[\AgdaEmptyExtraSkip]%
%
\>[2]\AgdaComment{-- should eta-expand multiple times}\<%
\\
%
\>[2]\AgdaFunction{test9}\AgdaSpace{}%
\AgdaSymbol{:}\AgdaSpace{}%
\AgdaDatatype{Const}\AgdaSpace{}%
\AgdaNumber{1}\<%
\\
%
\>[2]\AgdaFunction{test9}\AgdaSpace{}%
\AgdaSymbol{=}%
\>[493I]\AgdaFunction{normalize}\AgdaSpace{}%
\AgdaFunction{etarules}\AgdaSpace{}%
\AgdaFunction{betarules}\<%
\\
\>[.][@{}l@{}]\<[493I]%
\>[10]\AgdaSymbol{((}\AgdaFunction{infer}\AgdaSpace{}%
\AgdaFunction{rules}\AgdaSymbol{)}\AgdaSpace{}%
\AgdaOperator{\AgdaInductiveConstructor{,}}\AgdaSpace{}%
\AgdaFunction{PC}\AgdaSymbol{)}\AgdaSpace{}%
\AgdaSymbol{(}\AgdaInductiveConstructor{ε}\AgdaSpace{}%
\AgdaOperator{\AgdaInductiveConstructor{-,}}\AgdaSpace{}%
\AgdaSymbol{(}\AgdaFunction{α}\AgdaSpace{}%
\AgdaOperator{\AgdaFunction{⇨}}\AgdaSpace{}%
\AgdaFunction{α}\AgdaSpace{}%
\AgdaOperator{\AgdaFunction{⇨}}\AgdaSpace{}%
\AgdaFunction{α}\AgdaSymbol{))}\AgdaSpace{}%
\AgdaSymbol{(}\AgdaFunction{α}\AgdaSpace{}%
\AgdaOperator{\AgdaFunction{⇨}}\AgdaSpace{}%
\AgdaFunction{α}\AgdaSpace{}%
\AgdaOperator{\AgdaFunction{⇨}}\AgdaSpace{}%
\AgdaFunction{α}\AgdaSymbol{)}\AgdaSpace{}%
\AgdaSymbol{(}\AgdaInductiveConstructor{var}\AgdaSpace{}%
\AgdaInductiveConstructor{ze}\AgdaSymbol{)}\<%
\\
%
\\[\AgdaEmptyExtraSkip]%
%
\>[2]\AgdaFunction{\AgdaUnderscore{}}\AgdaSpace{}%
\AgdaSymbol{:}\AgdaSpace{}%
\AgdaFunction{test9}\AgdaSpace{}%
\AgdaOperator{\AgdaDatatype{≡}}\AgdaSpace{}%
\AgdaFunction{lam}\AgdaSpace{}%
\AgdaSymbol{(}\AgdaFunction{lam}\AgdaSpace{}%
\AgdaSymbol{(}\AgdaInductiveConstructor{thunk}\AgdaSpace{}%
\AgdaSymbol{(}\AgdaFunction{app}\AgdaSpace{}%
\AgdaSymbol{(}\AgdaFunction{app}\AgdaSpace{}%
\AgdaSymbol{(}\AgdaInductiveConstructor{var}\AgdaSpace{}%
\AgdaSymbol{(}\AgdaInductiveConstructor{su}\AgdaSpace{}%
\AgdaSymbol{(}\AgdaInductiveConstructor{su}\AgdaSpace{}%
\AgdaInductiveConstructor{ze}\AgdaSymbol{)))}\AgdaSpace{}%
\AgdaSymbol{(}\AgdaFunction{\textasciitilde{}}\AgdaSpace{}%
\AgdaSymbol{(}\AgdaInductiveConstructor{su}\AgdaSpace{}%
\AgdaInductiveConstructor{ze}\AgdaSymbol{)))}\AgdaSpace{}%
\AgdaSymbol{(}\AgdaFunction{\textasciitilde{}}\AgdaSpace{}%
\AgdaInductiveConstructor{ze}\AgdaSymbol{))))}\<%
\\
%
\>[2]\AgdaSymbol{\AgdaUnderscore{}}\AgdaSpace{}%
\AgdaSymbol{=}\AgdaSpace{}%
\AgdaInductiveConstructor{refl}\<%
\\
\>[0]\<%
\end{code}


\subsection{Eta expansion}
\hide{
\begin{code}%
\>[0]\AgdaKeyword{module}\AgdaSpace{}%
\AgdaModule{Test.EtaTest}\AgdaSpace{}%
\AgdaKeyword{where}\<%
\\
%
\\[\AgdaEmptyExtraSkip]%
\>[0]\AgdaKeyword{open}\AgdaSpace{}%
\AgdaKeyword{import}\AgdaSpace{}%
\AgdaModule{Test.Specs.STLC}\AgdaSpace{}%
\AgdaKeyword{using}\AgdaSpace{}%
\AgdaSymbol{(}\AgdaFunction{etarules}\AgdaSymbol{;}\AgdaSpace{}%
\AgdaFunction{lam-ηrule}\AgdaSymbol{;}\AgdaSpace{}%
\AgdaFunction{betarules}\AgdaSymbol{;}\AgdaSpace{}%
\AgdaFunction{rules}\AgdaSymbol{)}\<%
\\
\>[0]\AgdaKeyword{open}\AgdaSpace{}%
\AgdaModule{Test.Specs.STLC.combinators}\<%
\\
\>[0]\AgdaKeyword{open}\AgdaSpace{}%
\AgdaKeyword{import}\AgdaSpace{}%
\AgdaModule{CoreLanguage}\<%
\\
\>[0]\AgdaKeyword{open}\AgdaSpace{}%
\AgdaKeyword{import}\AgdaSpace{}%
\AgdaModule{Pattern}\<%
\\
\>[0]\AgdaKeyword{open}\AgdaSpace{}%
\AgdaKeyword{import}\AgdaSpace{}%
\AgdaModule{Relation.Binary.PropositionalEquality}\AgdaSpace{}%
\AgdaKeyword{using}\AgdaSpace{}%
\AgdaSymbol{(}\AgdaOperator{\AgdaDatatype{\AgdaUnderscore{}≡\AgdaUnderscore{}}}\AgdaSymbol{;}\AgdaSpace{}%
\AgdaInductiveConstructor{refl}\AgdaSymbol{)}\<%
\\
\>[0]\AgdaKeyword{open}\AgdaSpace{}%
\AgdaKeyword{import}\AgdaSpace{}%
\AgdaModule{EtaRule}\<%
\\
\>[0]\AgdaKeyword{open}\AgdaSpace{}%
\AgdaKeyword{import}\AgdaSpace{}%
\AgdaModule{Data.Maybe}\AgdaSpace{}%
\AgdaKeyword{using}\AgdaSpace{}%
\AgdaSymbol{(}\AgdaDatatype{Maybe}\AgdaSymbol{;}\AgdaSpace{}%
\AgdaInductiveConstructor{just}\AgdaSymbol{;}\AgdaSpace{}%
\AgdaInductiveConstructor{nothing}\AgdaSymbol{)}\<%
\\
\>[0]\AgdaKeyword{open}\AgdaSpace{}%
\AgdaModule{η-Rule}\<%
\\
\>[0]\AgdaKeyword{open}\AgdaSpace{}%
\AgdaKeyword{import}\AgdaSpace{}%
\AgdaModule{Thinning}\AgdaSpace{}%
\AgdaKeyword{using}\AgdaSpace{}%
\AgdaSymbol{(}\AgdaOperator{\AgdaFunction{\AgdaUnderscore{}\textasciicircum{}term}}\AgdaSymbol{)}\<%
\\
\>[0]\AgdaKeyword{open}\AgdaSpace{}%
\AgdaKeyword{import}\AgdaSpace{}%
\AgdaModule{Failable}\<%
\\
\>[0]\AgdaKeyword{open}\AgdaSpace{}%
\AgdaKeyword{import}\AgdaSpace{}%
\AgdaModule{Data.List}\AgdaSpace{}%
\AgdaKeyword{using}\AgdaSpace{}%
\AgdaSymbol{(}\AgdaInductiveConstructor{[]}\AgdaSymbol{)}\<%
\end{code}
}
\begin{code}%
\>[0]\AgdaComment{---------------------------------------------}\<%
\\
\>[0]\AgdaComment{-- Test 1:}\<%
\\
\>[0]\AgdaComment{-- η expanding λx.a = λy.(λx.a y)}\<%
\\
\>[0]\AgdaComment{---------------------------------------------}\<%
\\
%
\\[\AgdaEmptyExtraSkip]%
\>[0]\AgdaFunction{test1}\AgdaSpace{}%
\AgdaSymbol{:}\AgdaSpace{}%
\AgdaDatatype{Const}\AgdaSpace{}%
\AgdaNumber{0}\<%
\\
\>[0]\AgdaFunction{test1}\AgdaSpace{}%
\AgdaSymbol{=}\AgdaSpace{}%
\AgdaFunction{lam}\AgdaSpace{}%
\AgdaFunction{a}\<%
\\
%
\\[\AgdaEmptyExtraSkip]%
\>[0]\AgdaFunction{test1type}\AgdaSpace{}%
\AgdaSymbol{:}\AgdaSpace{}%
\AgdaDatatype{Const}\AgdaSpace{}%
\AgdaNumber{0}\<%
\\
\>[0]\AgdaFunction{test1type}\AgdaSpace{}%
\AgdaSymbol{=}\AgdaSpace{}%
\AgdaFunction{α}\AgdaSpace{}%
\AgdaOperator{\AgdaFunction{⇨}}\AgdaSpace{}%
\AgdaFunction{α}\<%
\\
%
\\[\AgdaEmptyExtraSkip]%
\>[0]\AgdaFunction{\AgdaUnderscore{}}\AgdaSpace{}%
\AgdaSymbol{:}%
\>[52I]\AgdaFunction{η-match}%
\>[53I]\AgdaFunction{lam-ηrule}\AgdaSpace{}%
\AgdaFunction{test1type}\<%
\\
\>[.][@{}l@{}]\<[53I]%
\>[12]\AgdaOperator{\AgdaDatatype{≡}}\<%
\\
\>[52I][@{}l@{\AgdaIndent{0}}]%
\>[5]\AgdaInductiveConstructor{succeed}\AgdaSpace{}%
\AgdaSymbol{((}\AgdaInductiveConstructor{thing}\AgdaSpace{}%
\AgdaFunction{α}\AgdaSymbol{)}\AgdaSpace{}%
\AgdaOperator{\AgdaInductiveConstructor{∙}}\AgdaSpace{}%
\AgdaSymbol{(}\AgdaInductiveConstructor{`}\AgdaSpace{}%
\AgdaOperator{\AgdaInductiveConstructor{∙}}\AgdaSpace{}%
\AgdaSymbol{(}\AgdaInductiveConstructor{thing}\AgdaSpace{}%
\AgdaFunction{α}\AgdaSymbol{)))}\<%
\\
\>[0]\AgdaSymbol{\AgdaUnderscore{}}\AgdaSpace{}%
\AgdaSymbol{=}\AgdaSpace{}%
\AgdaInductiveConstructor{refl}\<%
\\
%
\\[\AgdaEmptyExtraSkip]%
\>[0]\AgdaFunction{\AgdaUnderscore{}}\AgdaSpace{}%
\AgdaSymbol{:}%
\>[65I]\AgdaFunction{η-expand}%
\>[66I]\AgdaFunction{lam-ηrule}\AgdaSpace{}%
\AgdaFunction{test1type}\AgdaSpace{}%
\AgdaFunction{test1}\<%
\\
\>[.][@{}l@{}]\<[66I]%
\>[13]\AgdaOperator{\AgdaDatatype{≡}}\<%
\\
\>[65I][@{}l@{\AgdaIndent{0}}]%
\>[5]\AgdaFunction{lam}\AgdaSpace{}%
\AgdaSymbol{(}\AgdaInductiveConstructor{thunk}\AgdaSpace{}%
\AgdaSymbol{(}\AgdaInductiveConstructor{elim}\AgdaSpace{}%
\AgdaSymbol{((}\AgdaFunction{test1}\AgdaSpace{}%
\AgdaOperator{\AgdaFunction{\textasciicircum{}term}}\AgdaSymbol{)}\AgdaSpace{}%
\AgdaOperator{\AgdaInductiveConstructor{∷}}\AgdaSpace{}%
\AgdaSymbol{(}\AgdaFunction{test1type}\AgdaSpace{}%
\AgdaOperator{\AgdaFunction{\textasciicircum{}term}}\AgdaSymbol{))}\AgdaSpace{}%
\AgdaSymbol{(}\AgdaFunction{\textasciitilde{}}\AgdaSpace{}%
\AgdaInductiveConstructor{ze}\AgdaSymbol{)))}\<%
\\
\>[0]\AgdaSymbol{\AgdaUnderscore{}}\AgdaSpace{}%
\AgdaSymbol{=}\AgdaSpace{}%
\AgdaInductiveConstructor{refl}\<%
\\
%
\\[\AgdaEmptyExtraSkip]%
%
\\[\AgdaEmptyExtraSkip]%
\>[0]\AgdaComment{--------------------------------------------------}\<%
\\
\>[0]\AgdaComment{-- Test 2:}\<%
\\
\>[0]\AgdaComment{-- η expanding ((λx.λy.x) b) = λy.(((λx.λy.x) b) y)}\<%
\\
\>[0]\AgdaComment{---------------------------------------------------}\<%
\\
%
\\[\AgdaEmptyExtraSkip]%
\>[0]\AgdaFunction{test2}\AgdaSpace{}%
\AgdaSymbol{:}\AgdaSpace{}%
\AgdaDatatype{Const}\AgdaSpace{}%
\AgdaNumber{0}\<%
\\
\>[0]\AgdaFunction{test2}\AgdaSpace{}%
\AgdaSymbol{=}\AgdaSpace{}%
\AgdaInductiveConstructor{thunk}\AgdaSpace{}%
\AgdaSymbol{(}\AgdaInductiveConstructor{elim}\AgdaSpace{}%
\AgdaSymbol{(}\AgdaFunction{lam}\AgdaSpace{}%
\AgdaSymbol{(}\AgdaFunction{lam}\AgdaSpace{}%
\AgdaSymbol{(}\AgdaFunction{\textasciitilde{}}\AgdaSpace{}%
\AgdaSymbol{(}\AgdaInductiveConstructor{su}\AgdaSpace{}%
\AgdaInductiveConstructor{ze}\AgdaSymbol{)))}\AgdaSpace{}%
\AgdaOperator{\AgdaInductiveConstructor{∷}}\AgdaSpace{}%
\AgdaSymbol{(}\AgdaFunction{β}\AgdaSpace{}%
\AgdaOperator{\AgdaFunction{⇨}}\AgdaSpace{}%
\AgdaFunction{α}\AgdaSpace{}%
\AgdaOperator{\AgdaFunction{⇨}}\AgdaSpace{}%
\AgdaFunction{β}\AgdaSymbol{))}\AgdaSpace{}%
\AgdaFunction{b}\AgdaSymbol{)}\<%
\\
%
\\[\AgdaEmptyExtraSkip]%
\>[0]\AgdaFunction{test2type}\AgdaSpace{}%
\AgdaSymbol{:}\AgdaSpace{}%
\AgdaDatatype{Const}\AgdaSpace{}%
\AgdaNumber{0}\<%
\\
\>[0]\AgdaFunction{test2type}\AgdaSpace{}%
\AgdaSymbol{=}\AgdaSpace{}%
\AgdaFunction{α}\AgdaSpace{}%
\AgdaOperator{\AgdaFunction{⇨}}\AgdaSpace{}%
\AgdaFunction{β}\<%
\\
%
\\[\AgdaEmptyExtraSkip]%
\>[0]\AgdaFunction{\AgdaUnderscore{}}\AgdaSpace{}%
\AgdaSymbol{:}%
\>[106I]\AgdaFunction{η-match}%
\>[107I]\AgdaFunction{lam-ηrule}\AgdaSpace{}%
\AgdaFunction{test2type}\<%
\\
\>[.][@{}l@{}]\<[107I]%
\>[12]\AgdaOperator{\AgdaDatatype{≡}}\<%
\\
\>[106I][@{}l@{\AgdaIndent{0}}]%
\>[5]\AgdaInductiveConstructor{succeed}\AgdaSpace{}%
\AgdaSymbol{((}\AgdaInductiveConstructor{thing}\AgdaSpace{}%
\AgdaFunction{α}\AgdaSymbol{)}\AgdaSpace{}%
\AgdaOperator{\AgdaInductiveConstructor{∙}}\AgdaSpace{}%
\AgdaSymbol{(}\AgdaInductiveConstructor{`}\AgdaSpace{}%
\AgdaOperator{\AgdaInductiveConstructor{∙}}\AgdaSpace{}%
\AgdaSymbol{(}\AgdaInductiveConstructor{thing}\AgdaSpace{}%
\AgdaFunction{β}\AgdaSymbol{)))}\<%
\\
\>[0]\AgdaSymbol{\AgdaUnderscore{}}\AgdaSpace{}%
\AgdaSymbol{=}\AgdaSpace{}%
\AgdaInductiveConstructor{refl}\<%
\\
%
\\[\AgdaEmptyExtraSkip]%
\>[0]\AgdaFunction{\AgdaUnderscore{}}\AgdaSpace{}%
\AgdaSymbol{:}%
\>[119I]\AgdaFunction{η-expand}%
\>[120I]\AgdaFunction{lam-ηrule}\AgdaSpace{}%
\AgdaFunction{test2type}\AgdaSpace{}%
\AgdaFunction{test2}\<%
\\
\>[.][@{}l@{}]\<[120I]%
\>[13]\AgdaOperator{\AgdaDatatype{≡}}\<%
\\
\>[119I][@{}l@{\AgdaIndent{0}}]%
\>[5]\AgdaFunction{lam}\AgdaSpace{}%
\AgdaSymbol{(}\AgdaInductiveConstructor{thunk}\AgdaSpace{}%
\AgdaSymbol{(}\AgdaInductiveConstructor{elim}\AgdaSpace{}%
\AgdaSymbol{(}\AgdaFunction{↞↞}\AgdaSpace{}%
\AgdaSymbol{(}\AgdaFunction{test2}\AgdaSpace{}%
\AgdaOperator{\AgdaFunction{\textasciicircum{}term}}\AgdaSymbol{)}\AgdaSpace{}%
\AgdaSymbol{(}\AgdaFunction{test2type}\AgdaSpace{}%
\AgdaOperator{\AgdaFunction{\textasciicircum{}term}}\AgdaSymbol{))}\AgdaSpace{}%
\AgdaSymbol{(}\AgdaFunction{\textasciitilde{}}\AgdaSpace{}%
\AgdaInductiveConstructor{ze}\AgdaSymbol{)))}\<%
\\
\>[0]\AgdaSymbol{\AgdaUnderscore{}}\AgdaSpace{}%
\AgdaSymbol{=}\AgdaSpace{}%
\AgdaInductiveConstructor{refl}\<%
\\
\>[0]\<%
\end{code}


\subsection{Type checking STLC}
\hide{
\begin{code}%
\>[0]\AgdaKeyword{module}\AgdaSpace{}%
\AgdaModule{Test.STLCTest}\AgdaSpace{}%
\AgdaKeyword{where}\<%
\end{code}
}

\hide{
\begin{code}%
\>[0]\AgdaKeyword{import}\AgdaSpace{}%
\AgdaModule{Test.Specs.STLC}\AgdaSpace{}%
\AgdaSymbol{as}\AgdaSpace{}%
\AgdaModule{STLC}\<%
\\
\>[0]\AgdaKeyword{open}\AgdaSpace{}%
\AgdaKeyword{import}\AgdaSpace{}%
\AgdaModule{CoreLanguage}\<%
\\
\>[0]\AgdaKeyword{open}\AgdaSpace{}%
\AgdaKeyword{import}\AgdaSpace{}%
\AgdaModule{Failable}\<%
\\
\>[0]\AgdaKeyword{open}\AgdaSpace{}%
\AgdaModule{STLC}\AgdaSpace{}%
\AgdaKeyword{using}\AgdaSpace{}%
\AgdaSymbol{(}\AgdaFunction{rules}\AgdaSymbol{)}\<%
\\
\>[0]\AgdaKeyword{open}\AgdaSpace{}%
\AgdaModule{STLC.combinators}\<%
\\
\>[0]\AgdaKeyword{open}\AgdaSpace{}%
\AgdaKeyword{import}\AgdaSpace{}%
\AgdaModule{BwdVec}\AgdaSpace{}%
\AgdaKeyword{using}\AgdaSpace{}%
\AgdaSymbol{(}\AgdaInductiveConstructor{ε}\AgdaSymbol{;}\AgdaSpace{}%
\AgdaOperator{\AgdaInductiveConstructor{\AgdaUnderscore{}-,\AgdaUnderscore{}}}\AgdaSymbol{)}\<%
\\
\>[0]\AgdaKeyword{open}\AgdaSpace{}%
\AgdaKeyword{import}\AgdaSpace{}%
\AgdaModule{TypeChecker}\AgdaSpace{}%
\AgdaKeyword{using}\AgdaSpace{}%
\AgdaSymbol{(}\AgdaFunction{infer}\AgdaSymbol{)}\<%
\\
\>[0]\AgdaKeyword{open}\AgdaSpace{}%
\AgdaKeyword{import}\AgdaSpace{}%
\AgdaModule{Relation.Binary.PropositionalEquality}\AgdaSpace{}%
\AgdaKeyword{using}\AgdaSpace{}%
\AgdaSymbol{(}\AgdaOperator{\AgdaDatatype{\AgdaUnderscore{}≡\AgdaUnderscore{}}}\AgdaSymbol{;}\AgdaSpace{}%
\AgdaInductiveConstructor{refl}\AgdaSymbol{)}\<%
\end{code}
}

\begin{code}%
\>[0]\<%
\\
\>[0]\AgdaComment{-- should check annotated terms are typed:}\<%
\\
%
\\[\AgdaEmptyExtraSkip]%
\>[0]\AgdaFunction{\AgdaUnderscore{}}\AgdaSpace{}%
\AgdaSymbol{:}%
\>[28I]\AgdaFunction{infer}\AgdaSpace{}%
\AgdaFunction{rules}\AgdaSpace{}%
\AgdaInductiveConstructor{ε}\AgdaSpace{}%
\AgdaSymbol{(}\AgdaFunction{lam}\AgdaSpace{}%
\AgdaSymbol{(}\AgdaFunction{\textasciitilde{}}\AgdaSpace{}%
\AgdaInductiveConstructor{ze}\AgdaSymbol{)}\AgdaSpace{}%
\AgdaOperator{\AgdaInductiveConstructor{∷}}\AgdaSpace{}%
\AgdaSymbol{(}\AgdaFunction{α}\AgdaSpace{}%
\AgdaOperator{\AgdaFunction{⇨}}\AgdaSpace{}%
\AgdaFunction{α}\AgdaSymbol{))}\<%
\\
\>[.][@{}l@{}]\<[28I]%
\>[4]\AgdaOperator{\AgdaDatatype{≡}}\<%
\\
%
\>[4]\AgdaInductiveConstructor{succeed}\AgdaSpace{}%
\AgdaSymbol{(}\AgdaFunction{α}\AgdaSpace{}%
\AgdaOperator{\AgdaFunction{⇨}}\AgdaSpace{}%
\AgdaFunction{α}\AgdaSymbol{)}\<%
\\
\>[0]\AgdaSymbol{\AgdaUnderscore{}}\AgdaSpace{}%
\AgdaSymbol{=}\AgdaSpace{}%
\AgdaInductiveConstructor{refl}\<%
\\
%
\\[\AgdaEmptyExtraSkip]%
\>[0]\AgdaFunction{\AgdaUnderscore{}}\AgdaSpace{}%
\AgdaSymbol{:}%
\>[44I]\AgdaFunction{infer}\AgdaSpace{}%
\AgdaFunction{rules}\AgdaSpace{}%
\AgdaInductiveConstructor{ε}\AgdaSpace{}%
\AgdaSymbol{(}\AgdaFunction{α}\AgdaSpace{}%
\AgdaOperator{\AgdaInductiveConstructor{∷}}\AgdaSpace{}%
\AgdaSymbol{(}\AgdaFunction{α}\AgdaSpace{}%
\AgdaOperator{\AgdaFunction{⇨}}\AgdaSpace{}%
\AgdaFunction{α}\AgdaSymbol{))}\<%
\\
\>[.][@{}l@{}]\<[44I]%
\>[4]\AgdaOperator{\AgdaDatatype{≡}}\<%
\\
%
\>[4]\AgdaInductiveConstructor{fail}\AgdaSpace{}%
\AgdaString{"failed ∋-check: α→α ∋ α"}\<%
\\
\>[0]\AgdaSymbol{\AgdaUnderscore{}}\AgdaSpace{}%
\AgdaSymbol{=}\AgdaSpace{}%
\AgdaInductiveConstructor{refl}\<%
\\
%
\\[\AgdaEmptyExtraSkip]%
\>[0]\AgdaComment{-- should check applications are typed:}\<%
\\
%
\\[\AgdaEmptyExtraSkip]%
\>[0]\AgdaFunction{\AgdaUnderscore{}}\AgdaSpace{}%
\AgdaSymbol{:}%
\>[56I]\AgdaFunction{infer}\AgdaSpace{}%
\AgdaFunction{rules}\AgdaSpace{}%
\AgdaInductiveConstructor{ε}\AgdaSpace{}%
\AgdaSymbol{(}\AgdaFunction{app}\AgdaSpace{}%
\AgdaSymbol{(}\AgdaFunction{lam}\AgdaSpace{}%
\AgdaSymbol{(}\AgdaFunction{\textasciitilde{}}\AgdaSpace{}%
\AgdaInductiveConstructor{ze}\AgdaSymbol{)}\AgdaSpace{}%
\AgdaOperator{\AgdaInductiveConstructor{∷}}\AgdaSpace{}%
\AgdaSymbol{(}\AgdaFunction{α}\AgdaSpace{}%
\AgdaOperator{\AgdaFunction{⇨}}\AgdaSpace{}%
\AgdaFunction{α}\AgdaSymbol{))}\AgdaSpace{}%
\AgdaFunction{a}\AgdaSymbol{)}\<%
\\
\>[.][@{}l@{}]\<[56I]%
\>[4]\AgdaOperator{\AgdaDatatype{≡}}\<%
\\
%
\>[4]\AgdaInductiveConstructor{succeed}\AgdaSpace{}%
\AgdaFunction{α}\<%
\\
\>[0]\AgdaSymbol{\AgdaUnderscore{}}\AgdaSpace{}%
\AgdaSymbol{=}\AgdaSpace{}%
\AgdaInductiveConstructor{refl}\<%
\\
%
\\[\AgdaEmptyExtraSkip]%
\>[0]\AgdaFunction{\AgdaUnderscore{}}\AgdaSpace{}%
\AgdaSymbol{:}%
\>[72I]\AgdaFunction{infer}\AgdaSpace{}%
\AgdaFunction{rules}\AgdaSpace{}%
\AgdaInductiveConstructor{ε}\AgdaSpace{}%
\AgdaSymbol{(}\AgdaFunction{app}\AgdaSpace{}%
\AgdaSymbol{(}\AgdaFunction{lam}\AgdaSpace{}%
\AgdaSymbol{(}\AgdaFunction{\textasciitilde{}}\AgdaSpace{}%
\AgdaInductiveConstructor{ze}\AgdaSymbol{)}\AgdaSpace{}%
\AgdaOperator{\AgdaInductiveConstructor{∷}}\AgdaSpace{}%
\AgdaSymbol{((}\AgdaFunction{α}\AgdaSpace{}%
\AgdaOperator{\AgdaFunction{⇨}}\AgdaSpace{}%
\AgdaFunction{α}\AgdaSymbol{)}\AgdaSpace{}%
\AgdaOperator{\AgdaFunction{⇨}}\AgdaSpace{}%
\AgdaSymbol{(}\AgdaFunction{α}\AgdaSpace{}%
\AgdaOperator{\AgdaFunction{⇨}}\AgdaSpace{}%
\AgdaFunction{α}\AgdaSymbol{)))}\AgdaSpace{}%
\AgdaSymbol{(}\AgdaFunction{lam}\AgdaSpace{}%
\AgdaSymbol{(}\AgdaFunction{\textasciitilde{}}\AgdaSpace{}%
\AgdaInductiveConstructor{ze}\AgdaSymbol{)))}\<%
\\
\>[.][@{}l@{}]\<[72I]%
\>[4]\AgdaOperator{\AgdaDatatype{≡}}\<%
\\
%
\>[4]\AgdaInductiveConstructor{succeed}\AgdaSpace{}%
\AgdaSymbol{(}\AgdaFunction{α}\AgdaSpace{}%
\AgdaOperator{\AgdaFunction{⇨}}\AgdaSpace{}%
\AgdaFunction{α}\AgdaSymbol{)}\<%
\\
\>[0]\AgdaSymbol{\AgdaUnderscore{}}\AgdaSpace{}%
\AgdaSymbol{=}\AgdaSpace{}%
\AgdaInductiveConstructor{refl}\<%
\\
%
\\[\AgdaEmptyExtraSkip]%
\>[0]\AgdaFunction{\AgdaUnderscore{}}\AgdaSpace{}%
\AgdaSymbol{:}%
\>[96I]\AgdaFunction{infer}\AgdaSpace{}%
\AgdaFunction{rules}\AgdaSpace{}%
\AgdaInductiveConstructor{ε}\AgdaSpace{}%
\AgdaSymbol{(}\AgdaFunction{app}\AgdaSpace{}%
\AgdaSymbol{((}\AgdaFunction{lam}\AgdaSpace{}%
\AgdaFunction{b}\AgdaSymbol{)}\AgdaSpace{}%
\AgdaOperator{\AgdaInductiveConstructor{∷}}\AgdaSpace{}%
\AgdaSymbol{(}\AgdaFunction{α}\AgdaSpace{}%
\AgdaOperator{\AgdaFunction{⇨}}\AgdaSpace{}%
\AgdaFunction{β}\AgdaSymbol{))}\AgdaSpace{}%
\AgdaFunction{a}\AgdaSymbol{)}\<%
\\
\>[.][@{}l@{}]\<[96I]%
\>[4]\AgdaOperator{\AgdaDatatype{≡}}\<%
\\
%
\>[4]\AgdaInductiveConstructor{succeed}\AgdaSpace{}%
\AgdaFunction{β}\<%
\\
\>[0]\AgdaSymbol{\AgdaUnderscore{}}\AgdaSpace{}%
\AgdaSymbol{=}\AgdaSpace{}%
\AgdaInductiveConstructor{refl}\<%
\\
%
\\[\AgdaEmptyExtraSkip]%
\>[0]\AgdaFunction{\AgdaUnderscore{}}\AgdaSpace{}%
\AgdaSymbol{:}%
\>[111I]\AgdaFunction{infer}\AgdaSpace{}%
\AgdaFunction{rules}\AgdaSpace{}%
\AgdaInductiveConstructor{ε}\AgdaSpace{}%
\AgdaSymbol{(}\AgdaFunction{app}\<%
\\
\>[111I][@{}l@{\AgdaIndent{0}}]%
\>[9]\AgdaSymbol{((}\AgdaFunction{lam}\AgdaSpace{}%
\AgdaSymbol{(}\AgdaInductiveConstructor{thunk}\AgdaSpace{}%
\AgdaSymbol{(}\AgdaInductiveConstructor{elim}\AgdaSpace{}%
\AgdaSymbol{(}\AgdaInductiveConstructor{var}\AgdaSpace{}%
\AgdaInductiveConstructor{ze}\AgdaSymbol{)}\AgdaSpace{}%
\AgdaFunction{a}\AgdaSymbol{)))}\AgdaSpace{}%
\AgdaOperator{\AgdaInductiveConstructor{∷}}\AgdaSpace{}%
\AgdaSymbol{((}\AgdaFunction{α}\AgdaSpace{}%
\AgdaOperator{\AgdaFunction{⇨}}\AgdaSpace{}%
\AgdaFunction{α}\AgdaSymbol{)}\AgdaSpace{}%
\AgdaOperator{\AgdaFunction{⇨}}\AgdaSpace{}%
\AgdaFunction{α}\AgdaSymbol{))}\<%
\\
%
\>[9]\AgdaSymbol{(}\AgdaInductiveConstructor{thunk}\AgdaSpace{}%
\AgdaSymbol{(}\AgdaInductiveConstructor{elim}\AgdaSpace{}%
\AgdaSymbol{(}\AgdaFunction{lam}\AgdaSpace{}%
\AgdaSymbol{(}\AgdaFunction{\textasciitilde{}}\AgdaSpace{}%
\AgdaInductiveConstructor{ze}\AgdaSymbol{)}\AgdaSpace{}%
\AgdaOperator{\AgdaInductiveConstructor{∷}}\AgdaSpace{}%
\AgdaSymbol{((}\AgdaFunction{α}\AgdaSpace{}%
\AgdaOperator{\AgdaFunction{⇨}}\AgdaSpace{}%
\AgdaFunction{α}\AgdaSymbol{)}\AgdaSpace{}%
\AgdaOperator{\AgdaFunction{⇨}}\AgdaSpace{}%
\AgdaSymbol{(}\AgdaFunction{α}\AgdaSpace{}%
\AgdaOperator{\AgdaFunction{⇨}}\AgdaSpace{}%
\AgdaFunction{α}\AgdaSymbol{)))}\AgdaSpace{}%
\AgdaSymbol{(}\AgdaFunction{lam}\AgdaSpace{}%
\AgdaSymbol{(}\AgdaFunction{\textasciitilde{}}\AgdaSpace{}%
\AgdaInductiveConstructor{ze}\AgdaSymbol{)))))}\<%
\\
\>[.][@{}l@{}]\<[111I]%
\>[4]\AgdaOperator{\AgdaDatatype{≡}}\<%
\\
%
\>[4]\AgdaInductiveConstructor{succeed}\AgdaSpace{}%
\AgdaFunction{α}\<%
\\
\>[0]\AgdaSymbol{\AgdaUnderscore{}}\AgdaSpace{}%
\AgdaSymbol{=}\AgdaSpace{}%
\AgdaInductiveConstructor{refl}\<%
\\
%
\\[\AgdaEmptyExtraSkip]%
\>[0]\AgdaFunction{\AgdaUnderscore{}}\AgdaSpace{}%
\AgdaSymbol{:}%
\>[145I]\AgdaFunction{infer}\AgdaSpace{}%
\AgdaFunction{rules}\AgdaSpace{}%
\AgdaSymbol{(}\AgdaInductiveConstructor{ε}\AgdaSpace{}%
\AgdaOperator{\AgdaInductiveConstructor{-,}}\AgdaSpace{}%
\AgdaSymbol{(}\AgdaFunction{α}\AgdaSpace{}%
\AgdaOperator{\AgdaFunction{⇨}}\AgdaSpace{}%
\AgdaFunction{α}\AgdaSymbol{)}\AgdaSpace{}%
\AgdaOperator{\AgdaInductiveConstructor{-,}}\AgdaSpace{}%
\AgdaFunction{β}\AgdaSymbol{)}\AgdaSpace{}%
\AgdaSymbol{((}\AgdaFunction{lam}\AgdaSpace{}%
\AgdaSymbol{(}\AgdaInductiveConstructor{thunk}\AgdaSpace{}%
\AgdaSymbol{(}\AgdaInductiveConstructor{elim}\AgdaSpace{}%
\AgdaSymbol{(}\AgdaInductiveConstructor{var}\AgdaSpace{}%
\AgdaSymbol{(}\AgdaInductiveConstructor{su}\AgdaSpace{}%
\AgdaSymbol{(}\AgdaInductiveConstructor{su}\AgdaSpace{}%
\AgdaInductiveConstructor{ze}\AgdaSymbol{)))}\AgdaSpace{}%
\AgdaFunction{a}\AgdaSymbol{)))}\AgdaSpace{}%
\AgdaOperator{\AgdaInductiveConstructor{∷}}\AgdaSpace{}%
\AgdaSymbol{(}\AgdaFunction{β}\AgdaSpace{}%
\AgdaOperator{\AgdaFunction{⇨}}\AgdaSpace{}%
\AgdaFunction{α}\AgdaSymbol{))}\<%
\\
\>[.][@{}l@{}]\<[145I]%
\>[4]\AgdaOperator{\AgdaDatatype{≡}}\<%
\\
%
\>[4]\AgdaInductiveConstructor{succeed}\AgdaSpace{}%
\AgdaSymbol{(}\AgdaFunction{β}\AgdaSpace{}%
\AgdaOperator{\AgdaFunction{⇨}}\AgdaSpace{}%
\AgdaFunction{α}\AgdaSymbol{)}\<%
\\
\>[0]\AgdaSymbol{\AgdaUnderscore{}}\AgdaSpace{}%
\AgdaSymbol{=}\AgdaSpace{}%
\AgdaInductiveConstructor{refl}\<%
\\
%
\\[\AgdaEmptyExtraSkip]%
\>[0]\AgdaComment{-- should check the target of elimination first:}\<%
\\
%
\\[\AgdaEmptyExtraSkip]%
\>[0]\AgdaFunction{\AgdaUnderscore{}}\AgdaSpace{}%
\AgdaSymbol{:}%
\>[172I]\AgdaFunction{infer}\AgdaSpace{}%
\AgdaFunction{rules}\AgdaSpace{}%
\AgdaInductiveConstructor{ε}\AgdaSpace{}%
\AgdaSymbol{(}\AgdaFunction{app}\AgdaSpace{}%
\AgdaSymbol{((}\AgdaFunction{lam}\AgdaSpace{}%
\AgdaFunction{a}\AgdaSymbol{)}\AgdaSpace{}%
\AgdaOperator{\AgdaInductiveConstructor{∷}}\AgdaSpace{}%
\AgdaSymbol{(}\AgdaFunction{α}\AgdaSpace{}%
\AgdaOperator{\AgdaFunction{⇨}}\AgdaSpace{}%
\AgdaFunction{β}\AgdaSymbol{))}\AgdaSpace{}%
\AgdaFunction{b}\AgdaSymbol{)}\<%
\\
\>[.][@{}l@{}]\<[172I]%
\>[4]\AgdaOperator{\AgdaDatatype{≡}}\<%
\\
%
\>[4]\AgdaInductiveConstructor{fail}\AgdaSpace{}%
\AgdaString{"failed ∋-check: β ∋ a"}\<%
\\
\>[0]\AgdaSymbol{\AgdaUnderscore{}}\AgdaSpace{}%
\AgdaSymbol{=}\AgdaSpace{}%
\AgdaInductiveConstructor{refl}\<%
\\
%
\\[\AgdaEmptyExtraSkip]%
\>[0]\AgdaComment{-- if target of elimination passes typchecking, should check the eliminator:}\<%
\\
%
\\[\AgdaEmptyExtraSkip]%
\>[0]\AgdaFunction{\AgdaUnderscore{}}\AgdaSpace{}%
\AgdaSymbol{:}%
\>[187I]\AgdaFunction{infer}\AgdaSpace{}%
\AgdaFunction{rules}\AgdaSpace{}%
\AgdaInductiveConstructor{ε}\AgdaSpace{}%
\AgdaSymbol{(}\AgdaFunction{app}\AgdaSpace{}%
\AgdaSymbol{(}\AgdaFunction{lam}\AgdaSpace{}%
\AgdaSymbol{(}\AgdaInductiveConstructor{thunk}\AgdaSpace{}%
\AgdaSymbol{(}\AgdaInductiveConstructor{var}\AgdaSpace{}%
\AgdaInductiveConstructor{ze}\AgdaSymbol{))}\AgdaSpace{}%
\AgdaOperator{\AgdaInductiveConstructor{∷}}\AgdaSpace{}%
\AgdaSymbol{(}\AgdaFunction{α}\AgdaSpace{}%
\AgdaOperator{\AgdaFunction{⇨}}\AgdaSpace{}%
\AgdaFunction{α}\AgdaSymbol{))}\AgdaSpace{}%
\AgdaFunction{b}\AgdaSymbol{)}\<%
\\
\>[.][@{}l@{}]\<[187I]%
\>[4]\AgdaOperator{\AgdaDatatype{≡}}\<%
\\
%
\>[4]\AgdaInductiveConstructor{fail}\AgdaSpace{}%
\AgdaString{"failed ∋-check: α ∋ b"}\<%
\\
\>[0]\AgdaSymbol{\AgdaUnderscore{}}\AgdaSpace{}%
\AgdaSymbol{=}\AgdaSpace{}%
\AgdaInductiveConstructor{refl}\<%
\\
%
\\[\AgdaEmptyExtraSkip]%
\>[0]\AgdaComment{-- should correctly type nested eliminations}\<%
\\
%
\\[\AgdaEmptyExtraSkip]%
\>[0]\AgdaFunction{\AgdaUnderscore{}}\AgdaSpace{}%
\AgdaSymbol{:}%
\>[204I]\AgdaFunction{infer}%
\>[205I]\AgdaFunction{rules}\AgdaSpace{}%
\AgdaInductiveConstructor{ε}\<%
\\
\>[.][@{}l@{}]\<[205I]%
\>[10]\AgdaSymbol{(}\AgdaInductiveConstructor{elim}\<%
\\
\>[10][@{}l@{\AgdaIndent{0}}]%
\>[12]\AgdaSymbol{(}\AgdaFunction{lam}\AgdaSpace{}%
\AgdaSymbol{(}\AgdaInductiveConstructor{thunk}\<%
\\
\>[12][@{}l@{\AgdaIndent{0}}]%
\>[14]\AgdaSymbol{(}\AgdaInductiveConstructor{elim}\<%
\\
\>[14][@{}l@{\AgdaIndent{0}}]%
\>[16]\AgdaSymbol{(}\AgdaInductiveConstructor{var}\AgdaSpace{}%
\AgdaInductiveConstructor{ze}\AgdaSymbol{)}\<%
\\
\>[16][@{}l@{\AgdaIndent{0}}]%
\>[17]\AgdaFunction{a}\AgdaSymbol{))}\<%
\\
%
\>[14]\AgdaOperator{\AgdaInductiveConstructor{∷}}\AgdaSpace{}%
\AgdaSymbol{((}\AgdaFunction{α}\AgdaSpace{}%
\AgdaOperator{\AgdaFunction{⇨}}\AgdaSpace{}%
\AgdaFunction{α}\AgdaSymbol{)}\AgdaSpace{}%
\AgdaOperator{\AgdaFunction{⇨}}\AgdaSpace{}%
\AgdaFunction{α}\AgdaSymbol{))}\<%
\\
%
\>[12]\AgdaSymbol{(}\AgdaInductiveConstructor{thunk}\<%
\\
\>[12][@{}l@{\AgdaIndent{0}}]%
\>[14]\AgdaSymbol{(}\AgdaInductiveConstructor{elim}\<%
\\
\>[14][@{}l@{\AgdaIndent{0}}]%
\>[16]\AgdaSymbol{((}\AgdaFunction{lam}\AgdaSpace{}%
\AgdaSymbol{(}\AgdaFunction{\textasciitilde{}}\AgdaSpace{}%
\AgdaInductiveConstructor{ze}\AgdaSymbol{))}\AgdaSpace{}%
\AgdaOperator{\AgdaInductiveConstructor{∷}}\AgdaSpace{}%
\AgdaSymbol{((}\AgdaFunction{α}\AgdaSpace{}%
\AgdaOperator{\AgdaFunction{⇨}}\AgdaSpace{}%
\AgdaFunction{α}\AgdaSymbol{)}\AgdaSpace{}%
\AgdaOperator{\AgdaFunction{⇨}}\AgdaSpace{}%
\AgdaSymbol{(}\AgdaFunction{α}\AgdaSpace{}%
\AgdaOperator{\AgdaFunction{⇨}}\AgdaSpace{}%
\AgdaFunction{α}\AgdaSymbol{)))}\<%
\\
%
\>[16]\AgdaSymbol{((}\AgdaFunction{lam}\AgdaSpace{}%
\AgdaSymbol{(}\AgdaFunction{\textasciitilde{}}\AgdaSpace{}%
\AgdaInductiveConstructor{ze}\AgdaSymbol{))))))}\<%
\\
\>[.][@{}l@{}]\<[204I]%
\>[4]\AgdaOperator{\AgdaDatatype{≡}}\<%
\\
%
\>[4]\AgdaInductiveConstructor{succeed}\AgdaSpace{}%
\AgdaFunction{α}\<%
\\
\>[0]\AgdaSymbol{\AgdaUnderscore{}}\AgdaSpace{}%
\AgdaSymbol{=}\AgdaSpace{}%
\AgdaInductiveConstructor{refl}\<%
\end{code}

\clearpage
\section{Lattice meet/join algorithms}
\label{appendix-joinalgorithms}

We simplify these algorithms by ignoring scope, in practice we would
need to consider scope if we were to use these algorithms in our code.

The following algorithm may be used to calculate the join of a set
of patterns.

\begin{enumerate}
\item Consider all the topmost data constructors in the patterns
  (disregarding places).
  \begin{enumerate}
  \item if they are not all the same then the join is a \emph{place}
  \item if they are all \emph{atoms}
    \begin{enumerate}
    \item if they are all the same atom, then the join is that atom
      \item if they are all different atoms, then the join is a \emph{place}
    \end{enumerate}
  \item if they were all \emph{place}s then the join is a \emph{place}
  \item if the are all \emph{⊥}s then the join is \emph{⊥}  
  \item if they are all pairs
    \begin{enumerate}
      \item calculate the join of all the left elements $J_l$
      \item calulate the join of all the right elements $J_r$
      \item the join is the pair ($J_l$ ∙ $J_r$)
    \end{enumerate}
  \item if they are all binds
    \begin{enumerate}
    \item calculate the join of the terms under all the binders $J_b$
    \item the join is (bind $J_b$)
    \end{enumerate}
  \end{enumerate}
\end{enumerate}

The following algorithm may be used to calculate the meet of a set
of patterns.

\begin{enumerate}
\item Consider all the topmost data constructors in the patterns
  (disregarding places).
  \begin{enumerate}
  \item if they are not all the same then the meet is \emph{⊥}
  \item if they are all \emph{atoms}
    \begin{enumerate}
    \item if they are all the same atom, then the meet is that atom.
      \item if they are all different atoms, then the meet is \emph{⊥}
    \end{enumerate}
  \item if they were all \emph{place}s then the meet is a \emph{place}
  \item if the are all \emph{⊥}s then the meet is \emph{⊥}  
  \item if they are all pairs
    \begin{enumerate}
      \item calculate the meet of all the left elements $M_l$
      \item calulate the meet of all the right elements $M_r$
      \item the meet is the pair ($M_l$ ∙ $M_r$)
    \end{enumerate}
  \item if they are all binds
    \begin{enumerate}
    \item calculate the meet of the terms under all the binders $M_b$
    \item the meet is (bind $M_b$)
    \end{enumerate}
  \end{enumerate}
\end{enumerate}

\end{appendices}
