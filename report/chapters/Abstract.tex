\begin{abstract}

  Philip Wadler states that the Curry-Howard isomorphism is
  "a double-barrelled name that ensures the existence of other
  double-barrelled names"\cite{wadler2015}. By this he is of course
  referring to the deep isomorphism between logic and programming languages.
  
  It seems that many great programming-language type systems are
  developed (or discovered depending on your school of thought) not
  once, but twice: once by a logician and once by a computer
  scientist. Unfortunately for the computer scientists, the logicians
  seem to beat them to the punch more often than not.

  It is with this bitterly in mind that we aim to shorten the
  timeframe from the development of new type systems to their
  implementation by doing what computer scientists do best:
  automating. We will aim to abstract the parts of type checker
  software to such a degree where we can write software that creates
  type-checkers for us. 
  
  We begin by performing a generalized type theory literature review:
  taking the reader on a whirlwind tour of the Simply Typed Lambda Calculus
  \cite{church1940}, System F \cite{reynolds1974} and Hindley-Milner
  type theory \cite{hindley1969,milner1978,milner1982}
  before finally exploring Intuitionistic type theory
  \cite{martinlof1980} (sometimes known as Martin L\"{o}f type theory).

  We will then focus specifically on Hindley-Milner and Martin L\"{o}f
  type theory by introducing a programming language that uses each
  kind of type system at its core and perform a critical analysis
  and comparison of the systems, highlighting the differences by way
  of example where possible.

  Armed with our newly found knowledge we will develop a domain specific
  language (DSL) for describing type systems, paying special attention
  so as to not restrict the kinds of systems we can represent in this
  language.

  Finally we will put our DSL to good use by designing and
  implementing a type-checker-generator in Agda. This will take the
  form of a piece of software that receives, as its input, a
  specification of a type-system written in our previously devloped
  DSL and produces a type checker.

  We will conclude by evaluating our type-checker-generator and our
  design and implementation process. Finally, we speculate as to how
  we might improve our software and finish by outlining possible areas
  for future work.

\end{abstract}
