\chapter{Building a type checker generator}

Now that a user has the means of describing a type system, we will
discuss how the generic type-checking software is constructed.

We will describe key definitions and types to communicate a
high-level understanding of how the software works,
however, we will often omit implementation details in areas where the
code is less important, tedious or verbose.

The general tactic for type-checking some generic syntax will be to
parse the required typing, $\beta$ and $\eta$ rules from the
descriptions provided using our DSL. Using this information we can
then parse user source code and represent it in a fairly
language-agnostic internal object syntax before using the rules
to type-check our internal syntax.

We will begin by describing the internal syntax, talk about how we
might represent the various tools and rules that we need to
type check this syntax and define the type checking process before
briefly covering our approach to parsing this information from the
user-supplied input files. We purposely keep our parsing conversations
brief to prioritize the presentation of challenges in other areas.

Our approach in this section is heavily influenced by the work of
Conor McBride in \emph{The types who say 'ni'} \cite{TypesWhoSayNi} -
although the code we provide is strictly our own in all cases, there
are some areas where this code is a near-direct translation of
the mathematics in this paper. As we pointed out in the previous
chapter: patterns, expressions and the treatment of premises are
all concepts that we take from this work. In addition, our internal
syntax is very like that described in the paper and we use the same
approach for thinnings, substitution, contexts and much of our
treatment of scope.

\\\\
\section{A core language}

\hide{
\begin{code}%
\>[0]\AgdaKeyword{module}\AgdaSpace{}%
\AgdaModule{CoreLanguage}\AgdaSpace{}%
\AgdaKeyword{where}\<%
\end{code}

\begin{code}%
\>[0]\AgdaKeyword{open}\AgdaSpace{}%
\AgdaKeyword{import}\AgdaSpace{}%
\AgdaModule{Data.Product}\AgdaSpace{}%
\AgdaKeyword{using}\AgdaSpace{}%
\AgdaSymbol{(}\AgdaOperator{\AgdaFunction{\AgdaUnderscore{}×\AgdaUnderscore{}}}\AgdaSymbol{;}\AgdaSpace{}%
\AgdaFunction{Σ-syntax}\AgdaSymbol{;}\AgdaSpace{}%
\AgdaOperator{\AgdaInductiveConstructor{\AgdaUnderscore{},\AgdaUnderscore{}}}\AgdaSymbol{)}\<%
\\
\>[0]\AgdaKeyword{open}\AgdaSpace{}%
\AgdaKeyword{import}\AgdaSpace{}%
\AgdaModule{Data.Nat}\AgdaSpace{}%
\AgdaKeyword{using}\AgdaSpace{}%
\AgdaSymbol{(}\AgdaDatatype{ℕ}\AgdaSymbol{;}\AgdaSpace{}%
\AgdaOperator{\AgdaPrimitive{\AgdaUnderscore{}+\AgdaUnderscore{}}}\AgdaSymbol{;}\AgdaSpace{}%
\AgdaInductiveConstructor{suc}\AgdaSymbol{)}\<%
\\
\>[0]\AgdaKeyword{open}\AgdaSpace{}%
\AgdaKeyword{import}\AgdaSpace{}%
\AgdaModule{Data.Unit}\AgdaSpace{}%
\AgdaKeyword{using}\AgdaSpace{}%
\AgdaSymbol{(}\AgdaRecord{⊤}\AgdaSymbol{)}\<%
\\
\>[0]\AgdaKeyword{open}\AgdaSpace{}%
\AgdaKeyword{import}\AgdaSpace{}%
\AgdaModule{Data.Char}\AgdaSpace{}%
\AgdaKeyword{using}\AgdaSpace{}%
\AgdaSymbol{(}\AgdaPostulate{Char}\AgdaSymbol{)}\<%
\\
\>[0]\AgdaKeyword{open}\AgdaSpace{}%
\AgdaKeyword{import}\AgdaSpace{}%
\AgdaModule{Data.List}\<%
\end{code}
}

\begin{code}%
\>[0]\<%
\\
\>[0]\AgdaFunction{Scope}\AgdaSpace{}%
\AgdaSymbol{=}\AgdaSpace{}%
\AgdaDatatype{ℕ}\<%
\\
%
\\[\AgdaEmptyExtraSkip]%
\>[0]\AgdaKeyword{data}\AgdaSpace{}%
\AgdaDatatype{Var}\AgdaSpace{}%
\AgdaSymbol{:}\AgdaSpace{}%
\AgdaFunction{Scope}\AgdaSpace{}%
\AgdaSymbol{→}\AgdaSpace{}%
\AgdaPrimitiveType{Set}\AgdaSpace{}%
\AgdaKeyword{where}\<%
\\
\>[0][@{}l@{\AgdaIndent{0}}]%
\>[2]\AgdaInductiveConstructor{ze}\AgdaSpace{}%
\AgdaSymbol{:}\AgdaSpace{}%
\AgdaSymbol{\{}\AgdaBound{s}\AgdaSpace{}%
\AgdaSymbol{:}\AgdaSpace{}%
\AgdaFunction{Scope}\AgdaSymbol{\}}\AgdaSpace{}%
\AgdaSymbol{→}\AgdaSpace{}%
\AgdaDatatype{Var}\AgdaSpace{}%
\AgdaSymbol{(}\AgdaInductiveConstructor{suc}\AgdaSpace{}%
\AgdaBound{s}\AgdaSymbol{)}\<%
\\
%
\>[2]\AgdaInductiveConstructor{su}\AgdaSpace{}%
\AgdaSymbol{:}\AgdaSpace{}%
\AgdaSymbol{\{}\AgdaBound{s}\AgdaSpace{}%
\AgdaSymbol{:}\AgdaSpace{}%
\AgdaFunction{Scope}\AgdaSymbol{\}}\AgdaSpace{}%
\AgdaSymbol{→}\AgdaSpace{}%
\AgdaDatatype{Var}\AgdaSpace{}%
\AgdaBound{s}\AgdaSpace{}%
\AgdaSymbol{→}\AgdaSpace{}%
\AgdaDatatype{Var}\AgdaSpace{}%
\AgdaSymbol{(}\AgdaInductiveConstructor{suc}\AgdaSpace{}%
\AgdaBound{s}\AgdaSymbol{)}\<%
\\
%
\\[\AgdaEmptyExtraSkip]%
\>[0]\AgdaKeyword{data}\AgdaSpace{}%
\AgdaDatatype{Ess-Const}\AgdaSpace{}%
\AgdaSymbol{(}\AgdaBound{γ}\AgdaSpace{}%
\AgdaSymbol{:}\AgdaSpace{}%
\AgdaFunction{Scope}\AgdaSymbol{)}\AgdaSpace{}%
\AgdaSymbol{:}\AgdaSpace{}%
\AgdaPrimitiveType{Set}\<%
\\
\>[0]\AgdaKeyword{data}\AgdaSpace{}%
\AgdaDatatype{Lib-Const}\AgdaSpace{}%
\AgdaSymbol{(}\AgdaBound{γ}\AgdaSpace{}%
\AgdaSymbol{:}\AgdaSpace{}%
\AgdaFunction{Scope}\AgdaSymbol{)}\AgdaSpace{}%
\AgdaSymbol{:}\AgdaSpace{}%
\AgdaPrimitiveType{Set}\<%
\\
\>[0]\AgdaKeyword{data}\AgdaSpace{}%
\AgdaDatatype{Ess-Compu}\AgdaSpace{}%
\AgdaSymbol{(}\AgdaBound{γ}\AgdaSpace{}%
\AgdaSymbol{:}\AgdaSpace{}%
\AgdaFunction{Scope}\AgdaSymbol{)}\AgdaSpace{}%
\AgdaSymbol{:}\AgdaSpace{}%
\AgdaPrimitiveType{Set}\<%
\\
\>[0]\AgdaKeyword{data}\AgdaSpace{}%
\AgdaDatatype{Lib-Compu}\AgdaSpace{}%
\AgdaSymbol{(}\AgdaBound{γ}\AgdaSpace{}%
\AgdaSymbol{:}\AgdaSpace{}%
\AgdaFunction{Scope}\AgdaSymbol{)}\AgdaSpace{}%
\AgdaSymbol{:}\AgdaSpace{}%
\AgdaPrimitiveType{Set}\<%
\\
%
\\[\AgdaEmptyExtraSkip]%
\>[0]\AgdaKeyword{data}\AgdaSpace{}%
\AgdaDatatype{Ess-Const}\AgdaSpace{}%
\AgdaBound{γ}\AgdaSpace{}%
\AgdaKeyword{where}\<%
\\
\>[0][@{}l@{\AgdaIndent{0}}]%
\>[2]\AgdaInductiveConstructor{`}%
\>[9]\AgdaSymbol{:}\AgdaSpace{}%
\AgdaPostulate{Char}\AgdaSpace{}%
\AgdaSymbol{→}\AgdaSpace{}%
\AgdaDatatype{Ess-Const}\AgdaSpace{}%
\AgdaBound{γ}\<%
\\
%
\>[2]\AgdaOperator{\AgdaInductiveConstructor{\AgdaUnderscore{}∙\AgdaUnderscore{}}}%
\>[9]\AgdaSymbol{:}\AgdaSpace{}%
\AgdaDatatype{Lib-Const}\AgdaSpace{}%
\AgdaBound{γ}\AgdaSpace{}%
\AgdaSymbol{→}\AgdaSpace{}%
\AgdaDatatype{Lib-Const}\AgdaSpace{}%
\AgdaBound{γ}\AgdaSpace{}%
\AgdaSymbol{→}\AgdaSpace{}%
\AgdaDatatype{Ess-Const}\AgdaSpace{}%
\AgdaBound{γ}\<%
\\
%
\>[2]\AgdaInductiveConstructor{bind}%
\>[9]\AgdaSymbol{:}\AgdaSpace{}%
\AgdaDatatype{Lib-Const}\AgdaSpace{}%
\AgdaSymbol{(}\AgdaInductiveConstructor{suc}\AgdaSpace{}%
\AgdaBound{γ}\AgdaSymbol{)}\AgdaSpace{}%
\AgdaSymbol{→}\AgdaSpace{}%
\AgdaDatatype{Ess-Const}\AgdaSpace{}%
\AgdaBound{γ}\<%
\\
%
\\[\AgdaEmptyExtraSkip]%
\>[0]\AgdaKeyword{infixr}\AgdaSpace{}%
\AgdaNumber{20}\AgdaSpace{}%
\AgdaOperator{\AgdaInductiveConstructor{\AgdaUnderscore{}∙\AgdaUnderscore{}}}\<%
\\
%
\\[\AgdaEmptyExtraSkip]%
\>[0]\AgdaKeyword{data}\AgdaSpace{}%
\AgdaDatatype{Lib-Const}\AgdaSpace{}%
\AgdaBound{γ}\AgdaSpace{}%
\AgdaKeyword{where}\<%
\\
\>[0][@{}l@{\AgdaIndent{0}}]%
\>[2]\AgdaInductiveConstructor{ess}%
\>[9]\AgdaSymbol{:}\AgdaSpace{}%
\AgdaDatatype{Ess-Const}\AgdaSpace{}%
\AgdaBound{γ}\AgdaSpace{}%
\AgdaSymbol{→}\AgdaSpace{}%
\AgdaDatatype{Lib-Const}\AgdaSpace{}%
\AgdaBound{γ}\<%
\\
%
\>[2]\AgdaInductiveConstructor{thunk}%
\>[9]\AgdaSymbol{:}\AgdaSpace{}%
\AgdaDatatype{Ess-Compu}\AgdaSpace{}%
\AgdaBound{γ}\AgdaSpace{}%
\AgdaSymbol{→}\AgdaSpace{}%
\AgdaDatatype{Lib-Const}\AgdaSpace{}%
\AgdaBound{γ}\<%
\\
%
\\[\AgdaEmptyExtraSkip]%
\>[0]\AgdaKeyword{data}\AgdaSpace{}%
\AgdaDatatype{Ess-Compu}\AgdaSpace{}%
\AgdaBound{γ}\AgdaSpace{}%
\AgdaKeyword{where}\<%
\\
\>[0][@{}l@{\AgdaIndent{0}}]%
\>[2]\AgdaInductiveConstructor{var}%
\>[9]\AgdaSymbol{:}\AgdaSpace{}%
\AgdaDatatype{Var}\AgdaSpace{}%
\AgdaBound{γ}\AgdaSpace{}%
\AgdaSymbol{→}\AgdaSpace{}%
\AgdaDatatype{Ess-Compu}\AgdaSpace{}%
\AgdaBound{γ}\<%
\\
%
\>[2]\AgdaInductiveConstructor{elim}%
\>[9]\AgdaSymbol{:}\AgdaSpace{}%
\AgdaDatatype{Lib-Compu}\AgdaSpace{}%
\AgdaBound{γ}\AgdaSpace{}%
\AgdaSymbol{→}\AgdaSpace{}%
\AgdaDatatype{Lib-Const}\AgdaSpace{}%
\AgdaBound{γ}\AgdaSpace{}%
\AgdaSymbol{→}\AgdaSpace{}%
\AgdaDatatype{Ess-Compu}\AgdaSpace{}%
\AgdaBound{γ}\<%
\\
%
\\[\AgdaEmptyExtraSkip]%
\>[0]\AgdaKeyword{data}\AgdaSpace{}%
\AgdaDatatype{Lib-Compu}\AgdaSpace{}%
\AgdaBound{γ}\AgdaSpace{}%
\AgdaKeyword{where}\<%
\\
\>[0][@{}l@{\AgdaIndent{0}}]%
\>[2]\AgdaInductiveConstructor{ess}%
\>[9]\AgdaSymbol{:}\AgdaSpace{}%
\AgdaDatatype{Ess-Compu}\AgdaSpace{}%
\AgdaBound{γ}\AgdaSpace{}%
\AgdaSymbol{→}\AgdaSpace{}%
\AgdaDatatype{Lib-Compu}\AgdaSpace{}%
\AgdaBound{γ}\<%
\\
%
\>[2]\AgdaOperator{\AgdaInductiveConstructor{\AgdaUnderscore{}∷\AgdaUnderscore{}}}%
\>[9]\AgdaSymbol{:}\AgdaSpace{}%
\AgdaDatatype{Lib-Const}\AgdaSpace{}%
\AgdaBound{γ}\AgdaSpace{}%
\AgdaSymbol{→}\AgdaSpace{}%
\AgdaDatatype{Lib-Const}\AgdaSpace{}%
\AgdaBound{γ}\AgdaSpace{}%
\AgdaSymbol{→}\AgdaSpace{}%
\AgdaDatatype{Lib-Compu}\AgdaSpace{}%
\AgdaBound{γ}\<%
\\
%
\\[\AgdaEmptyExtraSkip]%
\>[0]\AgdaKeyword{data}\AgdaSpace{}%
\AgdaDatatype{Lib}\AgdaSpace{}%
\AgdaSymbol{:}\AgdaSpace{}%
\AgdaPrimitiveType{Set}\AgdaSpace{}%
\AgdaKeyword{where}\AgdaSpace{}%
\AgdaInductiveConstructor{ess}\AgdaSpace{}%
\AgdaInductiveConstructor{lib}\AgdaSpace{}%
\AgdaSymbol{:}\AgdaSpace{}%
\AgdaDatatype{Lib}\<%
\\
\>[0]\AgdaKeyword{data}\AgdaSpace{}%
\AgdaDatatype{Dir}\AgdaSpace{}%
\AgdaSymbol{:}\AgdaSpace{}%
\AgdaPrimitiveType{Set}\AgdaSpace{}%
\AgdaKeyword{where}\AgdaSpace{}%
\AgdaInductiveConstructor{const}\AgdaSpace{}%
\AgdaInductiveConstructor{compu}\AgdaSpace{}%
\AgdaSymbol{:}\AgdaSpace{}%
\AgdaDatatype{Dir}\<%
\\
%
\\[\AgdaEmptyExtraSkip]%
\>[0]\AgdaFunction{Term}\AgdaSpace{}%
\AgdaSymbol{:}\AgdaSpace{}%
\AgdaDatatype{Lib}\AgdaSpace{}%
\AgdaSymbol{→}\AgdaSpace{}%
\AgdaDatatype{Dir}\AgdaSpace{}%
\AgdaSymbol{→}\AgdaSpace{}%
\AgdaFunction{Scope}\AgdaSpace{}%
\AgdaSymbol{→}\AgdaSpace{}%
\AgdaPrimitiveType{Set}\<%
\\
\>[0]\AgdaFunction{Term}\AgdaSpace{}%
\AgdaInductiveConstructor{ess}\AgdaSpace{}%
\AgdaInductiveConstructor{const}\AgdaSpace{}%
\AgdaBound{γ}\AgdaSpace{}%
\AgdaSymbol{=}\AgdaSpace{}%
\AgdaDatatype{Ess-Const}\AgdaSpace{}%
\AgdaBound{γ}\<%
\\
\>[0]\AgdaFunction{Term}\AgdaSpace{}%
\AgdaInductiveConstructor{ess}\AgdaSpace{}%
\AgdaInductiveConstructor{compu}\AgdaSpace{}%
\AgdaBound{γ}\AgdaSpace{}%
\AgdaSymbol{=}\AgdaSpace{}%
\AgdaDatatype{Ess-Compu}\AgdaSpace{}%
\AgdaBound{γ}\<%
\\
\>[0]\AgdaFunction{Term}\AgdaSpace{}%
\AgdaInductiveConstructor{lib}\AgdaSpace{}%
\AgdaInductiveConstructor{const}\AgdaSpace{}%
\AgdaBound{γ}\AgdaSpace{}%
\AgdaSymbol{=}\AgdaSpace{}%
\AgdaDatatype{Lib-Const}\AgdaSpace{}%
\AgdaBound{γ}\<%
\\
\>[0]\AgdaFunction{Term}\AgdaSpace{}%
\AgdaInductiveConstructor{lib}\AgdaSpace{}%
\AgdaInductiveConstructor{compu}\AgdaSpace{}%
\AgdaBound{γ}\AgdaSpace{}%
\AgdaSymbol{=}\AgdaSpace{}%
\AgdaDatatype{Lib-Compu}\AgdaSpace{}%
\AgdaBound{γ}\<%
\end{code}

\begin{code}%
\>[0]\AgdaComment{--example construction (ΠS\textbackslash{}xT)}\<%
\\
\>[0]\AgdaComment{-- to remove}\<%
\\
\>[0]\AgdaFunction{dep-func}\AgdaSpace{}%
\AgdaSymbol{:}\AgdaSpace{}%
\AgdaFunction{Term}\AgdaSpace{}%
\AgdaInductiveConstructor{ess}\AgdaSpace{}%
\AgdaInductiveConstructor{const}\AgdaSpace{}%
\AgdaNumber{0}\<%
\\
\>[0]\AgdaFunction{dep-func}\AgdaSpace{}%
\AgdaSymbol{=}\AgdaSpace{}%
\AgdaInductiveConstructor{ess}\AgdaSpace{}%
\AgdaSymbol{(}\AgdaInductiveConstructor{`}\AgdaSpace{}%
\AgdaString{'Π'}\AgdaSymbol{)}\AgdaSpace{}%
\AgdaOperator{\AgdaInductiveConstructor{∙}}\AgdaSpace{}%
\AgdaInductiveConstructor{ess}\AgdaSpace{}%
\AgdaSymbol{(}\AgdaInductiveConstructor{ess}\AgdaSpace{}%
\AgdaSymbol{(}\AgdaInductiveConstructor{`}\AgdaSpace{}%
\AgdaString{'S'}\AgdaSymbol{)}\AgdaSpace{}%
\AgdaOperator{\AgdaInductiveConstructor{∙}}\AgdaSpace{}%
\AgdaInductiveConstructor{ess}\AgdaSpace{}%
\AgdaSymbol{(}\AgdaInductiveConstructor{bind}\AgdaSpace{}%
\AgdaSymbol{(}\AgdaInductiveConstructor{ess}\AgdaSpace{}%
\AgdaSymbol{(}\AgdaInductiveConstructor{`}\AgdaSpace{}%
\AgdaString{'T'}\AgdaSymbol{))))}\<%
\end{code}

\section{Thinnings}

\begin{code}%
\>[0]\AgdaKeyword{module}\AgdaSpace{}%
\AgdaModule{Thinning}\AgdaSpace{}%
\AgdaKeyword{where}\<%
\end{code}
\section{Substitution}

\hide{
\begin{code}%
\>[0]\AgdaKeyword{module}\AgdaSpace{}%
\AgdaModule{Substitution}\AgdaSpace{}%
\AgdaKeyword{where}\<%
\end{code}
}

\hide{
\begin{code}%
\>[0]\AgdaKeyword{open}\AgdaSpace{}%
\AgdaKeyword{import}\AgdaSpace{}%
\AgdaModule{CoreLanguage}\<%
\\
\>[0]\AgdaKeyword{open}\AgdaSpace{}%
\AgdaKeyword{import}\AgdaSpace{}%
\AgdaModule{BwdVec}\<%
\\
\>[0]\AgdaKeyword{open}\AgdaSpace{}%
\AgdaKeyword{import}\AgdaSpace{}%
\AgdaModule{Composition}\<%
\\
\>[0]\AgdaKeyword{open}\AgdaSpace{}%
\AgdaKeyword{import}\AgdaSpace{}%
\AgdaModule{Data.Nat}\AgdaSpace{}%
\AgdaKeyword{using}\AgdaSpace{}%
\AgdaSymbol{(}\AgdaInductiveConstructor{zero}\AgdaSymbol{;}\AgdaSpace{}%
\AgdaInductiveConstructor{suc}\AgdaSymbol{)}\<%
\end{code}
}

\hide{
\begin{code}%
\>[0]\AgdaKeyword{private}\<%
\\
\>[0][@{}l@{\AgdaIndent{0}}]%
\>[2]\AgdaKeyword{variable}\<%
\\
\>[2][@{}l@{\AgdaIndent{0}}]%
\>[4]\AgdaGeneralizable{δ}\AgdaSpace{}%
\AgdaSymbol{:}\AgdaSpace{}%
\AgdaFunction{Scope}\<%
\\
%
\>[4]\AgdaGeneralizable{γ}\AgdaSpace{}%
\AgdaSymbol{:}\AgdaSpace{}%
\AgdaFunction{Scope}\<%
\end{code}
}

Substitutions are defined as backward vectors, vectors that grow by appending
elements to the right hand side as opposed to the left. A substitution is
defined in terms of two scopes, the scope of the target of substitution, and
the scope of the entities we will substitute into the target.

We are able to look up individual variables in a substitution, later we will
see that this is just a special case of using a thinning to select from a
subsitution but having dedicated syntax is often more convenient.

We also capture the idea of Substitutable entities and define composition
for substitution across all such definitions.

\begin{code}%
\>[0]\AgdaOperator{\AgdaFunction{\AgdaUnderscore{}⇒[\AgdaUnderscore{}]\AgdaUnderscore{}}}\AgdaSpace{}%
\AgdaSymbol{:}\AgdaSpace{}%
\AgdaFunction{Scope}\AgdaSpace{}%
\AgdaSymbol{→}\AgdaSpace{}%
\AgdaFunction{Scoped}\AgdaSpace{}%
\AgdaSymbol{→}\AgdaSpace{}%
\AgdaFunction{Scope}\AgdaSpace{}%
\AgdaSymbol{→}\AgdaSpace{}%
\AgdaPrimitiveType{Set}\<%
\\
\>[0]\AgdaBound{γ}\AgdaSpace{}%
\AgdaOperator{\AgdaFunction{⇒[}}\AgdaSpace{}%
\AgdaBound{X}\AgdaSpace{}%
\AgdaOperator{\AgdaFunction{]}}\AgdaSpace{}%
\AgdaBound{δ}\AgdaSpace{}%
\AgdaSymbol{=}\AgdaSpace{}%
\AgdaDatatype{BwdVec}\AgdaSpace{}%
\AgdaSymbol{(}\AgdaBound{X}\AgdaSpace{}%
\AgdaBound{δ}\AgdaSymbol{)}\AgdaSpace{}%
\AgdaBound{γ}\<%
\\
%
\\[\AgdaEmptyExtraSkip]%
\>[0]\AgdaFunction{lookup}\AgdaSpace{}%
\AgdaSymbol{:}\AgdaSpace{}%
\AgdaSymbol{(}\AgdaBound{T}\AgdaSpace{}%
\AgdaSymbol{:}\AgdaSpace{}%
\AgdaFunction{Scoped}\AgdaSymbol{)}\AgdaSpace{}%
\AgdaSymbol{→}\AgdaSpace{}%
\AgdaGeneralizable{δ}\AgdaSpace{}%
\AgdaOperator{\AgdaFunction{⇒[}}\AgdaSpace{}%
\AgdaBound{T}\AgdaSpace{}%
\AgdaOperator{\AgdaFunction{]}}\AgdaSpace{}%
\AgdaGeneralizable{γ}\AgdaSpace{}%
\AgdaSymbol{→}\AgdaSpace{}%
\AgdaDatatype{Var}\AgdaSpace{}%
\AgdaGeneralizable{δ}\AgdaSpace{}%
\AgdaSymbol{→}\AgdaSpace{}%
\AgdaBound{T}\AgdaSpace{}%
\AgdaGeneralizable{γ}\<%
\\
%
\\[\AgdaEmptyExtraSkip]%
\>[0]\AgdaFunction{Substitutable}\AgdaSpace{}%
\AgdaSymbol{:}\AgdaSpace{}%
\AgdaFunction{Scoped}\AgdaSpace{}%
\AgdaSymbol{→}\AgdaSpace{}%
\AgdaPrimitiveType{Set}\<%
\\
\>[0]\AgdaFunction{Substitutable}\AgdaSpace{}%
\AgdaBound{T}\AgdaSpace{}%
\AgdaSymbol{=}\AgdaSpace{}%
\AgdaSymbol{∀}\AgdaSpace{}%
\AgdaSymbol{\{}\AgdaBound{γ}\AgdaSymbol{\}}\AgdaSpace{}%
\AgdaSymbol{\{}\AgdaBound{γ'}\AgdaSymbol{\}}\AgdaSpace{}%
\AgdaSymbol{→}\AgdaSpace{}%
\AgdaBound{T}\AgdaSpace{}%
\AgdaBound{γ}\AgdaSpace{}%
\AgdaSymbol{→}\AgdaSpace{}%
\AgdaBound{γ}\AgdaSpace{}%
\AgdaOperator{\AgdaFunction{⇒[}}\AgdaSpace{}%
\AgdaBound{T}\AgdaSpace{}%
\AgdaOperator{\AgdaFunction{]}}\AgdaSpace{}%
\AgdaBound{γ'}\AgdaSpace{}%
\AgdaSymbol{→}\AgdaSpace{}%
\AgdaBound{T}\AgdaSpace{}%
\AgdaBound{γ'}\<%
\\
%
\\[\AgdaEmptyExtraSkip]%
\>[0]\AgdaOperator{\AgdaFunction{[\AgdaUnderscore{}]\AgdaUnderscore{}∘σ\AgdaUnderscore{}}}\AgdaSpace{}%
\AgdaSymbol{:}\AgdaSpace{}%
\AgdaSymbol{∀}\AgdaSpace{}%
\AgdaSymbol{\{}\AgdaBound{T}\AgdaSymbol{\}}\AgdaSpace{}%
\AgdaSymbol{→}\AgdaSpace{}%
\AgdaFunction{Substitutable}\AgdaSpace{}%
\AgdaBound{T}\AgdaSpace{}%
\AgdaSymbol{→}\AgdaSpace{}%
\AgdaFunction{Composable}\AgdaSpace{}%
\AgdaOperator{\AgdaFunction{\AgdaUnderscore{}⇒[}}\AgdaSpace{}%
\AgdaBound{T}\AgdaSpace{}%
\AgdaOperator{\AgdaFunction{]\AgdaUnderscore{}}}\<%
\end{code}

\hide{
\begin{code}%
\>[0]\AgdaFunction{lookup}\AgdaSpace{}%
\AgdaBound{T}\AgdaSpace{}%
\AgdaSymbol{(}\AgdaBound{σ}\AgdaSpace{}%
\AgdaOperator{\AgdaInductiveConstructor{-,}}\AgdaSpace{}%
\AgdaBound{x}\AgdaSymbol{)}\AgdaSpace{}%
\AgdaInductiveConstructor{ze}\AgdaSpace{}%
\AgdaSymbol{=}\AgdaSpace{}%
\AgdaBound{x}\<%
\\
\>[0]\AgdaFunction{lookup}\AgdaSpace{}%
\AgdaBound{T}\AgdaSpace{}%
\AgdaSymbol{(}\AgdaBound{σ}\AgdaSpace{}%
\AgdaOperator{\AgdaInductiveConstructor{-,}}\AgdaSpace{}%
\AgdaBound{x}\AgdaSymbol{)}\AgdaSpace{}%
\AgdaSymbol{(}\AgdaInductiveConstructor{su}\AgdaSpace{}%
\AgdaBound{v}\AgdaSymbol{)}\AgdaSpace{}%
\AgdaSymbol{=}\AgdaSpace{}%
\AgdaFunction{lookup}\AgdaSpace{}%
\AgdaBound{T}\AgdaSpace{}%
\AgdaBound{σ}\AgdaSpace{}%
\AgdaBound{v}\<%
\\
%
\\[\AgdaEmptyExtraSkip]%
\>[0]\AgdaOperator{\AgdaFunction{[}}\AgdaSpace{}%
\AgdaBound{/}\AgdaSpace{}%
\AgdaOperator{\AgdaFunction{]}}%
\>[7]\AgdaInductiveConstructor{ε}%
\>[15]\AgdaOperator{\AgdaFunction{∘σ}}\AgdaSpace{}%
\AgdaBound{σ'}\AgdaSpace{}%
\AgdaSymbol{=}\AgdaSpace{}%
\AgdaInductiveConstructor{ε}\<%
\\
\>[0]\AgdaOperator{\AgdaFunction{[}}\AgdaSpace{}%
\AgdaBound{/}\AgdaSpace{}%
\AgdaOperator{\AgdaFunction{]}}\AgdaSpace{}%
\AgdaSymbol{(}\AgdaBound{σ}\AgdaSpace{}%
\AgdaOperator{\AgdaInductiveConstructor{-,}}\AgdaSpace{}%
\AgdaBound{x}\AgdaSymbol{)}\AgdaSpace{}%
\AgdaOperator{\AgdaFunction{∘σ}}\AgdaSpace{}%
\AgdaBound{σ'}\AgdaSpace{}%
\AgdaSymbol{=}\AgdaSpace{}%
\AgdaSymbol{(}\AgdaOperator{\AgdaFunction{[}}\AgdaSpace{}%
\AgdaBound{/}\AgdaSpace{}%
\AgdaOperator{\AgdaFunction{]}}\AgdaSpace{}%
\AgdaBound{σ}\AgdaSpace{}%
\AgdaOperator{\AgdaFunction{∘σ}}\AgdaSpace{}%
\AgdaBound{σ'}\AgdaSymbol{)}\AgdaSpace{}%
\AgdaOperator{\AgdaInductiveConstructor{-,}}\AgdaSpace{}%
\AgdaBound{/}\AgdaSpace{}%
\AgdaBound{x}\AgdaSpace{}%
\AgdaBound{σ'}\<%
\end{code}
}


\section{Term Substitution}

\hide{
\begin{code}%
\>[0]\AgdaKeyword{module}\AgdaSpace{}%
\AgdaModule{TermSubstitution}\AgdaSpace{}%
\AgdaKeyword{where}\<%
\end{code}
}

\hide{
\begin{code}%
\>[0]\AgdaKeyword{open}\AgdaSpace{}%
\AgdaKeyword{import}\AgdaSpace{}%
\AgdaModule{CoreLanguage}\<%
\\
\>[0]\AgdaKeyword{open}\AgdaSpace{}%
\AgdaKeyword{import}\AgdaSpace{}%
\AgdaModule{Substitution}\<%
\\
\>[0]\AgdaKeyword{open}\AgdaSpace{}%
\AgdaKeyword{import}\AgdaSpace{}%
\AgdaModule{Thinning}\AgdaSpace{}%
\AgdaKeyword{using}\AgdaSpace{}%
\AgdaSymbol{(}\AgdaFunction{Thinnable}\AgdaSymbol{;}\AgdaSpace{}%
\AgdaFunction{Weakenable}\AgdaSymbol{;}\AgdaSpace{}%
\AgdaFunction{weaken}\AgdaSymbol{;}\AgdaSpace{}%
\AgdaOperator{\AgdaFunction{\AgdaUnderscore{}⟨term\AgdaUnderscore{}}}\AgdaSymbol{;}\AgdaSpace{}%
\AgdaFunction{⟨sub}\AgdaSymbol{)}\<%
\\
\>[0]\AgdaKeyword{open}\AgdaSpace{}%
\AgdaKeyword{import}\AgdaSpace{}%
\AgdaModule{BwdVec}\<%
\\
\>[0]\AgdaKeyword{open}\AgdaSpace{}%
\AgdaKeyword{import}\AgdaSpace{}%
\AgdaModule{Data.Nat}\AgdaSpace{}%
\AgdaKeyword{using}\AgdaSpace{}%
\AgdaSymbol{(}\AgdaInductiveConstructor{zero}\AgdaSymbol{;}\AgdaSpace{}%
\AgdaInductiveConstructor{suc}\AgdaSymbol{)}\<%
\end{code}
}

\hide{
\begin{code}%
\>[0]\AgdaKeyword{private}\<%
\\
\>[0][@{}l@{\AgdaIndent{0}}]%
\>[2]\AgdaKeyword{variable}\<%
\\
\>[2][@{}l@{\AgdaIndent{0}}]%
\>[4]\AgdaGeneralizable{δ}\AgdaSpace{}%
\AgdaSymbol{:}\AgdaSpace{}%
\AgdaFunction{Scope}\<%
\\
%
\>[4]\AgdaGeneralizable{δ'}\AgdaSpace{}%
\AgdaSymbol{:}\AgdaSpace{}%
\AgdaFunction{Scope}\<%
\\
%
\>[4]\AgdaGeneralizable{γ}\AgdaSpace{}%
\AgdaSymbol{:}\AgdaSpace{}%
\AgdaFunction{Scope}\<%
\\
%
\>[4]\AgdaGeneralizable{d}\AgdaSpace{}%
\AgdaSymbol{:}\AgdaSpace{}%
\AgdaDatatype{Dir}\<%
\end{code}
}

Following our generic notion of substitution, we now specialise our
our definiton to substitute computations. We supply the usual thinning
and weakening mechanics that we rely on as well as defining the identity
substitution.

\begin{code}%
\>[0]\AgdaOperator{\AgdaFunction{\AgdaUnderscore{}⇒\AgdaUnderscore{}}}\AgdaSpace{}%
\AgdaSymbol{:}\AgdaSpace{}%
\AgdaFunction{Scope}\AgdaSpace{}%
\AgdaSymbol{→}\AgdaSpace{}%
\AgdaFunction{Scope}\AgdaSpace{}%
\AgdaSymbol{→}\AgdaSpace{}%
\AgdaPrimitiveType{Set}\<%
\\
\>[0]\AgdaBound{γ}\AgdaSpace{}%
\AgdaOperator{\AgdaFunction{⇒}}\AgdaSpace{}%
\AgdaBound{δ}\AgdaSpace{}%
\AgdaSymbol{=}\AgdaSpace{}%
\AgdaBound{γ}\AgdaSpace{}%
\AgdaOperator{\AgdaFunction{⇒[}}\AgdaSpace{}%
\AgdaFunction{Term}\AgdaSpace{}%
\AgdaInductiveConstructor{compu}\AgdaSpace{}%
\AgdaOperator{\AgdaFunction{]}}\AgdaSpace{}%
\AgdaBound{δ}\<%
\\
%
\\[\AgdaEmptyExtraSkip]%
\>[0]\AgdaOperator{\AgdaFunction{\AgdaUnderscore{}⟨σ\AgdaUnderscore{}}}%
\>[6]\AgdaSymbol{:}\AgdaSpace{}%
\AgdaFunction{Thinnable}\AgdaSpace{}%
\AgdaSymbol{(}\AgdaGeneralizable{γ}\AgdaSpace{}%
\AgdaOperator{\AgdaFunction{⇒\AgdaUnderscore{}}}\AgdaSymbol{)}\<%
\\
\>[0]\AgdaOperator{\AgdaFunction{\AgdaUnderscore{}\textasciicircum{}}}%
\>[6]\AgdaSymbol{:}\AgdaSpace{}%
\AgdaFunction{Weakenable}\AgdaSpace{}%
\AgdaSymbol{(}\AgdaGeneralizable{γ}\AgdaSpace{}%
\AgdaOperator{\AgdaFunction{⇒\AgdaUnderscore{}}}\AgdaSymbol{)}\<%
\\
\>[0]\AgdaFunction{id}%
\>[6]\AgdaSymbol{:}\AgdaSpace{}%
\AgdaGeneralizable{γ}\AgdaSpace{}%
\AgdaOperator{\AgdaFunction{⇒}}\AgdaSpace{}%
\AgdaGeneralizable{γ}\<%
\end{code}

\hide{
\begin{code}%
\>[0]\<%
\\
\>[0]\AgdaBound{σ}\AgdaSpace{}%
\AgdaOperator{\AgdaFunction{⟨σ}}\AgdaSpace{}%
\AgdaBound{θ}\AgdaSpace{}%
\AgdaSymbol{=}\AgdaSpace{}%
\AgdaFunction{⟨sub}\AgdaSpace{}%
\AgdaOperator{\AgdaFunction{\AgdaUnderscore{}⟨term\AgdaUnderscore{}}}\AgdaSpace{}%
\AgdaBound{σ}\AgdaSpace{}%
\AgdaBound{θ}\<%
\\
%
\\[\AgdaEmptyExtraSkip]%
\>[0]\AgdaOperator{\AgdaFunction{\AgdaUnderscore{}\textasciicircum{}}}\AgdaSpace{}%
\AgdaSymbol{=}\AgdaSpace{}%
\AgdaFunction{weaken}\AgdaSpace{}%
\AgdaOperator{\AgdaFunction{\AgdaUnderscore{}⟨σ\AgdaUnderscore{}}}\<%
\\
%
\\[\AgdaEmptyExtraSkip]%
\>[0]\AgdaFunction{id}\AgdaSpace{}%
\AgdaSymbol{\{}\AgdaInductiveConstructor{zero}\AgdaSymbol{\}}\AgdaSpace{}%
\AgdaSymbol{=}\AgdaSpace{}%
\AgdaInductiveConstructor{ε}\<%
\\
\>[0]\AgdaFunction{id}\AgdaSpace{}%
\AgdaSymbol{\{}\AgdaInductiveConstructor{suc}\AgdaSpace{}%
\AgdaBound{γ}\AgdaSymbol{\}}\AgdaSpace{}%
\AgdaSymbol{=}\AgdaSpace{}%
\AgdaSymbol{(}\AgdaFunction{id}\AgdaSpace{}%
\AgdaSymbol{\{}\AgdaBound{γ}\AgdaSymbol{\}}\AgdaSpace{}%
\AgdaOperator{\AgdaFunction{\textasciicircum{}}}\AgdaSymbol{)}\AgdaSpace{}%
\AgdaOperator{\AgdaInductiveConstructor{-,}}\AgdaSpace{}%
\AgdaInductiveConstructor{var}\AgdaSpace{}%
\AgdaInductiveConstructor{ze}\<%
\end{code}
}

We also define the action of such a substitution on a term, where most cases
recurse on direct substructures as one might expect except that we ensure we
alter the substitution accordingly as we pass under binders, introducing an
identity substitution for the newly bound variable. Here we do not use the
Substitutable type defined previously as the target of substitution may be
either a construction or a computation, however the objects we are substituting
in are always computations. This type is reserved for use later.

\begin{code}%
\>[0]\AgdaOperator{\AgdaFunction{\AgdaUnderscore{}/term\AgdaUnderscore{}}}\AgdaSpace{}%
\AgdaSymbol{:}\AgdaSpace{}%
\AgdaFunction{Term}\AgdaSpace{}%
\AgdaGeneralizable{d}\AgdaSpace{}%
\AgdaGeneralizable{γ}\AgdaSpace{}%
\AgdaSymbol{→}\AgdaSpace{}%
\AgdaGeneralizable{γ}\AgdaSpace{}%
\AgdaOperator{\AgdaFunction{⇒}}\AgdaSpace{}%
\AgdaGeneralizable{δ}\AgdaSpace{}%
\AgdaSymbol{→}\AgdaSpace{}%
\AgdaFunction{Term}\AgdaSpace{}%
\AgdaGeneralizable{d}\AgdaSpace{}%
\AgdaGeneralizable{δ}\<%
\\
\>[0]\AgdaComment{-- ...}\<%
\\
\>[0]\AgdaOperator{\AgdaFunction{\AgdaUnderscore{}/term\AgdaUnderscore{}}}\AgdaSpace{}%
\AgdaSymbol{\{}\AgdaInductiveConstructor{const}\AgdaSymbol{\}}\AgdaSpace{}%
\AgdaSymbol{(}\AgdaInductiveConstructor{bind}\AgdaSpace{}%
\AgdaBound{t}\AgdaSymbol{)}%
\>[28]\AgdaBound{σ}%
\>[31]\AgdaSymbol{=}\AgdaSpace{}%
\AgdaInductiveConstructor{bind}\AgdaSpace{}%
\AgdaSymbol{(}\AgdaBound{t}\AgdaSpace{}%
\AgdaOperator{\AgdaFunction{/term}}\AgdaSpace{}%
\AgdaSymbol{(}\AgdaBound{σ}\AgdaSpace{}%
\AgdaOperator{\AgdaFunction{\textasciicircum{}}}\AgdaSpace{}%
\AgdaOperator{\AgdaInductiveConstructor{-,}}\AgdaSpace{}%
\AgdaInductiveConstructor{var}\AgdaSpace{}%
\AgdaInductiveConstructor{ze}\AgdaSymbol{))}\<%
\\
\>[0]\AgdaOperator{\AgdaFunction{\AgdaUnderscore{}/term\AgdaUnderscore{}}}\AgdaSpace{}%
\AgdaSymbol{\{}\AgdaInductiveConstructor{compu}\AgdaSymbol{\}}\AgdaSpace{}%
\AgdaSymbol{(}\AgdaInductiveConstructor{var}\AgdaSpace{}%
\AgdaBound{v}\AgdaSymbol{)}%
\>[28]\AgdaBound{σ}%
\>[31]\AgdaSymbol{=}\AgdaSpace{}%
\AgdaFunction{lookup}\AgdaSpace{}%
\AgdaSymbol{(}\AgdaFunction{Term}\AgdaSpace{}%
\AgdaInductiveConstructor{compu}\AgdaSymbol{)}\AgdaSpace{}%
\AgdaBound{σ}\AgdaSpace{}%
\AgdaBound{v}\<%
\\
\>[0]\AgdaComment{-- ...}\<%
\end{code}

\hide{
\begin{code}%
\>[0]\AgdaOperator{\AgdaFunction{\AgdaUnderscore{}/term\AgdaUnderscore{}}}\AgdaSpace{}%
\AgdaSymbol{\{}\AgdaInductiveConstructor{const}\AgdaSymbol{\}}\AgdaSpace{}%
\AgdaSymbol{(}\AgdaInductiveConstructor{`}\AgdaSpace{}%
\AgdaBound{x}\AgdaSymbol{)}%
\>[28]\AgdaBound{σ}%
\>[31]\AgdaSymbol{=}\AgdaSpace{}%
\AgdaInductiveConstructor{`}\AgdaSpace{}%
\AgdaBound{x}\<%
\\
\>[0]\AgdaOperator{\AgdaFunction{\AgdaUnderscore{}/term\AgdaUnderscore{}}}\AgdaSpace{}%
\AgdaSymbol{\{}\AgdaInductiveConstructor{const}\AgdaSymbol{\}}\AgdaSpace{}%
\AgdaSymbol{(}\AgdaBound{s}\AgdaSpace{}%
\AgdaOperator{\AgdaInductiveConstructor{∙}}\AgdaSpace{}%
\AgdaBound{t}\AgdaSymbol{)}%
\>[28]\AgdaBound{σ}%
\>[31]\AgdaSymbol{=}\AgdaSpace{}%
\AgdaSymbol{(}\AgdaBound{s}\AgdaSpace{}%
\AgdaOperator{\AgdaFunction{/term}}\AgdaSpace{}%
\AgdaBound{σ}\AgdaSymbol{)}\AgdaSpace{}%
\AgdaOperator{\AgdaInductiveConstructor{∙}}\AgdaSpace{}%
\AgdaSymbol{(}\AgdaBound{t}\AgdaSpace{}%
\AgdaOperator{\AgdaFunction{/term}}\AgdaSpace{}%
\AgdaBound{σ}\AgdaSymbol{)}\<%
\\
\>[0]\AgdaOperator{\AgdaFunction{\AgdaUnderscore{}/term\AgdaUnderscore{}}}\AgdaSpace{}%
\AgdaSymbol{\{}\AgdaInductiveConstructor{const}\AgdaSymbol{\}}\AgdaSpace{}%
\AgdaSymbol{(}\AgdaInductiveConstructor{thunk}\AgdaSpace{}%
\AgdaBound{x}\AgdaSymbol{)}%
\>[28]\AgdaBound{σ}%
\>[31]\AgdaSymbol{=}\AgdaSpace{}%
\AgdaFunction{↠}\AgdaSpace{}%
\AgdaSymbol{(}\AgdaBound{x}\AgdaSpace{}%
\AgdaOperator{\AgdaFunction{/term}}\AgdaSpace{}%
\AgdaBound{σ}\AgdaSymbol{)}\<%
\\
\>[0]\AgdaOperator{\AgdaFunction{\AgdaUnderscore{}/term\AgdaUnderscore{}}}\AgdaSpace{}%
\AgdaSymbol{\{}\AgdaInductiveConstructor{compu}\AgdaSymbol{\}}\AgdaSpace{}%
\AgdaSymbol{(}\AgdaInductiveConstructor{elim}\AgdaSpace{}%
\AgdaBound{e}\AgdaSpace{}%
\AgdaBound{s}\AgdaSymbol{)}%
\>[28]\AgdaBound{σ}%
\>[31]\AgdaSymbol{=}\AgdaSpace{}%
\AgdaInductiveConstructor{elim}\AgdaSpace{}%
\AgdaSymbol{(}\AgdaBound{e}\AgdaSpace{}%
\AgdaOperator{\AgdaFunction{/term}}\AgdaSpace{}%
\AgdaBound{σ}\AgdaSymbol{)}\AgdaSpace{}%
\AgdaSymbol{(}\AgdaBound{s}\AgdaSpace{}%
\AgdaOperator{\AgdaFunction{/term}}\AgdaSpace{}%
\AgdaBound{σ}\AgdaSymbol{)}\<%
\\
\>[0]\AgdaOperator{\AgdaFunction{\AgdaUnderscore{}/term\AgdaUnderscore{}}}\AgdaSpace{}%
\AgdaSymbol{\{}\AgdaInductiveConstructor{compu}\AgdaSymbol{\}}\AgdaSpace{}%
\AgdaSymbol{(}\AgdaBound{t}\AgdaSpace{}%
\AgdaOperator{\AgdaInductiveConstructor{∷}}\AgdaSpace{}%
\AgdaBound{T}\AgdaSymbol{)}%
\>[28]\AgdaBound{σ}%
\>[31]\AgdaSymbol{=}\AgdaSpace{}%
\AgdaSymbol{(}\AgdaBound{t}\AgdaSpace{}%
\AgdaOperator{\AgdaFunction{/term}}\AgdaSpace{}%
\AgdaBound{σ}\AgdaSymbol{)}\AgdaSpace{}%
\AgdaOperator{\AgdaInductiveConstructor{∷}}\AgdaSpace{}%
\AgdaSymbol{(}\AgdaBound{T}\AgdaSpace{}%
\AgdaOperator{\AgdaFunction{/term}}\AgdaSpace{}%
\AgdaBound{σ}\AgdaSymbol{)}\<%
\end{code}
}

\section{A Notion of Context}

\hide{
\begin{code}%
\>[0]\AgdaKeyword{module}\AgdaSpace{}%
\AgdaModule{Context}\AgdaSpace{}%
\AgdaKeyword{where}\<%
\\
\>[0]\AgdaKeyword{open}\AgdaSpace{}%
\AgdaKeyword{import}\AgdaSpace{}%
\AgdaModule{Level}\<%
\end{code}
}

In our initial work, we begin with a simple notion of contexts, defining
them as backwards lists of types. We take care to produce a universe independent
implementation so that we might use it with our notion of Pattern.

\begin{code}%
\>[0]\AgdaKeyword{data}\AgdaSpace{}%
\AgdaDatatype{Bwd}\AgdaSpace{}%
\AgdaSymbol{\{}\AgdaBound{ℓ}\AgdaSymbol{\}}\AgdaSpace{}%
\AgdaSymbol{(}\AgdaBound{A}\AgdaSpace{}%
\AgdaSymbol{:}\AgdaSpace{}%
\AgdaPrimitiveType{Set}\AgdaSpace{}%
\AgdaBound{ℓ}\AgdaSymbol{)}\AgdaSpace{}%
\AgdaSymbol{:}\AgdaSpace{}%
\AgdaPrimitiveType{Set}\AgdaSpace{}%
\AgdaBound{ℓ}\AgdaSpace{}%
\AgdaKeyword{where}\<%
\\
\>[0][@{}l@{\AgdaIndent{0}}]%
\>[2]\AgdaInductiveConstructor{ε}%
\>[7]\AgdaSymbol{:}\AgdaSpace{}%
\AgdaDatatype{Bwd}\AgdaSpace{}%
\AgdaBound{A}\<%
\\
%
\>[2]\AgdaOperator{\AgdaInductiveConstructor{\AgdaUnderscore{}-,\AgdaUnderscore{}}}\AgdaSpace{}%
\AgdaSymbol{:}\AgdaSpace{}%
\AgdaDatatype{Bwd}\AgdaSpace{}%
\AgdaBound{A}\AgdaSpace{}%
\AgdaSymbol{→}\AgdaSpace{}%
\AgdaBound{A}\AgdaSpace{}%
\AgdaSymbol{→}\AgdaSpace{}%
\AgdaDatatype{Bwd}\AgdaSpace{}%
\AgdaBound{A}\<%
\end{code}

\hide{
\begin{code}%
\>[0]\AgdaKeyword{private}\<%
\\
\>[0][@{}l@{\AgdaIndent{0}}]%
\>[2]\AgdaKeyword{variable}\<%
\\
\>[2][@{}l@{\AgdaIndent{0}}]%
\>[4]\AgdaGeneralizable{ℓ}\AgdaSpace{}%
\AgdaSymbol{:}\AgdaSpace{}%
\AgdaPostulate{Level}\<%
\\
%
\>[4]\AgdaGeneralizable{X}\AgdaSpace{}%
\AgdaSymbol{:}\AgdaSpace{}%
\AgdaPrimitiveType{Set}\AgdaSpace{}%
\AgdaGeneralizable{ℓ}\<%
\\
%
\>[4]\AgdaGeneralizable{σ}\AgdaSpace{}%
\AgdaSymbol{:}\AgdaSpace{}%
\AgdaGeneralizable{X}\<%
\\
%
\>[4]\AgdaGeneralizable{Γ}\AgdaSpace{}%
\AgdaSymbol{:}\AgdaSpace{}%
\AgdaDatatype{Bwd}\AgdaSpace{}%
\AgdaGeneralizable{X}\<%
\end{code}
}

We use a de-bruijn index representation of variables.

\begin{code}%
\>[0]\AgdaKeyword{data}\AgdaSpace{}%
\AgdaDatatype{Var}\AgdaSpace{}%
\AgdaSymbol{(}\AgdaBound{τ}\AgdaSpace{}%
\AgdaSymbol{:}\AgdaSpace{}%
\AgdaGeneralizable{X}\AgdaSymbol{)}\AgdaSpace{}%
\AgdaSymbol{:}\AgdaSpace{}%
\AgdaDatatype{Bwd}\AgdaSpace{}%
\AgdaBound{X}\AgdaSpace{}%
\AgdaSymbol{→}\AgdaSpace{}%
\AgdaPrimitiveType{Set}\AgdaSpace{}%
\AgdaKeyword{where}\<%
\\
\>[0][@{}l@{\AgdaIndent{0}}]%
\>[2]\AgdaInductiveConstructor{ze}\AgdaSpace{}%
\AgdaSymbol{:}\AgdaSpace{}%
\AgdaDatatype{Var}\AgdaSpace{}%
\AgdaBound{τ}\AgdaSpace{}%
\AgdaSymbol{(}\AgdaGeneralizable{Γ}\AgdaSpace{}%
\AgdaOperator{\AgdaInductiveConstructor{-,}}\AgdaSpace{}%
\AgdaBound{τ}\AgdaSymbol{)}\<%
\\
%
\>[2]\AgdaInductiveConstructor{su}\AgdaSpace{}%
\AgdaSymbol{:}\AgdaSpace{}%
\AgdaDatatype{Var}\AgdaSpace{}%
\AgdaBound{τ}\AgdaSpace{}%
\AgdaGeneralizable{Γ}\AgdaSpace{}%
\AgdaSymbol{→}\AgdaSpace{}%
\AgdaDatatype{Var}\AgdaSpace{}%
\AgdaBound{τ}\AgdaSpace{}%
\AgdaSymbol{(}\AgdaGeneralizable{Γ}\AgdaSpace{}%
\AgdaOperator{\AgdaInductiveConstructor{-,}}\AgdaSpace{}%
\AgdaGeneralizable{σ}\AgdaSymbol{)}\<%
\end{code}

\section{Opening}

\hide{
\begin{code}%
\>[0]\AgdaSymbol{\{-\#}\AgdaSpace{}%
\AgdaKeyword{OPTIONS}\AgdaSpace{}%
\AgdaPragma{--rewriting}\AgdaSpace{}%
\AgdaSymbol{\#-\}}\<%
\\
\>[0]\AgdaKeyword{module}\AgdaSpace{}%
\AgdaModule{Opening}\AgdaSpace{}%
\AgdaKeyword{where}\<%
\end{code}
}

\hide{
\begin{code}%
\>[0]\AgdaKeyword{open}\AgdaSpace{}%
\AgdaKeyword{import}\AgdaSpace{}%
\AgdaModule{CoreLanguage}\AgdaSpace{}%
\AgdaKeyword{using}\AgdaSpace{}%
\AgdaSymbol{(}\AgdaFunction{Scope}\AgdaSymbol{;}\AgdaSpace{}%
\AgdaFunction{Scoped}\AgdaSymbol{)}\<%
\\
\>[0]\AgdaKeyword{open}\AgdaSpace{}%
\AgdaKeyword{import}\AgdaSpace{}%
\AgdaModule{Data.Nat}\AgdaSpace{}%
\AgdaKeyword{using}\AgdaSpace{}%
\AgdaSymbol{(}\AgdaOperator{\AgdaPrimitive{\AgdaUnderscore{}+\AgdaUnderscore{}}}\AgdaSymbol{)}\<%
\\
\>[0]\AgdaKeyword{open}\AgdaSpace{}%
\AgdaKeyword{import}\AgdaSpace{}%
\AgdaModule{Thinning}\AgdaSpace{}%
\AgdaKeyword{using}\AgdaSpace{}%
\AgdaSymbol{(}\AgdaFunction{Thinnable}\AgdaSymbol{;}\AgdaSpace{}%
\AgdaFunction{Ø}\AgdaSymbol{;}\AgdaSpace{}%
\AgdaFunction{ι}\AgdaSymbol{;}\AgdaSpace{}%
\AgdaOperator{\AgdaFunction{\AgdaUnderscore{}++\AgdaUnderscore{}}}\AgdaSymbol{;}\AgdaSpace{}%
\AgdaOperator{\AgdaFunction{\AgdaUnderscore{}▹\AgdaUnderscore{}}}\AgdaSymbol{)}\<%
\end{code}
}

\begin{code}%
\>[0]\AgdaComment{-- scoped things can be openable}\<%
\\
\>[0]\AgdaFunction{Openable}\AgdaSpace{}%
\AgdaSymbol{:}\AgdaSpace{}%
\AgdaSymbol{(}\AgdaBound{T}\AgdaSpace{}%
\AgdaSymbol{:}\AgdaSpace{}%
\AgdaFunction{Scoped}\AgdaSymbol{)}\AgdaSpace{}%
\AgdaSymbol{→}\AgdaSpace{}%
\AgdaPrimitiveType{Set}\<%
\\
\>[0]\AgdaFunction{Openable}\AgdaSpace{}%
\AgdaBound{T}\AgdaSpace{}%
\AgdaSymbol{=}\AgdaSpace{}%
\AgdaSymbol{∀}\AgdaSpace{}%
\AgdaSymbol{\{}\AgdaBound{δ}\AgdaSymbol{\}}\AgdaSpace{}%
\AgdaSymbol{→}\AgdaSpace{}%
\AgdaSymbol{(}\AgdaBound{γ}\AgdaSpace{}%
\AgdaSymbol{:}\AgdaSpace{}%
\AgdaFunction{Scope}\AgdaSymbol{)}\AgdaSpace{}%
\AgdaSymbol{→}\AgdaSpace{}%
\AgdaBound{T}\AgdaSpace{}%
\AgdaBound{δ}\AgdaSpace{}%
\AgdaSymbol{→}\AgdaSpace{}%
\AgdaBound{T}\AgdaSpace{}%
\AgdaSymbol{(}\AgdaBound{γ}\AgdaSpace{}%
\AgdaOperator{\AgdaPrimitive{+}}\AgdaSpace{}%
\AgdaBound{δ}\AgdaSymbol{)}\<%
\\
%
\\[\AgdaEmptyExtraSkip]%
\>[0]\AgdaComment{-- anything thinnable is automatically openable}\<%
\\
\>[0]\AgdaFunction{openable}\AgdaSpace{}%
\AgdaSymbol{:}\AgdaSpace{}%
\AgdaSymbol{∀}\AgdaSpace{}%
\AgdaSymbol{\{}\AgdaBound{T}\AgdaSymbol{\}}\AgdaSpace{}%
\AgdaSymbol{→}\AgdaSpace{}%
\AgdaFunction{Thinnable}\AgdaSpace{}%
\AgdaBound{T}\AgdaSpace{}%
\AgdaSymbol{→}\AgdaSpace{}%
\AgdaFunction{Openable}\AgdaSpace{}%
\AgdaBound{T}\<%
\\
\>[0]\AgdaFunction{openable}\AgdaSpace{}%
\AgdaBound{⟨}\AgdaSpace{}%
\AgdaSymbol{\{}\AgdaBound{δ}\AgdaSymbol{\}}\AgdaSpace{}%
\AgdaSymbol{=}\AgdaSpace{}%
\AgdaSymbol{λ}\AgdaSpace{}%
\AgdaBound{γ}\AgdaSpace{}%
\AgdaBound{Tδ}\AgdaSpace{}%
\AgdaSymbol{→}\AgdaSpace{}%
\AgdaBound{⟨}\AgdaSpace{}%
\AgdaBound{Tδ}\AgdaSpace{}%
\AgdaSymbol{(}\AgdaBound{γ}\AgdaSpace{}%
\AgdaOperator{\AgdaFunction{▹}}\AgdaSpace{}%
\AgdaBound{δ}\AgdaSymbol{)}\<%
\end{code}

\section{Patterns}

\hide{
\begin{code}%
\>[0]\AgdaSymbol{\{-\#}\AgdaSpace{}%
\AgdaKeyword{OPTIONS}\AgdaSpace{}%
\AgdaPragma{--rewriting}\AgdaSpace{}%
\AgdaSymbol{\#-\}}\<%
\\
\>[0]\AgdaKeyword{module}\AgdaSpace{}%
\AgdaModule{Pattern}\AgdaSpace{}%
\AgdaKeyword{where}\<%
\end{code}
}

\hide{
\begin{code}%
\>[0]\AgdaKeyword{open}\AgdaSpace{}%
\AgdaKeyword{import}\AgdaSpace{}%
\AgdaModule{CoreLanguage}\AgdaSpace{}%
\AgdaKeyword{renaming}\AgdaSpace{}%
\AgdaSymbol{(}\AgdaFunction{↠}\AgdaSpace{}%
\AgdaSymbol{to}\AgdaSpace{}%
\AgdaFunction{↠↠}\AgdaSymbol{)}\<%
\\
\>[0]\AgdaKeyword{open}\AgdaSpace{}%
\AgdaKeyword{import}\AgdaSpace{}%
\AgdaModule{Thinning}\AgdaSpace{}%
\AgdaKeyword{using}\AgdaSpace{}%
\AgdaSymbol{(}\AgdaOperator{\AgdaDatatype{\AgdaUnderscore{}⊑\AgdaUnderscore{}}}\AgdaSymbol{;}\AgdaSpace{}%
\AgdaFunction{Ø}\AgdaSymbol{;}\AgdaSpace{}%
\AgdaFunction{ι}\AgdaSymbol{;}\AgdaSpace{}%
\AgdaOperator{\AgdaFunction{\AgdaUnderscore{}++\AgdaUnderscore{}}}\AgdaSymbol{;}\AgdaSpace{}%
\AgdaOperator{\AgdaFunction{\AgdaUnderscore{}⟨term⊗\AgdaUnderscore{}}}\AgdaSymbol{;}\AgdaSpace{}%
\AgdaFunction{++-identityʳ}\AgdaSymbol{)}\<%
\\
\>[0]\AgdaKeyword{open}\AgdaSpace{}%
\AgdaKeyword{import}\AgdaSpace{}%
\AgdaModule{Data.Char}\AgdaSpace{}%
\AgdaKeyword{using}\AgdaSpace{}%
\AgdaSymbol{(}\AgdaPostulate{Char}\AgdaSymbol{)}\AgdaSpace{}%
\AgdaKeyword{renaming}\AgdaSpace{}%
\AgdaSymbol{(}\AgdaOperator{\AgdaFunction{\AgdaUnderscore{}≟\AgdaUnderscore{}}}\AgdaSpace{}%
\AgdaSymbol{to}\AgdaSpace{}%
\AgdaOperator{\AgdaFunction{\AgdaUnderscore{}is\AgdaUnderscore{}}}\AgdaSymbol{)}\<%
\\
\>[0]\AgdaKeyword{open}\AgdaSpace{}%
\AgdaKeyword{import}\AgdaSpace{}%
\AgdaModule{Data.Nat.Properties}\AgdaSpace{}%
\AgdaKeyword{using}\AgdaSpace{}%
\AgdaSymbol{(}\AgdaOperator{\AgdaFunction{\AgdaUnderscore{}≟\AgdaUnderscore{}}}\AgdaSymbol{)}\<%
\\
\>[0]\AgdaKeyword{open}\AgdaSpace{}%
\AgdaKeyword{import}\AgdaSpace{}%
\AgdaModule{Data.Maybe}\AgdaSpace{}%
\AgdaKeyword{using}\AgdaSpace{}%
\AgdaSymbol{(}\AgdaDatatype{Maybe}\AgdaSymbol{;}\AgdaSpace{}%
\AgdaInductiveConstructor{just}\AgdaSymbol{;}\AgdaSpace{}%
\AgdaInductiveConstructor{nothing}\AgdaSymbol{;}\AgdaSpace{}%
\AgdaOperator{\AgdaFunction{\AgdaUnderscore{}>>=\AgdaUnderscore{}}}\AgdaSymbol{)}\<%
\\
\>[0]\AgdaKeyword{open}\AgdaSpace{}%
\AgdaKeyword{import}\AgdaSpace{}%
\AgdaModule{Data.Bool}\AgdaSpace{}%
\AgdaKeyword{using}\AgdaSpace{}%
\AgdaSymbol{(}\AgdaInductiveConstructor{true}\AgdaSymbol{;}\AgdaSpace{}%
\AgdaInductiveConstructor{false}\AgdaSymbol{)}\<%
\\
\>[0]\AgdaKeyword{open}\AgdaSpace{}%
\AgdaKeyword{import}\AgdaSpace{}%
\AgdaModule{Relation.Nullary}\AgdaSpace{}%
\AgdaKeyword{using}\AgdaSpace{}%
\AgdaSymbol{(}\AgdaField{does}\AgdaSymbol{;}\AgdaSpace{}%
\AgdaOperator{\AgdaInductiveConstructor{\AgdaUnderscore{}because\AgdaUnderscore{}}}\AgdaSymbol{;}\AgdaSpace{}%
\AgdaField{proof}\AgdaSymbol{;}\AgdaSpace{}%
\AgdaInductiveConstructor{ofʸ}\AgdaSymbol{)}\<%
\\
\>[0]\AgdaKeyword{open}\AgdaSpace{}%
\AgdaKeyword{import}\AgdaSpace{}%
\AgdaModule{Relation.Binary.PropositionalEquality}\AgdaSpace{}%
\AgdaKeyword{using}\AgdaSpace{}%
\AgdaSymbol{(}\AgdaInductiveConstructor{refl}\AgdaSymbol{;}%
\>[64]\AgdaOperator{\AgdaDatatype{\AgdaUnderscore{}≡\AgdaUnderscore{}}}\AgdaSymbol{;}\AgdaSpace{}%
\AgdaFunction{cong}\AgdaSymbol{;}\AgdaSpace{}%
\AgdaFunction{cong₂}\AgdaSymbol{)}\<%
\\
\>[0]\AgdaKeyword{open}\AgdaSpace{}%
\AgdaKeyword{import}\AgdaSpace{}%
\AgdaModule{Data.Nat}\AgdaSpace{}%
\AgdaKeyword{using}\AgdaSpace{}%
\AgdaSymbol{(}\AgdaInductiveConstructor{zero}\AgdaSymbol{;}\AgdaSpace{}%
\AgdaInductiveConstructor{suc}\AgdaSymbol{;}\AgdaSpace{}%
\AgdaOperator{\AgdaPrimitive{\AgdaUnderscore{}+\AgdaUnderscore{}}}\AgdaSymbol{)}\<%
\\
\>[0]\AgdaKeyword{open}\AgdaSpace{}%
\AgdaKeyword{import}\AgdaSpace{}%
\AgdaModule{Opening}\AgdaSpace{}%
\AgdaKeyword{using}\AgdaSpace{}%
\AgdaSymbol{(}\AgdaFunction{Openable}\AgdaSymbol{)}\<%
\end{code}
}

\hide{
\begin{code}%
\>[0]\AgdaKeyword{private}\<%
\\
\>[0][@{}l@{\AgdaIndent{0}}]%
\>[2]\AgdaKeyword{variable}\<%
\\
\>[2][@{}l@{\AgdaIndent{0}}]%
\>[4]\AgdaGeneralizable{δ}\AgdaSpace{}%
\AgdaSymbol{:}\AgdaSpace{}%
\AgdaFunction{Scope}\<%
\\
%
\>[4]\AgdaGeneralizable{γ}\AgdaSpace{}%
\AgdaSymbol{:}\AgdaSpace{}%
\AgdaFunction{Scope}\<%
\\
%
\>[4]\AgdaGeneralizable{γ'}\AgdaSpace{}%
\AgdaSymbol{:}\AgdaSpace{}%
\AgdaFunction{Scope}\<%
\end{code}
}

Key to implementing our generic type-checker, is the concept of a pattern. Our
rules are defined not in terms of concrete pieces of syntax, but in terms of
patterns of constructions, which we then match against concrete syntax.

Our concept of a pattern is structurally identical to that of a construction,
except that we exclude thunks, and introduce the notion of a \emph{place} which
may stand for any arbitrary construction scoped in some $δ$ so long as we show
how it might be thinned to $γ$.

The dual concept of a pattern is that of an environment. It is structurally
similar to a pattern except where a pattern may have a \emph{place}, an
environment answers this call with a \emph{thing} that can fit in the place.
As always, we must be careful to handle scope correctly in the case of $bind$
when constructing environments so that the underlying entity is defined in
the weakened scope.

Environments are indexed by a pattern so that we can ensure that it always
matches exactly the pattern intended (in that it has an identical structure
and a \emph{thing} for every \emph{place}). Consequently this allows us a
non-failable operation to generate a term from from pattern $p$ and its
associated $p\mbox{-Env}$

\begin{code}%
\>[0]\AgdaKeyword{data}\AgdaSpace{}%
\AgdaDatatype{Pattern}\AgdaSpace{}%
\AgdaSymbol{(}\AgdaBound{γ}\AgdaSpace{}%
\AgdaSymbol{:}\AgdaSpace{}%
\AgdaFunction{Scope}\AgdaSymbol{)}\AgdaSpace{}%
\AgdaSymbol{:}\AgdaSpace{}%
\AgdaPrimitiveType{Set}\AgdaSpace{}%
\AgdaKeyword{where}\<%
\\
\>[0][@{}l@{\AgdaIndent{0}}]%
\>[2]\AgdaInductiveConstructor{`}%
\>[9]\AgdaSymbol{:}\AgdaSpace{}%
\AgdaPostulate{Char}\AgdaSpace{}%
\AgdaSymbol{→}\AgdaSpace{}%
\AgdaDatatype{Pattern}\AgdaSpace{}%
\AgdaBound{γ}\<%
\\
%
\>[2]\AgdaOperator{\AgdaInductiveConstructor{\AgdaUnderscore{}∙\AgdaUnderscore{}}}%
\>[9]\AgdaSymbol{:}\AgdaSpace{}%
\AgdaDatatype{Pattern}\AgdaSpace{}%
\AgdaBound{γ}\AgdaSpace{}%
\AgdaSymbol{→}\AgdaSpace{}%
\AgdaDatatype{Pattern}\AgdaSpace{}%
\AgdaBound{γ}\AgdaSpace{}%
\AgdaSymbol{→}\AgdaSpace{}%
\AgdaDatatype{Pattern}\AgdaSpace{}%
\AgdaBound{γ}\<%
\\
%
\>[2]\AgdaInductiveConstructor{bind}%
\>[9]\AgdaSymbol{:}\AgdaSpace{}%
\AgdaDatatype{Pattern}\AgdaSpace{}%
\AgdaSymbol{(}\AgdaInductiveConstructor{suc}\AgdaSpace{}%
\AgdaBound{γ}\AgdaSymbol{)}\AgdaSpace{}%
\AgdaSymbol{→}\AgdaSpace{}%
\AgdaDatatype{Pattern}\AgdaSpace{}%
\AgdaBound{γ}\<%
\\
%
\>[2]\AgdaInductiveConstructor{place}%
\>[9]\AgdaSymbol{:}\AgdaSpace{}%
\AgdaSymbol{\{}\AgdaBound{δ}\AgdaSpace{}%
\AgdaSymbol{:}\AgdaSpace{}%
\AgdaFunction{Scope}\AgdaSymbol{\}}\AgdaSpace{}%
\AgdaSymbol{→}\AgdaSpace{}%
\AgdaBound{δ}\AgdaSpace{}%
\AgdaOperator{\AgdaDatatype{⊑}}\AgdaSpace{}%
\AgdaBound{γ}\AgdaSpace{}%
\AgdaSymbol{→}\AgdaSpace{}%
\AgdaDatatype{Pattern}\AgdaSpace{}%
\AgdaBound{γ}\<%
\end{code}
\hide{
\begin{code}%
\>[0]\AgdaKeyword{infixr}\AgdaSpace{}%
\AgdaNumber{20}\AgdaSpace{}%
\AgdaOperator{\AgdaInductiveConstructor{\AgdaUnderscore{}∙\AgdaUnderscore{}}}\<%
\\
\>[0]\AgdaKeyword{private}\<%
\\
\>[0][@{}l@{\AgdaIndent{0}}]%
\>[2]\AgdaKeyword{variable}\<%
\\
\>[2][@{}l@{\AgdaIndent{0}}]%
\>[4]\AgdaGeneralizable{p}\AgdaSpace{}%
\AgdaSymbol{:}\AgdaSpace{}%
\AgdaDatatype{Pattern}\AgdaSpace{}%
\AgdaGeneralizable{γ}\<%
\\
%
\>[4]\AgdaGeneralizable{q}\AgdaSpace{}%
\AgdaSymbol{:}\AgdaSpace{}%
\AgdaDatatype{Pattern}\AgdaSpace{}%
\AgdaGeneralizable{γ}\<%
\\
%
\>[4]\AgdaGeneralizable{r}\AgdaSpace{}%
\AgdaSymbol{:}\AgdaSpace{}%
\AgdaDatatype{Pattern}\AgdaSpace{}%
\AgdaGeneralizable{γ}\<%
\\
%
\>[4]\AgdaGeneralizable{t}\AgdaSpace{}%
\AgdaSymbol{:}\AgdaSpace{}%
\AgdaDatatype{Pattern}\AgdaSpace{}%
\AgdaSymbol{(}\AgdaInductiveConstructor{suc}\AgdaSpace{}%
\AgdaGeneralizable{γ}\AgdaSymbol{)}\<%
\end{code}
}
\begin{code}%
\>[0]\AgdaKeyword{data}\AgdaSpace{}%
\AgdaOperator{\AgdaDatatype{\AgdaUnderscore{}-Env}}\AgdaSpace{}%
\AgdaSymbol{\{}\AgdaBound{γ}\AgdaSpace{}%
\AgdaSymbol{:}\AgdaSpace{}%
\AgdaFunction{Scope}\AgdaSymbol{\}}\AgdaSpace{}%
\AgdaSymbol{:}\AgdaSpace{}%
\AgdaDatatype{Pattern}\AgdaSpace{}%
\AgdaBound{γ}\AgdaSpace{}%
\AgdaSymbol{→}\AgdaSpace{}%
\AgdaPrimitiveType{Set}\AgdaSpace{}%
\AgdaKeyword{where}\<%
\\
\>[0][@{}l@{\AgdaIndent{0}}]%
\>[2]\AgdaInductiveConstructor{`}%
\>[9]\AgdaSymbol{:}\AgdaSpace{}%
\AgdaSymbol{\{}\AgdaBound{c}\AgdaSpace{}%
\AgdaSymbol{:}\AgdaSpace{}%
\AgdaPostulate{Char}\AgdaSymbol{\}}\AgdaSpace{}%
\AgdaSymbol{→}\AgdaSpace{}%
\AgdaSymbol{(}\AgdaInductiveConstructor{`}\AgdaSpace{}%
\AgdaBound{c}\AgdaSymbol{)}\AgdaSpace{}%
\AgdaOperator{\AgdaDatatype{-Env}}\<%
\\
%
\>[2]\AgdaOperator{\AgdaInductiveConstructor{\AgdaUnderscore{}∙\AgdaUnderscore{}}}%
\>[9]\AgdaSymbol{:}\AgdaSpace{}%
\AgdaGeneralizable{q}\AgdaSpace{}%
\AgdaOperator{\AgdaDatatype{-Env}}\AgdaSpace{}%
\AgdaSymbol{→}\AgdaSpace{}%
\AgdaGeneralizable{r}\AgdaSpace{}%
\AgdaOperator{\AgdaDatatype{-Env}}\AgdaSpace{}%
\AgdaSymbol{→}\AgdaSpace{}%
\AgdaSymbol{(}\AgdaGeneralizable{q}\AgdaSpace{}%
\AgdaOperator{\AgdaInductiveConstructor{∙}}\AgdaSpace{}%
\AgdaGeneralizable{r}\AgdaSymbol{)}\AgdaSpace{}%
\AgdaOperator{\AgdaDatatype{-Env}}\<%
\\
%
\>[2]\AgdaInductiveConstructor{bind}%
\>[9]\AgdaSymbol{:}\AgdaSpace{}%
\AgdaGeneralizable{t}\AgdaSpace{}%
\AgdaOperator{\AgdaDatatype{-Env}}\AgdaSpace{}%
\AgdaSymbol{→}\AgdaSpace{}%
\AgdaSymbol{(}\AgdaInductiveConstructor{bind}\AgdaSpace{}%
\AgdaGeneralizable{t}\AgdaSymbol{)}\AgdaSpace{}%
\AgdaOperator{\AgdaDatatype{-Env}}\<%
\\
%
\>[2]\AgdaInductiveConstructor{thing}%
\>[9]\AgdaSymbol{:}\AgdaSpace{}%
\AgdaSymbol{\{}\AgdaBound{θ}\AgdaSpace{}%
\AgdaSymbol{:}\AgdaSpace{}%
\AgdaGeneralizable{δ}\AgdaSpace{}%
\AgdaOperator{\AgdaDatatype{⊑}}\AgdaSpace{}%
\AgdaBound{γ}\AgdaSymbol{\}}\AgdaSpace{}%
\AgdaSymbol{→}\AgdaSpace{}%
\AgdaFunction{Term}\AgdaSpace{}%
\AgdaInductiveConstructor{const}\AgdaSpace{}%
\AgdaGeneralizable{δ}\AgdaSpace{}%
\AgdaSymbol{→}\AgdaSpace{}%
\AgdaSymbol{(}\AgdaInductiveConstructor{place}\AgdaSpace{}%
\AgdaBound{θ}\AgdaSymbol{)}\AgdaSpace{}%
\AgdaOperator{\AgdaDatatype{-Env}}\<%
\end{code}

We define the opening of a pattern by recursing structurally and opening
the thinnings of the places in the matter described in section \ref{sec:Opening}.

\begin{code}%
\>[0]\AgdaOperator{\AgdaFunction{\AgdaUnderscore{}⊗\AgdaUnderscore{}}}\AgdaSpace{}%
\AgdaSymbol{:}\AgdaSpace{}%
\AgdaFunction{Openable}\AgdaSpace{}%
\AgdaDatatype{Pattern}\<%
\end{code}

\hide{
\begin{code}%
\>[0]\AgdaComment{-- We can 'open' patterns}\<%
\\
\>[0]\AgdaBound{γ}\AgdaSpace{}%
\AgdaOperator{\AgdaFunction{⊗}}\AgdaSpace{}%
\AgdaInductiveConstructor{`}\AgdaSpace{}%
\AgdaBound{x}%
\>[13]\AgdaSymbol{=}\AgdaSpace{}%
\AgdaInductiveConstructor{`}\AgdaSpace{}%
\AgdaBound{x}\<%
\\
\>[0]\AgdaBound{γ}\AgdaSpace{}%
\AgdaOperator{\AgdaFunction{⊗}}\AgdaSpace{}%
\AgdaSymbol{(}\AgdaBound{s}\AgdaSpace{}%
\AgdaOperator{\AgdaInductiveConstructor{∙}}\AgdaSpace{}%
\AgdaBound{t}\AgdaSymbol{)}%
\>[13]\AgdaSymbol{=}\AgdaSpace{}%
\AgdaSymbol{(}\AgdaBound{γ}\AgdaSpace{}%
\AgdaOperator{\AgdaFunction{⊗}}\AgdaSpace{}%
\AgdaBound{s}\AgdaSymbol{)}\AgdaSpace{}%
\AgdaOperator{\AgdaInductiveConstructor{∙}}\AgdaSpace{}%
\AgdaSymbol{(}\AgdaBound{γ}\AgdaSpace{}%
\AgdaOperator{\AgdaFunction{⊗}}\AgdaSpace{}%
\AgdaBound{t}\AgdaSymbol{)}\<%
\\
\>[0]\AgdaBound{γ}\AgdaSpace{}%
\AgdaOperator{\AgdaFunction{⊗}}\AgdaSpace{}%
\AgdaSymbol{(}\AgdaInductiveConstructor{bind}\AgdaSpace{}%
\AgdaBound{t}\AgdaSymbol{)}\AgdaSpace{}%
\AgdaSymbol{=}\AgdaSpace{}%
\AgdaInductiveConstructor{bind}\AgdaSpace{}%
\AgdaSymbol{(}\AgdaBound{γ}\AgdaSpace{}%
\AgdaOperator{\AgdaFunction{⊗}}\AgdaSpace{}%
\AgdaBound{t}\AgdaSymbol{)}\<%
\\
\>[0]\AgdaBound{γ}\AgdaSpace{}%
\AgdaOperator{\AgdaFunction{⊗}}\AgdaSpace{}%
\AgdaInductiveConstructor{place}\AgdaSpace{}%
\AgdaBound{θ}%
\>[13]\AgdaSymbol{=}\AgdaSpace{}%
\AgdaInductiveConstructor{place}\AgdaSpace{}%
\AgdaSymbol{(}\AgdaFunction{ι}\AgdaSpace{}%
\AgdaSymbol{\{}\AgdaBound{γ}\AgdaSymbol{\}}\AgdaSpace{}%
\AgdaOperator{\AgdaFunction{++}}\AgdaSpace{}%
\AgdaBound{θ}\AgdaSymbol{)}\<%
\\
%
\\[\AgdaEmptyExtraSkip]%
\>[0]\AgdaComment{-- opening a pattern by 0 is just the pattern}\<%
\\
\>[0]\AgdaKeyword{open}\AgdaSpace{}%
\AgdaKeyword{import}\AgdaSpace{}%
\AgdaModule{Thinning}\AgdaSpace{}%
\AgdaKeyword{using}\AgdaSpace{}%
\AgdaSymbol{(}\AgdaInductiveConstructor{ε}\AgdaSymbol{;}\AgdaSpace{}%
\AgdaOperator{\AgdaInductiveConstructor{\AgdaUnderscore{}O}}\AgdaSymbol{;}\AgdaSpace{}%
\AgdaOperator{\AgdaInductiveConstructor{\AgdaUnderscore{}I}}\AgdaSymbol{)}\<%
\\
\>[0]\AgdaFunction{⊗-identityʳ}\AgdaSpace{}%
\AgdaSymbol{:}\AgdaSpace{}%
\AgdaNumber{0}\AgdaSpace{}%
\AgdaOperator{\AgdaFunction{⊗}}\AgdaSpace{}%
\AgdaGeneralizable{p}\AgdaSpace{}%
\AgdaOperator{\AgdaDatatype{≡}}\AgdaSpace{}%
\AgdaGeneralizable{p}\<%
\\
\>[0]\AgdaFunction{⊗-identityʳ}\AgdaSpace{}%
\AgdaSymbol{\{}\AgdaArgument{p}\AgdaSpace{}%
\AgdaSymbol{=}\AgdaSpace{}%
\AgdaInductiveConstructor{`}\AgdaSpace{}%
\AgdaBound{x}\AgdaSymbol{\}}%
\>[26]\AgdaSymbol{=}\AgdaSpace{}%
\AgdaInductiveConstructor{refl}\<%
\\
\>[0]\AgdaFunction{⊗-identityʳ}\AgdaSpace{}%
\AgdaSymbol{\{}\AgdaArgument{p}\AgdaSpace{}%
\AgdaSymbol{=}\AgdaSpace{}%
\AgdaBound{p}\AgdaSpace{}%
\AgdaOperator{\AgdaInductiveConstructor{∙}}\AgdaSpace{}%
\AgdaBound{p₁}\AgdaSymbol{\}}%
\>[26]\AgdaSymbol{=}\AgdaSpace{}%
\AgdaFunction{cong₂}\AgdaSpace{}%
\AgdaOperator{\AgdaInductiveConstructor{\AgdaUnderscore{}∙\AgdaUnderscore{}}}\AgdaSpace{}%
\AgdaFunction{⊗-identityʳ}\AgdaSpace{}%
\AgdaFunction{⊗-identityʳ}\<%
\\
\>[0]\AgdaFunction{⊗-identityʳ}\AgdaSpace{}%
\AgdaSymbol{\{}\AgdaArgument{p}\AgdaSpace{}%
\AgdaSymbol{=}\AgdaSpace{}%
\AgdaInductiveConstructor{bind}\AgdaSpace{}%
\AgdaBound{p}\AgdaSymbol{\}}%
\>[26]\AgdaSymbol{=}\AgdaSpace{}%
\AgdaFunction{cong}\AgdaSpace{}%
\AgdaInductiveConstructor{bind}\AgdaSpace{}%
\AgdaFunction{⊗-identityʳ}\<%
\\
\>[0]\AgdaFunction{⊗-identityʳ}\AgdaSpace{}%
\AgdaSymbol{\{}\AgdaArgument{p}\AgdaSpace{}%
\AgdaSymbol{=}\AgdaSpace{}%
\AgdaInductiveConstructor{place}\AgdaSpace{}%
\AgdaBound{θ}\AgdaSymbol{\}}\AgdaSpace{}%
\AgdaKeyword{rewrite}\AgdaSpace{}%
\AgdaFunction{++-identityʳ}\AgdaSpace{}%
\AgdaBound{θ}\AgdaSpace{}%
\AgdaSymbol{=}\AgdaSpace{}%
\AgdaInductiveConstructor{refl}\<%
\end{code}
}

We now have the required machinery to define pattern matching. We do not
define matching over some term and pattern scoped identially, but more 
generally over some term that might be operating in some wider scope. This
is crucial as a pattern is often defined in the empty scope so that we might
not refer to arbirary free variables when defining formal rules. When
type-checking, the terms we match against them may operate in a wider scope.

For this reason, our matching allows the matching of a term in some wider
scope, to a pattern in a potentially narrower scope and if it succeeds it
returns an environment for the \emph{opened} pattern.

\begin{code}%
\>[0]\AgdaFunction{match}\AgdaSpace{}%
\AgdaSymbol{:}\AgdaSpace{}%
\AgdaFunction{Term}\AgdaSpace{}%
\AgdaInductiveConstructor{const}\AgdaSpace{}%
\AgdaSymbol{(}\AgdaGeneralizable{δ}\AgdaSpace{}%
\AgdaOperator{\AgdaPrimitive{+}}\AgdaSpace{}%
\AgdaGeneralizable{γ}\AgdaSymbol{)}\AgdaSpace{}%
\AgdaSymbol{→}\AgdaSpace{}%
\AgdaSymbol{(}\AgdaBound{p}\AgdaSpace{}%
\AgdaSymbol{:}\AgdaSpace{}%
\AgdaDatatype{Pattern}\AgdaSpace{}%
\AgdaGeneralizable{γ}\AgdaSymbol{)}\AgdaSpace{}%
\AgdaSymbol{→}\AgdaSpace{}%
\AgdaDatatype{Maybe}\AgdaSpace{}%
\AgdaSymbol{((}\AgdaGeneralizable{δ}\AgdaSpace{}%
\AgdaOperator{\AgdaFunction{⊗}}\AgdaSpace{}%
\AgdaBound{p}\AgdaSymbol{)}\AgdaSpace{}%
\AgdaOperator{\AgdaDatatype{-Env}}\AgdaSymbol{)}\<%
\\
\>[0]\AgdaFunction{match}%
\>[7]\AgdaSymbol{\{}\AgdaArgument{γ}\AgdaSpace{}%
\AgdaSymbol{=}\AgdaSpace{}%
\AgdaBound{γ}\AgdaSymbol{\}}\AgdaSpace{}%
\AgdaBound{t}%
\>[19]\AgdaSymbol{(}\AgdaInductiveConstructor{place}\AgdaSpace{}%
\AgdaSymbol{\{}\AgdaBound{δ'}\AgdaSymbol{\}}\AgdaSpace{}%
\AgdaBound{θ}\AgdaSymbol{)}\AgdaSpace{}%
\AgdaKeyword{with}\AgdaSpace{}%
\AgdaBound{γ}\AgdaSpace{}%
\AgdaOperator{\AgdaFunction{≟}}\AgdaSpace{}%
\AgdaBound{δ'}\<%
\\
\>[0]\AgdaSymbol{...}\AgdaSpace{}%
\AgdaSymbol{|}\AgdaSpace{}%
\AgdaInductiveConstructor{true}\AgdaSpace{}%
\AgdaOperator{\AgdaInductiveConstructor{because}}\AgdaSpace{}%
\AgdaInductiveConstructor{ofʸ}\AgdaSpace{}%
\AgdaInductiveConstructor{refl}\AgdaSpace{}%
\AgdaSymbol{=}\AgdaSpace{}%
\AgdaInductiveConstructor{just}\AgdaSpace{}%
\AgdaSymbol{(}\AgdaInductiveConstructor{thing}\AgdaSpace{}%
\AgdaBound{t}\AgdaSymbol{)}\<%
\\
\>[0]\AgdaSymbol{...}\AgdaSpace{}%
\AgdaSymbol{|}\AgdaSpace{}%
\AgdaInductiveConstructor{false}\AgdaSpace{}%
\AgdaOperator{\AgdaInductiveConstructor{because}}\AgdaSpace{}%
\AgdaSymbol{\AgdaUnderscore{}}%
\>[28]\AgdaSymbol{=}\AgdaSpace{}%
\AgdaInductiveConstructor{nothing}\<%
\\
\>[0]\AgdaFunction{match}\AgdaSpace{}%
\AgdaSymbol{(}\AgdaInductiveConstructor{`}\AgdaSpace{}%
\AgdaBound{a}\AgdaSymbol{)}\AgdaSpace{}%
\AgdaSymbol{(}\AgdaInductiveConstructor{`}\AgdaSpace{}%
\AgdaBound{c}\AgdaSymbol{)}\AgdaSpace{}%
\AgdaKeyword{with}\AgdaSpace{}%
\AgdaBound{a}\AgdaSpace{}%
\AgdaOperator{\AgdaFunction{is}}\AgdaSpace{}%
\AgdaBound{c}\<%
\\
\>[0]\AgdaSymbol{...}\AgdaSpace{}%
\AgdaSymbol{|}\AgdaSpace{}%
\AgdaInductiveConstructor{true}\AgdaSpace{}%
\AgdaOperator{\AgdaInductiveConstructor{because}}\AgdaSpace{}%
\AgdaInductiveConstructor{ofʸ}\AgdaSpace{}%
\AgdaInductiveConstructor{refl}\AgdaSpace{}%
\AgdaSymbol{=}\AgdaSpace{}%
\AgdaInductiveConstructor{just}\AgdaSpace{}%
\AgdaInductiveConstructor{`}\<%
\\
\>[0]\AgdaSymbol{...}\AgdaSpace{}%
\AgdaSymbol{|}\AgdaSpace{}%
\AgdaInductiveConstructor{false}\AgdaSpace{}%
\AgdaOperator{\AgdaInductiveConstructor{because}}\AgdaSpace{}%
\AgdaSymbol{\AgdaUnderscore{}}%
\>[28]\AgdaSymbol{=}\AgdaSpace{}%
\AgdaInductiveConstructor{nothing}\<%
\\
\>[0]\AgdaFunction{match}\AgdaSpace{}%
\AgdaSymbol{(}\AgdaBound{s}\AgdaSpace{}%
\AgdaOperator{\AgdaInductiveConstructor{∙}}\AgdaSpace{}%
\AgdaBound{t}\AgdaSymbol{)}\AgdaSpace{}%
\AgdaSymbol{(}\AgdaBound{p}\AgdaSpace{}%
\AgdaOperator{\AgdaInductiveConstructor{∙}}\AgdaSpace{}%
\AgdaBound{q}\AgdaSymbol{)}%
\>[24]\AgdaSymbol{=}%
\>[313I]\AgdaKeyword{do}\<%
\\
\>[313I][@{}l@{\AgdaIndent{0}}]%
\>[28]\AgdaBound{x}\AgdaSpace{}%
\AgdaOperator{\AgdaFunction{←}}\AgdaSpace{}%
\AgdaFunction{match}\AgdaSpace{}%
\AgdaBound{s}\AgdaSpace{}%
\AgdaBound{p}\<%
\\
%
\>[28]\AgdaBound{y}\AgdaSpace{}%
\AgdaOperator{\AgdaFunction{←}}\AgdaSpace{}%
\AgdaFunction{match}\AgdaSpace{}%
\AgdaBound{t}\AgdaSpace{}%
\AgdaBound{q}\<%
\\
%
\>[28]\AgdaInductiveConstructor{just}\AgdaSpace{}%
\AgdaSymbol{(}\AgdaBound{x}\AgdaSpace{}%
\AgdaOperator{\AgdaInductiveConstructor{∙}}\AgdaSpace{}%
\AgdaBound{y}\AgdaSymbol{)}\<%
\\
\>[0]\AgdaFunction{match}\AgdaSpace{}%
\AgdaSymbol{(}\AgdaInductiveConstructor{bind}\AgdaSpace{}%
\AgdaBound{t}\AgdaSymbol{)}\AgdaSpace{}%
\AgdaSymbol{(}\AgdaInductiveConstructor{bind}\AgdaSpace{}%
\AgdaBound{p}\AgdaSymbol{)}\AgdaSpace{}%
\AgdaSymbol{=}%
\>[330I]\AgdaKeyword{do}\<%
\\
\>[330I][@{}l@{\AgdaIndent{0}}]%
\>[28]\AgdaBound{x}\AgdaSpace{}%
\AgdaOperator{\AgdaFunction{←}}\AgdaSpace{}%
\AgdaFunction{match}\AgdaSpace{}%
\AgdaBound{t}\AgdaSpace{}%
\AgdaBound{p}\<%
\\
%
\>[28]\AgdaInductiveConstructor{just}\AgdaSpace{}%
\AgdaSymbol{(}\AgdaInductiveConstructor{bind}\AgdaSpace{}%
\AgdaBound{x}\AgdaSymbol{)}\<%
\\
%
\\[\AgdaEmptyExtraSkip]%
\>[0]\AgdaComment{-- TO DO, THUNK MATCHING!! evaluate it to head normal form}\<%
\\
\>[0]\AgdaCatchallClause{\AgdaFunction{match}}\AgdaSpace{}%
\AgdaCatchallClause{\AgdaSymbol{\AgdaUnderscore{}}}\AgdaSpace{}%
\AgdaCatchallClause{\AgdaSymbol{\AgdaUnderscore{}}}%
\>[28]\AgdaSymbol{=}\AgdaSpace{}%
\AgdaInductiveConstructor{nothing}\<%
\end{code}

When contructing formal rules, it is critical that we are able to refer
to distinct places in a pattern. For this purpose we define a schematic
variable. This variable is able to trace a path through the pattern that
indexes it, terminating with a $⋆$ to mark the place we refer to. We
construct this concept with care to ensure it is well-scoped by construction.

Using a schematic variable, we are able to look up the term associated to
a place by an environment by merely proceeding structually down the path
described by the svar and extracting the term from the thing it points to.

\begin{code}%
\>[0]\AgdaKeyword{data}\AgdaSpace{}%
\AgdaDatatype{svar}\AgdaSpace{}%
\AgdaSymbol{:}\AgdaSpace{}%
\AgdaDatatype{Pattern}\AgdaSpace{}%
\AgdaGeneralizable{γ}\AgdaSpace{}%
\AgdaSymbol{→}\AgdaSpace{}%
\AgdaFunction{Scope}\AgdaSpace{}%
\AgdaSymbol{→}\AgdaSpace{}%
\AgdaPrimitiveType{Set}\AgdaSpace{}%
\AgdaKeyword{where}\<%
\\
\>[0][@{}l@{\AgdaIndent{0}}]%
\>[2]\AgdaInductiveConstructor{⋆}%
\>[8]\AgdaSymbol{:}\AgdaSpace{}%
\AgdaSymbol{\{}\AgdaBound{θ}\AgdaSpace{}%
\AgdaSymbol{:}\AgdaSpace{}%
\AgdaGeneralizable{δ}\AgdaSpace{}%
\AgdaOperator{\AgdaDatatype{⊑}}\AgdaSpace{}%
\AgdaGeneralizable{γ}\AgdaSymbol{\}}\AgdaSpace{}%
\AgdaSymbol{→}\AgdaSpace{}%
\AgdaDatatype{svar}\AgdaSpace{}%
\AgdaSymbol{(}\AgdaInductiveConstructor{place}\AgdaSpace{}%
\AgdaBound{θ}\AgdaSymbol{)}\AgdaSpace{}%
\AgdaGeneralizable{δ}\<%
\\
%
\>[2]\AgdaOperator{\AgdaInductiveConstructor{\AgdaUnderscore{}∙}}%
\>[8]\AgdaSymbol{:}\AgdaSpace{}%
\AgdaDatatype{svar}\AgdaSpace{}%
\AgdaGeneralizable{p}\AgdaSpace{}%
\AgdaGeneralizable{δ}\AgdaSpace{}%
\AgdaSymbol{→}\AgdaSpace{}%
\AgdaDatatype{svar}\AgdaSpace{}%
\AgdaSymbol{(}\AgdaGeneralizable{p}\AgdaSpace{}%
\AgdaOperator{\AgdaInductiveConstructor{∙}}\AgdaSpace{}%
\AgdaGeneralizable{q}\AgdaSymbol{)}\AgdaSpace{}%
\AgdaGeneralizable{δ}\<%
\\
%
\>[2]\AgdaOperator{\AgdaInductiveConstructor{∙\AgdaUnderscore{}}}%
\>[8]\AgdaSymbol{:}\AgdaSpace{}%
\AgdaDatatype{svar}\AgdaSpace{}%
\AgdaGeneralizable{q}\AgdaSpace{}%
\AgdaGeneralizable{δ}\AgdaSpace{}%
\AgdaSymbol{→}\AgdaSpace{}%
\AgdaDatatype{svar}\AgdaSpace{}%
\AgdaSymbol{(}\AgdaGeneralizable{p}\AgdaSpace{}%
\AgdaOperator{\AgdaInductiveConstructor{∙}}\AgdaSpace{}%
\AgdaGeneralizable{q}\AgdaSymbol{)}\AgdaSpace{}%
\AgdaGeneralizable{δ}\<%
\\
%
\>[2]\AgdaInductiveConstructor{bind}%
\>[8]\AgdaSymbol{:}\AgdaSpace{}%
\AgdaDatatype{svar}\AgdaSpace{}%
\AgdaGeneralizable{q}\AgdaSpace{}%
\AgdaGeneralizable{δ}\AgdaSpace{}%
\AgdaSymbol{→}\AgdaSpace{}%
\AgdaDatatype{svar}\AgdaSpace{}%
\AgdaSymbol{(}\AgdaInductiveConstructor{bind}\AgdaSpace{}%
\AgdaGeneralizable{q}\AgdaSymbol{)}\AgdaSpace{}%
\AgdaGeneralizable{δ}\<%
\\
%
\\[\AgdaEmptyExtraSkip]%
%
\\[\AgdaEmptyExtraSkip]%
\>[0]\AgdaOperator{\AgdaFunction{\AgdaUnderscore{}‼\AgdaUnderscore{}}}\AgdaSpace{}%
\AgdaSymbol{:}\AgdaSpace{}%
\AgdaDatatype{svar}\AgdaSpace{}%
\AgdaGeneralizable{p}\AgdaSpace{}%
\AgdaGeneralizable{δ}\AgdaSpace{}%
\AgdaSymbol{→}\AgdaSpace{}%
\AgdaSymbol{(}\AgdaGeneralizable{γ'}\AgdaSpace{}%
\AgdaOperator{\AgdaFunction{⊗}}\AgdaSpace{}%
\AgdaGeneralizable{p}\AgdaSymbol{)}\AgdaSpace{}%
\AgdaOperator{\AgdaDatatype{-Env}}\AgdaSpace{}%
\AgdaSymbol{→}\AgdaSpace{}%
\AgdaFunction{Term}\AgdaSpace{}%
\AgdaInductiveConstructor{const}\AgdaSpace{}%
\AgdaSymbol{(}\AgdaGeneralizable{γ'}\AgdaSpace{}%
\AgdaOperator{\AgdaPrimitive{+}}\AgdaSpace{}%
\AgdaGeneralizable{δ}\AgdaSymbol{)}\<%
\\
\>[0]\AgdaInductiveConstructor{⋆}%
\>[8]\AgdaOperator{\AgdaFunction{‼}}\AgdaSpace{}%
\AgdaInductiveConstructor{thing}\AgdaSpace{}%
\AgdaBound{x}%
\>[19]\AgdaSymbol{=}\AgdaSpace{}%
\AgdaBound{x}\<%
\\
\>[0]\AgdaSymbol{(}\AgdaBound{v}\AgdaSpace{}%
\AgdaOperator{\AgdaInductiveConstructor{∙}}\AgdaSymbol{)}%
\>[8]\AgdaOperator{\AgdaFunction{‼}}\AgdaSpace{}%
\AgdaSymbol{(}\AgdaBound{p}\AgdaSpace{}%
\AgdaOperator{\AgdaInductiveConstructor{∙}}\AgdaSpace{}%
\AgdaBound{q}\AgdaSymbol{)}%
\>[19]\AgdaSymbol{=}\AgdaSpace{}%
\AgdaBound{v}\AgdaSpace{}%
\AgdaOperator{\AgdaFunction{‼}}\AgdaSpace{}%
\AgdaBound{p}\<%
\\
\>[0]\AgdaSymbol{(}\AgdaOperator{\AgdaInductiveConstructor{∙}}\AgdaSpace{}%
\AgdaBound{v}\AgdaSymbol{)}%
\>[8]\AgdaOperator{\AgdaFunction{‼}}\AgdaSpace{}%
\AgdaSymbol{(}\AgdaBound{p}\AgdaSpace{}%
\AgdaOperator{\AgdaInductiveConstructor{∙}}\AgdaSpace{}%
\AgdaBound{q}\AgdaSymbol{)}%
\>[19]\AgdaSymbol{=}\AgdaSpace{}%
\AgdaBound{v}\AgdaSpace{}%
\AgdaOperator{\AgdaFunction{‼}}\AgdaSpace{}%
\AgdaBound{q}\<%
\\
\>[0]\AgdaInductiveConstructor{bind}\AgdaSpace{}%
\AgdaBound{v}%
\>[8]\AgdaOperator{\AgdaFunction{‼}}\AgdaSpace{}%
\AgdaInductiveConstructor{bind}\AgdaSpace{}%
\AgdaBound{t}%
\>[19]\AgdaSymbol{=}\AgdaSpace{}%
\AgdaBound{v}\AgdaSpace{}%
\AgdaOperator{\AgdaFunction{‼}}\AgdaSpace{}%
\AgdaBound{t}\<%
\end{code}

We define a few less interesting but critical utility functions for later
use. We give a means to remove a place from a pattern, replacing it with
a trivial atom. Similarly we extend the same functionality to environments.

We also define some openings and a method for retrieving a term from a pattern
and some opening of its environment. We are unable to use our previously defined
Openable for schematic variables as the type is a little difference due to it
having a pattern index which also needs to be opened in the return type.

\begin{code}%
\>[0]\AgdaOperator{\AgdaFunction{\AgdaUnderscore{}-\AgdaUnderscore{}}}%
\>[10]\AgdaSymbol{:}\AgdaSpace{}%
\AgdaSymbol{(}\AgdaBound{p}\AgdaSpace{}%
\AgdaSymbol{:}\AgdaSpace{}%
\AgdaDatatype{Pattern}\AgdaSpace{}%
\AgdaGeneralizable{γ}\AgdaSymbol{)}\AgdaSpace{}%
\AgdaSymbol{→}\AgdaSpace{}%
\AgdaDatatype{svar}\AgdaSpace{}%
\AgdaBound{p}\AgdaSpace{}%
\AgdaGeneralizable{δ}\AgdaSpace{}%
\AgdaSymbol{→}\AgdaSpace{}%
\AgdaDatatype{Pattern}\AgdaSpace{}%
\AgdaGeneralizable{γ}\<%
\\
\>[0]\AgdaOperator{\AgdaFunction{\AgdaUnderscore{}-penv\AgdaUnderscore{}}}%
\>[10]\AgdaSymbol{:}\AgdaSpace{}%
\AgdaGeneralizable{p}\AgdaSpace{}%
\AgdaOperator{\AgdaDatatype{-Env}}\AgdaSpace{}%
\AgdaSymbol{→}\AgdaSpace{}%
\AgdaSymbol{(}\AgdaBound{ξ}\AgdaSpace{}%
\AgdaSymbol{:}\AgdaSpace{}%
\AgdaDatatype{svar}\AgdaSpace{}%
\AgdaGeneralizable{p}\AgdaSpace{}%
\AgdaGeneralizable{δ}\AgdaSymbol{)}\AgdaSpace{}%
\AgdaSymbol{→}\AgdaSpace{}%
\AgdaSymbol{(}\AgdaGeneralizable{p}\AgdaSpace{}%
\AgdaOperator{\AgdaFunction{-}}\AgdaSpace{}%
\AgdaBound{ξ}\AgdaSymbol{)}\AgdaSpace{}%
\AgdaOperator{\AgdaDatatype{-Env}}\<%
\\
\>[0]\AgdaOperator{\AgdaFunction{\AgdaUnderscore{}⊗svar\AgdaUnderscore{}}}%
\>[10]\AgdaSymbol{:}\AgdaSpace{}%
\AgdaSymbol{(}\AgdaBound{γ}\AgdaSpace{}%
\AgdaSymbol{:}\AgdaSpace{}%
\AgdaFunction{Scope}\AgdaSymbol{)}\AgdaSpace{}%
\AgdaSymbol{→}\AgdaSpace{}%
\AgdaDatatype{svar}\AgdaSpace{}%
\AgdaGeneralizable{p}\AgdaSpace{}%
\AgdaGeneralizable{δ}\AgdaSpace{}%
\AgdaSymbol{→}\AgdaSpace{}%
\AgdaDatatype{svar}\AgdaSpace{}%
\AgdaSymbol{(}\AgdaBound{γ}\AgdaSpace{}%
\AgdaOperator{\AgdaFunction{⊗}}\AgdaSpace{}%
\AgdaGeneralizable{p}\AgdaSymbol{)}\AgdaSpace{}%
\AgdaSymbol{(}\AgdaBound{γ}\AgdaSpace{}%
\AgdaOperator{\AgdaPrimitive{+}}\AgdaSpace{}%
\AgdaGeneralizable{δ}\AgdaSymbol{)}\<%
\\
\>[0]\AgdaOperator{\AgdaFunction{\AgdaUnderscore{}⊗var\AgdaUnderscore{}}}%
\>[10]\AgdaSymbol{:}\AgdaSpace{}%
\AgdaFunction{Openable}\AgdaSpace{}%
\AgdaDatatype{Var}\<%
\\
\>[0]\AgdaFunction{termFrom}%
\>[10]\AgdaSymbol{:}\AgdaSpace{}%
\AgdaSymbol{(}\AgdaBound{p}\AgdaSpace{}%
\AgdaSymbol{:}\AgdaSpace{}%
\AgdaDatatype{Pattern}\AgdaSpace{}%
\AgdaGeneralizable{γ}\AgdaSymbol{)}\AgdaSpace{}%
\AgdaSymbol{→}\AgdaSpace{}%
\AgdaSymbol{(}\AgdaGeneralizable{δ}\AgdaSpace{}%
\AgdaOperator{\AgdaFunction{⊗}}\AgdaSpace{}%
\AgdaBound{p}\AgdaSymbol{)}\AgdaSpace{}%
\AgdaOperator{\AgdaDatatype{-Env}}\AgdaSpace{}%
\AgdaSymbol{→}\AgdaSpace{}%
\AgdaFunction{Term}\AgdaSpace{}%
\AgdaInductiveConstructor{const}\AgdaSpace{}%
\AgdaSymbol{(}\AgdaGeneralizable{δ}\AgdaSpace{}%
\AgdaOperator{\AgdaPrimitive{+}}\AgdaSpace{}%
\AgdaGeneralizable{γ}\AgdaSymbol{)}\<%
\end{code}
\hide{
\begin{code}%
\>[0]\AgdaSymbol{(}\AgdaBound{p}\AgdaSpace{}%
\AgdaOperator{\AgdaInductiveConstructor{∙}}\AgdaSpace{}%
\AgdaBound{q}\AgdaSymbol{)}\AgdaSpace{}%
\AgdaOperator{\AgdaFunction{-}}\AgdaSpace{}%
\AgdaSymbol{(}\AgdaBound{ξ}\AgdaSpace{}%
\AgdaOperator{\AgdaInductiveConstructor{∙}}\AgdaSymbol{)}%
\>[17]\AgdaSymbol{=}\AgdaSpace{}%
\AgdaSymbol{(}\AgdaBound{p}\AgdaSpace{}%
\AgdaOperator{\AgdaFunction{-}}\AgdaSpace{}%
\AgdaBound{ξ}\AgdaSymbol{)}\AgdaSpace{}%
\AgdaOperator{\AgdaInductiveConstructor{∙}}\AgdaSpace{}%
\AgdaBound{q}\<%
\\
\>[0]\AgdaSymbol{(}\AgdaBound{p}\AgdaSpace{}%
\AgdaOperator{\AgdaInductiveConstructor{∙}}\AgdaSpace{}%
\AgdaBound{q}\AgdaSymbol{)}\AgdaSpace{}%
\AgdaOperator{\AgdaFunction{-}}\AgdaSpace{}%
\AgdaSymbol{(}\AgdaOperator{\AgdaInductiveConstructor{∙}}\AgdaSpace{}%
\AgdaBound{ξ}\AgdaSymbol{)}%
\>[17]\AgdaSymbol{=}\AgdaSpace{}%
\AgdaBound{p}\AgdaSpace{}%
\AgdaOperator{\AgdaInductiveConstructor{∙}}\AgdaSpace{}%
\AgdaSymbol{(}\AgdaBound{q}\AgdaSpace{}%
\AgdaOperator{\AgdaFunction{-}}\AgdaSpace{}%
\AgdaBound{ξ}\AgdaSymbol{)}\<%
\\
\>[0]\AgdaInductiveConstructor{bind}\AgdaSpace{}%
\AgdaBound{p}%
\>[8]\AgdaOperator{\AgdaFunction{-}}\AgdaSpace{}%
\AgdaInductiveConstructor{bind}\AgdaSpace{}%
\AgdaBound{ξ}\AgdaSpace{}%
\AgdaSymbol{=}\AgdaSpace{}%
\AgdaInductiveConstructor{bind}\AgdaSpace{}%
\AgdaSymbol{(}\AgdaBound{p}\AgdaSpace{}%
\AgdaOperator{\AgdaFunction{-}}\AgdaSpace{}%
\AgdaBound{ξ}\AgdaSymbol{)}\<%
\\
\>[0]\AgdaInductiveConstructor{place}\AgdaSpace{}%
\AgdaBound{x}\AgdaSpace{}%
\AgdaOperator{\AgdaFunction{-}}\AgdaSpace{}%
\AgdaInductiveConstructor{⋆}%
\>[17]\AgdaSymbol{=}\AgdaSpace{}%
\AgdaInductiveConstructor{`}\AgdaSpace{}%
\AgdaString{'⊤'}\<%
\\
%
\\[\AgdaEmptyExtraSkip]%
\>[0]\AgdaSymbol{(}\AgdaBound{s}\AgdaSpace{}%
\AgdaOperator{\AgdaInductiveConstructor{∙}}\AgdaSpace{}%
\AgdaBound{t}\AgdaSymbol{)}\AgdaSpace{}%
\AgdaOperator{\AgdaFunction{-penv}}\AgdaSpace{}%
\AgdaSymbol{(}\AgdaBound{ξ}\AgdaSpace{}%
\AgdaOperator{\AgdaInductiveConstructor{∙}}\AgdaSymbol{)}\AgdaSpace{}%
\AgdaSymbol{=}\AgdaSpace{}%
\AgdaSymbol{(}\AgdaBound{s}\AgdaSpace{}%
\AgdaOperator{\AgdaFunction{-penv}}\AgdaSpace{}%
\AgdaBound{ξ}\AgdaSymbol{)}\AgdaSpace{}%
\AgdaOperator{\AgdaInductiveConstructor{∙}}\AgdaSpace{}%
\AgdaBound{t}\<%
\\
\>[0]\AgdaSymbol{(}\AgdaBound{s}\AgdaSpace{}%
\AgdaOperator{\AgdaInductiveConstructor{∙}}\AgdaSpace{}%
\AgdaBound{t}\AgdaSymbol{)}\AgdaSpace{}%
\AgdaOperator{\AgdaFunction{-penv}}\AgdaSpace{}%
\AgdaSymbol{(}\AgdaOperator{\AgdaInductiveConstructor{∙}}\AgdaSpace{}%
\AgdaBound{ξ}\AgdaSymbol{)}\AgdaSpace{}%
\AgdaSymbol{=}\AgdaSpace{}%
\AgdaBound{s}\AgdaSpace{}%
\AgdaOperator{\AgdaInductiveConstructor{∙}}\AgdaSpace{}%
\AgdaSymbol{(}\AgdaBound{t}\AgdaSpace{}%
\AgdaOperator{\AgdaFunction{-penv}}\AgdaSpace{}%
\AgdaBound{ξ}\AgdaSymbol{)}\<%
\\
\>[0]\AgdaInductiveConstructor{bind}\AgdaSpace{}%
\AgdaBound{e}\AgdaSpace{}%
\AgdaOperator{\AgdaFunction{-penv}}\AgdaSpace{}%
\AgdaInductiveConstructor{bind}\AgdaSpace{}%
\AgdaBound{ξ}\AgdaSpace{}%
\AgdaSymbol{=}\AgdaSpace{}%
\AgdaInductiveConstructor{bind}\AgdaSpace{}%
\AgdaSymbol{(}\AgdaBound{e}\AgdaSpace{}%
\AgdaOperator{\AgdaFunction{-penv}}\AgdaSpace{}%
\AgdaBound{ξ}\AgdaSymbol{)}\<%
\\
\>[0]\AgdaInductiveConstructor{thing}\AgdaSpace{}%
\AgdaBound{x}\AgdaSpace{}%
\AgdaOperator{\AgdaFunction{-penv}}\AgdaSpace{}%
\AgdaInductiveConstructor{⋆}%
\>[20]\AgdaSymbol{=}\AgdaSpace{}%
\AgdaInductiveConstructor{`}\<%
\\
%
\\[\AgdaEmptyExtraSkip]%
\>[0]\AgdaBound{γ}\AgdaSpace{}%
\AgdaOperator{\AgdaFunction{⊗svar}}\AgdaSpace{}%
\AgdaInductiveConstructor{⋆}%
\>[15]\AgdaSymbol{=}\AgdaSpace{}%
\AgdaInductiveConstructor{⋆}\<%
\\
\>[0]\AgdaBound{γ}\AgdaSpace{}%
\AgdaOperator{\AgdaFunction{⊗svar}}\AgdaSpace{}%
\AgdaSymbol{(}\AgdaBound{v}\AgdaSpace{}%
\AgdaOperator{\AgdaInductiveConstructor{∙}}\AgdaSymbol{)}%
\>[15]\AgdaSymbol{=}\AgdaSpace{}%
\AgdaSymbol{(}\AgdaBound{γ}\AgdaSpace{}%
\AgdaOperator{\AgdaFunction{⊗svar}}\AgdaSpace{}%
\AgdaBound{v}\AgdaSymbol{)}\AgdaSpace{}%
\AgdaOperator{\AgdaInductiveConstructor{∙}}\<%
\\
\>[0]\AgdaBound{γ}\AgdaSpace{}%
\AgdaOperator{\AgdaFunction{⊗svar}}\AgdaSpace{}%
\AgdaSymbol{(}\AgdaOperator{\AgdaInductiveConstructor{∙}}\AgdaSpace{}%
\AgdaBound{v}\AgdaSymbol{)}%
\>[15]\AgdaSymbol{=}\AgdaSpace{}%
\AgdaOperator{\AgdaInductiveConstructor{∙}}\AgdaSpace{}%
\AgdaSymbol{(}\AgdaBound{γ}\AgdaSpace{}%
\AgdaOperator{\AgdaFunction{⊗svar}}\AgdaSpace{}%
\AgdaBound{v}\AgdaSymbol{)}\<%
\\
\>[0]\AgdaBound{γ}\AgdaSpace{}%
\AgdaOperator{\AgdaFunction{⊗svar}}\AgdaSpace{}%
\AgdaInductiveConstructor{bind}\AgdaSpace{}%
\AgdaBound{v}\AgdaSpace{}%
\AgdaSymbol{=}\AgdaSpace{}%
\AgdaInductiveConstructor{bind}\AgdaSpace{}%
\AgdaSymbol{(}\AgdaBound{γ}\AgdaSpace{}%
\AgdaOperator{\AgdaFunction{⊗svar}}\AgdaSpace{}%
\AgdaBound{v}\AgdaSymbol{)}\<%
\\
%
\\[\AgdaEmptyExtraSkip]%
\>[0]\AgdaBound{γ}\AgdaSpace{}%
\AgdaOperator{\AgdaFunction{⊗var}}\AgdaSpace{}%
\AgdaInductiveConstructor{ze}\AgdaSpace{}%
\AgdaSymbol{=}\AgdaSpace{}%
\AgdaInductiveConstructor{ze}\<%
\\
\>[0]\AgdaBound{γ}\AgdaSpace{}%
\AgdaOperator{\AgdaFunction{⊗var}}\AgdaSpace{}%
\AgdaInductiveConstructor{su}\AgdaSpace{}%
\AgdaBound{v}\AgdaSpace{}%
\AgdaSymbol{=}\AgdaSpace{}%
\AgdaInductiveConstructor{su}\AgdaSpace{}%
\AgdaSymbol{(}\AgdaBound{γ}\AgdaSpace{}%
\AgdaOperator{\AgdaFunction{⊗var}}\AgdaSpace{}%
\AgdaBound{v}\AgdaSymbol{)}\<%
\\
%
\\[\AgdaEmptyExtraSkip]%
\>[0]\AgdaFunction{termFrom}\AgdaSpace{}%
\AgdaSymbol{(}\AgdaInductiveConstructor{`}\AgdaSpace{}%
\AgdaBound{x}\AgdaSymbol{)}\AgdaSpace{}%
\AgdaInductiveConstructor{`}%
\>[30]\AgdaSymbol{=}\AgdaSpace{}%
\AgdaInductiveConstructor{`}\AgdaSpace{}%
\AgdaBound{x}\<%
\\
\>[0]\AgdaFunction{termFrom}\AgdaSpace{}%
\AgdaSymbol{(}\AgdaBound{p}\AgdaSpace{}%
\AgdaOperator{\AgdaInductiveConstructor{∙}}\AgdaSpace{}%
\AgdaBound{p₁}\AgdaSymbol{)}\AgdaSpace{}%
\AgdaSymbol{(}\AgdaBound{e}\AgdaSpace{}%
\AgdaOperator{\AgdaInductiveConstructor{∙}}\AgdaSpace{}%
\AgdaBound{e₁}\AgdaSymbol{)}%
\>[30]\AgdaSymbol{=}\AgdaSpace{}%
\AgdaFunction{termFrom}\AgdaSpace{}%
\AgdaBound{p}\AgdaSpace{}%
\AgdaBound{e}\AgdaSpace{}%
\AgdaOperator{\AgdaInductiveConstructor{∙}}\AgdaSpace{}%
\AgdaFunction{termFrom}\AgdaSpace{}%
\AgdaBound{p₁}\AgdaSpace{}%
\AgdaBound{e₁}\<%
\\
\>[0]\AgdaFunction{termFrom}\AgdaSpace{}%
\AgdaSymbol{(}\AgdaInductiveConstructor{bind}\AgdaSpace{}%
\AgdaBound{p}\AgdaSymbol{)}\AgdaSpace{}%
\AgdaSymbol{(}\AgdaInductiveConstructor{bind}\AgdaSpace{}%
\AgdaBound{e}\AgdaSymbol{)}%
\>[30]\AgdaSymbol{=}\AgdaSpace{}%
\AgdaInductiveConstructor{bind}\AgdaSpace{}%
\AgdaSymbol{(}\AgdaFunction{termFrom}\AgdaSpace{}%
\AgdaBound{p}\AgdaSpace{}%
\AgdaBound{e}\AgdaSymbol{)}\<%
\\
\>[0]\AgdaFunction{termFrom}\AgdaSpace{}%
\AgdaSymbol{(}\AgdaInductiveConstructor{place}\AgdaSpace{}%
\AgdaBound{θ}\AgdaSymbol{)}\AgdaSpace{}%
\AgdaSymbol{(}\AgdaInductiveConstructor{thing}\AgdaSpace{}%
\AgdaBound{x₁}\AgdaSymbol{)}\AgdaSpace{}%
\AgdaSymbol{=}\AgdaSpace{}%
\AgdaBound{x₁}\AgdaSpace{}%
\AgdaOperator{\AgdaFunction{⟨term⊗}}\AgdaSpace{}%
\AgdaBound{θ}\<%
\end{code}
}

\section{Expressions}

\hide{
\begin{code}%
\>[0]\AgdaSymbol{\{-\#}\AgdaSpace{}%
\AgdaKeyword{OPTIONS}\AgdaSpace{}%
\AgdaPragma{--rewriting}\AgdaSpace{}%
\AgdaSymbol{\#-\}}\<%
\\
\>[0]\AgdaKeyword{module}\AgdaSpace{}%
\AgdaModule{Expression}\AgdaSpace{}%
\AgdaKeyword{where}\<%
\end{code}
}

\hide{
\begin{code}%
\>[0]\AgdaKeyword{open}\AgdaSpace{}%
\AgdaKeyword{import}\AgdaSpace{}%
\AgdaModule{CoreLanguage}\<%
\\
\>[0]\AgdaKeyword{open}\AgdaSpace{}%
\AgdaKeyword{import}\AgdaSpace{}%
\AgdaModule{Pattern}\AgdaSpace{}%
\AgdaKeyword{using}\<%
\\
\>[0][@{}l@{\AgdaIndent{0}}]%
\>[2]\AgdaSymbol{(}\AgdaDatatype{Pattern}\AgdaSymbol{;}\AgdaSpace{}%
\AgdaOperator{\AgdaDatatype{\AgdaUnderscore{}-Env}}\AgdaSymbol{;}\AgdaSpace{}%
\AgdaDatatype{svar}\AgdaSymbol{;}\AgdaSpace{}%
\AgdaOperator{\AgdaFunction{\AgdaUnderscore{}⊗\AgdaUnderscore{}}}\AgdaSymbol{;}\AgdaSpace{}%
\AgdaOperator{\AgdaFunction{\AgdaUnderscore{}⊗var\AgdaUnderscore{}}}\AgdaSymbol{;}\AgdaSpace{}%
\AgdaOperator{\AgdaFunction{\AgdaUnderscore{}⊗svar\AgdaUnderscore{}}}\AgdaSymbol{;}\AgdaSpace{}%
\AgdaOperator{\AgdaFunction{\AgdaUnderscore{}‼\AgdaUnderscore{}}}\AgdaSymbol{)}\<%
\\
\>[0]\AgdaKeyword{open}\AgdaSpace{}%
\AgdaKeyword{import}\AgdaSpace{}%
\AgdaModule{Thinning}\AgdaSpace{}%
\AgdaKeyword{hiding}\AgdaSpace{}%
\AgdaSymbol{(}\AgdaOperator{\AgdaFunction{\AgdaUnderscore{}++\AgdaUnderscore{}}}\AgdaSymbol{)}\<%
\\
\>[0]\AgdaKeyword{open}\AgdaSpace{}%
\AgdaKeyword{import}\AgdaSpace{}%
\AgdaModule{Substitution}\<%
\\
\>[0]\AgdaKeyword{open}\AgdaSpace{}%
\AgdaKeyword{import}\AgdaSpace{}%
\AgdaModule{TermSubstitution}\<%
\\
\>[0]\AgdaKeyword{open}\AgdaSpace{}%
\AgdaKeyword{import}\AgdaSpace{}%
\AgdaModule{Composition}\<%
\\
\>[0]\AgdaKeyword{open}\AgdaSpace{}%
\AgdaKeyword{import}\AgdaSpace{}%
\AgdaModule{Data.String}\AgdaSpace{}%
\AgdaKeyword{using}\AgdaSpace{}%
\AgdaSymbol{(}\AgdaPostulate{String}\AgdaSymbol{)}\<%
\\
\>[0]\AgdaKeyword{open}\AgdaSpace{}%
\AgdaKeyword{import}\AgdaSpace{}%
\AgdaModule{Data.Nat}\AgdaSpace{}%
\AgdaKeyword{using}\AgdaSpace{}%
\AgdaSymbol{(}\AgdaInductiveConstructor{zero}\AgdaSymbol{;}\AgdaSpace{}%
\AgdaInductiveConstructor{suc}\AgdaSymbol{;}\AgdaSpace{}%
\AgdaOperator{\AgdaPrimitive{\AgdaUnderscore{}+\AgdaUnderscore{}}}\AgdaSymbol{)}\<%
\\
\>[0]\AgdaKeyword{open}\AgdaSpace{}%
\AgdaKeyword{import}\AgdaSpace{}%
\AgdaModule{BwdVec}\AgdaSpace{}%
\AgdaKeyword{hiding}\AgdaSpace{}%
\AgdaSymbol{(}\AgdaFunction{map}\AgdaSymbol{)}\<%
\end{code}
}

\hide{
\begin{code}%
\>[0]\AgdaKeyword{private}\<%
\\
\>[0][@{}l@{\AgdaIndent{0}}]%
\>[2]\AgdaKeyword{variable}\<%
\\
\>[2][@{}l@{\AgdaIndent{0}}]%
\>[4]\AgdaGeneralizable{δ}\AgdaSpace{}%
\AgdaSymbol{:}\AgdaSpace{}%
\AgdaFunction{Scope}\<%
\\
%
\>[4]\AgdaGeneralizable{δ'}\AgdaSpace{}%
\AgdaSymbol{:}\AgdaSpace{}%
\AgdaFunction{Scope}\<%
\\
%
\>[4]\AgdaGeneralizable{γ}\AgdaSpace{}%
\AgdaSymbol{:}\AgdaSpace{}%
\AgdaFunction{Scope}\<%
\\
%
\>[4]\AgdaGeneralizable{p}\AgdaSpace{}%
\AgdaSymbol{:}\AgdaSpace{}%
\AgdaDatatype{Pattern}\AgdaSpace{}%
\AgdaGeneralizable{γ}\<%
\\
%
\>[4]\AgdaGeneralizable{q}\AgdaSpace{}%
\AgdaSymbol{:}\AgdaSpace{}%
\AgdaDatatype{Pattern}\AgdaSpace{}%
\AgdaGeneralizable{γ}\<%
\\
%
\>[4]\AgdaGeneralizable{γ'}\AgdaSpace{}%
\AgdaSymbol{:}\AgdaSpace{}%
\AgdaFunction{Scope}\<%
\\
%
\>[4]\AgdaGeneralizable{d}\AgdaSpace{}%
\AgdaSymbol{:}\AgdaSpace{}%
\AgdaDatatype{Dir}\<%
\\
%
\>[4]\AgdaGeneralizable{d'}\AgdaSpace{}%
\AgdaSymbol{:}\AgdaSpace{}%
\AgdaDatatype{Dir}\<%
\end{code}
}

Expressions allow us to define a way that we might construct a term from a
pattern that represents something that we trust. For example, if the expression
is showing how one might build the required term as the output of some
type-synthesis rule for elimination, what is trusted might be the type of the
target and whatever we learn to trust in the premises. These trusted patterns
contain places that sculpt out component parts in a piece of matched syntax
and allow us to use these component parts in constructing a new term.

It mirrors the structure of our terms except that it includes the option
to reference some place in a pattern and instantiate it with some substitution
of its free  variables.

\begin{code}%
\>[0]\AgdaFunction{Expr}\AgdaSpace{}%
\AgdaSymbol{:}\AgdaSpace{}%
\AgdaDatatype{Pattern}\AgdaSpace{}%
\AgdaGeneralizable{δ}\AgdaSpace{}%
\AgdaSymbol{→}\AgdaSpace{}%
\AgdaDatatype{Dir}\AgdaSpace{}%
\AgdaSymbol{→}\AgdaSpace{}%
\AgdaFunction{Scoped}\<%
\\
%
\\[\AgdaEmptyExtraSkip]%
\>[0]\AgdaKeyword{data}\AgdaSpace{}%
\AgdaDatatype{Con}\AgdaSpace{}%
\AgdaSymbol{(}\AgdaBound{p}\AgdaSpace{}%
\AgdaSymbol{:}\AgdaSpace{}%
\AgdaDatatype{Pattern}\AgdaSpace{}%
\AgdaGeneralizable{δ}\AgdaSymbol{)}\AgdaSpace{}%
\AgdaSymbol{(}\AgdaBound{γ}\AgdaSpace{}%
\AgdaSymbol{:}\AgdaSpace{}%
\AgdaFunction{Scope}\AgdaSymbol{)}\AgdaSpace{}%
\AgdaSymbol{:}\AgdaSpace{}%
\AgdaPrimitiveType{Set}\<%
\\
\>[0]\AgdaKeyword{data}\AgdaSpace{}%
\AgdaDatatype{Com}\AgdaSpace{}%
\AgdaSymbol{(}\AgdaBound{p}\AgdaSpace{}%
\AgdaSymbol{:}\AgdaSpace{}%
\AgdaDatatype{Pattern}\AgdaSpace{}%
\AgdaGeneralizable{δ}\AgdaSymbol{)}\AgdaSpace{}%
\AgdaSymbol{(}\AgdaBound{γ}\AgdaSpace{}%
\AgdaSymbol{:}\AgdaSpace{}%
\AgdaFunction{Scope}\AgdaSymbol{)}\AgdaSpace{}%
\AgdaSymbol{:}\AgdaSpace{}%
\AgdaPrimitiveType{Set}\<%
\\
%
\\[\AgdaEmptyExtraSkip]%
\>[0]\AgdaKeyword{data}\AgdaSpace{}%
\AgdaDatatype{Con}\AgdaSpace{}%
\AgdaBound{p}\AgdaSpace{}%
\AgdaBound{γ}\AgdaSpace{}%
\AgdaKeyword{where}\<%
\\
\>[0][@{}l@{\AgdaIndent{0}}]%
\>[2]\AgdaInductiveConstructor{`}%
\>[9]\AgdaSymbol{:}\AgdaSpace{}%
\AgdaPostulate{String}\AgdaSpace{}%
\AgdaSymbol{→}\AgdaSpace{}%
\AgdaDatatype{Con}\AgdaSpace{}%
\AgdaBound{p}\AgdaSpace{}%
\AgdaBound{γ}\<%
\\
%
\>[2]\AgdaOperator{\AgdaInductiveConstructor{\AgdaUnderscore{}∙\AgdaUnderscore{}}}%
\>[9]\AgdaSymbol{:}\AgdaSpace{}%
\AgdaDatatype{Con}\AgdaSpace{}%
\AgdaBound{p}\AgdaSpace{}%
\AgdaBound{γ}\AgdaSpace{}%
\AgdaSymbol{→}\AgdaSpace{}%
\AgdaDatatype{Con}\AgdaSpace{}%
\AgdaBound{p}\AgdaSpace{}%
\AgdaBound{γ}\AgdaSpace{}%
\AgdaSymbol{→}\AgdaSpace{}%
\AgdaDatatype{Con}\AgdaSpace{}%
\AgdaBound{p}\AgdaSpace{}%
\AgdaBound{γ}\<%
\\
%
\>[2]\AgdaInductiveConstructor{bind}%
\>[9]\AgdaSymbol{:}\AgdaSpace{}%
\AgdaDatatype{Con}\AgdaSpace{}%
\AgdaBound{p}\AgdaSpace{}%
\AgdaSymbol{(}\AgdaInductiveConstructor{suc}\AgdaSpace{}%
\AgdaBound{γ}\AgdaSymbol{)}\AgdaSpace{}%
\AgdaSymbol{→}\AgdaSpace{}%
\AgdaDatatype{Con}\AgdaSpace{}%
\AgdaBound{p}\AgdaSpace{}%
\AgdaBound{γ}\<%
\\
%
\>[2]\AgdaInductiveConstructor{thunk}%
\>[9]\AgdaSymbol{:}\AgdaSpace{}%
\AgdaDatatype{Com}\AgdaSpace{}%
\AgdaBound{p}\AgdaSpace{}%
\AgdaBound{γ}\AgdaSpace{}%
\AgdaSymbol{→}\AgdaSpace{}%
\AgdaDatatype{Con}\AgdaSpace{}%
\AgdaBound{p}\AgdaSpace{}%
\AgdaBound{γ}\<%
\\
%
\>[2]\AgdaOperator{\AgdaInductiveConstructor{\AgdaUnderscore{}/\AgdaUnderscore{}}}%
\>[9]\AgdaSymbol{:}\AgdaSpace{}%
\AgdaDatatype{svar}\AgdaSpace{}%
\AgdaBound{p}\AgdaSpace{}%
\AgdaGeneralizable{γ'}\AgdaSpace{}%
\AgdaSymbol{→}\AgdaSpace{}%
\AgdaGeneralizable{γ'}\AgdaSpace{}%
\AgdaOperator{\AgdaFunction{⇒[}}\AgdaSpace{}%
\AgdaDatatype{Com}\AgdaSpace{}%
\AgdaBound{p}\AgdaSpace{}%
\AgdaOperator{\AgdaFunction{]}}\AgdaSpace{}%
\AgdaBound{γ}\AgdaSpace{}%
\AgdaSymbol{→}\AgdaSpace{}%
\AgdaDatatype{Con}\AgdaSpace{}%
\AgdaBound{p}\AgdaSpace{}%
\AgdaBound{γ}\<%
\\
\>[0]\<%
\\
\>[0]\AgdaKeyword{data}\AgdaSpace{}%
\AgdaDatatype{Com}\AgdaSpace{}%
\AgdaBound{p}\AgdaSpace{}%
\AgdaBound{γ}\AgdaSpace{}%
\AgdaKeyword{where}\<%
\\
\>[0][@{}l@{\AgdaIndent{0}}]%
\>[2]\AgdaInductiveConstructor{var}%
\>[9]\AgdaSymbol{:}\AgdaSpace{}%
\AgdaDatatype{Var}\AgdaSpace{}%
\AgdaBound{γ}\AgdaSpace{}%
\AgdaSymbol{→}\AgdaSpace{}%
\AgdaDatatype{Com}\AgdaSpace{}%
\AgdaBound{p}\AgdaSpace{}%
\AgdaBound{γ}\<%
\\
%
\>[2]\AgdaInductiveConstructor{elim}%
\>[9]\AgdaSymbol{:}\AgdaSpace{}%
\AgdaDatatype{Com}\AgdaSpace{}%
\AgdaBound{p}\AgdaSpace{}%
\AgdaBound{γ}\AgdaSpace{}%
\AgdaSymbol{→}\AgdaSpace{}%
\AgdaDatatype{Con}\AgdaSpace{}%
\AgdaBound{p}\AgdaSpace{}%
\AgdaBound{γ}\AgdaSpace{}%
\AgdaSymbol{→}\AgdaSpace{}%
\AgdaDatatype{Com}\AgdaSpace{}%
\AgdaBound{p}\AgdaSpace{}%
\AgdaBound{γ}\<%
\\
%
\>[2]\AgdaOperator{\AgdaInductiveConstructor{\AgdaUnderscore{}∷\AgdaUnderscore{}}}%
\>[9]\AgdaSymbol{:}\AgdaSpace{}%
\AgdaDatatype{Con}\AgdaSpace{}%
\AgdaBound{p}\AgdaSpace{}%
\AgdaBound{γ}\AgdaSpace{}%
\AgdaSymbol{→}\AgdaSpace{}%
\AgdaDatatype{Con}\AgdaSpace{}%
\AgdaBound{p}\AgdaSpace{}%
\AgdaBound{γ}\AgdaSpace{}%
\AgdaSymbol{→}\AgdaSpace{}%
\AgdaDatatype{Com}\AgdaSpace{}%
\AgdaBound{p}\AgdaSpace{}%
\AgdaBound{γ}\<%
\\
%
\\[\AgdaEmptyExtraSkip]%
%
\\[\AgdaEmptyExtraSkip]%
\>[0]\AgdaFunction{Expr}\AgdaSpace{}%
\AgdaBound{p}\AgdaSpace{}%
\AgdaInductiveConstructor{const}\AgdaSpace{}%
\AgdaBound{γ}\AgdaSpace{}%
\AgdaSymbol{=}\AgdaSpace{}%
\AgdaDatatype{Con}\AgdaSpace{}%
\AgdaBound{p}\AgdaSpace{}%
\AgdaBound{γ}\<%
\\
\>[0]\AgdaFunction{Expr}\AgdaSpace{}%
\AgdaBound{p}\AgdaSpace{}%
\AgdaInductiveConstructor{compu}\AgdaSpace{}%
\AgdaBound{γ}\AgdaSpace{}%
\AgdaSymbol{=}\AgdaSpace{}%
\AgdaDatatype{Com}\AgdaSpace{}%
\AgdaBound{p}\AgdaSpace{}%
\AgdaBound{γ}\<%
\end{code}
As is expected by now, we omit the usual smattering of thinnings, weakenings
and openings.
\hide{
\begin{code}%
\>[0]\AgdaKeyword{infixr}\AgdaSpace{}%
\AgdaNumber{20}\AgdaSpace{}%
\AgdaOperator{\AgdaInductiveConstructor{\AgdaUnderscore{}∙\AgdaUnderscore{}}}\<%
\\
\>[0]\AgdaOperator{\AgdaFunction{\AgdaUnderscore{}⟨exp\AgdaUnderscore{}}}%
\>[9]\AgdaSymbol{:}\AgdaSpace{}%
\AgdaFunction{Thinnable}\AgdaSpace{}%
\AgdaSymbol{(}\AgdaFunction{Expr}\AgdaSpace{}%
\AgdaGeneralizable{p}\AgdaSpace{}%
\AgdaGeneralizable{d}\AgdaSymbol{)}\<%
\\
\>[0]\AgdaOperator{\AgdaFunction{\AgdaUnderscore{}⟨exp\AgdaUnderscore{}}}\AgdaSpace{}%
\AgdaSymbol{\{}\AgdaArgument{d}\AgdaSpace{}%
\AgdaSymbol{=}\AgdaSpace{}%
\AgdaInductiveConstructor{const}\AgdaSymbol{\}}\AgdaSpace{}%
\AgdaSymbol{(}\AgdaInductiveConstructor{`}\AgdaSpace{}%
\AgdaBound{x}\AgdaSymbol{)}%
\>[28]\AgdaBound{θ}\AgdaSpace{}%
\AgdaSymbol{=}\AgdaSpace{}%
\AgdaInductiveConstructor{`}\AgdaSpace{}%
\AgdaBound{x}\<%
\\
\>[0]\AgdaOperator{\AgdaFunction{\AgdaUnderscore{}⟨exp\AgdaUnderscore{}}}\AgdaSpace{}%
\AgdaSymbol{\{}\AgdaArgument{d}\AgdaSpace{}%
\AgdaSymbol{=}\AgdaSpace{}%
\AgdaInductiveConstructor{const}\AgdaSymbol{\}}\AgdaSpace{}%
\AgdaSymbol{(}\AgdaBound{s}\AgdaSpace{}%
\AgdaOperator{\AgdaInductiveConstructor{∙}}\AgdaSpace{}%
\AgdaBound{t}\AgdaSymbol{)}%
\>[28]\AgdaBound{θ}\AgdaSpace{}%
\AgdaSymbol{=}\AgdaSpace{}%
\AgdaSymbol{(}\AgdaBound{s}\AgdaSpace{}%
\AgdaOperator{\AgdaFunction{⟨exp}}\AgdaSpace{}%
\AgdaBound{θ}\AgdaSymbol{)}\AgdaSpace{}%
\AgdaOperator{\AgdaInductiveConstructor{∙}}\AgdaSpace{}%
\AgdaSymbol{(}\AgdaBound{t}\AgdaSpace{}%
\AgdaOperator{\AgdaFunction{⟨exp}}\AgdaSpace{}%
\AgdaBound{θ}\AgdaSymbol{)}\<%
\\
\>[0]\AgdaOperator{\AgdaFunction{\AgdaUnderscore{}⟨exp\AgdaUnderscore{}}}\AgdaSpace{}%
\AgdaSymbol{\{}\AgdaArgument{d}\AgdaSpace{}%
\AgdaSymbol{=}\AgdaSpace{}%
\AgdaInductiveConstructor{const}\AgdaSymbol{\}}\AgdaSpace{}%
\AgdaSymbol{(}\AgdaInductiveConstructor{bind}\AgdaSpace{}%
\AgdaBound{t}\AgdaSymbol{)}\AgdaSpace{}%
\AgdaBound{θ}\AgdaSpace{}%
\AgdaSymbol{=}\AgdaSpace{}%
\AgdaInductiveConstructor{bind}\AgdaSpace{}%
\AgdaSymbol{(}\AgdaBound{t}\AgdaSpace{}%
\AgdaOperator{\AgdaFunction{⟨exp}}\AgdaSpace{}%
\AgdaSymbol{(}\AgdaBound{θ}\AgdaSpace{}%
\AgdaOperator{\AgdaInductiveConstructor{I}}\AgdaSymbol{))}\<%
\\
\>[0]\AgdaOperator{\AgdaFunction{\AgdaUnderscore{}⟨exp\AgdaUnderscore{}}}\AgdaSpace{}%
\AgdaSymbol{\{}\AgdaArgument{d}\AgdaSpace{}%
\AgdaSymbol{=}\AgdaSpace{}%
\AgdaInductiveConstructor{const}\AgdaSymbol{\}}\AgdaSpace{}%
\AgdaSymbol{(}\AgdaInductiveConstructor{thunk}\AgdaSpace{}%
\AgdaBound{x}\AgdaSymbol{)}\AgdaSpace{}%
\AgdaBound{θ}\AgdaSpace{}%
\AgdaSymbol{=}\AgdaSpace{}%
\AgdaInductiveConstructor{thunk}\AgdaSpace{}%
\AgdaSymbol{(}\AgdaBound{x}\AgdaSpace{}%
\AgdaOperator{\AgdaFunction{⟨exp}}\AgdaSpace{}%
\AgdaBound{θ}\AgdaSymbol{)}\<%
\\
\>[0]\AgdaOperator{\AgdaFunction{\AgdaUnderscore{}⟨exp\AgdaUnderscore{}}}\AgdaSpace{}%
\AgdaSymbol{\{}\AgdaArgument{d}\AgdaSpace{}%
\AgdaSymbol{=}\AgdaSpace{}%
\AgdaInductiveConstructor{const}\AgdaSymbol{\}}\AgdaSpace{}%
\AgdaSymbol{(}\AgdaBound{ξ}\AgdaSpace{}%
\AgdaOperator{\AgdaInductiveConstructor{/}}\AgdaSpace{}%
\AgdaBound{σ}\AgdaSymbol{)}%
\>[29]\AgdaBound{θ}\AgdaSpace{}%
\AgdaSymbol{=}\AgdaSpace{}%
\AgdaBound{ξ}\AgdaSpace{}%
\AgdaOperator{\AgdaInductiveConstructor{/}}\AgdaSpace{}%
\AgdaSymbol{(}\AgdaBound{σ}\AgdaSpace{}%
\AgdaOperator{\AgdaFunction{⟨}}\AgdaSpace{}%
\AgdaBound{θ}\AgdaSymbol{)}\<%
\\
\>[0][@{}l@{\AgdaIndent{0}}]%
\>[2]\AgdaKeyword{where}\<%
\\
\>[2][@{}l@{\AgdaIndent{0}}]%
\>[4]\AgdaComment{-- had to inline ⟨sub for the termination checker}\<%
\\
%
\>[4]\AgdaOperator{\AgdaFunction{\AgdaUnderscore{}⟨\AgdaUnderscore{}}}\AgdaSpace{}%
\AgdaSymbol{:}\AgdaSpace{}%
\AgdaFunction{Thinnable}\AgdaSpace{}%
\AgdaSymbol{(}\AgdaGeneralizable{γ'}\AgdaSpace{}%
\AgdaOperator{\AgdaFunction{⇒[}}\AgdaSpace{}%
\AgdaSymbol{(λ}\AgdaSpace{}%
\AgdaBound{γ}\AgdaSpace{}%
\AgdaSymbol{→}\AgdaSpace{}%
\AgdaDatatype{Com}\AgdaSpace{}%
\AgdaGeneralizable{p}\AgdaSpace{}%
\AgdaBound{γ}\AgdaSymbol{)}\AgdaSpace{}%
\AgdaOperator{\AgdaFunction{]\AgdaUnderscore{}}}\AgdaSymbol{)}\<%
\\
%
\>[4]\AgdaOperator{\AgdaFunction{\AgdaUnderscore{}⟨\AgdaUnderscore{}}}%
\>[9]\AgdaInductiveConstructor{ε}%
\>[18]\AgdaBound{θ'}\AgdaSpace{}%
\AgdaSymbol{=}\AgdaSpace{}%
\AgdaInductiveConstructor{ε}\<%
\\
%
\>[4]\AgdaOperator{\AgdaFunction{\AgdaUnderscore{}⟨\AgdaUnderscore{}}}\AgdaSpace{}%
\AgdaSymbol{(}\AgdaBound{σ'}\AgdaSpace{}%
\AgdaOperator{\AgdaInductiveConstructor{-,}}\AgdaSpace{}%
\AgdaBound{x}\AgdaSymbol{)}\AgdaSpace{}%
\AgdaBound{θ'}\AgdaSpace{}%
\AgdaSymbol{=}\AgdaSpace{}%
\AgdaSymbol{(}\AgdaBound{σ'}\AgdaSpace{}%
\AgdaOperator{\AgdaFunction{⟨}}\AgdaSpace{}%
\AgdaBound{θ'}\AgdaSymbol{)}\AgdaSpace{}%
\AgdaOperator{\AgdaInductiveConstructor{-,}}\AgdaSpace{}%
\AgdaSymbol{(}\AgdaBound{x}\AgdaSpace{}%
\AgdaOperator{\AgdaFunction{⟨exp}}\AgdaSpace{}%
\AgdaBound{θ'}\AgdaSymbol{)}\<%
\\
%
\\[\AgdaEmptyExtraSkip]%
\>[0]\AgdaOperator{\AgdaFunction{\AgdaUnderscore{}⟨exp\AgdaUnderscore{}}}\AgdaSpace{}%
\AgdaSymbol{\{}\AgdaArgument{d}\AgdaSpace{}%
\AgdaSymbol{=}\AgdaSpace{}%
\AgdaInductiveConstructor{compu}\AgdaSymbol{\}}\AgdaSpace{}%
\AgdaSymbol{(}\AgdaInductiveConstructor{var}\AgdaSpace{}%
\AgdaBound{x}\AgdaSymbol{)}%
\>[30]\AgdaBound{θ}%
\>[33]\AgdaSymbol{=}\AgdaSpace{}%
\AgdaInductiveConstructor{var}\AgdaSpace{}%
\AgdaSymbol{(}\AgdaBound{x}\AgdaSpace{}%
\AgdaOperator{\AgdaFunction{⟨var}}\AgdaSpace{}%
\AgdaBound{θ}\AgdaSymbol{)}\<%
\\
\>[0]\AgdaOperator{\AgdaFunction{\AgdaUnderscore{}⟨exp\AgdaUnderscore{}}}\AgdaSpace{}%
\AgdaSymbol{\{}\AgdaArgument{d}\AgdaSpace{}%
\AgdaSymbol{=}\AgdaSpace{}%
\AgdaInductiveConstructor{compu}\AgdaSymbol{\}}\AgdaSpace{}%
\AgdaSymbol{(}\AgdaInductiveConstructor{elim}\AgdaSpace{}%
\AgdaBound{e}\AgdaSpace{}%
\AgdaBound{s}\AgdaSymbol{)}\AgdaSpace{}%
\AgdaBound{θ}%
\>[33]\AgdaSymbol{=}\AgdaSpace{}%
\AgdaInductiveConstructor{elim}\AgdaSpace{}%
\AgdaSymbol{(}\AgdaBound{e}\AgdaSpace{}%
\AgdaOperator{\AgdaFunction{⟨exp}}\AgdaSpace{}%
\AgdaBound{θ}\AgdaSymbol{)}\AgdaSpace{}%
\AgdaSymbol{(}\AgdaBound{s}\AgdaSpace{}%
\AgdaOperator{\AgdaFunction{⟨exp}}\AgdaSpace{}%
\AgdaBound{θ}\AgdaSymbol{)}\<%
\\
\>[0]\AgdaOperator{\AgdaFunction{\AgdaUnderscore{}⟨exp\AgdaUnderscore{}}}\AgdaSpace{}%
\AgdaSymbol{\{}\AgdaArgument{d}\AgdaSpace{}%
\AgdaSymbol{=}\AgdaSpace{}%
\AgdaInductiveConstructor{compu}\AgdaSymbol{\}}\AgdaSpace{}%
\AgdaSymbol{(}\AgdaBound{t}\AgdaSpace{}%
\AgdaOperator{\AgdaInductiveConstructor{∷}}\AgdaSpace{}%
\AgdaBound{T}\AgdaSymbol{)}\AgdaSpace{}%
\AgdaBound{θ}%
\>[30]\AgdaSymbol{=}\AgdaSpace{}%
\AgdaSymbol{(}\AgdaBound{t}\AgdaSpace{}%
\AgdaOperator{\AgdaFunction{⟨exp}}\AgdaSpace{}%
\AgdaBound{θ}\AgdaSymbol{)}\AgdaSpace{}%
\AgdaOperator{\AgdaInductiveConstructor{∷}}\AgdaSpace{}%
\AgdaSymbol{(}\AgdaBound{T}\AgdaSpace{}%
\AgdaOperator{\AgdaFunction{⟨exp}}\AgdaSpace{}%
\AgdaBound{θ}\AgdaSymbol{)}\<%
\\
%
\\[\AgdaEmptyExtraSkip]%
\>[0]\AgdaOperator{\AgdaFunction{\AgdaUnderscore{}\textasciicircum{}exp}}%
\>[9]\AgdaSymbol{:}\AgdaSpace{}%
\AgdaFunction{Weakenable}\AgdaSpace{}%
\AgdaSymbol{(}\AgdaFunction{Expr}\AgdaSpace{}%
\AgdaGeneralizable{p}\AgdaSpace{}%
\AgdaGeneralizable{d}\AgdaSymbol{)}\<%
\\
\>[0]\AgdaOperator{\AgdaFunction{\AgdaUnderscore{}\textasciicircum{}exp}}\AgdaSpace{}%
\AgdaSymbol{\{}\AgdaArgument{p}\AgdaSpace{}%
\AgdaSymbol{=}\AgdaSpace{}%
\AgdaBound{p}\AgdaSymbol{\}}\AgdaSpace{}%
\AgdaSymbol{\{}\AgdaArgument{d}\AgdaSpace{}%
\AgdaSymbol{=}\AgdaSpace{}%
\AgdaBound{d}\AgdaSymbol{\}}\AgdaSpace{}%
\AgdaSymbol{=}\AgdaSpace{}%
\AgdaFunction{weaken}\AgdaSpace{}%
\AgdaSymbol{\{}\AgdaArgument{T}\AgdaSpace{}%
\AgdaSymbol{=}\AgdaSpace{}%
\AgdaFunction{Expr}\AgdaSpace{}%
\AgdaBound{p}\AgdaSpace{}%
\AgdaBound{d}\AgdaSymbol{\}}\AgdaSpace{}%
\AgdaOperator{\AgdaFunction{\AgdaUnderscore{}⟨exp\AgdaUnderscore{}}}\<%
\\
%
\\[\AgdaEmptyExtraSkip]%
\>[0]\AgdaOperator{\AgdaFunction{\AgdaUnderscore{}\textasciicircum{}/exp}}%
\>[9]\AgdaSymbol{:}\AgdaSpace{}%
\AgdaFunction{Weakenable}\AgdaSpace{}%
\AgdaSymbol{(}\AgdaGeneralizable{γ}\AgdaSpace{}%
\AgdaOperator{\AgdaFunction{⇒[}}\AgdaSpace{}%
\AgdaFunction{Expr}\AgdaSpace{}%
\AgdaGeneralizable{p}\AgdaSpace{}%
\AgdaGeneralizable{d}\AgdaSpace{}%
\AgdaOperator{\AgdaFunction{]\AgdaUnderscore{}}}\AgdaSymbol{)}\<%
\\
\>[0]\AgdaOperator{\AgdaFunction{\AgdaUnderscore{}\textasciicircum{}/exp}}\AgdaSpace{}%
\AgdaSymbol{\{}\AgdaArgument{p}\AgdaSpace{}%
\AgdaSymbol{=}\AgdaSpace{}%
\AgdaBound{p}\AgdaSymbol{\}}\AgdaSpace{}%
\AgdaSymbol{\{}\AgdaArgument{d}\AgdaSpace{}%
\AgdaSymbol{=}\AgdaSpace{}%
\AgdaBound{d}\AgdaSymbol{\}}%
\>[24]\AgdaSymbol{=}\AgdaSpace{}%
\AgdaFunction{\textasciicircum{}sub}\AgdaSpace{}%
\AgdaSymbol{\{}\AgdaArgument{T}\AgdaSpace{}%
\AgdaSymbol{=}\AgdaSpace{}%
\AgdaFunction{Expr}\AgdaSpace{}%
\AgdaBound{p}\AgdaSpace{}%
\AgdaBound{d}\AgdaSymbol{\}}%
\>[47]\AgdaOperator{\AgdaFunction{\AgdaUnderscore{}⟨exp\AgdaUnderscore{}}}\<%
\\
%
\\[\AgdaEmptyExtraSkip]%
\>[0]\AgdaFunction{idexpr}\AgdaSpace{}%
\AgdaSymbol{:}\AgdaSpace{}%
\AgdaGeneralizable{γ}\AgdaSpace{}%
\AgdaOperator{\AgdaFunction{⇒[}}\AgdaSpace{}%
\AgdaFunction{Expr}\AgdaSpace{}%
\AgdaGeneralizable{p}\AgdaSpace{}%
\AgdaInductiveConstructor{compu}\AgdaSpace{}%
\AgdaOperator{\AgdaFunction{]}}\AgdaSpace{}%
\AgdaGeneralizable{γ}\<%
\\
\>[0]\AgdaFunction{idexpr}\AgdaSpace{}%
\AgdaSymbol{\{}\AgdaInductiveConstructor{zero}\AgdaSymbol{\}}%
\>[23]\AgdaSymbol{=}\AgdaSpace{}%
\AgdaInductiveConstructor{ε}\<%
\\
\>[0]\AgdaFunction{idexpr}\AgdaSpace{}%
\AgdaSymbol{\{}\AgdaInductiveConstructor{suc}\AgdaSpace{}%
\AgdaBound{γ}\AgdaSymbol{\}}\AgdaSpace{}%
\AgdaSymbol{\{}\AgdaArgument{p}\AgdaSpace{}%
\AgdaSymbol{=}\AgdaSpace{}%
\AgdaBound{p}\AgdaSymbol{\}}\<%
\\
\>[0][@{}l@{\AgdaIndent{0}}]%
\>[2]\AgdaSymbol{=}\AgdaSpace{}%
\AgdaFunction{\textasciicircum{}sub}\AgdaSpace{}%
\AgdaSymbol{\{}\AgdaArgument{T}\AgdaSpace{}%
\AgdaSymbol{=}\AgdaSpace{}%
\AgdaFunction{Expr}\AgdaSpace{}%
\AgdaBound{p}\AgdaSpace{}%
\AgdaInductiveConstructor{compu}\AgdaSymbol{\}}\AgdaSpace{}%
\AgdaOperator{\AgdaFunction{\AgdaUnderscore{}⟨exp\AgdaUnderscore{}}}\AgdaSpace{}%
\AgdaFunction{idexpr}\AgdaSpace{}%
\AgdaOperator{\AgdaInductiveConstructor{-,}}\AgdaSpace{}%
\AgdaInductiveConstructor{var}\AgdaSpace{}%
\AgdaSymbol{(}\AgdaFunction{fromNum}\AgdaSpace{}%
\AgdaBound{γ}\AgdaSymbol{)}\<%
\\
%
\\[\AgdaEmptyExtraSkip]%
\>[0]\AgdaOperator{\AgdaFunction{\AgdaUnderscore{}⊗expr\AgdaUnderscore{}}}%
\>[9]\AgdaSymbol{:}\AgdaSpace{}%
\AgdaSymbol{(}\AgdaBound{γ}\AgdaSpace{}%
\AgdaSymbol{:}\AgdaSpace{}%
\AgdaFunction{Scope}\AgdaSymbol{)}\AgdaSpace{}%
\AgdaSymbol{→}\AgdaSpace{}%
\AgdaFunction{Expr}\AgdaSpace{}%
\AgdaGeneralizable{p}\AgdaSpace{}%
\AgdaGeneralizable{d}\AgdaSpace{}%
\AgdaGeneralizable{δ}\AgdaSpace{}%
\AgdaSymbol{→}\AgdaSpace{}%
\AgdaFunction{Expr}\AgdaSpace{}%
\AgdaSymbol{(}\AgdaBound{γ}\AgdaSpace{}%
\AgdaOperator{\AgdaFunction{⊗}}\AgdaSpace{}%
\AgdaGeneralizable{p}\AgdaSymbol{)}\AgdaSpace{}%
\AgdaGeneralizable{d}\AgdaSpace{}%
\AgdaSymbol{(}\AgdaBound{γ}\AgdaSpace{}%
\AgdaOperator{\AgdaPrimitive{+}}\AgdaSpace{}%
\AgdaGeneralizable{δ}\AgdaSymbol{)}\<%
\\
\>[0]\AgdaOperator{\AgdaFunction{\AgdaUnderscore{}⊗sub\AgdaUnderscore{}}}\AgdaSpace{}%
\AgdaSymbol{:}%
\>[387I]\AgdaSymbol{(}\AgdaBound{δ'}\AgdaSpace{}%
\AgdaSymbol{:}\AgdaSpace{}%
\AgdaFunction{Scope}\AgdaSymbol{)}\AgdaSpace{}%
\AgdaSymbol{→}\<%
\\
\>[.][@{}l@{}]\<[387I]%
\>[9]\AgdaGeneralizable{δ}\AgdaSpace{}%
\AgdaOperator{\AgdaFunction{⇒[}}\AgdaSpace{}%
\AgdaFunction{Expr}\AgdaSpace{}%
\AgdaGeneralizable{p}\AgdaSpace{}%
\AgdaInductiveConstructor{compu}\AgdaSpace{}%
\AgdaOperator{\AgdaFunction{]}}\AgdaSpace{}%
\AgdaGeneralizable{γ}\AgdaSpace{}%
\AgdaSymbol{→}\<%
\\
%
\>[9]\AgdaSymbol{(}\AgdaBound{δ'}\AgdaSpace{}%
\AgdaOperator{\AgdaPrimitive{+}}\AgdaSpace{}%
\AgdaGeneralizable{δ}\AgdaSymbol{)}\AgdaSpace{}%
\AgdaOperator{\AgdaFunction{⇒[}}\AgdaSpace{}%
\AgdaFunction{Expr}\AgdaSpace{}%
\AgdaSymbol{(}\AgdaBound{δ'}\AgdaSpace{}%
\AgdaOperator{\AgdaFunction{⊗}}%
\>[33]\AgdaGeneralizable{p}\AgdaSymbol{)}\AgdaSpace{}%
\AgdaInductiveConstructor{compu}\AgdaSpace{}%
\AgdaOperator{\AgdaFunction{]}}\AgdaSpace{}%
\AgdaSymbol{(}\AgdaBound{δ'}\AgdaSpace{}%
\AgdaOperator{\AgdaPrimitive{+}}\AgdaSpace{}%
\AgdaGeneralizable{γ}\AgdaSymbol{)}\<%
\\
%
\\[\AgdaEmptyExtraSkip]%
\>[0]\AgdaOperator{\AgdaFunction{\AgdaUnderscore{}⊗expr\AgdaUnderscore{}}}\AgdaSpace{}%
\AgdaSymbol{\{}\AgdaArgument{d}\AgdaSpace{}%
\AgdaSymbol{=}\AgdaSpace{}%
\AgdaInductiveConstructor{const}\AgdaSymbol{\}}\AgdaSpace{}%
\AgdaBound{γ}\AgdaSpace{}%
\AgdaSymbol{(}\AgdaInductiveConstructor{`}\AgdaSpace{}%
\AgdaBound{x}\AgdaSymbol{)}%
\>[34]\AgdaSymbol{=}\AgdaSpace{}%
\AgdaInductiveConstructor{`}\AgdaSpace{}%
\AgdaBound{x}\<%
\\
\>[0]\AgdaOperator{\AgdaFunction{\AgdaUnderscore{}⊗expr\AgdaUnderscore{}}}\AgdaSpace{}%
\AgdaSymbol{\{}\AgdaArgument{d}\AgdaSpace{}%
\AgdaSymbol{=}\AgdaSpace{}%
\AgdaInductiveConstructor{const}\AgdaSymbol{\}}\AgdaSpace{}%
\AgdaBound{γ}\AgdaSpace{}%
\AgdaSymbol{(}\AgdaBound{s}\AgdaSpace{}%
\AgdaOperator{\AgdaInductiveConstructor{∙}}\AgdaSpace{}%
\AgdaBound{t}\AgdaSymbol{)}%
\>[34]\AgdaSymbol{=}\AgdaSpace{}%
\AgdaSymbol{(}\AgdaBound{γ}\AgdaSpace{}%
\AgdaOperator{\AgdaFunction{⊗expr}}\AgdaSpace{}%
\AgdaBound{s}\AgdaSymbol{)}\AgdaSpace{}%
\AgdaOperator{\AgdaInductiveConstructor{∙}}\AgdaSpace{}%
\AgdaSymbol{(}\AgdaBound{γ}\AgdaSpace{}%
\AgdaOperator{\AgdaFunction{⊗expr}}\AgdaSpace{}%
\AgdaBound{t}\AgdaSymbol{)}\<%
\\
\>[0]\AgdaOperator{\AgdaFunction{\AgdaUnderscore{}⊗expr\AgdaUnderscore{}}}\AgdaSpace{}%
\AgdaSymbol{\{}\AgdaArgument{d}\AgdaSpace{}%
\AgdaSymbol{=}\AgdaSpace{}%
\AgdaInductiveConstructor{const}\AgdaSymbol{\}}\AgdaSpace{}%
\AgdaBound{γ}\AgdaSpace{}%
\AgdaSymbol{(}\AgdaInductiveConstructor{bind}\AgdaSpace{}%
\AgdaBound{x}\AgdaSymbol{)}%
\>[34]\AgdaSymbol{=}\AgdaSpace{}%
\AgdaInductiveConstructor{bind}\AgdaSpace{}%
\AgdaSymbol{(}\AgdaBound{γ}\AgdaSpace{}%
\AgdaOperator{\AgdaFunction{⊗expr}}\AgdaSpace{}%
\AgdaBound{x}\AgdaSymbol{)}\<%
\\
\>[0]\AgdaOperator{\AgdaFunction{\AgdaUnderscore{}⊗expr\AgdaUnderscore{}}}\AgdaSpace{}%
\AgdaSymbol{\{}\AgdaArgument{d}\AgdaSpace{}%
\AgdaSymbol{=}\AgdaSpace{}%
\AgdaInductiveConstructor{const}\AgdaSymbol{\}}\AgdaSpace{}%
\AgdaBound{γ}\AgdaSpace{}%
\AgdaSymbol{(}\AgdaInductiveConstructor{thunk}\AgdaSpace{}%
\AgdaBound{x}\AgdaSymbol{)}%
\>[34]\AgdaSymbol{=}\AgdaSpace{}%
\AgdaInductiveConstructor{thunk}\AgdaSpace{}%
\AgdaSymbol{(}\AgdaBound{γ}\AgdaSpace{}%
\AgdaOperator{\AgdaFunction{⊗expr}}\AgdaSpace{}%
\AgdaBound{x}\AgdaSymbol{)}\<%
\\
\>[0]\AgdaOperator{\AgdaFunction{\AgdaUnderscore{}⊗expr\AgdaUnderscore{}}}\AgdaSpace{}%
\AgdaSymbol{\{}\AgdaArgument{d}\AgdaSpace{}%
\AgdaSymbol{=}\AgdaSpace{}%
\AgdaInductiveConstructor{const}\AgdaSymbol{\}}\AgdaSpace{}%
\AgdaBound{γ}\AgdaSpace{}%
\AgdaSymbol{(}\AgdaBound{ξ}\AgdaSpace{}%
\AgdaOperator{\AgdaInductiveConstructor{/}}\AgdaSpace{}%
\AgdaBound{σ}\AgdaSymbol{)}%
\>[34]\AgdaSymbol{=}\AgdaSpace{}%
\AgdaSymbol{(}\AgdaBound{γ}\AgdaSpace{}%
\AgdaOperator{\AgdaFunction{⊗svar}}\AgdaSpace{}%
\AgdaBound{ξ}\AgdaSymbol{)}\AgdaSpace{}%
\AgdaOperator{\AgdaInductiveConstructor{/}}\AgdaSpace{}%
\AgdaSymbol{(}\AgdaBound{γ}\AgdaSpace{}%
\AgdaOperator{\AgdaFunction{⊗sub}}\AgdaSpace{}%
\AgdaBound{σ}\AgdaSymbol{)}\<%
\\
\>[0]\AgdaOperator{\AgdaFunction{\AgdaUnderscore{}⊗expr\AgdaUnderscore{}}}\AgdaSpace{}%
\AgdaSymbol{\{}\AgdaArgument{d}\AgdaSpace{}%
\AgdaSymbol{=}\AgdaSpace{}%
\AgdaInductiveConstructor{compu}\AgdaSymbol{\}}\AgdaSpace{}%
\AgdaBound{γ}\AgdaSpace{}%
\AgdaSymbol{(}\AgdaInductiveConstructor{var}\AgdaSpace{}%
\AgdaBound{x}\AgdaSymbol{)}%
\>[34]\AgdaSymbol{=}\AgdaSpace{}%
\AgdaInductiveConstructor{var}\AgdaSpace{}%
\AgdaSymbol{(}\AgdaBound{γ}\AgdaSpace{}%
\AgdaOperator{\AgdaFunction{⊗var}}\AgdaSpace{}%
\AgdaBound{x}\AgdaSymbol{)}\<%
\\
\>[0]\AgdaOperator{\AgdaFunction{\AgdaUnderscore{}⊗expr\AgdaUnderscore{}}}\AgdaSpace{}%
\AgdaSymbol{\{}\AgdaArgument{d}\AgdaSpace{}%
\AgdaSymbol{=}\AgdaSpace{}%
\AgdaInductiveConstructor{compu}\AgdaSymbol{\}}\AgdaSpace{}%
\AgdaBound{γ}\AgdaSpace{}%
\AgdaSymbol{(}\AgdaInductiveConstructor{elim}\AgdaSpace{}%
\AgdaBound{e}\AgdaSpace{}%
\AgdaBound{s}\AgdaSymbol{)}%
\>[34]\AgdaSymbol{=}\AgdaSpace{}%
\AgdaInductiveConstructor{elim}\AgdaSpace{}%
\AgdaSymbol{(}\AgdaBound{γ}\AgdaSpace{}%
\AgdaOperator{\AgdaFunction{⊗expr}}\AgdaSpace{}%
\AgdaBound{e}\AgdaSymbol{)}\AgdaSpace{}%
\AgdaSymbol{(}\AgdaBound{γ}\AgdaSpace{}%
\AgdaOperator{\AgdaFunction{⊗expr}}\AgdaSpace{}%
\AgdaBound{s}\AgdaSymbol{)}\<%
\\
\>[0]\AgdaOperator{\AgdaFunction{\AgdaUnderscore{}⊗expr\AgdaUnderscore{}}}\AgdaSpace{}%
\AgdaSymbol{\{}\AgdaArgument{d}\AgdaSpace{}%
\AgdaSymbol{=}\AgdaSpace{}%
\AgdaInductiveConstructor{compu}\AgdaSymbol{\}}\AgdaSpace{}%
\AgdaBound{γ}\AgdaSpace{}%
\AgdaSymbol{(}\AgdaBound{t}\AgdaSpace{}%
\AgdaOperator{\AgdaInductiveConstructor{∷}}\AgdaSpace{}%
\AgdaBound{T}\AgdaSymbol{)}%
\>[34]\AgdaSymbol{=}\AgdaSpace{}%
\AgdaSymbol{(}\AgdaBound{γ}\AgdaSpace{}%
\AgdaOperator{\AgdaFunction{⊗expr}}\AgdaSpace{}%
\AgdaBound{t}\AgdaSymbol{)}\AgdaSpace{}%
\AgdaOperator{\AgdaInductiveConstructor{∷}}\AgdaSpace{}%
\AgdaSymbol{(}\AgdaBound{γ}\AgdaSpace{}%
\AgdaOperator{\AgdaFunction{⊗expr}}\AgdaSpace{}%
\AgdaBound{T}\AgdaSymbol{)}\<%
\\
%
\\[\AgdaEmptyExtraSkip]%
\>[0]\AgdaOperator{\AgdaFunction{\AgdaUnderscore{}⊗sub\AgdaUnderscore{}}}\AgdaSpace{}%
\AgdaSymbol{\{}\AgdaArgument{p}\AgdaSpace{}%
\AgdaSymbol{=}\AgdaSpace{}%
\AgdaBound{p}\AgdaSymbol{\}}\AgdaSpace{}%
\AgdaSymbol{\{}\AgdaArgument{γ}\AgdaSpace{}%
\AgdaSymbol{=}\AgdaSpace{}%
\AgdaBound{γ}\AgdaSymbol{\}}\AgdaSpace{}%
\AgdaBound{δ'}\AgdaSpace{}%
\AgdaInductiveConstructor{ε}\<%
\\
\>[0][@{}l@{\AgdaIndent{0}}]%
\>[2]\AgdaSymbol{=}\AgdaSpace{}%
\AgdaFunction{⟨sub}\AgdaSpace{}%
\AgdaSymbol{\{}\AgdaArgument{T}\AgdaSpace{}%
\AgdaSymbol{=}\AgdaSpace{}%
\AgdaFunction{Expr}\AgdaSpace{}%
\AgdaSymbol{(}\AgdaBound{δ'}\AgdaSpace{}%
\AgdaOperator{\AgdaFunction{⊗}}\AgdaSpace{}%
\AgdaBound{p}\AgdaSymbol{)}\AgdaSpace{}%
\AgdaInductiveConstructor{compu}\AgdaSymbol{\}}\AgdaSpace{}%
\AgdaOperator{\AgdaFunction{\AgdaUnderscore{}⟨exp\AgdaUnderscore{}}}\AgdaSpace{}%
\AgdaSymbol{(}\AgdaFunction{idexpr}\AgdaSpace{}%
\AgdaSymbol{\{}\AgdaBound{δ'}\AgdaSymbol{\})}\AgdaSpace{}%
\AgdaSymbol{(}\AgdaBound{δ'}\AgdaSpace{}%
\AgdaOperator{\AgdaFunction{◃}}\AgdaSpace{}%
\AgdaBound{γ}\AgdaSymbol{)}\<%
\\
\>[0]\AgdaBound{δ'}\AgdaSpace{}%
\AgdaOperator{\AgdaFunction{⊗sub}}\AgdaSpace{}%
\AgdaSymbol{(}\AgdaBound{σ}\AgdaSpace{}%
\AgdaOperator{\AgdaInductiveConstructor{-,}}\AgdaSpace{}%
\AgdaBound{x}\AgdaSymbol{)}\AgdaSpace{}%
\AgdaSymbol{=}\AgdaSpace{}%
\AgdaSymbol{(}\AgdaBound{δ'}\AgdaSpace{}%
\AgdaOperator{\AgdaFunction{⊗sub}}\AgdaSpace{}%
\AgdaBound{σ}\AgdaSymbol{)}\AgdaSpace{}%
\AgdaOperator{\AgdaInductiveConstructor{-,}}\AgdaSpace{}%
\AgdaSymbol{(}\AgdaBound{δ'}\AgdaSpace{}%
\AgdaOperator{\AgdaFunction{⊗expr}}\AgdaSpace{}%
\AgdaBound{x}\AgdaSymbol{)}\<%
\\
%
\\[\AgdaEmptyExtraSkip]%
\>[0]\AgdaOperator{\AgdaFunction{\AgdaUnderscore{}⟨esub\AgdaUnderscore{}}}\AgdaSpace{}%
\AgdaSymbol{:}\AgdaSpace{}%
\AgdaFunction{Thinnable}\AgdaSpace{}%
\AgdaSymbol{(}\AgdaGeneralizable{γ}\AgdaSpace{}%
\AgdaOperator{\AgdaFunction{⇒[}}\AgdaSpace{}%
\AgdaFunction{Expr}\AgdaSpace{}%
\AgdaGeneralizable{p}\AgdaSpace{}%
\AgdaGeneralizable{d}\AgdaSpace{}%
\AgdaOperator{\AgdaFunction{]\AgdaUnderscore{}}}\AgdaSymbol{)}\<%
\\
\>[0]\AgdaOperator{\AgdaFunction{\AgdaUnderscore{}⟨esub\AgdaUnderscore{}}}\AgdaSpace{}%
\AgdaSymbol{=}\AgdaSpace{}%
\AgdaFunction{⟨sub}\AgdaSpace{}%
\AgdaOperator{\AgdaFunction{\AgdaUnderscore{}⟨exp\AgdaUnderscore{}}}\<%
\\
%
\\[\AgdaEmptyExtraSkip]%
%
\\[\AgdaEmptyExtraSkip]%
\>[0]\AgdaFunction{map}\AgdaSpace{}%
\AgdaSymbol{:}\AgdaSpace{}%
\AgdaSymbol{(}\AgdaBound{f}\AgdaSpace{}%
\AgdaSymbol{:}\AgdaSpace{}%
\AgdaSymbol{∀\{}\AgdaBound{δ}\AgdaSymbol{\}}\AgdaSpace{}%
\AgdaSymbol{→}\AgdaSpace{}%
\AgdaDatatype{svar}\AgdaSpace{}%
\AgdaGeneralizable{p}\AgdaSpace{}%
\AgdaBound{δ}\AgdaSpace{}%
\AgdaSymbol{→}\AgdaSpace{}%
\AgdaDatatype{svar}\AgdaSpace{}%
\AgdaGeneralizable{q}\AgdaSpace{}%
\AgdaBound{δ}\AgdaSymbol{)}\AgdaSpace{}%
\AgdaSymbol{→}\AgdaSpace{}%
\AgdaFunction{Expr}\AgdaSpace{}%
\AgdaGeneralizable{p}\AgdaSpace{}%
\AgdaGeneralizable{d}\AgdaSpace{}%
\AgdaGeneralizable{γ'}\AgdaSpace{}%
\AgdaSymbol{→}\AgdaSpace{}%
\AgdaFunction{Expr}\AgdaSpace{}%
\AgdaGeneralizable{q}\AgdaSpace{}%
\AgdaGeneralizable{d}\AgdaSpace{}%
\AgdaGeneralizable{γ'}\<%
\\
\>[0]\AgdaFunction{map}\AgdaSpace{}%
\AgdaSymbol{\{}\AgdaArgument{d}\AgdaSpace{}%
\AgdaSymbol{=}\AgdaSpace{}%
\AgdaInductiveConstructor{const}\AgdaSymbol{\}}\AgdaSpace{}%
\AgdaBound{f}\AgdaSpace{}%
\AgdaSymbol{(}\AgdaInductiveConstructor{`}\AgdaSpace{}%
\AgdaBound{x}\AgdaSymbol{)}\AgdaSpace{}%
\AgdaSymbol{=}\AgdaSpace{}%
\AgdaInductiveConstructor{`}\AgdaSpace{}%
\AgdaBound{x}\<%
\\
\>[0]\AgdaFunction{map}\AgdaSpace{}%
\AgdaSymbol{\{}\AgdaArgument{d}\AgdaSpace{}%
\AgdaSymbol{=}\AgdaSpace{}%
\AgdaInductiveConstructor{const}\AgdaSymbol{\}}\AgdaSpace{}%
\AgdaBound{f}\AgdaSpace{}%
\AgdaSymbol{(}\AgdaBound{s}\AgdaSpace{}%
\AgdaOperator{\AgdaInductiveConstructor{∙}}\AgdaSpace{}%
\AgdaBound{t}\AgdaSymbol{)}\AgdaSpace{}%
\AgdaSymbol{=}\AgdaSpace{}%
\AgdaFunction{map}\AgdaSpace{}%
\AgdaBound{f}\AgdaSpace{}%
\AgdaBound{s}\AgdaSpace{}%
\AgdaOperator{\AgdaInductiveConstructor{∙}}\AgdaSpace{}%
\AgdaFunction{map}\AgdaSpace{}%
\AgdaBound{f}\AgdaSpace{}%
\AgdaBound{t}\<%
\\
\>[0]\AgdaFunction{map}\AgdaSpace{}%
\AgdaSymbol{\{}\AgdaArgument{d}\AgdaSpace{}%
\AgdaSymbol{=}\AgdaSpace{}%
\AgdaInductiveConstructor{const}\AgdaSymbol{\}}\AgdaSpace{}%
\AgdaBound{f}\AgdaSpace{}%
\AgdaSymbol{(}\AgdaInductiveConstructor{bind}\AgdaSpace{}%
\AgdaBound{t}\AgdaSymbol{)}\AgdaSpace{}%
\AgdaSymbol{=}\AgdaSpace{}%
\AgdaInductiveConstructor{bind}\AgdaSpace{}%
\AgdaSymbol{(}\AgdaFunction{map}\AgdaSpace{}%
\AgdaBound{f}\AgdaSpace{}%
\AgdaBound{t}\AgdaSymbol{)}\<%
\\
\>[0]\AgdaFunction{map}\AgdaSpace{}%
\AgdaSymbol{\{}\AgdaArgument{d}\AgdaSpace{}%
\AgdaSymbol{=}\AgdaSpace{}%
\AgdaInductiveConstructor{const}\AgdaSymbol{\}}\AgdaSpace{}%
\AgdaBound{f}\AgdaSpace{}%
\AgdaSymbol{(}\AgdaInductiveConstructor{thunk}\AgdaSpace{}%
\AgdaBound{x}\AgdaSymbol{)}\AgdaSpace{}%
\AgdaSymbol{=}\AgdaSpace{}%
\AgdaInductiveConstructor{thunk}\AgdaSpace{}%
\AgdaSymbol{(}\AgdaFunction{map}\AgdaSpace{}%
\AgdaBound{f}\AgdaSpace{}%
\AgdaBound{x}\AgdaSymbol{)}\<%
\\
\>[0]\AgdaFunction{map}\AgdaSpace{}%
\AgdaSymbol{\{}\AgdaArgument{p}\AgdaSpace{}%
\AgdaSymbol{=}\AgdaSpace{}%
\AgdaBound{p}\AgdaSymbol{\}}\AgdaSpace{}%
\AgdaSymbol{\{}\AgdaArgument{q}\AgdaSpace{}%
\AgdaSymbol{=}\AgdaSpace{}%
\AgdaBound{q}\AgdaSymbol{\}}\AgdaSpace{}%
\AgdaSymbol{\{}\AgdaArgument{d}\AgdaSpace{}%
\AgdaSymbol{=}\AgdaSpace{}%
\AgdaInductiveConstructor{const}\AgdaSymbol{\}}\AgdaSpace{}%
\AgdaBound{f}\AgdaSpace{}%
\AgdaSymbol{(}\AgdaBound{ξ}\AgdaSpace{}%
\AgdaOperator{\AgdaInductiveConstructor{/}}\AgdaSpace{}%
\AgdaBound{σ}\AgdaSymbol{)}\AgdaSpace{}%
\AgdaSymbol{=}\AgdaSpace{}%
\AgdaBound{f}\AgdaSpace{}%
\AgdaBound{ξ}\AgdaSpace{}%
\AgdaOperator{\AgdaInductiveConstructor{/}}\AgdaSpace{}%
\AgdaFunction{mapself}\AgdaSpace{}%
\AgdaBound{σ}\<%
\\
\>[0][@{}l@{\AgdaIndent{0}}]%
\>[2]\AgdaKeyword{where}\<%
\\
\>[2][@{}l@{\AgdaIndent{0}}]%
\>[4]\AgdaComment{-- inline to shut up termination checker}\<%
\\
%
\>[4]\AgdaFunction{mapself}\AgdaSpace{}%
\AgdaSymbol{:}\AgdaSpace{}%
\AgdaSymbol{∀}\AgdaSpace{}%
\AgdaSymbol{\{}\AgdaBound{n}\AgdaSymbol{\}}\AgdaSpace{}%
\AgdaSymbol{→}\AgdaSpace{}%
\AgdaDatatype{BwdVec}\AgdaSpace{}%
\AgdaSymbol{(}\AgdaDatatype{Com}\AgdaSpace{}%
\AgdaBound{p}\AgdaSpace{}%
\AgdaGeneralizable{γ'}\AgdaSymbol{)}\AgdaSpace{}%
\AgdaBound{n}\AgdaSpace{}%
\AgdaSymbol{→}\AgdaSpace{}%
\AgdaDatatype{BwdVec}\AgdaSpace{}%
\AgdaSymbol{(}\AgdaDatatype{Com}\AgdaSpace{}%
\AgdaBound{q}\AgdaSpace{}%
\AgdaGeneralizable{γ'}\AgdaSymbol{)}\AgdaSpace{}%
\AgdaBound{n}\<%
\\
%
\>[4]\AgdaFunction{mapself}\AgdaSpace{}%
\AgdaInductiveConstructor{ε}\AgdaSpace{}%
\AgdaSymbol{=}\AgdaSpace{}%
\AgdaInductiveConstructor{ε}\<%
\\
%
\>[4]\AgdaFunction{mapself}\AgdaSpace{}%
\AgdaSymbol{(}\AgdaBound{xs}\AgdaSpace{}%
\AgdaOperator{\AgdaInductiveConstructor{-,}}\AgdaSpace{}%
\AgdaBound{x}\AgdaSymbol{)}\AgdaSpace{}%
\AgdaSymbol{=}\AgdaSpace{}%
\AgdaFunction{mapself}\AgdaSpace{}%
\AgdaBound{xs}\AgdaSpace{}%
\AgdaOperator{\AgdaInductiveConstructor{-,}}\AgdaSpace{}%
\AgdaFunction{map}\AgdaSpace{}%
\AgdaBound{f}\AgdaSpace{}%
\AgdaBound{x}\<%
\\
\>[0]\AgdaFunction{map}\AgdaSpace{}%
\AgdaSymbol{\{}\AgdaArgument{d}\AgdaSpace{}%
\AgdaSymbol{=}\AgdaSpace{}%
\AgdaInductiveConstructor{compu}\AgdaSymbol{\}}\AgdaSpace{}%
\AgdaBound{f}\AgdaSpace{}%
\AgdaSymbol{(}\AgdaInductiveConstructor{var}\AgdaSpace{}%
\AgdaBound{x}\AgdaSymbol{)}\AgdaSpace{}%
\AgdaSymbol{=}\AgdaSpace{}%
\AgdaInductiveConstructor{var}\AgdaSpace{}%
\AgdaBound{x}\<%
\\
\>[0]\AgdaFunction{map}\AgdaSpace{}%
\AgdaSymbol{\{}\AgdaArgument{d}\AgdaSpace{}%
\AgdaSymbol{=}\AgdaSpace{}%
\AgdaInductiveConstructor{compu}\AgdaSymbol{\}}\AgdaSpace{}%
\AgdaBound{f}\AgdaSpace{}%
\AgdaSymbol{(}\AgdaInductiveConstructor{elim}\AgdaSpace{}%
\AgdaBound{e}\AgdaSpace{}%
\AgdaBound{x}\AgdaSymbol{)}\AgdaSpace{}%
\AgdaSymbol{=}\AgdaSpace{}%
\AgdaInductiveConstructor{elim}\AgdaSpace{}%
\AgdaSymbol{(}\AgdaFunction{map}\AgdaSpace{}%
\AgdaBound{f}\AgdaSpace{}%
\AgdaBound{e}\AgdaSymbol{)}\AgdaSpace{}%
\AgdaSymbol{(}\AgdaFunction{map}\AgdaSpace{}%
\AgdaBound{f}\AgdaSpace{}%
\AgdaBound{x}\AgdaSymbol{)}\<%
\\
\>[0]\AgdaFunction{map}\AgdaSpace{}%
\AgdaSymbol{\{}\AgdaArgument{d}\AgdaSpace{}%
\AgdaSymbol{=}\AgdaSpace{}%
\AgdaInductiveConstructor{compu}\AgdaSymbol{\}}\AgdaSpace{}%
\AgdaBound{f}\AgdaSpace{}%
\AgdaSymbol{(}\AgdaBound{t}\AgdaSpace{}%
\AgdaOperator{\AgdaInductiveConstructor{∷}}\AgdaSpace{}%
\AgdaBound{T}\AgdaSymbol{)}\AgdaSpace{}%
\AgdaSymbol{=}\AgdaSpace{}%
\AgdaFunction{map}\AgdaSpace{}%
\AgdaBound{f}\AgdaSpace{}%
\AgdaBound{t}\AgdaSpace{}%
\AgdaOperator{\AgdaInductiveConstructor{∷}}\AgdaSpace{}%
\AgdaFunction{map}\AgdaSpace{}%
\AgdaBound{f}\AgdaSpace{}%
\AgdaBound{T}\<%
\end{code}
}

Finally we are able to generate a term from an expression and some opening
of the appropriate environment. Most of the cases recurse structurally, while
variables are simply opened to encompassing scope, however we can note here
how it is that we process our svar instantiations

\begin{code}%
\>[0]\AgdaFunction{toTerm}%
\>[8]\AgdaSymbol{:}\AgdaSpace{}%
\AgdaSymbol{(}\AgdaGeneralizable{γ}\AgdaSpace{}%
\AgdaOperator{\AgdaFunction{⊗}}\AgdaSpace{}%
\AgdaGeneralizable{p}\AgdaSymbol{)}\AgdaSpace{}%
\AgdaOperator{\AgdaDatatype{-Env}}\AgdaSpace{}%
\AgdaSymbol{→}\AgdaSpace{}%
\AgdaFunction{Expr}\AgdaSpace{}%
\AgdaGeneralizable{p}\AgdaSpace{}%
\AgdaGeneralizable{d}\AgdaSpace{}%
\AgdaGeneralizable{γ'}\AgdaSpace{}%
\AgdaSymbol{→}\AgdaSpace{}%
\AgdaFunction{Term}\AgdaSpace{}%
\AgdaGeneralizable{d}\AgdaSpace{}%
\AgdaSymbol{(}\AgdaGeneralizable{γ}\AgdaSpace{}%
\AgdaOperator{\AgdaPrimitive{+}}\AgdaSpace{}%
\AgdaGeneralizable{γ'}\AgdaSymbol{)}\<%
\\
\>[0]\AgdaComment{-- ...}\<%
\\
\>[0]\AgdaFunction{toTerm}\AgdaSpace{}%
\AgdaSymbol{\{}\AgdaArgument{γ}\AgdaSpace{}%
\AgdaSymbol{=}\AgdaSpace{}%
\AgdaBound{γ}\AgdaSymbol{\}}\AgdaSpace{}%
\AgdaSymbol{\{}\AgdaArgument{d}\AgdaSpace{}%
\AgdaSymbol{=}\AgdaSpace{}%
\AgdaInductiveConstructor{const}\AgdaSymbol{\}}\AgdaSpace{}%
\AgdaSymbol{\{}\AgdaArgument{γ'}\AgdaSpace{}%
\AgdaSymbol{=}\AgdaSpace{}%
\AgdaBound{γ'}\AgdaSymbol{\}}\AgdaSpace{}%
\AgdaBound{penv}\AgdaSpace{}%
\AgdaSymbol{(}\AgdaBound{ξ}\AgdaSpace{}%
\AgdaOperator{\AgdaInductiveConstructor{/}}\AgdaSpace{}%
\AgdaBound{σ}\AgdaSymbol{)}\<%
\\
\>[0][@{}l@{\AgdaIndent{0}}]%
\>[2]\AgdaSymbol{=}%
\>[730I]\AgdaKeyword{let}\AgdaSpace{}%
\AgdaBound{σ'}%
\>[15]\AgdaSymbol{=}\AgdaSpace{}%
\AgdaFunction{map-toTerm}\AgdaSpace{}%
\AgdaBound{σ}\AgdaSpace{}%
\AgdaBound{penv}%
\>[36]\AgdaKeyword{in}\<%
\\
\>[.][@{}l@{}]\<[730I]%
\>[4]\AgdaKeyword{let}\AgdaSpace{}%
\AgdaBound{id-fv}%
\>[15]\AgdaSymbol{=}\AgdaSpace{}%
\AgdaFunction{id}\AgdaSpace{}%
\AgdaOperator{\AgdaFunction{⟨σ}}\AgdaSpace{}%
\AgdaSymbol{(}\AgdaBound{γ}\AgdaSpace{}%
\AgdaOperator{\AgdaFunction{◃}}\AgdaSpace{}%
\AgdaBound{γ'}\AgdaSymbol{)}%
\>[36]\AgdaKeyword{in}\<%
\\
\>[4][@{}l@{\AgdaIndent{0}}]%
\>[6]\AgdaSymbol{(}\AgdaBound{ξ}\AgdaSpace{}%
\AgdaOperator{\AgdaFunction{‼}}\AgdaSpace{}%
\AgdaBound{penv}\AgdaSymbol{)}\AgdaSpace{}%
\AgdaOperator{\AgdaFunction{/term}}\AgdaSpace{}%
\AgdaSymbol{((}\AgdaBound{id-fv}\AgdaSpace{}%
\AgdaOperator{\AgdaFunction{++}}\AgdaSpace{}%
\AgdaBound{σ'}\AgdaSymbol{))}\<%
\\
\>[0]\AgdaComment{-- ...}\<%
\end{code}

That is to say, we first resolve the substitution of expressions to one of
terms (here we must inline a specialisation of map in order to satisfy Agda's
termination checker) before extending it with the identity substitution for
all the free variables and finally performing the substitution on the term
taken from the environment.

In short, we substitute the bound variables in the term but ensure not to
affect any free variables that may exist.

\hide{
\begin{code}%
\>[0][@{}l@{\AgdaIndent{1}}]%
\>[4]\AgdaKeyword{where}\<%
\\
\>[4][@{}l@{\AgdaIndent{0}}]%
\>[6]\AgdaFunction{map-toTerm}\AgdaSpace{}%
\AgdaSymbol{:}\AgdaSpace{}%
\AgdaSymbol{∀}\AgdaSpace{}%
\AgdaSymbol{\{}\AgdaBound{γ}\AgdaSymbol{\}}\AgdaSpace{}%
\AgdaSymbol{→}\AgdaSpace{}%
\AgdaGeneralizable{δ'}\AgdaSpace{}%
\AgdaOperator{\AgdaFunction{⇒[}}\AgdaSpace{}%
\AgdaFunction{Expr}\AgdaSpace{}%
\AgdaGeneralizable{p}\AgdaSpace{}%
\AgdaGeneralizable{d}\AgdaSpace{}%
\AgdaOperator{\AgdaFunction{]}}\AgdaSpace{}%
\AgdaBound{γ'}\AgdaSpace{}%
\AgdaSymbol{→}\AgdaSpace{}%
\AgdaSymbol{((}\AgdaBound{γ}\AgdaSpace{}%
\AgdaOperator{\AgdaFunction{⊗}}\AgdaSpace{}%
\AgdaGeneralizable{p}\AgdaSymbol{)}\AgdaSpace{}%
\AgdaOperator{\AgdaDatatype{-Env}}\AgdaSymbol{)}%
\>[65]\AgdaSymbol{→}\AgdaSpace{}%
\AgdaGeneralizable{δ'}\AgdaSpace{}%
\AgdaOperator{\AgdaFunction{⇒[}}\AgdaSpace{}%
\AgdaFunction{Term}\AgdaSpace{}%
\AgdaGeneralizable{d}\AgdaSpace{}%
\AgdaOperator{\AgdaFunction{]}}\AgdaSpace{}%
\AgdaSymbol{(}\AgdaBound{γ}\AgdaSpace{}%
\AgdaOperator{\AgdaPrimitive{+}}\AgdaSpace{}%
\AgdaBound{γ'}\AgdaSymbol{)}\<%
\\
%
\>[6]\AgdaFunction{map-toTerm}\AgdaSpace{}%
\AgdaInductiveConstructor{ε}\AgdaSpace{}%
\AgdaBound{env}\AgdaSpace{}%
\AgdaSymbol{=}\AgdaSpace{}%
\AgdaInductiveConstructor{ε}\<%
\\
%
\>[6]\AgdaFunction{map-toTerm}\AgdaSpace{}%
\AgdaSymbol{(}\AgdaBound{σ}\AgdaSpace{}%
\AgdaOperator{\AgdaInductiveConstructor{-,}}\AgdaSpace{}%
\AgdaBound{x}\AgdaSymbol{)}\AgdaSpace{}%
\AgdaBound{env}\AgdaSpace{}%
\AgdaSymbol{=}\AgdaSpace{}%
\AgdaFunction{map-toTerm}\AgdaSpace{}%
\AgdaBound{σ}\AgdaSpace{}%
\AgdaBound{env}\AgdaSpace{}%
\AgdaOperator{\AgdaInductiveConstructor{-,}}\AgdaSpace{}%
\AgdaFunction{toTerm}\AgdaSpace{}%
\AgdaBound{env}\AgdaSpace{}%
\AgdaBound{x}\<%
\\
\>[0]\AgdaFunction{toTerm}\AgdaSpace{}%
\AgdaSymbol{\{}\AgdaArgument{d}\AgdaSpace{}%
\AgdaSymbol{=}\AgdaSpace{}%
\AgdaInductiveConstructor{const}\AgdaSymbol{\}}\AgdaSpace{}%
\AgdaBound{penv}\AgdaSpace{}%
\AgdaSymbol{(}\AgdaInductiveConstructor{`}\AgdaSpace{}%
\AgdaBound{x}\AgdaSymbol{)}%
\>[35]\AgdaSymbol{=}\AgdaSpace{}%
\AgdaInductiveConstructor{`}\AgdaSpace{}%
\AgdaBound{x}\<%
\\
\>[0]\AgdaFunction{toTerm}\AgdaSpace{}%
\AgdaSymbol{\{}\AgdaArgument{d}\AgdaSpace{}%
\AgdaSymbol{=}\AgdaSpace{}%
\AgdaInductiveConstructor{const}\AgdaSymbol{\}}\AgdaSpace{}%
\AgdaBound{penv}\AgdaSpace{}%
\AgdaSymbol{(}\AgdaBound{s}\AgdaSpace{}%
\AgdaOperator{\AgdaInductiveConstructor{∙}}\AgdaSpace{}%
\AgdaBound{t}\AgdaSymbol{)}%
\>[35]\AgdaSymbol{=}\AgdaSpace{}%
\AgdaSymbol{(}\AgdaFunction{toTerm}\AgdaSpace{}%
\AgdaBound{penv}\AgdaSpace{}%
\AgdaBound{s}\AgdaSymbol{)}\AgdaSpace{}%
\AgdaOperator{\AgdaInductiveConstructor{∙}}\AgdaSpace{}%
\AgdaSymbol{(}\AgdaFunction{toTerm}\AgdaSpace{}%
\AgdaBound{penv}\AgdaSpace{}%
\AgdaBound{t}\AgdaSymbol{)}\<%
\\
\>[0]\AgdaFunction{toTerm}\AgdaSpace{}%
\AgdaSymbol{\{}\AgdaArgument{d}\AgdaSpace{}%
\AgdaSymbol{=}\AgdaSpace{}%
\AgdaInductiveConstructor{const}\AgdaSymbol{\}}\AgdaSpace{}%
\AgdaBound{penv}\AgdaSpace{}%
\AgdaSymbol{(}\AgdaInductiveConstructor{bind}\AgdaSpace{}%
\AgdaBound{t}\AgdaSymbol{)}%
\>[35]\AgdaSymbol{=}\AgdaSpace{}%
\AgdaInductiveConstructor{bind}%
\>[43]\AgdaSymbol{(}\AgdaFunction{toTerm}\AgdaSpace{}%
\AgdaBound{penv}\AgdaSpace{}%
\AgdaBound{t}\AgdaSymbol{)}\<%
\\
\>[0]\AgdaFunction{toTerm}\AgdaSpace{}%
\AgdaSymbol{\{}\AgdaArgument{d}\AgdaSpace{}%
\AgdaSymbol{=}\AgdaSpace{}%
\AgdaInductiveConstructor{const}\AgdaSymbol{\}}\AgdaSpace{}%
\AgdaBound{penv}\AgdaSpace{}%
\AgdaSymbol{(}\AgdaInductiveConstructor{thunk}\AgdaSpace{}%
\AgdaBound{x}\AgdaSymbol{)}%
\>[35]\AgdaSymbol{=}\AgdaSpace{}%
\AgdaFunction{↠}\AgdaSpace{}%
\AgdaSymbol{(}\AgdaFunction{toTerm}\AgdaSpace{}%
\AgdaBound{penv}\AgdaSpace{}%
\AgdaBound{x}\AgdaSymbol{)}\<%
\\
\>[0]\AgdaFunction{toTerm}\AgdaSpace{}%
\AgdaSymbol{\{}\AgdaArgument{γ}\AgdaSpace{}%
\AgdaSymbol{=}\AgdaSpace{}%
\AgdaBound{γ}\AgdaSymbol{\}}\AgdaSpace{}%
\AgdaSymbol{\{}\AgdaArgument{d}\AgdaSpace{}%
\AgdaSymbol{=}\AgdaSpace{}%
\AgdaInductiveConstructor{compu}\AgdaSymbol{\}}\AgdaSpace{}%
\AgdaSymbol{\{}\AgdaArgument{γ'}\AgdaSpace{}%
\AgdaSymbol{=}%
\>[836I]\AgdaBound{γ'}\AgdaSymbol{\}}\AgdaSpace{}%
\AgdaBound{penv}\AgdaSpace{}%
\AgdaSymbol{(}\AgdaInductiveConstructor{var}\AgdaSpace{}%
\AgdaBound{x}\AgdaSymbol{)}\<%
\\
\>[836I][@{}l@{\AgdaIndent{0}}]%
\>[35]\AgdaSymbol{=}\AgdaSpace{}%
\AgdaInductiveConstructor{var}\AgdaSpace{}%
\AgdaSymbol{(}\AgdaBound{γ}\AgdaSpace{}%
\AgdaOperator{\AgdaFunction{⊗var}}\AgdaSpace{}%
\AgdaBound{x}\AgdaSymbol{)}\<%
\\
\>[0]\AgdaFunction{toTerm}\AgdaSpace{}%
\AgdaSymbol{\{}\AgdaArgument{d}\AgdaSpace{}%
\AgdaSymbol{=}\AgdaSpace{}%
\AgdaInductiveConstructor{compu}\AgdaSymbol{\}}\AgdaSpace{}%
\AgdaBound{penv}\AgdaSpace{}%
\AgdaSymbol{(}\AgdaInductiveConstructor{elim}\AgdaSpace{}%
\AgdaBound{e}\AgdaSpace{}%
\AgdaBound{s}\AgdaSymbol{)}\AgdaSpace{}%
\AgdaSymbol{=}\AgdaSpace{}%
\AgdaInductiveConstructor{elim}\AgdaSpace{}%
\AgdaSymbol{(}\AgdaFunction{toTerm}\AgdaSpace{}%
\AgdaBound{penv}\AgdaSpace{}%
\AgdaBound{e}\AgdaSymbol{)}\AgdaSpace{}%
\AgdaSymbol{(}\AgdaFunction{toTerm}\AgdaSpace{}%
\AgdaBound{penv}\AgdaSpace{}%
\AgdaBound{s}\AgdaSymbol{)}\<%
\\
\>[0]\AgdaFunction{toTerm}\AgdaSpace{}%
\AgdaSymbol{\{}\AgdaArgument{d}\AgdaSpace{}%
\AgdaSymbol{=}\AgdaSpace{}%
\AgdaInductiveConstructor{compu}\AgdaSymbol{\}}\AgdaSpace{}%
\AgdaBound{penv}\AgdaSpace{}%
\AgdaSymbol{(}\AgdaBound{t}\AgdaSpace{}%
\AgdaOperator{\AgdaInductiveConstructor{∷}}\AgdaSpace{}%
\AgdaBound{T}\AgdaSymbol{)}%
\>[35]\AgdaSymbol{=}\AgdaSpace{}%
\AgdaFunction{toTerm}\AgdaSpace{}%
\AgdaBound{penv}\AgdaSpace{}%
\AgdaBound{t}\AgdaSpace{}%
\AgdaOperator{\AgdaInductiveConstructor{∷}}\AgdaSpace{}%
\AgdaFunction{toTerm}\AgdaSpace{}%
\AgdaBound{penv}\AgdaSpace{}%
\AgdaBound{T}\<%
\end{code}
}

\section{Representing Judgements and Rules}

\hide{
\begin{code}%
\>[0]\AgdaSymbol{\{-\#}\AgdaSpace{}%
\AgdaKeyword{OPTIONS}\AgdaSpace{}%
\AgdaPragma{--rewriting}\AgdaSpace{}%
\AgdaSymbol{\#-\}}\<%
\\
\>[0]\AgdaKeyword{module}\AgdaSpace{}%
\AgdaModule{Rules}\AgdaSpace{}%
\AgdaKeyword{where}\<%
\end{code}
}

\begin{code}%
\>[0]\AgdaKeyword{open}\AgdaSpace{}%
\AgdaKeyword{import}\AgdaSpace{}%
\AgdaModule{CoreLanguage}\<%
\\
\>[0]\AgdaKeyword{open}\AgdaSpace{}%
\AgdaKeyword{import}\AgdaSpace{}%
\AgdaModule{Thinning}\AgdaSpace{}%
\AgdaKeyword{using}\AgdaSpace{}%
\AgdaSymbol{(}\AgdaOperator{\AgdaDatatype{\AgdaUnderscore{}⊑\AgdaUnderscore{}}}\AgdaSymbol{;}\AgdaSpace{}%
\AgdaFunction{ι}\AgdaSymbol{;}\AgdaSpace{}%
\AgdaOperator{\AgdaFunction{\AgdaUnderscore{}++\AgdaUnderscore{}}}\AgdaSymbol{)}\<%
\\
\>[0]\AgdaKeyword{open}\AgdaSpace{}%
\AgdaKeyword{import}\AgdaSpace{}%
\AgdaModule{Pattern}\AgdaSpace{}%
\AgdaKeyword{using}\AgdaSpace{}%
\AgdaSymbol{(}\AgdaDatatype{Pattern}\AgdaSymbol{;}\AgdaSpace{}%
\AgdaDatatype{svar}\AgdaSymbol{;}\AgdaSpace{}%
\AgdaInductiveConstructor{bind}\AgdaSymbol{;}\AgdaSpace{}%
\AgdaOperator{\AgdaInductiveConstructor{\AgdaUnderscore{}∙}}\AgdaSymbol{;}\AgdaSpace{}%
\AgdaOperator{\AgdaInductiveConstructor{∙\AgdaUnderscore{}}}\AgdaSymbol{;}\AgdaSpace{}%
\AgdaInductiveConstructor{place}\AgdaSymbol{;}\AgdaSpace{}%
\AgdaInductiveConstructor{⋆}\AgdaSymbol{;}\AgdaSpace{}%
\AgdaOperator{\AgdaInductiveConstructor{\AgdaUnderscore{}∙\AgdaUnderscore{}}}\AgdaSymbol{;}\AgdaSpace{}%
\AgdaInductiveConstructor{`}\AgdaSymbol{;}\AgdaSpace{}%
\AgdaInductiveConstructor{⊥}\AgdaSymbol{;}\AgdaSpace{}%
\AgdaOperator{\AgdaDatatype{\AgdaUnderscore{}-Env}}\AgdaSymbol{;}\AgdaSpace{}%
\AgdaFunction{match}\AgdaSymbol{;}\AgdaSpace{}%
\AgdaOperator{\AgdaFunction{\AgdaUnderscore{}-\AgdaUnderscore{}}}\AgdaSymbol{;}\AgdaSpace{}%
\AgdaOperator{\AgdaFunction{\AgdaUnderscore{}⊗\AgdaUnderscore{}}}\AgdaSymbol{;}\AgdaSpace{}%
\AgdaOperator{\AgdaFunction{\AgdaUnderscore{}⊗svar\AgdaUnderscore{}}}\AgdaSymbol{)}\<%
\\
\>[0]\AgdaKeyword{open}\AgdaSpace{}%
\AgdaKeyword{import}\AgdaSpace{}%
\AgdaModule{Expression}\AgdaSpace{}%
\AgdaKeyword{using}\AgdaSpace{}%
\AgdaSymbol{(}\AgdaFunction{Expr}\AgdaSymbol{;}\AgdaSpace{}%
\AgdaOperator{\AgdaFunction{\AgdaUnderscore{}⊗expr\AgdaUnderscore{}}}\AgdaSymbol{)}\<%
\\
\>[0]\AgdaKeyword{open}\AgdaSpace{}%
\AgdaKeyword{import}\AgdaSpace{}%
\AgdaModule{Data.Product}\AgdaSpace{}%
\AgdaKeyword{using}\AgdaSpace{}%
\AgdaSymbol{(}\AgdaOperator{\AgdaFunction{\AgdaUnderscore{}×\AgdaUnderscore{}}}\AgdaSymbol{;}\AgdaSpace{}%
\AgdaOperator{\AgdaInductiveConstructor{\AgdaUnderscore{},\AgdaUnderscore{}}}\AgdaSymbol{;}\AgdaSpace{}%
\AgdaField{proj₁}\AgdaSymbol{;}\AgdaSpace{}%
\AgdaFunction{Σ-syntax}\AgdaSymbol{)}\<%
\\
\>[0]\AgdaKeyword{open}\AgdaSpace{}%
\AgdaKeyword{import}\AgdaSpace{}%
\AgdaModule{Data.List}\AgdaSpace{}%
\AgdaKeyword{using}\AgdaSpace{}%
\AgdaSymbol{(}\AgdaDatatype{List}\AgdaSymbol{)}\<%
\\
\>[0]\AgdaKeyword{open}\AgdaSpace{}%
\AgdaKeyword{import}\AgdaSpace{}%
\AgdaModule{Data.Char}\<%
\\
\>[0]\AgdaKeyword{open}\AgdaSpace{}%
\AgdaKeyword{import}\AgdaSpace{}%
\AgdaModule{Data.Nat}\AgdaSpace{}%
\AgdaKeyword{using}\AgdaSpace{}%
\AgdaSymbol{(}\AgdaDatatype{ℕ}\AgdaSymbol{;}\AgdaSpace{}%
\AgdaInductiveConstructor{suc}\AgdaSymbol{;}\AgdaSpace{}%
\AgdaOperator{\AgdaPrimitive{\AgdaUnderscore{}+\AgdaUnderscore{}}}\AgdaSymbol{)}\<%
\\
\>[0]\AgdaKeyword{open}\AgdaSpace{}%
\AgdaKeyword{import}\AgdaSpace{}%
\AgdaModule{Data.Maybe}\AgdaSpace{}%
\AgdaKeyword{using}\AgdaSpace{}%
\AgdaSymbol{(}\AgdaDatatype{Maybe}\AgdaSymbol{;}\AgdaSpace{}%
\AgdaInductiveConstructor{just}\AgdaSymbol{;}\AgdaSpace{}%
\AgdaOperator{\AgdaFunction{\AgdaUnderscore{}>>=\AgdaUnderscore{}}}\AgdaSymbol{)}\<%
\\
\>[0]\AgdaKeyword{open}\AgdaSpace{}%
\AgdaKeyword{import}\AgdaSpace{}%
\AgdaModule{Relation.Binary.PropositionalEquality}\AgdaSpace{}%
\AgdaKeyword{using}\AgdaSpace{}%
\AgdaSymbol{(}\AgdaFunction{cong}\AgdaSymbol{;}\AgdaSpace{}%
\AgdaFunction{sym}\AgdaSymbol{;}\AgdaSpace{}%
\AgdaOperator{\AgdaDatatype{\AgdaUnderscore{}≡\AgdaUnderscore{}}}\AgdaSymbol{;}\AgdaSpace{}%
\AgdaInductiveConstructor{refl}\AgdaSymbol{)}\<%
\end{code}

\begin{code}%
\>[0]\AgdaKeyword{private}\<%
\\
\>[0][@{}l@{\AgdaIndent{0}}]%
\>[2]\AgdaKeyword{variable}\<%
\\
\>[2][@{}l@{\AgdaIndent{0}}]%
\>[4]\AgdaGeneralizable{δ}\AgdaSpace{}%
\AgdaSymbol{:}\AgdaSpace{}%
\AgdaFunction{Scope}\<%
\\
%
\>[4]\AgdaGeneralizable{δ'}\AgdaSpace{}%
\AgdaSymbol{:}\AgdaSpace{}%
\AgdaFunction{Scope}\<%
\\
%
\>[4]\AgdaGeneralizable{γ}\AgdaSpace{}%
\AgdaSymbol{:}\AgdaSpace{}%
\AgdaFunction{Scope}\<%
\\
%
\>[4]\AgdaGeneralizable{p}\AgdaSpace{}%
\AgdaSymbol{:}\AgdaSpace{}%
\AgdaDatatype{Pattern}\AgdaSpace{}%
\AgdaNumber{0}\<%
\\
%
\>[4]\AgdaGeneralizable{pᵍ}\AgdaSpace{}%
\AgdaSymbol{:}\AgdaSpace{}%
\AgdaDatatype{Pattern}\AgdaSpace{}%
\AgdaGeneralizable{γ}\<%
\\
%
\>[4]\AgdaGeneralizable{p`}\AgdaSpace{}%
\AgdaSymbol{:}\AgdaSpace{}%
\AgdaDatatype{Pattern}\AgdaSpace{}%
\AgdaSymbol{(}\AgdaInductiveConstructor{suc}\AgdaSpace{}%
\AgdaGeneralizable{γ}\AgdaSymbol{)}\<%
\\
%
\>[4]\AgdaGeneralizable{q`}\AgdaSpace{}%
\AgdaSymbol{:}\AgdaSpace{}%
\AgdaDatatype{Pattern}\AgdaSpace{}%
\AgdaNumber{0}\<%
\\
%
\>[4]\AgdaGeneralizable{q``}\AgdaSpace{}%
\AgdaSymbol{:}\AgdaSpace{}%
\AgdaDatatype{Pattern}\AgdaSpace{}%
\AgdaGeneralizable{δ}\<%
\\
%
\>[4]\AgdaGeneralizable{q}\AgdaSpace{}%
\AgdaSymbol{:}\AgdaSpace{}%
\AgdaDatatype{Pattern}\AgdaSpace{}%
\AgdaNumber{0}\<%
\\
%
\>[4]\AgdaGeneralizable{q'}\AgdaSpace{}%
\AgdaSymbol{:}\AgdaSpace{}%
\AgdaDatatype{Pattern}\AgdaSpace{}%
\AgdaNumber{0}\<%
\\
%
\\[\AgdaEmptyExtraSkip]%
\>[0]\AgdaKeyword{data}\AgdaSpace{}%
\AgdaDatatype{Prem}\AgdaSpace{}%
\AgdaSymbol{(}\AgdaBound{p}\AgdaSpace{}%
\AgdaSymbol{:}\AgdaSpace{}%
\AgdaDatatype{Pattern}\AgdaSpace{}%
\AgdaGeneralizable{δ}\AgdaSymbol{)}\AgdaSpace{}%
\AgdaSymbol{(}\AgdaBound{q}\AgdaSpace{}%
\AgdaSymbol{:}\AgdaSpace{}%
\AgdaDatatype{Pattern}\AgdaSpace{}%
\AgdaGeneralizable{δ}\AgdaSymbol{)}\AgdaSpace{}%
\AgdaSymbol{(}\AgdaBound{γ}\AgdaSpace{}%
\AgdaSymbol{:}\AgdaSpace{}%
\AgdaFunction{Scope}\AgdaSymbol{)}\AgdaSpace{}%
\AgdaSymbol{:}\AgdaSpace{}%
\AgdaSymbol{(}\AgdaBound{p'}\AgdaSpace{}%
\AgdaSymbol{:}\AgdaSpace{}%
\AgdaDatatype{Pattern}\AgdaSpace{}%
\AgdaBound{γ}\AgdaSymbol{)}\AgdaSpace{}%
\AgdaSymbol{→}\AgdaSpace{}%
\AgdaSymbol{(}\AgdaBound{q'}\AgdaSpace{}%
\AgdaSymbol{:}\AgdaSpace{}%
\AgdaDatatype{Pattern}\AgdaSpace{}%
\AgdaBound{δ}\AgdaSymbol{)}\AgdaSpace{}%
\AgdaSymbol{→}\AgdaSpace{}%
\AgdaPrimitiveType{Set}\AgdaSpace{}%
\AgdaKeyword{where}\<%
\\
\>[0][@{}l@{\AgdaIndent{0}}]%
\>[3]\AgdaInductiveConstructor{type}%
\>[11]\AgdaSymbol{:}\AgdaSpace{}%
\AgdaSymbol{(}\AgdaBound{ξ}\AgdaSpace{}%
\AgdaSymbol{:}\AgdaSpace{}%
\AgdaDatatype{svar}\AgdaSpace{}%
\AgdaBound{q}\AgdaSpace{}%
\AgdaGeneralizable{δ'}\AgdaSymbol{)}\AgdaSpace{}%
\AgdaSymbol{→}\AgdaSpace{}%
\AgdaSymbol{(}\AgdaBound{θ}\AgdaSpace{}%
\AgdaSymbol{:}\AgdaSpace{}%
\AgdaGeneralizable{δ'}\AgdaSpace{}%
\AgdaOperator{\AgdaDatatype{⊑}}\AgdaSpace{}%
\AgdaBound{γ}\AgdaSymbol{)}\AgdaSpace{}%
\AgdaSymbol{→}\AgdaSpace{}%
\AgdaDatatype{Prem}\AgdaSpace{}%
\AgdaBound{p}\AgdaSpace{}%
\AgdaBound{q}\AgdaSpace{}%
\AgdaBound{γ}\AgdaSpace{}%
\AgdaSymbol{(}\AgdaInductiveConstructor{place}\AgdaSpace{}%
\AgdaBound{θ}\AgdaSymbol{)}\AgdaSpace{}%
\AgdaSymbol{(}\AgdaBound{q}\AgdaSpace{}%
\AgdaOperator{\AgdaFunction{-}}\AgdaSpace{}%
\AgdaBound{ξ}\AgdaSymbol{)}\<%
\\
%
\>[3]\AgdaOperator{\AgdaInductiveConstructor{\AgdaUnderscore{}∋'\AgdaUnderscore{}[\AgdaUnderscore{}]}}\AgdaSpace{}%
\AgdaSymbol{:}\AgdaSpace{}%
\AgdaSymbol{(}\AgdaBound{T}\AgdaSpace{}%
\AgdaSymbol{:}\AgdaSpace{}%
\AgdaFunction{Expr}\AgdaSpace{}%
\AgdaBound{p}\AgdaSpace{}%
\AgdaInductiveConstructor{const}\AgdaSpace{}%
\AgdaBound{γ}\AgdaSymbol{)}\AgdaSpace{}%
\AgdaSymbol{→}\AgdaSpace{}%
\AgdaSymbol{(}\AgdaBound{ξ}\AgdaSpace{}%
\AgdaSymbol{:}\AgdaSpace{}%
\AgdaDatatype{svar}\AgdaSpace{}%
\AgdaBound{q}\AgdaSpace{}%
\AgdaGeneralizable{δ'}\AgdaSymbol{)}\AgdaSpace{}%
\AgdaSymbol{→}\AgdaSpace{}%
\AgdaSymbol{(}\AgdaBound{θ}\AgdaSpace{}%
\AgdaSymbol{:}\AgdaSpace{}%
\AgdaGeneralizable{δ'}\AgdaSpace{}%
\AgdaOperator{\AgdaDatatype{⊑}}\AgdaSpace{}%
\AgdaBound{γ}\AgdaSymbol{)}%
\>[68]\AgdaSymbol{→}\AgdaSpace{}%
\AgdaDatatype{Prem}\AgdaSpace{}%
\AgdaBound{p}\AgdaSpace{}%
\AgdaBound{q}\AgdaSpace{}%
\AgdaBound{γ}\AgdaSpace{}%
\AgdaSymbol{(}\AgdaInductiveConstructor{place}\AgdaSpace{}%
\AgdaBound{θ}\AgdaSymbol{)}\AgdaSpace{}%
\AgdaSymbol{(}\AgdaBound{q}\AgdaSpace{}%
\AgdaOperator{\AgdaFunction{-}}\AgdaSpace{}%
\AgdaBound{ξ}\AgdaSymbol{)}\<%
\\
%
\>[3]\AgdaOperator{\AgdaInductiveConstructor{\AgdaUnderscore{}≡'\AgdaUnderscore{}}}%
\>[11]\AgdaSymbol{:}\AgdaSpace{}%
\AgdaFunction{Expr}\AgdaSpace{}%
\AgdaBound{p}\AgdaSpace{}%
\AgdaInductiveConstructor{const}\AgdaSpace{}%
\AgdaBound{γ}\AgdaSpace{}%
\AgdaSymbol{→}\AgdaSpace{}%
\AgdaFunction{Expr}\AgdaSpace{}%
\AgdaBound{p}\AgdaSpace{}%
\AgdaInductiveConstructor{const}\AgdaSpace{}%
\AgdaBound{γ}\AgdaSpace{}%
\AgdaSymbol{→}\AgdaSpace{}%
\AgdaDatatype{Prem}\AgdaSpace{}%
\AgdaBound{p}\AgdaSpace{}%
\AgdaBound{q}\AgdaSpace{}%
\AgdaBound{γ}\AgdaSpace{}%
\AgdaSymbol{(}\AgdaInductiveConstructor{`}\AgdaSpace{}%
\AgdaString{'⊤'}\AgdaSymbol{)}\AgdaSpace{}%
\AgdaBound{q}\<%
\\
%
\>[3]\AgdaInductiveConstructor{univ}%
\>[11]\AgdaSymbol{:}\AgdaSpace{}%
\AgdaFunction{Expr}\AgdaSpace{}%
\AgdaBound{p}\AgdaSpace{}%
\AgdaInductiveConstructor{const}\AgdaSpace{}%
\AgdaBound{γ}\AgdaSpace{}%
\AgdaSymbol{→}\AgdaSpace{}%
\AgdaDatatype{Prem}\AgdaSpace{}%
\AgdaBound{p}\AgdaSpace{}%
\AgdaBound{q}\AgdaSpace{}%
\AgdaBound{γ}\AgdaSpace{}%
\AgdaSymbol{(}\AgdaInductiveConstructor{`}\AgdaSpace{}%
\AgdaString{'⊤'}\AgdaSymbol{)}\AgdaSpace{}%
\AgdaBound{q}\<%
\\
%
\>[3]\AgdaOperator{\AgdaInductiveConstructor{\AgdaUnderscore{}⊢'\AgdaUnderscore{}}}%
\>[11]\AgdaSymbol{:}\AgdaSpace{}%
\AgdaFunction{Expr}\AgdaSpace{}%
\AgdaBound{p}\AgdaSpace{}%
\AgdaInductiveConstructor{const}\AgdaSpace{}%
\AgdaBound{γ}\AgdaSpace{}%
\AgdaSymbol{→}\AgdaSpace{}%
\AgdaDatatype{Prem}\AgdaSpace{}%
\AgdaBound{p}\AgdaSpace{}%
\AgdaBound{q}\AgdaSpace{}%
\AgdaSymbol{(}\AgdaInductiveConstructor{suc}\AgdaSpace{}%
\AgdaBound{γ}\AgdaSymbol{)}\AgdaSpace{}%
\AgdaGeneralizable{p`}\AgdaSpace{}%
\AgdaGeneralizable{q``}\AgdaSpace{}%
\AgdaSymbol{→}\AgdaSpace{}%
\AgdaDatatype{Prem}\AgdaSpace{}%
\AgdaBound{p}\AgdaSpace{}%
\AgdaBound{q}\AgdaSpace{}%
\AgdaBound{γ}\AgdaSpace{}%
\AgdaSymbol{(}\AgdaInductiveConstructor{bind}\AgdaSpace{}%
\AgdaGeneralizable{p`}\AgdaSymbol{)}\AgdaSpace{}%
\AgdaGeneralizable{q``}\<%
\\
%
\\[\AgdaEmptyExtraSkip]%
%
\\[\AgdaEmptyExtraSkip]%
%
\\[\AgdaEmptyExtraSkip]%
\>[0]\AgdaFunction{helper}\AgdaSpace{}%
\AgdaSymbol{:}\AgdaSpace{}%
\AgdaSymbol{∀}\AgdaSpace{}%
\AgdaSymbol{\{}\AgdaBound{δ'}\AgdaSymbol{\}}\AgdaSpace{}%
\AgdaSymbol{(}\AgdaBound{δ}\AgdaSpace{}%
\AgdaSymbol{:}\AgdaSpace{}%
\AgdaFunction{Scope}\AgdaSymbol{)}\AgdaSpace{}%
\AgdaSymbol{→}\AgdaSpace{}%
\AgdaSymbol{(}\AgdaBound{q}\AgdaSpace{}%
\AgdaSymbol{:}\AgdaSpace{}%
\AgdaDatatype{Pattern}\AgdaSpace{}%
\AgdaGeneralizable{γ}\AgdaSymbol{)}\AgdaSpace{}%
\AgdaSymbol{→}\AgdaSpace{}%
\AgdaSymbol{(}\AgdaBound{ξ}\AgdaSpace{}%
\AgdaSymbol{:}\AgdaSpace{}%
\AgdaDatatype{svar}\AgdaSpace{}%
\AgdaBound{q}\AgdaSpace{}%
\AgdaBound{δ'}\AgdaSymbol{)}\AgdaSpace{}%
\AgdaSymbol{→}\AgdaSpace{}%
\AgdaSymbol{(}\AgdaBound{δ}\AgdaSpace{}%
\AgdaOperator{\AgdaFunction{⊗}}\AgdaSpace{}%
\AgdaBound{q}\AgdaSymbol{)}\AgdaSpace{}%
\AgdaOperator{\AgdaFunction{-}}\AgdaSpace{}%
\AgdaSymbol{(}\AgdaBound{ξ}\AgdaSpace{}%
\AgdaOperator{\AgdaFunction{⊗svar}}\AgdaSpace{}%
\AgdaBound{δ}\AgdaSymbol{)}\AgdaSpace{}%
\AgdaOperator{\AgdaDatatype{≡}}\AgdaSpace{}%
\AgdaBound{δ}\AgdaSpace{}%
\AgdaOperator{\AgdaFunction{⊗}}\AgdaSpace{}%
\AgdaSymbol{(}\AgdaBound{q}\AgdaSpace{}%
\AgdaOperator{\AgdaFunction{-}}\AgdaSpace{}%
\AgdaBound{ξ}\AgdaSymbol{)}\<%
\\
\>[0]\AgdaFunction{helper}\AgdaSpace{}%
\AgdaBound{δ}\AgdaSpace{}%
\AgdaSymbol{(}\AgdaBound{s}\AgdaSpace{}%
\AgdaOperator{\AgdaInductiveConstructor{∙}}\AgdaSpace{}%
\AgdaBound{t}\AgdaSymbol{)}%
\>[18]\AgdaSymbol{(}\AgdaBound{ξ}\AgdaSpace{}%
\AgdaOperator{\AgdaInductiveConstructor{∙}}\AgdaSymbol{)}%
\>[28]\AgdaSymbol{=}\AgdaSpace{}%
\AgdaFunction{cong}\AgdaSpace{}%
\AgdaSymbol{(λ}\AgdaSpace{}%
\AgdaBound{x}\AgdaSpace{}%
\AgdaSymbol{→}\AgdaSpace{}%
\AgdaOperator{\AgdaInductiveConstructor{Pattern.\AgdaUnderscore{}∙\AgdaUnderscore{}}}\AgdaSpace{}%
\AgdaBound{x}\AgdaSpace{}%
\AgdaSymbol{(}\AgdaBound{δ}\AgdaSpace{}%
\AgdaOperator{\AgdaFunction{⊗}}\AgdaSpace{}%
\AgdaBound{t}\AgdaSymbol{))}\AgdaSpace{}%
\AgdaSymbol{(}\AgdaFunction{helper}\AgdaSpace{}%
\AgdaBound{δ}\AgdaSpace{}%
\AgdaBound{s}\AgdaSpace{}%
\AgdaBound{ξ}\AgdaSymbol{)}\<%
\\
\>[0]\AgdaFunction{helper}\AgdaSpace{}%
\AgdaBound{δ}\AgdaSpace{}%
\AgdaSymbol{(}\AgdaBound{s}\AgdaSpace{}%
\AgdaOperator{\AgdaInductiveConstructor{∙}}\AgdaSpace{}%
\AgdaBound{t}\AgdaSymbol{)}\AgdaSpace{}%
\AgdaSymbol{(}\AgdaOperator{\AgdaInductiveConstructor{∙}}\AgdaSpace{}%
\AgdaBound{ξ}\AgdaSymbol{)}%
\>[28]\AgdaSymbol{=}\AgdaSpace{}%
\AgdaFunction{cong}\AgdaSpace{}%
\AgdaSymbol{(}\AgdaOperator{\AgdaInductiveConstructor{Pattern.\AgdaUnderscore{}∙\AgdaUnderscore{}}}\AgdaSpace{}%
\AgdaSymbol{(}\AgdaBound{δ}\AgdaSpace{}%
\AgdaOperator{\AgdaFunction{⊗}}\AgdaSpace{}%
\AgdaBound{s}\AgdaSymbol{))}\AgdaSpace{}%
\AgdaSymbol{(}\AgdaFunction{helper}\AgdaSpace{}%
\AgdaBound{δ}\AgdaSpace{}%
\AgdaBound{t}\AgdaSpace{}%
\AgdaBound{ξ}\AgdaSymbol{)}\<%
\\
\>[0]\AgdaFunction{helper}\AgdaSpace{}%
\AgdaBound{δ}\AgdaSpace{}%
\AgdaSymbol{(}\AgdaInductiveConstructor{bind}\AgdaSpace{}%
\AgdaBound{q}\AgdaSymbol{)}\AgdaSpace{}%
\AgdaSymbol{(}\AgdaInductiveConstructor{bind}\AgdaSpace{}%
\AgdaBound{ξ}\AgdaSymbol{)}%
\>[28]\AgdaSymbol{=}\AgdaSpace{}%
\AgdaFunction{cong}\AgdaSpace{}%
\AgdaInductiveConstructor{bind}\AgdaSpace{}%
\AgdaSymbol{(}\AgdaFunction{helper}\AgdaSpace{}%
\AgdaBound{δ}\AgdaSpace{}%
\AgdaBound{q}\AgdaSpace{}%
\AgdaBound{ξ}\AgdaSymbol{)}\<%
\\
\>[0]\AgdaFunction{helper}\AgdaSpace{}%
\AgdaBound{δ}\AgdaSpace{}%
\AgdaSymbol{(}\AgdaInductiveConstructor{place}\AgdaSpace{}%
\AgdaBound{x}\AgdaSymbol{)}\AgdaSpace{}%
\AgdaInductiveConstructor{⋆}%
\>[28]\AgdaSymbol{=}\AgdaSpace{}%
\AgdaInductiveConstructor{refl}\<%
\\
%
\\[\AgdaEmptyExtraSkip]%
\>[0]\AgdaFunction{⊗premise}\AgdaSpace{}%
\AgdaSymbol{:}\AgdaSpace{}%
\AgdaSymbol{∀}\AgdaSpace{}%
\AgdaSymbol{\{}\AgdaBound{p'}%
\>[302I]\AgdaSymbol{:}\AgdaSpace{}%
\AgdaDatatype{Pattern}\AgdaSpace{}%
\AgdaGeneralizable{γ}\AgdaSymbol{\}}\AgdaSpace{}%
\AgdaSymbol{→}\<%
\\
\>[.][@{}l@{}]\<[302I]%
\>[17]\AgdaSymbol{(}\AgdaBound{δ}\AgdaSpace{}%
\AgdaSymbol{:}\AgdaSpace{}%
\AgdaFunction{Scope}\AgdaSymbol{)}\AgdaSpace{}%
\AgdaSymbol{→}\<%
\\
%
\>[17]\AgdaDatatype{Prem}\AgdaSpace{}%
\AgdaGeneralizable{p}\AgdaSpace{}%
\AgdaGeneralizable{q}\AgdaSpace{}%
\AgdaGeneralizable{γ}\AgdaSpace{}%
\AgdaBound{p'}\AgdaSpace{}%
\AgdaGeneralizable{q'}\AgdaSpace{}%
\AgdaSymbol{→}\<%
\\
%
\>[17]\AgdaDatatype{Prem}\AgdaSpace{}%
\AgdaSymbol{(}\AgdaBound{δ}\AgdaSpace{}%
\AgdaOperator{\AgdaFunction{⊗}}\AgdaSpace{}%
\AgdaGeneralizable{p}\AgdaSymbol{)}\AgdaSpace{}%
\AgdaSymbol{(}\AgdaBound{δ}\AgdaSpace{}%
\AgdaOperator{\AgdaFunction{⊗}}\AgdaSpace{}%
\AgdaGeneralizable{q}\AgdaSymbol{)}\AgdaSpace{}%
\AgdaSymbol{(}\AgdaBound{δ}\AgdaSpace{}%
\AgdaOperator{\AgdaPrimitive{+}}\AgdaSpace{}%
\AgdaGeneralizable{γ}\AgdaSymbol{)}\AgdaSpace{}%
\AgdaSymbol{(}\AgdaBound{δ}\AgdaSpace{}%
\AgdaOperator{\AgdaFunction{⊗}}\AgdaSpace{}%
\AgdaBound{p'}\AgdaSymbol{)}%
\>[56]\AgdaSymbol{(}\AgdaBound{δ}\AgdaSpace{}%
\AgdaOperator{\AgdaFunction{⊗}}\AgdaSpace{}%
\AgdaGeneralizable{q'}\AgdaSymbol{)}\<%
\\
\>[0]\AgdaFunction{⊗premise}\AgdaSpace{}%
\AgdaSymbol{\{}\AgdaArgument{q}\AgdaSpace{}%
\AgdaSymbol{=}\AgdaSpace{}%
\AgdaBound{q}\AgdaSymbol{\}}\AgdaSpace{}%
\AgdaBound{δ}\AgdaSpace{}%
\AgdaSymbol{(}\AgdaInductiveConstructor{type}\AgdaSpace{}%
\AgdaBound{ξ}\AgdaSpace{}%
\AgdaBound{θ}\AgdaSymbol{)}%
\>[34]\AgdaKeyword{rewrite}\AgdaSpace{}%
\AgdaFunction{sym}\AgdaSpace{}%
\AgdaSymbol{(}\AgdaFunction{helper}\AgdaSpace{}%
\AgdaBound{δ}\AgdaSpace{}%
\AgdaBound{q}\AgdaSpace{}%
\AgdaBound{ξ}\AgdaSymbol{)}\AgdaSpace{}%
\AgdaSymbol{=}\AgdaSpace{}%
\AgdaInductiveConstructor{type}\AgdaSpace{}%
\AgdaSymbol{(}\AgdaBound{ξ}\AgdaSpace{}%
\AgdaOperator{\AgdaFunction{⊗svar}}\AgdaSpace{}%
\AgdaBound{δ}\AgdaSymbol{)}\AgdaSpace{}%
\AgdaSymbol{(}\AgdaFunction{ι}\AgdaSpace{}%
\AgdaOperator{\AgdaFunction{++}}\AgdaSpace{}%
\AgdaBound{θ}\AgdaSymbol{)}\<%
\\
\>[0]\AgdaFunction{⊗premise}\AgdaSpace{}%
\AgdaSymbol{\{}\AgdaArgument{q}\AgdaSpace{}%
\AgdaSymbol{=}\AgdaSpace{}%
\AgdaBound{q}\AgdaSymbol{\}}\AgdaSpace{}%
\AgdaBound{δ}\AgdaSpace{}%
\AgdaSymbol{(}\AgdaBound{T}\AgdaSpace{}%
\AgdaOperator{\AgdaInductiveConstructor{∋'}}\AgdaSpace{}%
\AgdaBound{ξ}\AgdaSpace{}%
\AgdaOperator{\AgdaInductiveConstructor{[}}\AgdaSpace{}%
\AgdaBound{θ}\AgdaSpace{}%
\AgdaOperator{\AgdaInductiveConstructor{]}}\AgdaSymbol{)}\AgdaSpace{}%
\AgdaKeyword{rewrite}\AgdaSpace{}%
\AgdaFunction{sym}\AgdaSpace{}%
\AgdaSymbol{(}\AgdaFunction{helper}\AgdaSpace{}%
\AgdaBound{δ}\AgdaSpace{}%
\AgdaBound{q}\AgdaSpace{}%
\AgdaBound{ξ}\AgdaSymbol{)}\AgdaSpace{}%
\AgdaSymbol{=}\AgdaSpace{}%
\AgdaSymbol{(}\AgdaBound{T}\AgdaSpace{}%
\AgdaOperator{\AgdaFunction{⊗expr}}\AgdaSpace{}%
\AgdaBound{δ}\AgdaSymbol{)}\AgdaSpace{}%
\AgdaOperator{\AgdaInductiveConstructor{∋'}}\AgdaSpace{}%
\AgdaBound{ξ}\AgdaSpace{}%
\AgdaOperator{\AgdaFunction{⊗svar}}\AgdaSpace{}%
\AgdaBound{δ}\AgdaSpace{}%
\AgdaOperator{\AgdaInductiveConstructor{[}}\AgdaSpace{}%
\AgdaFunction{ι}\AgdaSpace{}%
\AgdaOperator{\AgdaFunction{++}}\AgdaSpace{}%
\AgdaBound{θ}\AgdaSpace{}%
\AgdaOperator{\AgdaInductiveConstructor{]}}\<%
\\
\>[0]\AgdaFunction{⊗premise}\AgdaSpace{}%
\AgdaBound{δ}\AgdaSpace{}%
\AgdaSymbol{(}\AgdaBound{x}\AgdaSpace{}%
\AgdaOperator{\AgdaInductiveConstructor{≡'}}\AgdaSpace{}%
\AgdaBound{x₁}\AgdaSymbol{)}%
\>[26]\AgdaSymbol{=}\AgdaSpace{}%
\AgdaSymbol{(}\AgdaBound{x}\AgdaSpace{}%
\AgdaOperator{\AgdaFunction{⊗expr}}\AgdaSpace{}%
\AgdaBound{δ}\AgdaSymbol{)}\AgdaSpace{}%
\AgdaOperator{\AgdaInductiveConstructor{≡'}}\AgdaSpace{}%
\AgdaSymbol{(}\AgdaBound{x₁}\AgdaSpace{}%
\AgdaOperator{\AgdaFunction{⊗expr}}\AgdaSpace{}%
\AgdaBound{δ}\AgdaSymbol{)}\<%
\\
\>[0]\AgdaFunction{⊗premise}\AgdaSpace{}%
\AgdaBound{δ}\AgdaSpace{}%
\AgdaSymbol{(}\AgdaInductiveConstructor{univ}\AgdaSpace{}%
\AgdaBound{x}\AgdaSymbol{)}%
\>[26]\AgdaSymbol{=}\AgdaSpace{}%
\AgdaInductiveConstructor{univ}\AgdaSpace{}%
\AgdaSymbol{(}\AgdaBound{x}\AgdaSpace{}%
\AgdaOperator{\AgdaFunction{⊗expr}}\AgdaSpace{}%
\AgdaBound{δ}\AgdaSymbol{)}\<%
\\
\>[0]\AgdaFunction{⊗premise}\AgdaSpace{}%
\AgdaBound{δ}\AgdaSpace{}%
\AgdaSymbol{(}\AgdaOperator{\AgdaInductiveConstructor{\AgdaUnderscore{}⊢'\AgdaUnderscore{}}}\AgdaSpace{}%
\AgdaSymbol{\{}\AgdaArgument{p`}\AgdaSpace{}%
\AgdaSymbol{=}\AgdaSpace{}%
\AgdaBound{p`}\AgdaSymbol{\}}\AgdaSpace{}%
\AgdaBound{x}\AgdaSpace{}%
\AgdaBound{prem}\AgdaSymbol{)}\AgdaSpace{}%
\AgdaSymbol{=}\AgdaSpace{}%
\AgdaSymbol{(}\AgdaBound{x}\AgdaSpace{}%
\AgdaOperator{\AgdaFunction{⊗expr}}\AgdaSpace{}%
\AgdaBound{δ}\AgdaSymbol{)}\AgdaSpace{}%
\AgdaOperator{\AgdaInductiveConstructor{⊢'}}\AgdaSpace{}%
\AgdaFunction{⊗premise}\AgdaSpace{}%
\AgdaBound{δ}\AgdaSpace{}%
\AgdaBound{prem}\<%
\\
%
\\[\AgdaEmptyExtraSkip]%
\>[0]\AgdaComment{-- We have a concept of a placeless thing, which represents any}\<%
\\
\>[0]\AgdaComment{-- pattern that contains no places}\<%
\\
\>[0]\AgdaKeyword{data}\AgdaSpace{}%
\AgdaOperator{\AgdaDatatype{\AgdaUnderscore{}Placeless}}\AgdaSpace{}%
\AgdaSymbol{\{}\AgdaBound{γ}\AgdaSpace{}%
\AgdaSymbol{:}\AgdaSpace{}%
\AgdaFunction{Scope}\AgdaSymbol{\}}\AgdaSpace{}%
\AgdaSymbol{:}\AgdaSpace{}%
\AgdaDatatype{Pattern}\AgdaSpace{}%
\AgdaBound{γ}\AgdaSpace{}%
\AgdaSymbol{→}\AgdaSpace{}%
\AgdaPrimitiveType{Set}\AgdaSpace{}%
\AgdaKeyword{where}\<%
\\
\>[0][@{}l@{\AgdaIndent{0}}]%
\>[2]\AgdaInductiveConstructor{`}%
\>[7]\AgdaSymbol{:}\AgdaSpace{}%
\AgdaSymbol{(}\AgdaBound{c}\AgdaSpace{}%
\AgdaSymbol{:}\AgdaSpace{}%
\AgdaPostulate{Char}\AgdaSymbol{)}\AgdaSpace{}%
\AgdaSymbol{→}\AgdaSpace{}%
\AgdaInductiveConstructor{`}\AgdaSpace{}%
\AgdaBound{c}\AgdaSpace{}%
\AgdaOperator{\AgdaDatatype{Placeless}}\<%
\\
%
\>[2]\AgdaInductiveConstructor{⊥}%
\>[7]\AgdaSymbol{:}\AgdaSpace{}%
\AgdaInductiveConstructor{⊥}\AgdaSpace{}%
\AgdaOperator{\AgdaDatatype{Placeless}}\<%
\\
%
\>[2]\AgdaOperator{\AgdaInductiveConstructor{\AgdaUnderscore{}∙\AgdaUnderscore{}}}%
\>[7]\AgdaSymbol{:}\AgdaSpace{}%
\AgdaSymbol{\{}\AgdaBound{l}\AgdaSpace{}%
\AgdaBound{r}\AgdaSpace{}%
\AgdaSymbol{:}\AgdaSpace{}%
\AgdaDatatype{Pattern}\AgdaSpace{}%
\AgdaBound{γ}\AgdaSymbol{\}}\AgdaSpace{}%
\AgdaSymbol{→}\AgdaSpace{}%
\AgdaSymbol{(}\AgdaBound{l}\AgdaSpace{}%
\AgdaOperator{\AgdaDatatype{Placeless}}\AgdaSymbol{)}\AgdaSpace{}%
\AgdaSymbol{→}\AgdaSpace{}%
\AgdaSymbol{(}\AgdaBound{r}\AgdaSpace{}%
\AgdaOperator{\AgdaDatatype{Placeless}}\AgdaSymbol{)}\AgdaSpace{}%
\AgdaSymbol{→}\AgdaSpace{}%
\AgdaSymbol{(}\AgdaBound{l}\AgdaSpace{}%
\AgdaOperator{\AgdaInductiveConstructor{∙}}\AgdaSpace{}%
\AgdaBound{r}\AgdaSymbol{)}\AgdaSpace{}%
\AgdaOperator{\AgdaDatatype{Placeless}}\<%
\\
%
\>[2]\AgdaInductiveConstructor{bind}\AgdaSpace{}%
\AgdaSymbol{:}\AgdaSpace{}%
\AgdaSymbol{\{}\AgdaBound{t}\AgdaSpace{}%
\AgdaSymbol{:}\AgdaSpace{}%
\AgdaDatatype{Pattern}\AgdaSpace{}%
\AgdaSymbol{(}\AgdaInductiveConstructor{suc}\AgdaSpace{}%
\AgdaBound{γ}\AgdaSymbol{)\}}\AgdaSpace{}%
\AgdaSymbol{→}\AgdaSpace{}%
\AgdaSymbol{(}\AgdaBound{t}\AgdaSpace{}%
\AgdaOperator{\AgdaDatatype{Placeless}}\AgdaSymbol{)}\AgdaSpace{}%
\AgdaSymbol{→}\AgdaSpace{}%
\AgdaSymbol{(}\AgdaInductiveConstructor{bind}\AgdaSpace{}%
\AgdaBound{t}\AgdaSymbol{)}\AgdaSpace{}%
\AgdaOperator{\AgdaDatatype{Placeless}}\<%
\\
%
\\[\AgdaEmptyExtraSkip]%
\>[0]\AgdaComment{-- we can remove places from a pattern and replace them with ` '⊤'}\<%
\\
\>[0]\AgdaOperator{\AgdaFunction{\AgdaUnderscore{}-places}}\AgdaSpace{}%
\AgdaSymbol{:}\AgdaSpace{}%
\AgdaDatatype{Pattern}\AgdaSpace{}%
\AgdaGeneralizable{γ}\AgdaSpace{}%
\AgdaSymbol{→}\AgdaSpace{}%
\AgdaDatatype{Pattern}\AgdaSpace{}%
\AgdaGeneralizable{γ}\<%
\\
\>[0]\AgdaInductiveConstructor{`}\AgdaSpace{}%
\AgdaBound{x}%
\>[9]\AgdaOperator{\AgdaFunction{-places}}\AgdaSpace{}%
\AgdaSymbol{=}\AgdaSpace{}%
\AgdaInductiveConstructor{`}\AgdaSpace{}%
\AgdaBound{x}\<%
\\
\>[0]\AgdaSymbol{(}\AgdaBound{s}\AgdaSpace{}%
\AgdaOperator{\AgdaInductiveConstructor{∙}}\AgdaSpace{}%
\AgdaBound{t}\AgdaSymbol{)}%
\>[9]\AgdaOperator{\AgdaFunction{-places}}\AgdaSpace{}%
\AgdaSymbol{=}\AgdaSpace{}%
\AgdaSymbol{(}\AgdaBound{s}\AgdaSpace{}%
\AgdaOperator{\AgdaFunction{-places}}\AgdaSymbol{)}\AgdaSpace{}%
\AgdaOperator{\AgdaInductiveConstructor{∙}}\AgdaSpace{}%
\AgdaSymbol{(}\AgdaBound{t}\AgdaSpace{}%
\AgdaOperator{\AgdaFunction{-places}}\AgdaSymbol{)}\<%
\\
\>[0]\AgdaInductiveConstructor{bind}\AgdaSpace{}%
\AgdaBound{t}%
\>[9]\AgdaOperator{\AgdaFunction{-places}}\AgdaSpace{}%
\AgdaSymbol{=}\AgdaSpace{}%
\AgdaInductiveConstructor{bind}\AgdaSpace{}%
\AgdaSymbol{(}\AgdaBound{t}\AgdaSpace{}%
\AgdaOperator{\AgdaFunction{-places}}\AgdaSymbol{)}\<%
\\
\>[0]\AgdaInductiveConstructor{place}\AgdaSpace{}%
\AgdaBound{x}%
\>[9]\AgdaOperator{\AgdaFunction{-places}}\AgdaSpace{}%
\AgdaSymbol{=}\AgdaSpace{}%
\AgdaInductiveConstructor{`}\AgdaSpace{}%
\AgdaString{'⊤'}\<%
\\
\>[0]\AgdaInductiveConstructor{⊥}%
\>[9]\AgdaOperator{\AgdaFunction{-places}}\AgdaSpace{}%
\AgdaSymbol{=}\AgdaSpace{}%
\AgdaInductiveConstructor{⊥}\<%
\\
%
\\[\AgdaEmptyExtraSkip]%
\>[0]\AgdaComment{-- hence we can make a placeless thing from any pattern }\<%
\\
\>[0]\AgdaOperator{\AgdaFunction{\AgdaUnderscore{}placeless}}\AgdaSpace{}%
\AgdaSymbol{:}\AgdaSpace{}%
\AgdaSymbol{(}\AgdaBound{p}\AgdaSpace{}%
\AgdaSymbol{:}\AgdaSpace{}%
\AgdaDatatype{Pattern}\AgdaSpace{}%
\AgdaGeneralizable{γ}\AgdaSymbol{)}\AgdaSpace{}%
\AgdaSymbol{→}\AgdaSpace{}%
\AgdaSymbol{(}\AgdaBound{p}\AgdaSpace{}%
\AgdaOperator{\AgdaFunction{-places}}\AgdaSymbol{)}\AgdaSpace{}%
\AgdaOperator{\AgdaDatatype{Placeless}}\<%
\\
\>[0]\AgdaInductiveConstructor{`}\AgdaSpace{}%
\AgdaBound{x}\AgdaSpace{}%
\AgdaOperator{\AgdaFunction{placeless}}%
\>[19]\AgdaSymbol{=}\AgdaSpace{}%
\AgdaInductiveConstructor{`}\AgdaSpace{}%
\AgdaBound{x}\<%
\\
\>[0]\AgdaSymbol{(}\AgdaBound{s}\AgdaSpace{}%
\AgdaOperator{\AgdaInductiveConstructor{∙}}\AgdaSpace{}%
\AgdaBound{t}\AgdaSymbol{)}\AgdaSpace{}%
\AgdaOperator{\AgdaFunction{placeless}}%
\>[19]\AgdaSymbol{=}\AgdaSpace{}%
\AgdaSymbol{(}\AgdaBound{s}\AgdaSpace{}%
\AgdaOperator{\AgdaFunction{placeless}}\AgdaSymbol{)}\AgdaSpace{}%
\AgdaOperator{\AgdaInductiveConstructor{∙}}\AgdaSpace{}%
\AgdaSymbol{(}\AgdaBound{t}\AgdaSpace{}%
\AgdaOperator{\AgdaFunction{placeless}}\AgdaSymbol{)}\<%
\\
\>[0]\AgdaInductiveConstructor{bind}\AgdaSpace{}%
\AgdaBound{p}\AgdaSpace{}%
\AgdaOperator{\AgdaFunction{placeless}}%
\>[19]\AgdaSymbol{=}\AgdaSpace{}%
\AgdaInductiveConstructor{bind}\AgdaSpace{}%
\AgdaSymbol{(}\AgdaBound{p}\AgdaSpace{}%
\AgdaOperator{\AgdaFunction{placeless}}\AgdaSymbol{)}\<%
\\
\>[0]\AgdaInductiveConstructor{place}\AgdaSpace{}%
\AgdaBound{x}\AgdaSpace{}%
\AgdaOperator{\AgdaFunction{placeless}}%
\>[19]\AgdaSymbol{=}\AgdaSpace{}%
\AgdaInductiveConstructor{`}\AgdaSpace{}%
\AgdaString{'⊤'}\<%
\\
\>[0]\AgdaInductiveConstructor{⊥}\AgdaSpace{}%
\AgdaOperator{\AgdaFunction{placeless}}%
\>[19]\AgdaSymbol{=}\AgdaSpace{}%
\AgdaInductiveConstructor{⊥}\<%
\\
%
\\[\AgdaEmptyExtraSkip]%
\>[0]\AgdaComment{-- we can also 'open' placeless things trivially, just fixing up the type}\<%
\\
\>[0]\AgdaOperator{\AgdaFunction{\AgdaUnderscore{}⊗pl\AgdaUnderscore{}}}\AgdaSpace{}%
\AgdaSymbol{:}\AgdaSpace{}%
\AgdaSymbol{∀}\AgdaSpace{}%
\AgdaSymbol{\{}\AgdaBound{p}\AgdaSpace{}%
\AgdaSymbol{:}\AgdaSpace{}%
\AgdaDatatype{Pattern}\AgdaSpace{}%
\AgdaGeneralizable{γ}\AgdaSymbol{\}}\AgdaSpace{}%
\AgdaSymbol{→}\AgdaSpace{}%
\AgdaBound{p}\AgdaSpace{}%
\AgdaOperator{\AgdaDatatype{Placeless}}\AgdaSpace{}%
\AgdaSymbol{→}\AgdaSpace{}%
\AgdaSymbol{(}\AgdaBound{δ}\AgdaSpace{}%
\AgdaSymbol{:}\AgdaSpace{}%
\AgdaFunction{Scope}\AgdaSymbol{)}\AgdaSpace{}%
\AgdaSymbol{→}\AgdaSpace{}%
\AgdaSymbol{(}\AgdaBound{δ}\AgdaSpace{}%
\AgdaOperator{\AgdaFunction{⊗}}\AgdaSpace{}%
\AgdaBound{p}\AgdaSymbol{)}\AgdaSpace{}%
\AgdaOperator{\AgdaDatatype{Placeless}}\<%
\\
\>[0]\AgdaInductiveConstructor{`}\AgdaSpace{}%
\AgdaBound{c}%
\>[8]\AgdaOperator{\AgdaFunction{⊗pl}}\AgdaSpace{}%
\AgdaBound{δ}\AgdaSpace{}%
\AgdaSymbol{=}\AgdaSpace{}%
\AgdaInductiveConstructor{`}\AgdaSpace{}%
\AgdaBound{c}\<%
\\
\>[0]\AgdaInductiveConstructor{⊥}%
\>[8]\AgdaOperator{\AgdaFunction{⊗pl}}\AgdaSpace{}%
\AgdaBound{δ}\AgdaSpace{}%
\AgdaSymbol{=}\AgdaSpace{}%
\AgdaInductiveConstructor{⊥}\<%
\\
\>[0]\AgdaSymbol{(}\AgdaBound{s}\AgdaSpace{}%
\AgdaOperator{\AgdaInductiveConstructor{∙}}\AgdaSpace{}%
\AgdaBound{t}\AgdaSymbol{)}\AgdaSpace{}%
\AgdaOperator{\AgdaFunction{⊗pl}}\AgdaSpace{}%
\AgdaBound{δ}\AgdaSpace{}%
\AgdaSymbol{=}\AgdaSpace{}%
\AgdaSymbol{(}\AgdaBound{s}\AgdaSpace{}%
\AgdaOperator{\AgdaFunction{⊗pl}}\AgdaSpace{}%
\AgdaBound{δ}\AgdaSymbol{)}\AgdaSpace{}%
\AgdaOperator{\AgdaInductiveConstructor{∙}}\AgdaSpace{}%
\AgdaSymbol{(}\AgdaBound{t}\AgdaSpace{}%
\AgdaOperator{\AgdaFunction{⊗pl}}\AgdaSpace{}%
\AgdaBound{δ}\AgdaSymbol{)}\<%
\\
\>[0]\AgdaInductiveConstructor{bind}\AgdaSpace{}%
\AgdaBound{t}%
\>[8]\AgdaOperator{\AgdaFunction{⊗pl}}\AgdaSpace{}%
\AgdaBound{δ}\AgdaSpace{}%
\AgdaSymbol{=}\AgdaSpace{}%
\AgdaInductiveConstructor{bind}\AgdaSpace{}%
\AgdaSymbol{(}\AgdaBound{t}\AgdaSpace{}%
\AgdaOperator{\AgdaFunction{⊗pl}}\AgdaSpace{}%
\AgdaBound{δ}\AgdaSymbol{)}\<%
\\
%
\\[\AgdaEmptyExtraSkip]%
\>[0]\AgdaKeyword{private}\<%
\\
\>[0][@{}l@{\AgdaIndent{0}}]%
\>[2]\AgdaKeyword{variable}\<%
\\
\>[2][@{}l@{\AgdaIndent{0}}]%
\>[4]\AgdaGeneralizable{p'}\AgdaSpace{}%
\AgdaSymbol{:}\AgdaSpace{}%
\AgdaDatatype{Pattern}\AgdaSpace{}%
\AgdaNumber{0}\<%
\\
%
\>[4]\AgdaGeneralizable{q₁}\AgdaSpace{}%
\AgdaSymbol{:}\AgdaSpace{}%
\AgdaDatatype{Pattern}\AgdaSpace{}%
\AgdaNumber{0}\<%
\\
%
\>[4]\AgdaGeneralizable{p₂}\AgdaSpace{}%
\AgdaSymbol{:}\AgdaSpace{}%
\AgdaDatatype{Pattern}\AgdaSpace{}%
\AgdaNumber{0}\<%
\\
%
\\[\AgdaEmptyExtraSkip]%
\>[0]\AgdaComment{-- and a chain of Premises}\<%
\\
\>[0]\AgdaKeyword{private}\<%
\\
\>[0][@{}l@{\AgdaIndent{0}}]%
\>[2]\AgdaKeyword{variable}\<%
\\
\>[2][@{}l@{\AgdaIndent{0}}]%
\>[4]\AgdaGeneralizable{q₁`}\AgdaSpace{}%
\AgdaSymbol{:}\AgdaSpace{}%
\AgdaDatatype{Pattern}\AgdaSpace{}%
\AgdaGeneralizable{γ}\<%
\\
%
\>[4]\AgdaGeneralizable{p₂`}\AgdaSpace{}%
\AgdaSymbol{:}\AgdaSpace{}%
\AgdaDatatype{Pattern}\AgdaSpace{}%
\AgdaGeneralizable{γ}\<%
\\
%
\\[\AgdaEmptyExtraSkip]%
\>[0]\AgdaKeyword{data}\AgdaSpace{}%
\AgdaDatatype{Prems}\AgdaSpace{}%
\AgdaSymbol{(}\AgdaBound{p₀}\AgdaSpace{}%
\AgdaSymbol{:}\AgdaSpace{}%
\AgdaDatatype{Pattern}\AgdaSpace{}%
\AgdaGeneralizable{γ}\AgdaSymbol{)}\AgdaSpace{}%
\AgdaSymbol{(}\AgdaBound{q₀}\AgdaSpace{}%
\AgdaSymbol{:}\AgdaSpace{}%
\AgdaDatatype{Pattern}\AgdaSpace{}%
\AgdaGeneralizable{γ}\AgdaSymbol{)}\AgdaSpace{}%
\AgdaSymbol{:}\AgdaSpace{}%
\AgdaSymbol{(}\AgdaBound{p₂}\AgdaSpace{}%
\AgdaSymbol{:}\AgdaSpace{}%
\AgdaDatatype{Pattern}\AgdaSpace{}%
\AgdaBound{γ}\AgdaSymbol{)}\AgdaSpace{}%
\AgdaSymbol{→}\AgdaSpace{}%
\AgdaPrimitiveType{Set}\AgdaSpace{}%
\AgdaKeyword{where}\<%
\\
\>[0][@{}l@{\AgdaIndent{0}}]%
\>[2]\AgdaInductiveConstructor{ε}\AgdaSpace{}%
\AgdaSymbol{:}\AgdaSpace{}%
\AgdaSymbol{(}\AgdaBound{q₀}\AgdaSpace{}%
\AgdaOperator{\AgdaDatatype{Placeless}}\AgdaSymbol{)}\AgdaSpace{}%
\AgdaSymbol{→}\AgdaSpace{}%
\AgdaDatatype{Prems}\AgdaSpace{}%
\AgdaBound{p₀}\AgdaSpace{}%
\AgdaBound{q₀}\AgdaSpace{}%
\AgdaBound{p₀}\<%
\\
%
\>[2]\AgdaOperator{\AgdaInductiveConstructor{\AgdaUnderscore{}⇉\AgdaUnderscore{}}}\AgdaSpace{}%
\AgdaSymbol{:}%
\>[606I]\AgdaDatatype{Prem}\AgdaSpace{}%
\AgdaBound{p₀}\AgdaSpace{}%
\AgdaBound{q₀}\AgdaSpace{}%
\AgdaBound{γ}\AgdaSpace{}%
\AgdaGeneralizable{pᵍ}\AgdaSpace{}%
\AgdaGeneralizable{q₁`}\AgdaSpace{}%
\AgdaSymbol{→}\<%
\\
\>[.][@{}l@{}]\<[606I]%
\>[8]\AgdaDatatype{Prems}\AgdaSpace{}%
\AgdaSymbol{(}\AgdaBound{p₀}\AgdaSpace{}%
\AgdaOperator{\AgdaInductiveConstructor{∙}}\AgdaSpace{}%
\AgdaGeneralizable{pᵍ}\AgdaSymbol{)}\AgdaSpace{}%
\AgdaGeneralizable{q₁`}\AgdaSpace{}%
\AgdaGeneralizable{p₂`}\AgdaSpace{}%
\AgdaSymbol{→}\<%
\\
%
\>[8]\AgdaDatatype{Prems}\AgdaSpace{}%
\AgdaBound{p₀}\AgdaSpace{}%
\AgdaBound{q₀}\AgdaSpace{}%
\AgdaGeneralizable{p₂`}\<%
\\
\>[0]\AgdaKeyword{infixr}\AgdaSpace{}%
\AgdaNumber{20}\AgdaSpace{}%
\AgdaOperator{\AgdaInductiveConstructor{\AgdaUnderscore{}⇉\AgdaUnderscore{}}}\<%
\\
%
\\[\AgdaEmptyExtraSkip]%
\>[0]\AgdaFunction{⊗premises}\AgdaSpace{}%
\AgdaSymbol{:}\AgdaSpace{}%
\AgdaSymbol{(}\AgdaBound{δ}\AgdaSpace{}%
\AgdaSymbol{:}\AgdaSpace{}%
\AgdaFunction{Scope}\AgdaSymbol{)}\AgdaSpace{}%
\AgdaSymbol{→}\AgdaSpace{}%
\AgdaDatatype{Prems}\AgdaSpace{}%
\AgdaGeneralizable{p}\AgdaSpace{}%
\AgdaGeneralizable{q}\AgdaSpace{}%
\AgdaGeneralizable{p₂}\AgdaSpace{}%
\AgdaSymbol{→}\AgdaSpace{}%
\AgdaDatatype{Prems}\AgdaSpace{}%
\AgdaSymbol{(}\AgdaBound{δ}\AgdaSpace{}%
\AgdaOperator{\AgdaFunction{⊗}}\AgdaSpace{}%
\AgdaGeneralizable{p}\AgdaSymbol{)}\AgdaSpace{}%
\AgdaSymbol{(}\AgdaBound{δ}\AgdaSpace{}%
\AgdaOperator{\AgdaFunction{⊗}}\AgdaSpace{}%
\AgdaGeneralizable{q}\AgdaSymbol{)}\AgdaSpace{}%
\AgdaSymbol{(}\AgdaBound{δ}\AgdaSpace{}%
\AgdaOperator{\AgdaFunction{⊗}}\AgdaSpace{}%
\AgdaGeneralizable{p₂}\AgdaSymbol{)}\<%
\\
\>[0]\AgdaFunction{⊗premises}\AgdaSpace{}%
\AgdaBound{δ}\AgdaSpace{}%
\AgdaSymbol{(}\AgdaInductiveConstructor{ε}\AgdaSpace{}%
\AgdaBound{x}\AgdaSymbol{)}%
\>[28]\AgdaSymbol{=}\AgdaSpace{}%
\AgdaInductiveConstructor{ε}\AgdaSpace{}%
\AgdaSymbol{(}\AgdaBound{x}\AgdaSpace{}%
\AgdaOperator{\AgdaFunction{⊗pl}}\AgdaSpace{}%
\AgdaBound{δ}\AgdaSymbol{)}\<%
\\
\>[0]\AgdaFunction{⊗premises}\AgdaSpace{}%
\AgdaBound{δ}\AgdaSpace{}%
\AgdaSymbol{(}\AgdaBound{prem}\AgdaSpace{}%
\AgdaOperator{\AgdaInductiveConstructor{⇉}}\AgdaSpace{}%
\AgdaBound{prems}\AgdaSymbol{)}%
\>[28]\AgdaSymbol{=}\AgdaSpace{}%
\AgdaFunction{⊗premise}\AgdaSpace{}%
\AgdaBound{δ}\AgdaSpace{}%
\AgdaBound{prem}\AgdaSpace{}%
\AgdaOperator{\AgdaInductiveConstructor{⇉}}\AgdaSpace{}%
\AgdaFunction{⊗premises}\AgdaSpace{}%
\AgdaBound{δ}\AgdaSpace{}%
\AgdaBound{prems}\<%
\\
%
\\[\AgdaEmptyExtraSkip]%
\>[0]\AgdaKeyword{record}\AgdaSpace{}%
\AgdaRecord{TypeRule}\AgdaSpace{}%
\AgdaSymbol{:}\AgdaSpace{}%
\AgdaPrimitiveType{Set}\AgdaSpace{}%
\AgdaKeyword{where}\<%
\\
\>[0][@{}l@{\AgdaIndent{0}}]%
\>[2]\AgdaKeyword{field}\<%
\\
\>[2][@{}l@{\AgdaIndent{0}}]%
\>[4]\AgdaField{subject}%
\>[13]\AgdaSymbol{:}\AgdaSpace{}%
\AgdaDatatype{Pattern}\AgdaSpace{}%
\AgdaNumber{0}\<%
\\
%
\>[4]\AgdaField{premises}\AgdaSpace{}%
\AgdaSymbol{:}\AgdaSpace{}%
\AgdaFunction{Σ[}\AgdaSpace{}%
\AgdaBound{p'}\AgdaSpace{}%
\AgdaFunction{∈}\AgdaSpace{}%
\AgdaDatatype{Pattern}\AgdaSpace{}%
\AgdaNumber{0}\AgdaSpace{}%
\AgdaFunction{]}\AgdaSpace{}%
\AgdaDatatype{Prems}\AgdaSpace{}%
\AgdaSymbol{(}\AgdaInductiveConstructor{`}\AgdaSpace{}%
\AgdaString{'⊤'}\AgdaSymbol{)}\AgdaSpace{}%
\AgdaField{subject}\AgdaSpace{}%
\AgdaBound{p'}\<%
\\
\>[0]\AgdaKeyword{open}\AgdaSpace{}%
\AgdaModule{TypeRule}\<%
\\
%
\\[\AgdaEmptyExtraSkip]%
\>[0]\AgdaFunction{match-typerule}\AgdaSpace{}%
\AgdaSymbol{:}\AgdaSpace{}%
\AgdaSymbol{(}\AgdaBound{rule}\AgdaSpace{}%
\AgdaSymbol{:}\AgdaSpace{}%
\AgdaRecord{TypeRule}\AgdaSymbol{)}\AgdaSpace{}%
\AgdaSymbol{→}\AgdaSpace{}%
\AgdaFunction{Term}\AgdaSpace{}%
\AgdaInductiveConstructor{const}\AgdaSpace{}%
\AgdaGeneralizable{γ}\AgdaSpace{}%
\AgdaSymbol{→}\AgdaSpace{}%
\AgdaDatatype{Maybe}\AgdaSpace{}%
\AgdaSymbol{((}\AgdaGeneralizable{γ}\AgdaSpace{}%
\AgdaOperator{\AgdaFunction{⊗}}\AgdaSpace{}%
\AgdaSymbol{(}\AgdaField{subject}\AgdaSpace{}%
\AgdaBound{rule}\AgdaSymbol{))}\AgdaSpace{}%
\AgdaOperator{\AgdaDatatype{-Env}}\AgdaSymbol{)}\<%
\\
\>[0]\AgdaFunction{match-typerule}\AgdaSpace{}%
\AgdaBound{rule}\AgdaSpace{}%
\AgdaBound{term}\AgdaSpace{}%
\AgdaSymbol{=}\AgdaSpace{}%
\AgdaFunction{match}\AgdaSpace{}%
\AgdaBound{term}\AgdaSpace{}%
\AgdaSymbol{(}\AgdaField{subject}\AgdaSpace{}%
\AgdaBound{rule}\AgdaSymbol{)}\<%
\\
%
\\[\AgdaEmptyExtraSkip]%
\>[0]\AgdaKeyword{record}\AgdaSpace{}%
\AgdaRecord{UnivRule}\AgdaSpace{}%
\AgdaSymbol{:}\AgdaSpace{}%
\AgdaPrimitiveType{Set}\AgdaSpace{}%
\AgdaKeyword{where}\<%
\\
\>[0][@{}l@{\AgdaIndent{0}}]%
\>[2]\AgdaKeyword{field}\<%
\\
\>[2][@{}l@{\AgdaIndent{0}}]%
\>[4]\AgdaField{input}%
\>[13]\AgdaSymbol{:}\AgdaSpace{}%
\AgdaDatatype{Pattern}\AgdaSpace{}%
\AgdaNumber{0}\<%
\\
%
\>[4]\AgdaField{premises}\AgdaSpace{}%
\AgdaSymbol{:}\AgdaSpace{}%
\AgdaFunction{Σ[}\AgdaSpace{}%
\AgdaBound{p'}\AgdaSpace{}%
\AgdaFunction{∈}\AgdaSpace{}%
\AgdaDatatype{Pattern}\AgdaSpace{}%
\AgdaNumber{0}\AgdaSpace{}%
\AgdaFunction{]}\AgdaSpace{}%
\AgdaDatatype{Prems}\AgdaSpace{}%
\AgdaField{input}\AgdaSpace{}%
\AgdaSymbol{(}\AgdaInductiveConstructor{`}\AgdaSpace{}%
\AgdaString{'⊤'}\AgdaSymbol{)}\AgdaSpace{}%
\AgdaBound{p'}\<%
\\
\>[0]\AgdaKeyword{open}\AgdaSpace{}%
\AgdaModule{UnivRule}\<%
\\
%
\\[\AgdaEmptyExtraSkip]%
\>[0]\AgdaFunction{match-univrule}\AgdaSpace{}%
\AgdaSymbol{:}\AgdaSpace{}%
\AgdaSymbol{(}\AgdaBound{rule}\AgdaSpace{}%
\AgdaSymbol{:}\AgdaSpace{}%
\AgdaRecord{UnivRule}\AgdaSymbol{)}\AgdaSpace{}%
\AgdaSymbol{→}\AgdaSpace{}%
\AgdaFunction{Term}\AgdaSpace{}%
\AgdaInductiveConstructor{const}\AgdaSpace{}%
\AgdaGeneralizable{γ}\AgdaSpace{}%
\AgdaSymbol{→}\AgdaSpace{}%
\AgdaDatatype{Maybe}\AgdaSpace{}%
\AgdaSymbol{((}\AgdaGeneralizable{γ}\AgdaSpace{}%
\AgdaOperator{\AgdaFunction{⊗}}\AgdaSpace{}%
\AgdaSymbol{(}\AgdaField{input}\AgdaSpace{}%
\AgdaBound{rule}\AgdaSymbol{))}\AgdaSpace{}%
\AgdaOperator{\AgdaDatatype{-Env}}\AgdaSymbol{)}\<%
\\
\>[0]\AgdaFunction{match-univrule}\AgdaSpace{}%
\AgdaBound{rule}\AgdaSpace{}%
\AgdaBound{term}\AgdaSpace{}%
\AgdaSymbol{=}\AgdaSpace{}%
\AgdaFunction{match}\AgdaSpace{}%
\AgdaBound{term}\AgdaSpace{}%
\AgdaSymbol{(}\AgdaField{input}\AgdaSpace{}%
\AgdaBound{rule}\AgdaSymbol{)}\<%
\\
%
\\[\AgdaEmptyExtraSkip]%
\>[0]\AgdaKeyword{record}\AgdaSpace{}%
\AgdaRecord{∋rule}\AgdaSpace{}%
\AgdaSymbol{:}\AgdaSpace{}%
\AgdaPrimitiveType{Set}\AgdaSpace{}%
\AgdaKeyword{where}\<%
\\
\>[0][@{}l@{\AgdaIndent{0}}]%
\>[2]\AgdaKeyword{field}\<%
\\
\>[2][@{}l@{\AgdaIndent{0}}]%
\>[4]\AgdaField{subject}%
\>[13]\AgdaSymbol{:}\AgdaSpace{}%
\AgdaDatatype{Pattern}\AgdaSpace{}%
\AgdaNumber{0}\<%
\\
%
\>[4]\AgdaField{input}%
\>[13]\AgdaSymbol{:}\AgdaSpace{}%
\AgdaDatatype{Pattern}\AgdaSpace{}%
\AgdaNumber{0}\<%
\\
%
\>[4]\AgdaField{premises}\AgdaSpace{}%
\AgdaSymbol{:}\AgdaSpace{}%
\AgdaFunction{Σ[}\AgdaSpace{}%
\AgdaBound{p'}\AgdaSpace{}%
\AgdaFunction{∈}\AgdaSpace{}%
\AgdaDatatype{Pattern}\AgdaSpace{}%
\AgdaNumber{0}\AgdaSpace{}%
\AgdaFunction{]}\AgdaSpace{}%
\AgdaDatatype{Prems}\AgdaSpace{}%
\AgdaField{input}\AgdaSpace{}%
\AgdaField{subject}\AgdaSpace{}%
\AgdaBound{p'}\<%
\\
\>[0]\AgdaKeyword{open}\AgdaSpace{}%
\AgdaModule{∋rule}\<%
\\
%
\\[\AgdaEmptyExtraSkip]%
\>[0]\AgdaFunction{match-∋rule}\AgdaSpace{}%
\AgdaSymbol{:}%
\>[765I]\AgdaSymbol{(}\AgdaBound{rule}\AgdaSpace{}%
\AgdaSymbol{:}\AgdaSpace{}%
\AgdaRecord{∋rule}\AgdaSymbol{)}\AgdaSpace{}%
\AgdaSymbol{→}\AgdaSpace{}%
\AgdaFunction{Term}\AgdaSpace{}%
\AgdaInductiveConstructor{const}\AgdaSpace{}%
\AgdaGeneralizable{γ}\AgdaSpace{}%
\AgdaSymbol{→}\AgdaSpace{}%
\AgdaFunction{Term}\AgdaSpace{}%
\AgdaInductiveConstructor{const}\AgdaSpace{}%
\AgdaGeneralizable{γ}\AgdaSpace{}%
\AgdaSymbol{→}\<%
\\
\>[.][@{}l@{}]\<[765I]%
\>[14]\AgdaSymbol{(}\AgdaDatatype{Maybe}\AgdaSpace{}%
\AgdaSymbol{(((}\AgdaGeneralizable{γ}\AgdaSpace{}%
\AgdaOperator{\AgdaFunction{⊗}}\AgdaSpace{}%
\AgdaSymbol{(}\AgdaField{input}\AgdaSpace{}%
\AgdaBound{rule}\AgdaSymbol{))}\AgdaSpace{}%
\AgdaOperator{\AgdaDatatype{-Env}}\AgdaSymbol{)}\AgdaSpace{}%
\AgdaOperator{\AgdaFunction{×}}\AgdaSpace{}%
\AgdaSymbol{((}\AgdaGeneralizable{γ}\AgdaSpace{}%
\AgdaOperator{\AgdaFunction{⊗}}\AgdaSpace{}%
\AgdaSymbol{(}\AgdaField{subject}\AgdaSpace{}%
\AgdaBound{rule}\AgdaSymbol{))}\AgdaSpace{}%
\AgdaOperator{\AgdaDatatype{-Env}}\AgdaSymbol{)))}\<%
\\
\>[0]\AgdaFunction{match-∋rule}\AgdaSpace{}%
\AgdaBound{rule}\AgdaSpace{}%
\AgdaBound{Tterm}\AgdaSpace{}%
\AgdaBound{tterm}\<%
\\
\>[0][@{}l@{\AgdaIndent{0}}]%
\>[2]\AgdaSymbol{=}%
\>[791I]\AgdaKeyword{do}\<%
\\
\>[791I][@{}l@{\AgdaIndent{0}}]%
\>[6]\AgdaBound{inenv}%
\>[13]\AgdaOperator{\AgdaFunction{←}}\AgdaSpace{}%
\AgdaFunction{match}\AgdaSpace{}%
\AgdaBound{Tterm}\AgdaSpace{}%
\AgdaSymbol{(}\AgdaField{input}\AgdaSpace{}%
\AgdaBound{rule}\AgdaSymbol{)}\<%
\\
%
\>[6]\AgdaBound{subenv}\AgdaSpace{}%
\AgdaOperator{\AgdaFunction{←}}\AgdaSpace{}%
\AgdaFunction{match}\AgdaSpace{}%
\AgdaBound{tterm}\AgdaSpace{}%
\AgdaSymbol{(}\AgdaField{subject}\AgdaSpace{}%
\AgdaBound{rule}\AgdaSymbol{)}\<%
\\
%
\>[6]\AgdaInductiveConstructor{just}\AgdaSpace{}%
\AgdaSymbol{(}\AgdaBound{inenv}\AgdaSpace{}%
\AgdaOperator{\AgdaInductiveConstructor{,}}\AgdaSpace{}%
\AgdaBound{subenv}\AgdaSymbol{)}\<%
\\
%
\\[\AgdaEmptyExtraSkip]%
\>[0]\AgdaKeyword{record}\AgdaSpace{}%
\AgdaRecord{ElimRule}\AgdaSpace{}%
\AgdaSymbol{:}\AgdaSpace{}%
\AgdaPrimitiveType{Set}\AgdaSpace{}%
\AgdaKeyword{where}\<%
\\
\>[0][@{}l@{\AgdaIndent{0}}]%
\>[2]\AgdaKeyword{field}\<%
\\
\>[2][@{}l@{\AgdaIndent{0}}]%
\>[4]\AgdaField{targetPat}%
\>[15]\AgdaSymbol{:}\AgdaSpace{}%
\AgdaDatatype{Pattern}\AgdaSpace{}%
\AgdaNumber{0}\<%
\\
%
\>[4]\AgdaField{eliminator}\AgdaSpace{}%
\AgdaSymbol{:}\AgdaSpace{}%
\AgdaDatatype{Pattern}\AgdaSpace{}%
\AgdaNumber{0}\<%
\\
%
\>[4]\AgdaField{premises}%
\>[15]\AgdaSymbol{:}\AgdaSpace{}%
\AgdaFunction{Σ[}\AgdaSpace{}%
\AgdaBound{p'}\AgdaSpace{}%
\AgdaFunction{∈}\AgdaSpace{}%
\AgdaDatatype{Pattern}\AgdaSpace{}%
\AgdaNumber{0}\AgdaSpace{}%
\AgdaFunction{]}\AgdaSpace{}%
\AgdaDatatype{Prems}\AgdaSpace{}%
\AgdaField{targetPat}\AgdaSpace{}%
\AgdaField{eliminator}\AgdaSpace{}%
\AgdaBound{p'}\<%
\\
%
\>[4]\AgdaField{output}%
\>[15]\AgdaSymbol{:}\AgdaSpace{}%
\AgdaFunction{Expr}\AgdaSpace{}%
\AgdaSymbol{(}\AgdaField{proj₁}\AgdaSpace{}%
\AgdaField{premises}\AgdaSymbol{)}\AgdaSpace{}%
\AgdaInductiveConstructor{const}\AgdaSpace{}%
\AgdaNumber{0}\<%
\\
%
\\[\AgdaEmptyExtraSkip]%
\>[0]\AgdaFunction{match-erule}\AgdaSpace{}%
\AgdaSymbol{:}%
\>[829I]\AgdaSymbol{(}\AgdaBound{rule}\AgdaSpace{}%
\AgdaSymbol{:}\AgdaSpace{}%
\AgdaRecord{ElimRule}\AgdaSymbol{)}\AgdaSpace{}%
\AgdaSymbol{→}\<%
\\
\>[.][@{}l@{}]\<[829I]%
\>[14]\AgdaSymbol{(}\AgdaBound{T}\AgdaSpace{}%
\AgdaSymbol{:}\AgdaSpace{}%
\AgdaFunction{Term}\AgdaSpace{}%
\AgdaInductiveConstructor{const}\AgdaSpace{}%
\AgdaGeneralizable{γ}\AgdaSymbol{)}\AgdaSpace{}%
\AgdaSymbol{→}\<%
\\
%
\>[14]\AgdaSymbol{(}\AgdaBound{s}\AgdaSpace{}%
\AgdaSymbol{:}\AgdaSpace{}%
\AgdaFunction{Term}\AgdaSpace{}%
\AgdaInductiveConstructor{const}\AgdaSpace{}%
\AgdaGeneralizable{γ}\AgdaSymbol{)}\AgdaSpace{}%
\AgdaSymbol{→}\<%
\\
%
\>[14]\AgdaDatatype{Maybe}\AgdaSpace{}%
\AgdaSymbol{(((}\AgdaGeneralizable{γ}\AgdaSpace{}%
\AgdaOperator{\AgdaFunction{⊗}}\AgdaSpace{}%
\AgdaSymbol{(}\AgdaField{ElimRule.targetPat}\AgdaSpace{}%
\AgdaBound{rule}\AgdaSymbol{))}\AgdaSpace{}%
\AgdaOperator{\AgdaDatatype{-Env}}\AgdaSymbol{)}\AgdaSpace{}%
\AgdaOperator{\AgdaFunction{×}}\AgdaSpace{}%
\AgdaSymbol{((}\AgdaGeneralizable{γ}\AgdaSpace{}%
\AgdaOperator{\AgdaFunction{⊗}}\AgdaSpace{}%
\AgdaSymbol{(}\AgdaField{ElimRule.eliminator}\AgdaSpace{}%
\AgdaBound{rule}\AgdaSymbol{))}\AgdaSpace{}%
\AgdaOperator{\AgdaDatatype{-Env}}\AgdaSymbol{))}\<%
\\
\>[0]\AgdaFunction{match-erule}\AgdaSpace{}%
\AgdaBound{rule}\AgdaSpace{}%
\AgdaBound{T}\AgdaSpace{}%
\AgdaBound{s}\AgdaSpace{}%
\AgdaSymbol{=}%
\>[858I]\AgdaKeyword{do}\<%
\\
\>[858I][@{}l@{\AgdaIndent{0}}]%
\>[25]\AgdaBound{T-env}\AgdaSpace{}%
\AgdaOperator{\AgdaFunction{←}}\AgdaSpace{}%
\AgdaFunction{match}\AgdaSpace{}%
\AgdaBound{T}\AgdaSpace{}%
\AgdaSymbol{(}\AgdaField{targetPat}\AgdaSpace{}%
\AgdaBound{rule}\AgdaSymbol{)}\<%
\\
%
\>[25]\AgdaBound{s-env}\AgdaSpace{}%
\AgdaOperator{\AgdaFunction{←}}\AgdaSpace{}%
\AgdaFunction{match}\AgdaSpace{}%
\AgdaBound{s}\AgdaSpace{}%
\AgdaSymbol{(}\AgdaField{eliminator}\AgdaSpace{}%
\AgdaBound{rule}\AgdaSymbol{)}\<%
\\
%
\>[25]\AgdaInductiveConstructor{just}\AgdaSpace{}%
\AgdaSymbol{(}\AgdaBound{T-env}\AgdaSpace{}%
\AgdaOperator{\AgdaInductiveConstructor{,}}\AgdaSpace{}%
\AgdaBound{s-env}\AgdaSymbol{)}\<%
\\
\>[.][@{}l@{}]\<[858I]%
\>[23]\AgdaKeyword{where}\<%
\\
\>[23][@{}l@{\AgdaIndent{0}}]%
\>[25]\AgdaKeyword{open}\AgdaSpace{}%
\AgdaModule{ElimRule}\<%
\\
%
\\[\AgdaEmptyExtraSkip]%
\>[0]\AgdaKeyword{data}\AgdaSpace{}%
\AgdaDatatype{Rules}\AgdaSpace{}%
\AgdaSymbol{:}\AgdaSpace{}%
\AgdaPrimitiveType{Set}\AgdaSpace{}%
\AgdaKeyword{where}\<%
\\
\>[0][@{}l@{\AgdaIndent{0}}]%
\>[2]\AgdaInductiveConstructor{rs}\AgdaSpace{}%
\AgdaSymbol{:}\AgdaSpace{}%
\AgdaDatatype{List}\AgdaSpace{}%
\AgdaRecord{TypeRule}\AgdaSpace{}%
\AgdaSymbol{→}\AgdaSpace{}%
\AgdaDatatype{List}\AgdaSpace{}%
\AgdaRecord{UnivRule}\AgdaSpace{}%
\AgdaSymbol{→}\AgdaSpace{}%
\AgdaDatatype{List}\AgdaSpace{}%
\AgdaRecord{∋rule}\AgdaSpace{}%
\AgdaSymbol{→}\AgdaSpace{}%
\AgdaDatatype{List}\AgdaSpace{}%
\AgdaRecord{ElimRule}\AgdaSpace{}%
\AgdaSymbol{→}\AgdaSpace{}%
\AgdaDatatype{Rules}\<%
\end{code}

\section{η-Rules}
\hide{
\begin{code}%
\>[0]\AgdaSymbol{\{-\#}\AgdaSpace{}%
\AgdaKeyword{OPTIONS}\AgdaSpace{}%
\AgdaPragma{--rewriting}\AgdaSpace{}%
\AgdaSymbol{\#-\}}\<%
\\
\>[0]\AgdaKeyword{module}\AgdaSpace{}%
\AgdaModule{EtaRule}\AgdaSpace{}%
\AgdaKeyword{where}\<%
\end{code}
}
\hide{
\begin{code}%
\>[0]\AgdaComment{--open import Pattern using (Pattern)}\<%
\\
\>[0]\AgdaKeyword{open}\AgdaSpace{}%
\AgdaKeyword{import}\AgdaSpace{}%
\AgdaModule{CoreLanguage}\<%
\\
\>[0]\AgdaKeyword{open}\AgdaSpace{}%
\AgdaKeyword{import}\AgdaSpace{}%
\AgdaModule{Rules}\AgdaSpace{}%
\AgdaKeyword{using}\AgdaSpace{}%
\AgdaSymbol{(}\AgdaRecord{∋rule}\AgdaSymbol{;}\AgdaSpace{}%
\AgdaFunction{match-∋rule}\AgdaSymbol{)}\<%
\\
\>[0]\AgdaKeyword{open}\AgdaSpace{}%
\AgdaKeyword{import}\AgdaSpace{}%
\AgdaModule{Pattern}\AgdaSpace{}%
\AgdaKeyword{using}%
\>[15I]\AgdaSymbol{(}\AgdaDatatype{Pattern}\AgdaSymbol{;}\AgdaSpace{}%
\AgdaOperator{\AgdaDatatype{\AgdaUnderscore{}-Env}}\AgdaSymbol{;}\AgdaSpace{}%
\AgdaFunction{termFrom}\AgdaSymbol{;}\AgdaSpace{}%
\AgdaOperator{\AgdaFunction{\AgdaUnderscore{}⊗\AgdaUnderscore{}}}\AgdaSymbol{;}\AgdaSpace{}%
\AgdaInductiveConstructor{`}\AgdaSymbol{;}\AgdaSpace{}%
\AgdaOperator{\AgdaInductiveConstructor{\AgdaUnderscore{}∙\AgdaUnderscore{}}}\AgdaSymbol{;}\AgdaSpace{}%
\AgdaOperator{\AgdaFunction{\AgdaUnderscore{}⊗penv\AgdaUnderscore{}}}\AgdaSymbol{;}\AgdaSpace{}%
\AgdaOperator{\AgdaFunction{\AgdaUnderscore{}⊗term\AgdaUnderscore{}}}\AgdaSymbol{;}\<%
\\
\>[15I][@{}l@{\AgdaIndent{0}}]%
\>[27]\AgdaInductiveConstructor{bind}\AgdaSymbol{;}\AgdaSpace{}%
\AgdaInductiveConstructor{place}\AgdaSymbol{;}\AgdaSpace{}%
\AgdaFunction{match}\AgdaSymbol{;}\AgdaSpace{}%
\AgdaInductiveConstructor{thing}\AgdaSymbol{;}\AgdaSpace{}%
\AgdaFunction{map}\AgdaSymbol{)}\<%
\\
\>[0]\AgdaKeyword{open}\AgdaSpace{}%
\AgdaKeyword{import}\AgdaSpace{}%
\AgdaModule{Thinning}\AgdaSpace{}%
\AgdaKeyword{using}\AgdaSpace{}%
\AgdaSymbol{(}\AgdaOperator{\AgdaFunction{\AgdaUnderscore{}⟨term\AgdaUnderscore{}}}\AgdaSymbol{;}\AgdaSpace{}%
\AgdaOperator{\AgdaFunction{\AgdaUnderscore{}\textasciicircum{}term}}\AgdaSymbol{;}\AgdaSpace{}%
\AgdaOperator{\AgdaDatatype{\AgdaUnderscore{}⊑\AgdaUnderscore{}}}\AgdaSymbol{;}\AgdaSpace{}%
\AgdaOperator{\AgdaFunction{\AgdaUnderscore{}◃\AgdaUnderscore{}}}\AgdaSymbol{)}\<%
\\
\>[0]\AgdaKeyword{open}\AgdaSpace{}%
\AgdaKeyword{import}\AgdaSpace{}%
\AgdaModule{Data.List}\AgdaSpace{}%
\AgdaKeyword{using}\AgdaSpace{}%
\AgdaSymbol{(}\AgdaDatatype{List}\AgdaSymbol{)}\<%
\\
\>[0]\AgdaKeyword{open}\AgdaSpace{}%
\AgdaKeyword{import}\AgdaSpace{}%
\AgdaModule{Data.Product}\AgdaSpace{}%
\AgdaKeyword{using}\AgdaSpace{}%
\AgdaSymbol{(}\AgdaField{proj₂}\AgdaSymbol{)}\<%
\\
\>[0]\AgdaKeyword{open}\AgdaSpace{}%
\AgdaKeyword{import}\AgdaSpace{}%
\AgdaModule{Relation.Binary.PropositionalEquality}\AgdaSpace{}%
\AgdaKeyword{using}\AgdaSpace{}%
\AgdaSymbol{(}\AgdaOperator{\AgdaDatatype{\AgdaUnderscore{}≡\AgdaUnderscore{}}}\AgdaSymbol{)}\<%
\\
\>[0]\AgdaKeyword{open}\AgdaSpace{}%
\AgdaKeyword{import}\AgdaSpace{}%
\AgdaModule{Function}\AgdaSpace{}%
\AgdaKeyword{using}\AgdaSpace{}%
\AgdaSymbol{(}\AgdaOperator{\AgdaFunction{\AgdaUnderscore{}∘\AgdaUnderscore{}}}\AgdaSymbol{)}\<%
\\
\>[0]\AgdaKeyword{import}\AgdaSpace{}%
\AgdaModule{Data.Maybe}\<%
\\
\>[0]\AgdaKeyword{open}\AgdaSpace{}%
\AgdaKeyword{import}\AgdaSpace{}%
\AgdaModule{Data.List}\AgdaSpace{}%
\AgdaKeyword{using}\AgdaSpace{}%
\AgdaSymbol{(}\AgdaInductiveConstructor{[]}\AgdaSymbol{;}\AgdaSpace{}%
\AgdaOperator{\AgdaInductiveConstructor{\AgdaUnderscore{}∷\AgdaUnderscore{}}}\AgdaSymbol{)}\<%
\\
\>[0]\AgdaKeyword{open}\AgdaSpace{}%
\AgdaKeyword{import}\AgdaSpace{}%
\AgdaModule{Data.String}\AgdaSpace{}%
\AgdaKeyword{using}\AgdaSpace{}%
\AgdaSymbol{(}\AgdaOperator{\AgdaFunction{\AgdaUnderscore{}++\AgdaUnderscore{}}}\AgdaSymbol{)}\<%
\\
%
\\[\AgdaEmptyExtraSkip]%
\>[0]\AgdaComment{-- remove me:}\<%
\\
\>[0]\AgdaKeyword{open}\AgdaSpace{}%
\AgdaKeyword{import}\AgdaSpace{}%
\AgdaModule{Data.Nat}\AgdaSpace{}%
\AgdaKeyword{using}\AgdaSpace{}%
\AgdaSymbol{(}\AgdaOperator{\AgdaPrimitive{\AgdaUnderscore{}+\AgdaUnderscore{}}}\AgdaSymbol{)}\<%
\end{code}
}
\hide{
\begin{code}%
\>[0]\AgdaKeyword{private}\<%
\\
\>[0][@{}l@{\AgdaIndent{0}}]%
\>[2]\AgdaKeyword{variable}\<%
\\
\>[2][@{}l@{\AgdaIndent{0}}]%
\>[4]\AgdaGeneralizable{δ}\AgdaSpace{}%
\AgdaSymbol{:}\AgdaSpace{}%
\AgdaFunction{Scope}\<%
\\
%
\>[4]\AgdaGeneralizable{γ}\AgdaSpace{}%
\AgdaSymbol{:}\AgdaSpace{}%
\AgdaFunction{Scope}\<%
\end{code}
}

Later we will show how we might fully normalize a term using a technique
known as normalization by evaluation. In order to do this, we will find
that we require the ability to perform $η$-expansion on our language
constructions.

In a future draft of this work, we may include a way that we can
synthesize such rules from other information that is given. However,
for now, we merely give the type of such rules, and a method of performing
the expansion according to such a rule.

In our η-rule, we store only the eliminator for each place in the pattern,
then to generate the eta expanded form, we map the elimination of the target
over this environment of eliminators to get the full eliminations that
go in each place in the pattern. This is very straightforward as a concept
but we have to fix-up out types a little in order to satisfy the
well-scopedness.

\begin{code}%
\>[0]\AgdaKeyword{record}\AgdaSpace{}%
\AgdaRecord{η-Rule}\AgdaSpace{}%
\AgdaSymbol{:}\AgdaSpace{}%
\AgdaPrimitiveType{Set}\AgdaSpace{}%
\AgdaKeyword{where}\<%
\\
\>[0][@{}l@{\AgdaIndent{0}}]%
\>[2]\AgdaKeyword{open}\AgdaSpace{}%
\AgdaModule{∋rule}\<%
\\
%
\>[2]\AgdaKeyword{open}\AgdaSpace{}%
\AgdaModule{Data.Maybe}\AgdaSpace{}%
\AgdaKeyword{using}\AgdaSpace{}%
\AgdaSymbol{(}\AgdaDatatype{Maybe}\AgdaSymbol{;}\AgdaSpace{}%
\AgdaInductiveConstructor{just}\AgdaSymbol{;}\AgdaSpace{}%
\AgdaOperator{\AgdaFunction{\AgdaUnderscore{}>>=\AgdaUnderscore{}}}\AgdaSymbol{)}\<%
\\
%
\\[\AgdaEmptyExtraSkip]%
%
\>[2]\AgdaKeyword{field}\<%
\\
\>[2][@{}l@{\AgdaIndent{0}}]%
\>[4]\AgdaField{checkRule}%
\>[16]\AgdaSymbol{:}\AgdaSpace{}%
\AgdaRecord{∋rule}\<%
\\
%
\>[4]\AgdaField{eliminators}\AgdaSpace{}%
\AgdaSymbol{:}\AgdaSpace{}%
\AgdaField{subject}\AgdaSpace{}%
\AgdaField{checkRule}\AgdaSpace{}%
\AgdaOperator{\AgdaDatatype{-Env}}\<%
\\
%
\\[\AgdaEmptyExtraSkip]%
%
\>[2]\AgdaComment{-- perhaps in we don't actually need the environment returned the}\<%
\\
%
\>[2]\AgdaComment{-- way we use this, but it is left for now just incase}\<%
\\
%
\>[2]\AgdaFunction{η-match}\AgdaSpace{}%
\AgdaSymbol{:}\AgdaSpace{}%
\AgdaSymbol{(}\AgdaBound{type}\AgdaSpace{}%
\AgdaSymbol{:}\AgdaSpace{}%
\AgdaDatatype{Const}\AgdaSpace{}%
\AgdaGeneralizable{γ}\AgdaSymbol{)}\AgdaSpace{}%
\AgdaSymbol{→}\AgdaSpace{}%
\AgdaDatatype{Maybe}\AgdaSpace{}%
\AgdaSymbol{((}\AgdaGeneralizable{γ}\AgdaSpace{}%
\AgdaOperator{\AgdaFunction{⊗}}\AgdaSpace{}%
\AgdaField{input}\AgdaSpace{}%
\AgdaField{checkRule}\AgdaSymbol{)}\AgdaSpace{}%
\AgdaOperator{\AgdaDatatype{-Env}}\AgdaSymbol{)}\<%
\\
%
\>[2]\AgdaFunction{η-match}\AgdaSpace{}%
\AgdaBound{ty}\AgdaSpace{}%
\AgdaSymbol{=}\AgdaSpace{}%
\AgdaFunction{match}\AgdaSpace{}%
\AgdaBound{ty}\AgdaSpace{}%
\AgdaSymbol{(}\AgdaField{input}\AgdaSpace{}%
\AgdaField{checkRule}\AgdaSymbol{)}\<%
\\
%
\\[\AgdaEmptyExtraSkip]%
%
\>[2]\AgdaFunction{eliminations}\AgdaSpace{}%
\AgdaSymbol{:}\AgdaSpace{}%
\AgdaSymbol{(}\AgdaBound{type}\AgdaSpace{}%
\AgdaBound{target}\AgdaSpace{}%
\AgdaSymbol{:}\AgdaSpace{}%
\AgdaDatatype{Const}\AgdaSpace{}%
\AgdaGeneralizable{γ}\AgdaSymbol{)}\AgdaSpace{}%
\AgdaSymbol{→}\AgdaSpace{}%
\AgdaSymbol{(}\AgdaGeneralizable{γ}\AgdaSpace{}%
\AgdaOperator{\AgdaFunction{⊗}}\AgdaSpace{}%
\AgdaField{subject}\AgdaSpace{}%
\AgdaField{checkRule}\AgdaSymbol{)}\AgdaSpace{}%
\AgdaOperator{\AgdaDatatype{-Env}}\<%
\\
%
\>[2]\AgdaFunction{eliminations}\AgdaSpace{}%
\AgdaSymbol{\{}\AgdaBound{γ}\AgdaSymbol{\}}\AgdaSpace{}%
\AgdaBound{type}\AgdaSpace{}%
\AgdaBound{target}\<%
\\
\>[2][@{}l@{\AgdaIndent{0}}]%
\>[4]\AgdaSymbol{=}%
\>[116I]\AgdaFunction{map}\<%
\\
\>[116I][@{}l@{\AgdaIndent{0}}]%
\>[8]\AgdaSymbol{(λ}\AgdaSpace{}%
\AgdaSymbol{\{}\AgdaBound{δ}\AgdaSymbol{\}}\AgdaSpace{}%
\AgdaBound{const}\AgdaSpace{}%
\AgdaSymbol{→}\AgdaSpace{}%
\AgdaInductiveConstructor{thunk}\AgdaSpace{}%
\AgdaSymbol{(}\AgdaInductiveConstructor{elim}\AgdaSpace{}%
\AgdaSymbol{((}\AgdaBound{target}\AgdaSpace{}%
\AgdaOperator{\AgdaInductiveConstructor{∷}}\AgdaSpace{}%
\AgdaBound{type}\AgdaSymbol{)}\AgdaSpace{}%
\AgdaOperator{\AgdaFunction{⟨term}}\AgdaSpace{}%
\AgdaSymbol{(}\AgdaBound{γ}\AgdaSpace{}%
\AgdaOperator{\AgdaFunction{◃}}\AgdaSpace{}%
\AgdaBound{δ}\AgdaSymbol{))}\AgdaSpace{}%
\AgdaSymbol{(}\AgdaBound{γ}\AgdaSpace{}%
\AgdaOperator{\AgdaFunction{⊗term}}\AgdaSpace{}%
\AgdaBound{const}\AgdaSymbol{)))}\<%
\\
%
\>[8]\AgdaField{eliminators}\<%
\\
%
\\[\AgdaEmptyExtraSkip]%
%
\>[2]\AgdaFunction{η-expand}\AgdaSpace{}%
\AgdaSymbol{:}\AgdaSpace{}%
\AgdaSymbol{(}\AgdaBound{type}\AgdaSpace{}%
\AgdaBound{term}\AgdaSpace{}%
\AgdaSymbol{:}\AgdaSpace{}%
\AgdaDatatype{Const}\AgdaSpace{}%
\AgdaGeneralizable{γ}\AgdaSymbol{)}\AgdaSpace{}%
\AgdaSymbol{→}\AgdaSpace{}%
\AgdaDatatype{Const}\AgdaSpace{}%
\AgdaGeneralizable{γ}\<%
\\
%
\>[2]\AgdaFunction{η-expand}\AgdaSpace{}%
\AgdaSymbol{=}\AgdaSpace{}%
\AgdaSymbol{(}\AgdaFunction{termFrom}\AgdaSpace{}%
\AgdaSymbol{(}\AgdaField{subject}\AgdaSpace{}%
\AgdaField{checkRule}\AgdaSymbol{)}\AgdaSpace{}%
\AgdaOperator{\AgdaFunction{∘\AgdaUnderscore{}}}\AgdaSymbol{)}\AgdaSpace{}%
\AgdaOperator{\AgdaFunction{∘}}\AgdaSpace{}%
\AgdaFunction{eliminations}\<%
\\
\>[0]\AgdaKeyword{open}\AgdaSpace{}%
\AgdaModule{η-Rule}\<%
\end{code}

We can now provide a function to expand a term given a list of η rules, so
long as we are able to provide the terms type.

\hide{
\begin{code}%
\>[0]\AgdaKeyword{open}\AgdaSpace{}%
\AgdaKeyword{import}\AgdaSpace{}%
\AgdaModule{Failable}\<%
\\
\>[0]\AgdaKeyword{open}\AgdaSpace{}%
\AgdaKeyword{import}\AgdaSpace{}%
\AgdaModule{Data.Maybe}\AgdaSpace{}%
\AgdaKeyword{using}\AgdaSpace{}%
\AgdaSymbol{(}\AgdaInductiveConstructor{just}\AgdaSymbol{;}\AgdaSpace{}%
\AgdaInductiveConstructor{nothing}\AgdaSymbol{)}\<%
\end{code}
}
\begin{code}%
\>[0]\AgdaFunction{expand}\AgdaSpace{}%
\AgdaSymbol{:}\AgdaSpace{}%
\AgdaDatatype{List}\AgdaSpace{}%
\AgdaRecord{η-Rule}\AgdaSpace{}%
\AgdaSymbol{→}\AgdaSpace{}%
\AgdaSymbol{(}\AgdaBound{tm}\AgdaSpace{}%
\AgdaBound{ty}\AgdaSpace{}%
\AgdaSymbol{:}\AgdaSpace{}%
\AgdaDatatype{Const}\AgdaSpace{}%
\AgdaGeneralizable{γ}\AgdaSymbol{)}\AgdaSpace{}%
\AgdaSymbol{→}\AgdaSpace{}%
\AgdaDatatype{Failable}\AgdaSpace{}%
\AgdaSymbol{(}\AgdaDatatype{Const}\AgdaSpace{}%
\AgdaGeneralizable{γ}\AgdaSymbol{)}\<%
\end{code}
\hide{
\begin{code}%
\>[0]\AgdaFunction{findRule}\AgdaSpace{}%
\AgdaSymbol{:}\AgdaSpace{}%
\AgdaDatatype{List}\AgdaSpace{}%
\AgdaRecord{η-Rule}\AgdaSpace{}%
\AgdaSymbol{→}\AgdaSpace{}%
\AgdaSymbol{(}\AgdaBound{ty}\AgdaSpace{}%
\AgdaSymbol{:}\AgdaSpace{}%
\AgdaDatatype{Const}\AgdaSpace{}%
\AgdaGeneralizable{γ}\AgdaSymbol{)}\AgdaSpace{}%
\AgdaSymbol{→}\AgdaSpace{}%
\AgdaDatatype{Failable}\AgdaSpace{}%
\AgdaRecord{η-Rule}\<%
\\
\>[0]\AgdaFunction{findRule}\AgdaSpace{}%
\AgdaInductiveConstructor{[]}\AgdaSpace{}%
\AgdaBound{ty}\AgdaSpace{}%
\AgdaSymbol{=}\AgdaSpace{}%
\AgdaInductiveConstructor{fail}\AgdaSpace{}%
\AgdaSymbol{(}\AgdaString{"no η-rule match for: "}\AgdaSpace{}%
\AgdaOperator{\AgdaFunction{++}}\AgdaSpace{}%
\AgdaFunction{print}\AgdaSpace{}%
\AgdaBound{ty}\AgdaSymbol{)}\<%
\\
\>[0]\AgdaFunction{findRule}\AgdaSpace{}%
\AgdaSymbol{(}\AgdaBound{r}\AgdaSpace{}%
\AgdaOperator{\AgdaInductiveConstructor{∷}}\AgdaSpace{}%
\AgdaBound{rs}\AgdaSymbol{)}\AgdaSpace{}%
\AgdaBound{ty}\AgdaSpace{}%
\AgdaKeyword{with}\AgdaSpace{}%
\AgdaFunction{η-match}\AgdaSpace{}%
\AgdaBound{r}\AgdaSpace{}%
\AgdaBound{ty}\<%
\\
\>[0]\AgdaSymbol{...}\AgdaSpace{}%
\AgdaSymbol{|}\AgdaSpace{}%
\AgdaInductiveConstructor{nothing}\AgdaSpace{}%
\AgdaSymbol{=}\AgdaSpace{}%
\AgdaFunction{findRule}\AgdaSpace{}%
\AgdaBound{rs}\AgdaSpace{}%
\AgdaBound{ty}\<%
\\
\>[0]\AgdaSymbol{...}\AgdaSpace{}%
\AgdaSymbol{|}\AgdaSpace{}%
\AgdaInductiveConstructor{just}\AgdaSpace{}%
\AgdaBound{x}\AgdaSpace{}%
\AgdaSymbol{=}\AgdaSpace{}%
\AgdaInductiveConstructor{succeed}\AgdaSpace{}%
\AgdaBound{r}\<%
\\
%
\\[\AgdaEmptyExtraSkip]%
%
\\[\AgdaEmptyExtraSkip]%
\>[0]\AgdaFunction{expand}\AgdaSpace{}%
\AgdaBound{rs}\AgdaSpace{}%
\AgdaBound{tm}\AgdaSpace{}%
\AgdaBound{ty}\AgdaSpace{}%
\AgdaSymbol{=}%
\>[212I]\AgdaKeyword{do}\<%
\\
\>[212I][@{}l@{\AgdaIndent{0}}]%
\>[20]\AgdaBound{r}\AgdaSpace{}%
\AgdaOperator{\AgdaFunction{←}}\AgdaSpace{}%
\AgdaFunction{findRule}\AgdaSpace{}%
\AgdaBound{rs}\AgdaSpace{}%
\AgdaBound{ty}\<%
\\
%
\>[20]\AgdaInductiveConstructor{succeed}\AgdaSpace{}%
\AgdaSymbol{(}\AgdaFunction{η-expand}\AgdaSpace{}%
\AgdaBound{r}\AgdaSpace{}%
\AgdaBound{ty}\AgdaSpace{}%
\AgdaBound{tm}\AgdaSymbol{)}\<%
\end{code}
}

\section{Semantics}

\hide{
\begin{code}%
\>[0]\AgdaSymbol{\{-\#}\AgdaSpace{}%
\AgdaKeyword{OPTIONS}\AgdaSpace{}%
\AgdaPragma{--rewriting}\AgdaSpace{}%
\AgdaSymbol{\#-\}}\<%
\\
\>[0]\AgdaKeyword{module}\AgdaSpace{}%
\AgdaModule{Semantics}\AgdaSpace{}%
\AgdaKeyword{where}\<%
\end{code}
}

\hide{
\begin{code}%
\>[0]\AgdaKeyword{open}\AgdaSpace{}%
\AgdaKeyword{import}\AgdaSpace{}%
\AgdaModule{CoreLanguage}\<%
\\
\>[0]\AgdaKeyword{open}\AgdaSpace{}%
\AgdaKeyword{import}\AgdaSpace{}%
\AgdaModule{Pattern}\AgdaSpace{}%
\AgdaKeyword{using}\AgdaSpace{}%
\AgdaSymbol{(}\AgdaDatatype{Pattern}\AgdaSymbol{;}\AgdaSpace{}%
\AgdaOperator{\AgdaInductiveConstructor{\AgdaUnderscore{}∙\AgdaUnderscore{}}}\AgdaSymbol{;}\AgdaSpace{}%
\AgdaOperator{\AgdaFunction{\AgdaUnderscore{}⊗\AgdaUnderscore{}}}\AgdaSymbol{;}\AgdaSpace{}%
\AgdaOperator{\AgdaDatatype{\AgdaUnderscore{}-Env}}\AgdaSymbol{;}\AgdaSpace{}%
\AgdaFunction{match}\AgdaSymbol{)}\<%
\\
\>[0]\AgdaKeyword{open}\AgdaSpace{}%
\AgdaKeyword{import}\AgdaSpace{}%
\AgdaModule{Context}\AgdaSpace{}%
\AgdaKeyword{using}\AgdaSpace{}%
\AgdaSymbol{(}\AgdaFunction{Context}\AgdaSymbol{)}\AgdaSpace{}%
\AgdaKeyword{renaming}\AgdaSpace{}%
\AgdaSymbol{(}\AgdaOperator{\AgdaFunction{\AgdaUnderscore{},\AgdaUnderscore{}}}\AgdaSpace{}%
\AgdaSymbol{to}\AgdaSpace{}%
\AgdaOperator{\AgdaFunction{\AgdaUnderscore{}-,\AgdaUnderscore{}}}\AgdaSymbol{)}\<%
\\
\>[0]\AgdaKeyword{open}\AgdaSpace{}%
\AgdaKeyword{import}\AgdaSpace{}%
\AgdaModule{Data.String}\AgdaSpace{}%
\AgdaKeyword{using}\AgdaSpace{}%
\AgdaSymbol{(}\AgdaOperator{\AgdaFunction{\AgdaUnderscore{}++\AgdaUnderscore{}}}\AgdaSymbol{)}\<%
\\
\>[0]\AgdaKeyword{open}\AgdaSpace{}%
\AgdaKeyword{import}\AgdaSpace{}%
\AgdaModule{Expression}\AgdaSpace{}%
\AgdaKeyword{using}\AgdaSpace{}%
\AgdaSymbol{(}\AgdaFunction{toTerm}\AgdaSymbol{;}\AgdaSpace{}%
\AgdaDatatype{Con}\AgdaSymbol{)}\<%
\\
\>[0]\AgdaKeyword{open}\AgdaSpace{}%
\AgdaKeyword{import}\AgdaSpace{}%
\AgdaModule{Data.Product}\AgdaSpace{}%
\AgdaKeyword{using}\AgdaSpace{}%
\AgdaSymbol{(}\AgdaOperator{\AgdaFunction{\AgdaUnderscore{}×\AgdaUnderscore{}}}\AgdaSymbol{;}\AgdaSpace{}%
\AgdaOperator{\AgdaInductiveConstructor{\AgdaUnderscore{},\AgdaUnderscore{}}}\AgdaSymbol{;}\AgdaSpace{}%
\AgdaFunction{Σ-syntax}\AgdaSymbol{;}\AgdaSpace{}%
\AgdaField{proj₂}\AgdaSymbol{)}\<%
\\
\>[0]\AgdaKeyword{open}\AgdaSpace{}%
\AgdaKeyword{import}\AgdaSpace{}%
\AgdaModule{Data.List}\AgdaSpace{}%
\AgdaKeyword{using}\AgdaSpace{}%
\AgdaSymbol{(}\AgdaDatatype{List}\AgdaSymbol{;}\AgdaSpace{}%
\AgdaInductiveConstructor{[]}\AgdaSymbol{;}\AgdaSpace{}%
\AgdaOperator{\AgdaInductiveConstructor{\AgdaUnderscore{}∷\AgdaUnderscore{}}}\AgdaSymbol{)}\<%
\\
\>[0]\AgdaKeyword{open}\AgdaSpace{}%
\AgdaKeyword{import}\AgdaSpace{}%
\AgdaModule{Data.Maybe}\AgdaSpace{}%
\AgdaKeyword{using}\AgdaSpace{}%
\AgdaSymbol{(}\AgdaDatatype{Maybe}\AgdaSymbol{;}\AgdaSpace{}%
\AgdaInductiveConstructor{just}\AgdaSymbol{;}\AgdaSpace{}%
\AgdaInductiveConstructor{nothing}\AgdaSymbol{)}\<%
\\
\>[0]\AgdaKeyword{open}\AgdaSpace{}%
\AgdaKeyword{import}\AgdaSpace{}%
\AgdaModule{Failable}\AgdaSpace{}%
\AgdaKeyword{using}\AgdaSpace{}%
\AgdaSymbol{(}\AgdaDatatype{Failable}\AgdaSymbol{;}\AgdaSpace{}%
\AgdaInductiveConstructor{succeed}\AgdaSymbol{;}\AgdaSpace{}%
\AgdaInductiveConstructor{fail}\AgdaSymbol{)}\<%
\\
\>[0]\AgdaKeyword{open}\AgdaSpace{}%
\AgdaKeyword{import}\AgdaSpace{}%
\AgdaModule{Data.Nat}\AgdaSpace{}%
\AgdaKeyword{using}\AgdaSpace{}%
\AgdaSymbol{(}\AgdaOperator{\AgdaPrimitive{\AgdaUnderscore{}+\AgdaUnderscore{}}}\AgdaSymbol{;}\AgdaSpace{}%
\AgdaOperator{\AgdaFunction{\AgdaUnderscore{}≟\AgdaUnderscore{}}}\AgdaSymbol{)}\<%
\\
\>[0]\AgdaKeyword{open}\AgdaSpace{}%
\AgdaKeyword{import}\AgdaSpace{}%
\AgdaModule{Thinning}\AgdaSpace{}%
\AgdaKeyword{using}\AgdaSpace{}%
\AgdaSymbol{(}\AgdaOperator{\AgdaFunction{\AgdaUnderscore{}⟨term\AgdaUnderscore{}}}\AgdaSymbol{)}\<%
\\
\>[0]\AgdaKeyword{open}\AgdaSpace{}%
\AgdaKeyword{import}\AgdaSpace{}%
\AgdaModule{Rules}\AgdaSpace{}%
\AgdaKeyword{using}\AgdaSpace{}%
\AgdaSymbol{(}\AgdaRecord{∋rule}\AgdaSymbol{;}\AgdaSpace{}%
\AgdaFunction{typeOf}\AgdaSymbol{;}\AgdaSpace{}%
\AgdaFunction{bind-count}\AgdaSymbol{)}\<%
\\
\>[0]\AgdaKeyword{open}\AgdaSpace{}%
\AgdaKeyword{import}\AgdaSpace{}%
\AgdaModule{Pattern}\AgdaSpace{}%
\AgdaKeyword{using}%
\>[75I]\AgdaSymbol{(}\AgdaInductiveConstructor{`}\AgdaSymbol{;}\AgdaSpace{}%
\AgdaOperator{\AgdaInductiveConstructor{\AgdaUnderscore{}∙\AgdaUnderscore{}}}\AgdaSymbol{;}\AgdaSpace{}%
\AgdaInductiveConstructor{bind}\AgdaSymbol{;}\AgdaSpace{}%
\AgdaInductiveConstructor{place}\AgdaSymbol{;}\AgdaSpace{}%
\AgdaInductiveConstructor{thing}\AgdaSymbol{;}\AgdaSpace{}%
\AgdaDatatype{svar}\AgdaSymbol{;}\AgdaSpace{}%
\AgdaDatatype{svar-builder}\AgdaSymbol{;}\<%
\\
\>[75I][@{}l@{\AgdaIndent{0}}]%
\>[27]\AgdaInductiveConstructor{X}\AgdaSymbol{;}\AgdaSpace{}%
\AgdaFunction{⇚}\AgdaSymbol{;}\AgdaSpace{}%
\AgdaFunction{⇛}\AgdaSymbol{;}\AgdaSpace{}%
\AgdaFunction{↳}\AgdaSymbol{;}\AgdaSpace{}%
\AgdaFunction{build}\AgdaSymbol{;}\AgdaSpace{}%
\AgdaFunction{bind-count-bl}\AgdaSymbol{)}\<%
\\
\>[0]\AgdaKeyword{open}\AgdaSpace{}%
\AgdaKeyword{import}\AgdaSpace{}%
\AgdaModule{Relation.Nullary}\AgdaSpace{}%
\AgdaKeyword{using}\AgdaSpace{}%
\AgdaSymbol{(}\AgdaInductiveConstructor{yes}\AgdaSymbol{;}\AgdaSpace{}%
\AgdaInductiveConstructor{no}\AgdaSymbol{;}\AgdaSpace{}%
\AgdaOperator{\AgdaFunction{¬\AgdaUnderscore{}}}\AgdaSymbol{)}\<%
\\
\>[0]\AgdaKeyword{open}\AgdaSpace{}%
\AgdaKeyword{import}\AgdaSpace{}%
\AgdaModule{Function}\AgdaSpace{}%
\AgdaKeyword{using}\AgdaSpace{}%
\AgdaSymbol{(}\AgdaOperator{\AgdaFunction{\AgdaUnderscore{}∘\AgdaUnderscore{}}}\AgdaSymbol{)}\<%
\\
\>[0]\AgdaKeyword{open}\AgdaSpace{}%
\AgdaKeyword{import}\AgdaSpace{}%
\AgdaModule{Relation.Binary.PropositionalEquality}\AgdaSpace{}%
\AgdaKeyword{using}\AgdaSpace{}%
\AgdaSymbol{(}\AgdaFunction{sym}\AgdaSymbol{;}\AgdaSpace{}%
\AgdaFunction{subst}\AgdaSymbol{;}\AgdaSpace{}%
\AgdaOperator{\AgdaDatatype{\AgdaUnderscore{}≡\AgdaUnderscore{}}}\AgdaSymbol{;}\AgdaSpace{}%
\AgdaInductiveConstructor{refl}\AgdaSymbol{)}\<%
\\
\>[0]\AgdaKeyword{open}\AgdaSpace{}%
\AgdaModule{∋rule}\<%
\end{code}
}

\hide{
\begin{code}%
\>[0]\AgdaKeyword{private}\<%
\\
\>[0][@{}l@{\AgdaIndent{0}}]%
\>[2]\AgdaKeyword{variable}\<%
\\
\>[2][@{}l@{\AgdaIndent{0}}]%
\>[4]\AgdaGeneralizable{γ}\AgdaSpace{}%
\AgdaSymbol{:}\AgdaSpace{}%
\AgdaFunction{Scope}\<%
\\
%
\>[4]\AgdaGeneralizable{δ}\AgdaSpace{}%
\AgdaSymbol{:}\AgdaSpace{}%
\AgdaFunction{Scope}\<%
\\
%
\>[4]\AgdaGeneralizable{d}\AgdaSpace{}%
\AgdaSymbol{:}\AgdaSpace{}%
\AgdaDatatype{Dir}\<%
\end{code}
}

Given the way that we represent typing rules, our type for $β$-rules
should not look unfamiliar. We match a rule by matching patterns for the
target, target type and eliminator and construct the reduced term
and its type from an expression and the resulting environment we
computed in the matching process.

Since we are matching a rule by matching patterns, the target, type and
eliminator in question must be in weak head normal form \hl{verify} and
so care should be taken to compute these forms before attempting to match
a β-rule.
\begin{code}%
\>[0]\AgdaKeyword{record}\AgdaSpace{}%
\AgdaRecord{β-rule}\AgdaSpace{}%
\AgdaSymbol{:}\AgdaSpace{}%
\AgdaPrimitiveType{Set}\AgdaSpace{}%
\AgdaKeyword{where}\<%
\\
\>[0][@{}l@{\AgdaIndent{0}}]%
\>[2]\AgdaKeyword{field}\<%
\\
\>[2][@{}l@{\AgdaIndent{0}}]%
\>[4]\AgdaField{target}\AgdaSpace{}%
\AgdaField{targetType}\AgdaSpace{}%
\AgdaField{eliminator}\AgdaSpace{}%
\AgdaSymbol{:}\AgdaSpace{}%
\AgdaDatatype{Pattern}\AgdaSpace{}%
\AgdaNumber{0}\<%
\\
%
\>[4]\AgdaField{redTerm}\AgdaSpace{}%
\AgdaField{redType}\AgdaSpace{}%
\AgdaSymbol{:}\AgdaSpace{}%
\AgdaDatatype{Con}\AgdaSpace{}%
\AgdaSymbol{(}\AgdaField{target}\AgdaSpace{}%
\AgdaOperator{\AgdaInductiveConstructor{∙}}\AgdaSpace{}%
\AgdaField{targetType}\AgdaSpace{}%
\AgdaOperator{\AgdaInductiveConstructor{∙}}\AgdaSpace{}%
\AgdaField{eliminator}\AgdaSymbol{)}\AgdaSpace{}%
\AgdaNumber{0}\<%
\\
%
\\[\AgdaEmptyExtraSkip]%
%
\>[2]\AgdaKeyword{open}\AgdaSpace{}%
\AgdaModule{Data.Maybe}\AgdaSpace{}%
\AgdaKeyword{using}\AgdaSpace{}%
\AgdaSymbol{(}\AgdaOperator{\AgdaFunction{\AgdaUnderscore{}>>=\AgdaUnderscore{}}}\AgdaSymbol{)}\<%
\\
%
\\[\AgdaEmptyExtraSkip]%
%
\>[2]\AgdaFunction{Rule-Env}\AgdaSpace{}%
\AgdaSymbol{:}\AgdaSpace{}%
\AgdaSymbol{\{}\AgdaBound{γ}\AgdaSpace{}%
\AgdaSymbol{:}\AgdaSpace{}%
\AgdaFunction{Scope}\AgdaSymbol{\}}\AgdaSpace{}%
\AgdaSymbol{→}\AgdaSpace{}%
\AgdaPrimitiveType{Set}\<%
\\
%
\>[2]\AgdaFunction{Rule-Env}\AgdaSpace{}%
\AgdaSymbol{\{}\AgdaBound{γ}\AgdaSymbol{\}}\AgdaSpace{}%
\AgdaSymbol{=}%
\>[140I]\AgdaSymbol{((}\AgdaBound{γ}\AgdaSpace{}%
\AgdaOperator{\AgdaFunction{⊗}}\AgdaSpace{}%
\AgdaField{target}\AgdaSymbol{)}%
\>[36]\AgdaOperator{\AgdaInductiveConstructor{∙}}\<%
\\
\>[140I][@{}l@{\AgdaIndent{0}}]%
\>[18]\AgdaSymbol{(}\AgdaBound{γ}\AgdaSpace{}%
\AgdaOperator{\AgdaFunction{⊗}}\AgdaSpace{}%
\AgdaField{targetType}\AgdaSymbol{)}%
\>[36]\AgdaOperator{\AgdaInductiveConstructor{∙}}\<%
\\
%
\>[18]\AgdaSymbol{(}\AgdaBound{γ}\AgdaSpace{}%
\AgdaOperator{\AgdaFunction{⊗}}\AgdaSpace{}%
\AgdaField{eliminator}\AgdaSymbol{))}\AgdaSpace{}%
\AgdaOperator{\AgdaDatatype{-Env}}\<%
\\
%
\\[\AgdaEmptyExtraSkip]%
%
\>[2]\AgdaFunction{β-match}\AgdaSpace{}%
\AgdaSymbol{:}\AgdaSpace{}%
\AgdaSymbol{(}\AgdaBound{targ}\AgdaSpace{}%
\AgdaBound{type}\AgdaSpace{}%
\AgdaBound{elim}\AgdaSpace{}%
\AgdaSymbol{:}\AgdaSpace{}%
\AgdaDatatype{Const}\AgdaSpace{}%
\AgdaGeneralizable{γ}\AgdaSymbol{)}\AgdaSpace{}%
\AgdaSymbol{→}\AgdaSpace{}%
\AgdaDatatype{Maybe}\AgdaSpace{}%
\AgdaFunction{Rule-Env}\<%
\\
%
\>[2]\AgdaFunction{β-match}\AgdaSpace{}%
\AgdaBound{tar}\AgdaSpace{}%
\AgdaBound{ty}\AgdaSpace{}%
\AgdaBound{el}\AgdaSpace{}%
\AgdaSymbol{=}%
\>[162I]\AgdaKeyword{do}\<%
\\
\>[162I][@{}l@{\AgdaIndent{0}}]%
\>[24]\AgdaBound{t-env}%
\>[31]\AgdaOperator{\AgdaFunction{←}}\AgdaSpace{}%
\AgdaFunction{match}\AgdaSpace{}%
\AgdaBound{tar}\AgdaSpace{}%
\AgdaField{target}\<%
\\
%
\>[24]\AgdaBound{ty-env}\AgdaSpace{}%
\AgdaOperator{\AgdaFunction{←}}\AgdaSpace{}%
\AgdaFunction{match}\AgdaSpace{}%
\AgdaBound{ty}\AgdaSpace{}%
\AgdaField{targetType}\<%
\\
%
\>[24]\AgdaBound{e-env}%
\>[31]\AgdaOperator{\AgdaFunction{←}}\AgdaSpace{}%
\AgdaFunction{match}\AgdaSpace{}%
\AgdaBound{el}\AgdaSpace{}%
\AgdaField{eliminator}\<%
\\
%
\>[24]\AgdaInductiveConstructor{just}\AgdaSpace{}%
\AgdaSymbol{(}\AgdaBound{t-env}\AgdaSpace{}%
\AgdaOperator{\AgdaInductiveConstructor{∙}}\AgdaSpace{}%
\AgdaBound{ty-env}\AgdaSpace{}%
\AgdaOperator{\AgdaInductiveConstructor{∙}}\AgdaSpace{}%
\AgdaBound{e-env}\AgdaSymbol{)}\<%
\\
\>[0]\<%
\\
%
\\[\AgdaEmptyExtraSkip]%
%
\>[2]\AgdaFunction{β-reduce}%
\>[12]\AgdaSymbol{:}%
\>[15]\AgdaFunction{Rule-Env}\AgdaSpace{}%
\AgdaSymbol{\{}\AgdaGeneralizable{γ}\AgdaSymbol{\}}\AgdaSpace{}%
\AgdaSymbol{→}\AgdaSpace{}%
\AgdaDatatype{Compu}\AgdaSpace{}%
\AgdaGeneralizable{γ}\<%
\\
%
\>[2]\AgdaFunction{β-reduce}\AgdaSpace{}%
\AgdaBound{env}\<%
\\
\>[2][@{}l@{\AgdaIndent{0}}]%
\>[4]\AgdaSymbol{=}\AgdaSpace{}%
\AgdaFunction{↞↞}\AgdaSpace{}%
\AgdaSymbol{(}\AgdaFunction{toTerm}\AgdaSpace{}%
\AgdaBound{env}\AgdaSpace{}%
\AgdaField{redTerm}\AgdaSymbol{)}\AgdaSpace{}%
\AgdaSymbol{(}\AgdaFunction{toTerm}\AgdaSpace{}%
\AgdaBound{env}\AgdaSpace{}%
\AgdaField{redType}\AgdaSymbol{)}\<%
\\
\>[0]\AgdaKeyword{open}\AgdaSpace{}%
\AgdaModule{β-rule}\<%
\end{code}
We then define a function that will attempt a reduction with regards
to a list of β-rules by trying to match and apply a rule.
\hide{
\begin{code}%
\>[0]\AgdaKeyword{open}\AgdaSpace{}%
\AgdaKeyword{import}\AgdaSpace{}%
\AgdaModule{Failable}\AgdaSpace{}%
\AgdaKeyword{using}\AgdaSpace{}%
\AgdaSymbol{(}\AgdaOperator{\AgdaFunction{\AgdaUnderscore{}>>=\AgdaUnderscore{}}}\AgdaSymbol{)}\<%
\end{code}
}
\begin{code}%
\>[0]\AgdaFunction{findRule}\AgdaSpace{}%
\AgdaSymbol{:}%
\>[196I]\AgdaDatatype{List}\AgdaSpace{}%
\AgdaRecord{β-rule}\AgdaSpace{}%
\AgdaSymbol{→}\<%
\\
\>[.][@{}l@{}]\<[196I]%
\>[11]\AgdaSymbol{(}\AgdaBound{tar}\AgdaSpace{}%
\AgdaBound{type}\AgdaSpace{}%
\AgdaBound{elim}\AgdaSpace{}%
\AgdaSymbol{:}\AgdaSpace{}%
\AgdaDatatype{Const}\AgdaSpace{}%
\AgdaGeneralizable{γ}\AgdaSymbol{)}%
\>[38]\AgdaSymbol{→}\<%
\\
%
\>[11]\AgdaDatatype{Failable}\AgdaSpace{}%
\AgdaSymbol{(}\AgdaSpace{}%
\AgdaFunction{Σ[}\AgdaSpace{}%
\AgdaBound{r}\AgdaSpace{}%
\AgdaFunction{∈}\AgdaSpace{}%
\AgdaRecord{β-rule}\AgdaSpace{}%
\AgdaFunction{]}\AgdaSpace{}%
\AgdaFunction{Rule-Env}\AgdaSpace{}%
\AgdaBound{r}\AgdaSpace{}%
\AgdaSymbol{\{}\AgdaGeneralizable{γ}\AgdaSymbol{\}}\AgdaSpace{}%
\AgdaSymbol{)}\<%
\\
\>[0]\AgdaFunction{findRule}\AgdaSpace{}%
\AgdaInductiveConstructor{[]}\AgdaSpace{}%
\AgdaBound{t}\AgdaSpace{}%
\AgdaBound{ty}\AgdaSpace{}%
\AgdaBound{e}\AgdaSpace{}%
\AgdaSymbol{=}\AgdaSpace{}%
\AgdaInductiveConstructor{fail}%
\>[220I]\AgdaSymbol{(}\AgdaString{"No matching β-rule found for "}\AgdaSpace{}%
\AgdaOperator{\AgdaFunction{++}}\<%
\\
\>[220I][@{}l@{\AgdaIndent{0}}]%
\>[27]\AgdaFunction{print}\AgdaSpace{}%
\AgdaBound{t}\AgdaSpace{}%
\AgdaOperator{\AgdaFunction{++}}\AgdaSpace{}%
\AgdaString{" : "}\AgdaSpace{}%
\AgdaOperator{\AgdaFunction{++}}\AgdaSpace{}%
\AgdaFunction{print}\AgdaSpace{}%
\AgdaBound{ty}\AgdaSpace{}%
\AgdaOperator{\AgdaFunction{++}}\<%
\\
%
\>[27]\AgdaString{" eliminated by "}\AgdaSpace{}%
\AgdaOperator{\AgdaFunction{++}}\AgdaSpace{}%
\AgdaFunction{print}\AgdaSpace{}%
\AgdaBound{e}\AgdaSymbol{)}\<%
\\
\>[0]\AgdaFunction{findRule}\AgdaSpace{}%
\AgdaSymbol{(}\AgdaBound{r}\AgdaSpace{}%
\AgdaOperator{\AgdaInductiveConstructor{∷}}\AgdaSpace{}%
\AgdaBound{rs}\AgdaSymbol{)}\AgdaSpace{}%
\AgdaBound{t}\AgdaSpace{}%
\AgdaBound{ty}\AgdaSpace{}%
\AgdaBound{e}\AgdaSpace{}%
\AgdaKeyword{with}\AgdaSpace{}%
\AgdaFunction{β-match}\AgdaSpace{}%
\AgdaBound{r}\AgdaSpace{}%
\AgdaBound{t}\AgdaSpace{}%
\AgdaBound{ty}\AgdaSpace{}%
\AgdaBound{e}\<%
\\
\>[0]\AgdaSymbol{...}\AgdaSpace{}%
\AgdaSymbol{|}\AgdaSpace{}%
\AgdaInductiveConstructor{nothing}%
\>[16]\AgdaSymbol{=}\AgdaSpace{}%
\AgdaFunction{findRule}\AgdaSpace{}%
\AgdaBound{rs}\AgdaSpace{}%
\AgdaBound{t}\AgdaSpace{}%
\AgdaBound{ty}\AgdaSpace{}%
\AgdaBound{e}\<%
\\
\>[0]\AgdaSymbol{...}\AgdaSpace{}%
\AgdaSymbol{|}\AgdaSpace{}%
\AgdaInductiveConstructor{just}\AgdaSpace{}%
\AgdaBound{env}%
\>[16]\AgdaSymbol{=}\AgdaSpace{}%
\AgdaInductiveConstructor{succeed}\AgdaSpace{}%
\AgdaSymbol{(}\AgdaBound{r}\AgdaSpace{}%
\AgdaOperator{\AgdaInductiveConstructor{,}}\AgdaSpace{}%
\AgdaBound{env}\AgdaSymbol{)}\<%
\\
%
\\[\AgdaEmptyExtraSkip]%
\>[0]\AgdaFunction{reduce}\AgdaSpace{}%
\AgdaSymbol{:}%
\>[259I]\AgdaDatatype{List}\AgdaSpace{}%
\AgdaRecord{β-rule}%
\>[34]\AgdaSymbol{→}\<%
\\
\>[.][@{}l@{}]\<[259I]%
\>[9]\AgdaSymbol{(}\AgdaBound{tar}\AgdaSpace{}%
\AgdaBound{type}\AgdaSpace{}%
\AgdaBound{elim}\AgdaSpace{}%
\AgdaSymbol{:}\AgdaSpace{}%
\AgdaDatatype{Const}\AgdaSpace{}%
\AgdaGeneralizable{γ}\AgdaSymbol{)}%
\>[36]\AgdaSymbol{→}\<%
\\
%
\>[9]\AgdaDatatype{Failable}\AgdaSpace{}%
\AgdaSymbol{(}\AgdaDatatype{Compu}\AgdaSpace{}%
\AgdaGeneralizable{γ}\AgdaSymbol{)}\<%
\\
\>[0]\AgdaFunction{reduce}\AgdaSpace{}%
\AgdaBound{rules}\AgdaSpace{}%
\AgdaBound{ta}\AgdaSpace{}%
\AgdaBound{ty}\AgdaSpace{}%
\AgdaBound{el}\<%
\\
\>[0][@{}l@{\AgdaIndent{0}}]%
\>[2]\AgdaSymbol{=}%
\>[272I]\AgdaKeyword{do}\<%
\\
\>[272I][@{}l@{\AgdaIndent{0}}]%
\>[6]\AgdaSymbol{(}\AgdaBound{rule}\AgdaSpace{}%
\AgdaOperator{\AgdaInductiveConstructor{,}}\AgdaSpace{}%
\AgdaBound{env}\AgdaSymbol{)}\AgdaSpace{}%
\AgdaOperator{\AgdaFunction{←}}\AgdaSpace{}%
\AgdaFunction{findRule}\AgdaSpace{}%
\AgdaBound{rules}\AgdaSpace{}%
\AgdaBound{ta}\AgdaSpace{}%
\AgdaBound{ty}\AgdaSpace{}%
\AgdaBound{el}\<%
\\
%
\>[6]\AgdaInductiveConstructor{succeed}\AgdaSpace{}%
\AgdaSymbol{(}\AgdaFunction{β-reduce}\AgdaSpace{}%
\AgdaBound{rule}\AgdaSpace{}%
\AgdaBound{env}\AgdaSymbol{)}\<%
\end{code}
Finally, we implement normalization by evaluation. We first define an evaluation
function that works in terms of a generic means of reduction and type synthesis
then provide our implementation of \emph{reflect} which eta-expands sub-terms of
the term so that the head of the sub-term matches the constructor of its given
type. Normalisation is then the composition of these functions - reducing as far
as possible before building the correct head form at each type. \hl{expand?}
\hide{
\begin{code}%
\>[0]\AgdaKeyword{open}\AgdaSpace{}%
\AgdaKeyword{import}\AgdaSpace{}%
\AgdaModule{EtaRule}\<%
\\
\>[0]\AgdaKeyword{open}\AgdaSpace{}%
\AgdaModule{η-Rule}\<%
\\
\>[0]\AgdaSymbol{\{-\#}\AgdaSpace{}%
\AgdaKeyword{TERMINATING}\AgdaSpace{}%
\AgdaSymbol{\#-\}}\<%
\end{code}
}
\begin{code}%
\>[0]\AgdaOperator{\AgdaFunction{\AgdaUnderscore{}-\AgdaUnderscore{}-\AgdaUnderscore{}∥\AgdaUnderscore{}∥}}%
\>[289I]\AgdaSymbol{:}%
\>[12]\AgdaSymbol{(}\AgdaBound{reducer}\AgdaSpace{}%
\AgdaSymbol{:}\AgdaSpace{}%
\AgdaSymbol{∀}\AgdaSpace{}%
\AgdaSymbol{\{}\AgdaBound{γ}\AgdaSymbol{\}}\AgdaSpace{}%
\AgdaSymbol{→}\AgdaSpace{}%
\AgdaSymbol{(}\AgdaBound{tar}\AgdaSpace{}%
\AgdaBound{type}\AgdaSpace{}%
\AgdaBound{elim}\AgdaSpace{}%
\AgdaSymbol{:}\AgdaSpace{}%
\AgdaDatatype{Const}\AgdaSpace{}%
\AgdaBound{γ}\AgdaSymbol{)}\AgdaSpace{}%
\AgdaSymbol{→}\AgdaSpace{}%
\AgdaDatatype{Failable}\AgdaSpace{}%
\AgdaSymbol{(}\AgdaDatatype{Compu}\AgdaSpace{}%
\AgdaBound{γ}\AgdaSymbol{))}\AgdaSpace{}%
\AgdaSymbol{→}\<%
\\
\>[289I][@{}l@{\AgdaIndent{0}}]%
\>[11]\AgdaSymbol{(}\AgdaBound{inferer}\AgdaSpace{}%
\AgdaSymbol{:}\AgdaSpace{}%
\AgdaSymbol{∀}\AgdaSpace{}%
\AgdaSymbol{\{}\AgdaBound{γ}\AgdaSymbol{\}}\AgdaSpace{}%
\AgdaSymbol{→}\AgdaSpace{}%
\AgdaFunction{Context}\AgdaSpace{}%
\AgdaBound{γ}\AgdaSpace{}%
\AgdaSymbol{→}\AgdaSpace{}%
\AgdaSymbol{(}\AgdaBound{term}\AgdaSpace{}%
\AgdaSymbol{:}\AgdaSpace{}%
\AgdaFunction{Term}\AgdaSpace{}%
\AgdaInductiveConstructor{compu}\AgdaSpace{}%
\AgdaBound{γ}\AgdaSymbol{)}\AgdaSpace{}%
\AgdaSymbol{→}\AgdaSpace{}%
\AgdaDatatype{Failable}\AgdaSpace{}%
\AgdaSymbol{(}\AgdaFunction{Term}\AgdaSpace{}%
\AgdaInductiveConstructor{const}\AgdaSpace{}%
\AgdaBound{γ}\AgdaSymbol{))}\AgdaSpace{}%
\AgdaSymbol{→}\<%
\\
%
\>[11]\AgdaFunction{Context}\AgdaSpace{}%
\AgdaGeneralizable{γ}\AgdaSpace{}%
\AgdaSymbol{→}\<%
\\
%
\>[11]\AgdaFunction{Term}\AgdaSpace{}%
\AgdaGeneralizable{d}\AgdaSpace{}%
\AgdaGeneralizable{γ}\AgdaSpace{}%
\AgdaSymbol{→}\<%
\\
%
\>[11]\AgdaDatatype{Const}\AgdaSpace{}%
\AgdaGeneralizable{γ}\<%
\\
\>[0]\AgdaBound{rd}\AgdaSpace{}%
\AgdaOperator{\AgdaFunction{-}}\AgdaSpace{}%
\AgdaBound{inf}\AgdaSpace{}%
\AgdaOperator{\AgdaFunction{-}}\AgdaSpace{}%
\AgdaBound{Γ}\AgdaSpace{}%
\AgdaOperator{\AgdaFunction{∥}}\AgdaSpace{}%
\AgdaBound{T}\AgdaSpace{}%
\AgdaOperator{\AgdaFunction{∥}}\AgdaSpace{}%
\AgdaSymbol{=}\AgdaSpace{}%
\AgdaOperator{\AgdaFunction{⟦}}\AgdaSpace{}%
\AgdaBound{T}\AgdaSpace{}%
\AgdaOperator{\AgdaFunction{⟧}}\AgdaSpace{}%
\AgdaBound{Γ}\<%
\\
\>[0][@{}l@{\AgdaIndent{0}}]%
\>[2]\AgdaKeyword{where}\<%
\\
\>[2][@{}l@{\AgdaIndent{0}}]%
\>[4]\AgdaOperator{\AgdaFunction{⟦\AgdaUnderscore{}⟧}}\AgdaSpace{}%
\AgdaSymbol{:}\AgdaSpace{}%
\AgdaFunction{Term}\AgdaSpace{}%
\AgdaGeneralizable{d}\AgdaSpace{}%
\AgdaGeneralizable{γ}\AgdaSpace{}%
\AgdaSymbol{→}\AgdaSpace{}%
\AgdaFunction{Context}\AgdaSpace{}%
\AgdaGeneralizable{γ}\AgdaSpace{}%
\AgdaSymbol{→}\AgdaSpace{}%
\AgdaDatatype{Const}\AgdaSpace{}%
\AgdaGeneralizable{γ}\<%
\\
%
\>[4]\AgdaOperator{\AgdaFunction{⟦\AgdaUnderscore{}⟧}}\AgdaSpace{}%
\AgdaSymbol{\{}\AgdaInductiveConstructor{const}\AgdaSymbol{\}}\AgdaSpace{}%
\AgdaSymbol{(}\AgdaInductiveConstructor{`}\AgdaSpace{}%
\AgdaBound{x}\AgdaSymbol{)}\AgdaSpace{}%
\AgdaBound{Γ}%
\>[30]\AgdaSymbol{=}\AgdaSpace{}%
\AgdaInductiveConstructor{`}\AgdaSpace{}%
\AgdaBound{x}\<%
\\
%
\>[4]\AgdaOperator{\AgdaFunction{⟦\AgdaUnderscore{}⟧}}\AgdaSpace{}%
\AgdaSymbol{\{}\AgdaInductiveConstructor{const}\AgdaSymbol{\}}\AgdaSpace{}%
\AgdaSymbol{(}\AgdaBound{s}\AgdaSpace{}%
\AgdaOperator{\AgdaInductiveConstructor{∙}}\AgdaSpace{}%
\AgdaBound{t}\AgdaSymbol{)}\AgdaSpace{}%
\AgdaBound{Γ}%
\>[30]\AgdaSymbol{=}\AgdaSpace{}%
\AgdaOperator{\AgdaFunction{⟦}}\AgdaSpace{}%
\AgdaBound{s}\AgdaSpace{}%
\AgdaOperator{\AgdaFunction{⟧}}\AgdaSpace{}%
\AgdaBound{Γ}\AgdaSpace{}%
\AgdaOperator{\AgdaInductiveConstructor{∙}}\AgdaSpace{}%
\AgdaOperator{\AgdaFunction{⟦}}\AgdaSpace{}%
\AgdaBound{t}\AgdaSpace{}%
\AgdaOperator{\AgdaFunction{⟧}}\AgdaSpace{}%
\AgdaBound{Γ}\<%
\\
%
\>[4]\AgdaOperator{\AgdaFunction{⟦\AgdaUnderscore{}⟧}}\AgdaSpace{}%
\AgdaSymbol{\{}\AgdaInductiveConstructor{const}\AgdaSymbol{\}}\AgdaSpace{}%
\AgdaSymbol{(}\AgdaInductiveConstructor{bind}\AgdaSpace{}%
\AgdaBound{t}\AgdaSymbol{)}\AgdaSpace{}%
\AgdaBound{Γ}%
\>[30]\AgdaSymbol{=}\AgdaSpace{}%
\AgdaInductiveConstructor{bind}\AgdaSpace{}%
\AgdaSymbol{(}\AgdaOperator{\AgdaFunction{⟦}}\AgdaSpace{}%
\AgdaBound{t}\AgdaSpace{}%
\AgdaOperator{\AgdaFunction{⟧}}\AgdaSpace{}%
\AgdaSymbol{(}\AgdaBound{Γ}\AgdaSpace{}%
\AgdaOperator{\AgdaFunction{Context.,}}\AgdaSpace{}%
\AgdaInductiveConstructor{`}\AgdaSpace{}%
\AgdaString{"unknown"}\AgdaSymbol{)}\AgdaSpace{}%
\AgdaSymbol{)}\<%
\\
%
\>[4]\AgdaOperator{\AgdaFunction{⟦\AgdaUnderscore{}⟧}}\AgdaSpace{}%
\AgdaSymbol{\{}\AgdaInductiveConstructor{const}\AgdaSymbol{\}}\AgdaSpace{}%
\AgdaSymbol{(}\AgdaInductiveConstructor{thunk}\AgdaSpace{}%
\AgdaBound{x}\AgdaSymbol{)}\AgdaSpace{}%
\AgdaBound{Γ}%
\>[30]\AgdaSymbol{=}\AgdaSpace{}%
\AgdaOperator{\AgdaFunction{⟦}}\AgdaSpace{}%
\AgdaBound{x}\AgdaSpace{}%
\AgdaOperator{\AgdaFunction{⟧}}\AgdaSpace{}%
\AgdaBound{Γ}\<%
\\
%
\>[4]\AgdaOperator{\AgdaFunction{⟦\AgdaUnderscore{}⟧}}\AgdaSpace{}%
\AgdaSymbol{\{}\AgdaInductiveConstructor{compu}\AgdaSymbol{\}}\AgdaSpace{}%
\AgdaSymbol{(}\AgdaInductiveConstructor{var}\AgdaSpace{}%
\AgdaBound{x}\AgdaSymbol{)}\AgdaSpace{}%
\AgdaBound{Γ}%
\>[30]\AgdaSymbol{=}\AgdaSpace{}%
\AgdaInductiveConstructor{thunk}\AgdaSpace{}%
\AgdaSymbol{(}\AgdaInductiveConstructor{var}\AgdaSpace{}%
\AgdaBound{x}\AgdaSymbol{)}\<%
\\
%
\>[4]\AgdaOperator{\AgdaFunction{⟦\AgdaUnderscore{}⟧}}\AgdaSpace{}%
\AgdaSymbol{\{}\AgdaInductiveConstructor{compu}\AgdaSymbol{\}}\AgdaSpace{}%
\AgdaSymbol{(}\AgdaInductiveConstructor{elim}\AgdaSpace{}%
\AgdaBound{t}\AgdaSpace{}%
\AgdaBound{e}\AgdaSymbol{)}\AgdaSpace{}%
\AgdaBound{Γ}\AgdaSpace{}%
\AgdaKeyword{with}\AgdaSpace{}%
\AgdaBound{inf}\AgdaSpace{}%
\AgdaBound{Γ}\AgdaSpace{}%
\AgdaBound{t}\<%
\\
%
\>[4]\AgdaSymbol{...}\AgdaSpace{}%
\AgdaSymbol{|}\AgdaSpace{}%
\AgdaInductiveConstructor{fail}%
\>[18]\AgdaBound{n}\AgdaSpace{}%
\AgdaSymbol{=}\AgdaSpace{}%
\AgdaInductiveConstructor{thunk}\AgdaSpace{}%
\AgdaSymbol{(}\AgdaInductiveConstructor{elim}\AgdaSpace{}%
\AgdaSymbol{(}\AgdaFunction{↞↞}\AgdaSpace{}%
\AgdaSymbol{(}\AgdaOperator{\AgdaFunction{⟦}}\AgdaSpace{}%
\AgdaBound{t}\AgdaSpace{}%
\AgdaOperator{\AgdaFunction{⟧}}\AgdaSpace{}%
\AgdaBound{Γ}\AgdaSymbol{)}\AgdaSpace{}%
\AgdaSymbol{(}\AgdaInductiveConstructor{`}\AgdaSpace{}%
\AgdaString{"unknown"}\AgdaSymbol{))}\AgdaSpace{}%
\AgdaSymbol{(}\AgdaOperator{\AgdaFunction{⟦}}\AgdaSpace{}%
\AgdaBound{e}\AgdaSpace{}%
\AgdaOperator{\AgdaFunction{⟧}}\AgdaSpace{}%
\AgdaBound{Γ}\AgdaSymbol{))}\<%
\\
%
\>[4]\AgdaSymbol{...}\AgdaSpace{}%
\AgdaSymbol{|}\AgdaSpace{}%
\AgdaInductiveConstructor{succeed}\AgdaSpace{}%
\AgdaBound{ty}\AgdaSpace{}%
\AgdaKeyword{with}\AgdaSpace{}%
\AgdaBound{rd}\AgdaSpace{}%
\AgdaSymbol{(}\AgdaOperator{\AgdaFunction{⟦}}\AgdaSpace{}%
\AgdaBound{t}\AgdaSpace{}%
\AgdaOperator{\AgdaFunction{⟧}}\AgdaSpace{}%
\AgdaBound{Γ}\AgdaSymbol{)}\AgdaSpace{}%
\AgdaBound{ty}\AgdaSpace{}%
\AgdaSymbol{(}\AgdaOperator{\AgdaFunction{⟦}}\AgdaSpace{}%
\AgdaBound{e}\AgdaSpace{}%
\AgdaOperator{\AgdaFunction{⟧}}\AgdaSpace{}%
\AgdaBound{Γ}\AgdaSymbol{)}\<%
\\
%
\>[4]\AgdaSymbol{...}\AgdaSpace{}%
\AgdaSymbol{|}\AgdaSpace{}%
\AgdaInductiveConstructor{succeed}\AgdaSpace{}%
\AgdaBound{x}\AgdaSpace{}%
\AgdaSymbol{=}\AgdaSpace{}%
\AgdaOperator{\AgdaFunction{⟦}}\AgdaSpace{}%
\AgdaBound{x}\AgdaSpace{}%
\AgdaOperator{\AgdaFunction{⟧}}\AgdaSpace{}%
\AgdaBound{Γ}\<%
\\
%
\>[4]\AgdaSymbol{...}\AgdaSpace{}%
\AgdaSymbol{|}\AgdaSpace{}%
\AgdaInductiveConstructor{fail}\AgdaSpace{}%
\AgdaBound{x}%
\>[20]\AgdaSymbol{=}\AgdaSpace{}%
\AgdaInductiveConstructor{thunk}\AgdaSpace{}%
\AgdaSymbol{(}\AgdaInductiveConstructor{elim}\AgdaSpace{}%
\AgdaSymbol{(}\AgdaFunction{↞↞}\AgdaSpace{}%
\AgdaSymbol{(}\AgdaOperator{\AgdaFunction{⟦}}\AgdaSpace{}%
\AgdaBound{t}\AgdaSpace{}%
\AgdaOperator{\AgdaFunction{⟧}}\AgdaSpace{}%
\AgdaBound{Γ}\AgdaSymbol{)}\AgdaSpace{}%
\AgdaBound{ty}\AgdaSymbol{)}\AgdaSpace{}%
\AgdaSymbol{(}\AgdaOperator{\AgdaFunction{⟦}}\AgdaSpace{}%
\AgdaBound{e}\AgdaSpace{}%
\AgdaOperator{\AgdaFunction{⟧}}\AgdaSpace{}%
\AgdaBound{Γ}\AgdaSymbol{))}\<%
\\
%
\>[4]\AgdaOperator{\AgdaFunction{⟦\AgdaUnderscore{}⟧}}\AgdaSpace{}%
\AgdaSymbol{\{}\AgdaInductiveConstructor{compu}\AgdaSymbol{\}}\AgdaSpace{}%
\AgdaSymbol{(}\AgdaBound{t}\AgdaSpace{}%
\AgdaOperator{\AgdaInductiveConstructor{∷}}\AgdaSpace{}%
\AgdaBound{T}\AgdaSymbol{)}\AgdaSpace{}%
\AgdaBound{Γ}%
\>[28]\AgdaSymbol{=}\AgdaSpace{}%
\AgdaOperator{\AgdaFunction{⟦}}\AgdaSpace{}%
\AgdaBound{t}\AgdaSpace{}%
\AgdaOperator{\AgdaFunction{⟧}}\AgdaSpace{}%
\AgdaBound{Γ}\<%
\end{code}
\hide{
\begin{code}%
\>[0]\AgdaSymbol{\{-\#}\AgdaSpace{}%
\AgdaKeyword{TERMINATING}\AgdaSpace{}%
\AgdaSymbol{\#-\}}\<%
\end{code}
}
\begin{code}%
\>[0]\AgdaFunction{qt}\AgdaSpace{}%
\AgdaSymbol{:}\AgdaSpace{}%
\AgdaDatatype{List}\AgdaSpace{}%
\AgdaRecord{η-Rule}\AgdaSpace{}%
\AgdaSymbol{→}\AgdaSpace{}%
\AgdaSymbol{(}\AgdaBound{ty}\AgdaSpace{}%
\AgdaBound{tm}\AgdaSpace{}%
\AgdaSymbol{:}\AgdaSpace{}%
\AgdaDatatype{Const}\AgdaSpace{}%
\AgdaGeneralizable{γ}\AgdaSymbol{)}\AgdaSpace{}%
\AgdaSymbol{→}\AgdaSpace{}%
\AgdaDatatype{Const}\AgdaSpace{}%
\AgdaGeneralizable{γ}\<%
\\
\>[0]\AgdaFunction{qt}\AgdaSpace{}%
\AgdaSymbol{\{}\AgdaArgument{γ}\AgdaSpace{}%
\AgdaSymbol{=}\AgdaSpace{}%
\AgdaBound{γ}\AgdaSymbol{\}}\AgdaSpace{}%
\AgdaBound{rs}\AgdaSpace{}%
\AgdaBound{ty}\AgdaSpace{}%
\AgdaBound{v}\AgdaSpace{}%
\AgdaKeyword{with}\AgdaSpace{}%
\AgdaFunction{EtaRule.findRule}\AgdaSpace{}%
\AgdaBound{rs}\AgdaSpace{}%
\AgdaBound{ty}\<%
\\
\>[0]\AgdaSymbol{...}\AgdaSpace{}%
\AgdaSymbol{|}\AgdaSpace{}%
\AgdaInductiveConstructor{fail}\AgdaSpace{}%
\AgdaBound{x}%
\>[16]\AgdaSymbol{=}\AgdaSpace{}%
\AgdaBound{v}\<%
\\
\>[0]\AgdaSymbol{...}\AgdaSpace{}%
\AgdaSymbol{|}\AgdaSpace{}%
\AgdaInductiveConstructor{succeed}\AgdaSpace{}%
\AgdaSymbol{(}\AgdaBound{r}\AgdaSpace{}%
\AgdaOperator{\AgdaInductiveConstructor{,}}\AgdaSpace{}%
\AgdaBound{i}\AgdaSymbol{)}\AgdaSpace{}%
\AgdaSymbol{=}\AgdaSpace{}%
\AgdaFunction{helper}\AgdaSpace{}%
\AgdaSymbol{(}\AgdaBound{i}\AgdaSpace{}%
\AgdaOperator{\AgdaInductiveConstructor{∙}}\AgdaSpace{}%
\AgdaSymbol{(}\AgdaFunction{eliminations}\AgdaSpace{}%
\AgdaBound{r}\AgdaSpace{}%
\AgdaBound{ty}\AgdaSpace{}%
\AgdaBound{v}\AgdaSymbol{))}\AgdaSpace{}%
\AgdaInductiveConstructor{X}\AgdaSpace{}%
\AgdaSymbol{(}\AgdaFunction{eliminations}\AgdaSpace{}%
\AgdaBound{r}\AgdaSpace{}%
\AgdaBound{ty}\AgdaSpace{}%
\AgdaBound{v}\AgdaSymbol{)}\<%
\\
\>[0][@{}l@{\AgdaIndent{0}}]%
\>[2]\AgdaKeyword{where}\<%
\\
\>[2][@{}l@{\AgdaIndent{0}}]%
\>[4]\AgdaFunction{helper}\AgdaSpace{}%
\AgdaSymbol{:}%
\>[517I]\AgdaSymbol{∀}\AgdaSpace{}%
\AgdaSymbol{\{}\AgdaBound{γ'}\AgdaSymbol{\}\{}\AgdaBound{q}\AgdaSpace{}%
\AgdaSymbol{:}\AgdaSpace{}%
\AgdaDatatype{Pattern}\AgdaSpace{}%
\AgdaBound{γ'}\AgdaSymbol{\}}\AgdaSpace{}%
\AgdaSymbol{→}\<%
\\
\>[.][@{}l@{}]\<[517I]%
\>[13]\AgdaSymbol{((}\AgdaBound{γ}\AgdaSpace{}%
\AgdaOperator{\AgdaFunction{⊗}}\AgdaSpace{}%
\AgdaField{input}\AgdaSpace{}%
\AgdaSymbol{(}\AgdaField{checkRule}\AgdaSpace{}%
\AgdaBound{r}\AgdaSymbol{))}\AgdaSpace{}%
\AgdaOperator{\AgdaInductiveConstructor{∙}}\AgdaSpace{}%
\AgdaSymbol{(}\AgdaBound{γ}\AgdaSpace{}%
\AgdaOperator{\AgdaFunction{⊗}}\AgdaSpace{}%
\AgdaSymbol{(}\AgdaField{subject}\AgdaSpace{}%
\AgdaSymbol{(}\AgdaField{checkRule}\AgdaSpace{}%
\AgdaBound{r}\AgdaSymbol{))))}\AgdaOperator{\AgdaDatatype{-Env}}\AgdaSpace{}%
\AgdaSymbol{→}\<%
\\
%
\>[13]\AgdaSymbol{(}\AgdaBound{v}\AgdaSpace{}%
\AgdaSymbol{:}\AgdaSpace{}%
\AgdaDatatype{svar-builder}\AgdaSpace{}%
\AgdaSymbol{(}\AgdaBound{γ}\AgdaSpace{}%
\AgdaOperator{\AgdaFunction{⊗}}\AgdaSpace{}%
\AgdaSymbol{(}\AgdaField{subject}\AgdaSpace{}%
\AgdaSymbol{(}\AgdaField{checkRule}\AgdaSpace{}%
\AgdaBound{r}\AgdaSymbol{)))}\AgdaSpace{}%
\AgdaBound{q}\AgdaSymbol{)}\AgdaSpace{}%
\AgdaSymbol{→}\<%
\\
%
\>[13]\AgdaBound{q}\AgdaSpace{}%
\AgdaOperator{\AgdaDatatype{-Env}}\AgdaSpace{}%
\AgdaSymbol{→}\<%
\\
%
\>[13]\AgdaDatatype{Const}\AgdaSpace{}%
\AgdaSymbol{((}\AgdaFunction{bind-count-bl}\AgdaSpace{}%
\AgdaBound{v}\AgdaSymbol{)}\AgdaSpace{}%
\AgdaOperator{\AgdaPrimitive{+}}\AgdaSpace{}%
\AgdaBound{γ}\AgdaSymbol{)}\<%
\\
%
\>[4]\AgdaFunction{helper}\AgdaSpace{}%
\AgdaSymbol{\{}\AgdaArgument{γ'}\AgdaSpace{}%
\AgdaSymbol{=}\AgdaSpace{}%
\AgdaBound{γ'}\AgdaSymbol{\}\{}\AgdaArgument{q}\AgdaSpace{}%
\AgdaSymbol{=}\AgdaSpace{}%
\AgdaBound{q}\AgdaSymbol{\}}\AgdaSpace{}%
\AgdaSymbol{(}\AgdaBound{i}\AgdaSpace{}%
\AgdaOperator{\AgdaInductiveConstructor{∙}}\AgdaSpace{}%
\AgdaBound{s}\AgdaSymbol{)}\AgdaSpace{}%
\AgdaBound{v}\AgdaSpace{}%
\AgdaBound{elims}\AgdaSpace{}%
\AgdaKeyword{with}\AgdaSpace{}%
\AgdaBound{q}\AgdaSpace{}%
\AgdaSymbol{|}\AgdaSpace{}%
\AgdaBound{elims}\<%
\\
%
\>[4]\AgdaSymbol{...}\AgdaSpace{}%
\AgdaSymbol{|}\AgdaSpace{}%
\AgdaInductiveConstructor{`}\AgdaSpace{}%
\AgdaBound{x}%
\>[18]\AgdaSymbol{|}\AgdaSpace{}%
\AgdaInductiveConstructor{`}\AgdaSpace{}%
\AgdaSymbol{=}\AgdaSpace{}%
\AgdaInductiveConstructor{`}\AgdaSpace{}%
\AgdaBound{x}\<%
\\
%
\>[4]\AgdaSymbol{...}\AgdaSpace{}%
\AgdaSymbol{|}\AgdaSpace{}%
\AgdaBound{ps}\AgdaSpace{}%
\AgdaOperator{\AgdaInductiveConstructor{∙}}\AgdaSpace{}%
\AgdaBound{pt}\AgdaSpace{}%
\AgdaSymbol{|}\AgdaSpace{}%
\AgdaBound{es}\AgdaSpace{}%
\AgdaOperator{\AgdaInductiveConstructor{∙}}\AgdaSpace{}%
\AgdaBound{et}\<%
\\
\>[4][@{}l@{\AgdaIndent{0}}]%
\>[6]\AgdaSymbol{=}\AgdaSpace{}%
\AgdaFunction{helper}\AgdaSpace{}%
\AgdaSymbol{(}\AgdaBound{i}\AgdaSpace{}%
\AgdaOperator{\AgdaInductiveConstructor{∙}}\AgdaSpace{}%
\AgdaBound{s}\AgdaSymbol{)}\AgdaSpace{}%
\AgdaSymbol{(}\AgdaFunction{⇚}\AgdaSpace{}%
\AgdaBound{v}\AgdaSymbol{)}\AgdaSpace{}%
\AgdaBound{es}\AgdaSpace{}%
\AgdaOperator{\AgdaInductiveConstructor{∙}}\AgdaSpace{}%
\AgdaFunction{helper}\AgdaSpace{}%
\AgdaSymbol{(}\AgdaBound{i}\AgdaSpace{}%
\AgdaOperator{\AgdaInductiveConstructor{∙}}\AgdaSpace{}%
\AgdaBound{s}\AgdaSymbol{)}\AgdaSpace{}%
\AgdaSymbol{(}\AgdaFunction{⇛}\AgdaSpace{}%
\AgdaBound{v}\AgdaSymbol{)}\AgdaSpace{}%
\AgdaBound{et}\<%
\\
%
\>[4]\AgdaSymbol{...}\AgdaSpace{}%
\AgdaSymbol{|}\AgdaSpace{}%
\AgdaInductiveConstructor{bind}\AgdaSpace{}%
\AgdaBound{pt}\AgdaSpace{}%
\AgdaSymbol{|}\AgdaSpace{}%
\AgdaInductiveConstructor{bind}\AgdaSpace{}%
\AgdaBound{et}\<%
\\
\>[4][@{}l@{\AgdaIndent{0}}]%
\>[6]\AgdaSymbol{=}\AgdaSpace{}%
\AgdaInductiveConstructor{bind}\AgdaSpace{}%
\AgdaSymbol{(}\AgdaFunction{helper}\AgdaSpace{}%
\AgdaSymbol{(}\AgdaBound{i}\AgdaSpace{}%
\AgdaOperator{\AgdaInductiveConstructor{∙}}\AgdaSpace{}%
\AgdaBound{s}\AgdaSymbol{)}\AgdaSpace{}%
\AgdaSymbol{(}\AgdaFunction{↳}\AgdaSpace{}%
\AgdaBound{v}\AgdaSymbol{)}\AgdaSpace{}%
\AgdaBound{et}\AgdaSymbol{)}\<%
\\
%
\>[4]\AgdaSymbol{...}\AgdaSpace{}%
\AgdaSymbol{|}\AgdaSpace{}%
\AgdaInductiveConstructor{place}\AgdaSpace{}%
\AgdaBound{θ}\AgdaSpace{}%
\AgdaSymbol{|}\AgdaSpace{}%
\AgdaInductiveConstructor{thing}\AgdaSpace{}%
\AgdaBound{el}\<%
\\
\>[4][@{}l@{\AgdaIndent{0}}]%
\>[6]\AgdaKeyword{with}\AgdaSpace{}%
\AgdaFunction{bind-count}\AgdaSpace{}%
\AgdaSymbol{(}\AgdaFunction{build}\AgdaSpace{}%
\AgdaBound{v}\AgdaSymbol{)}\AgdaSpace{}%
\AgdaOperator{\AgdaPrimitive{+}}\AgdaSpace{}%
\AgdaBound{γ}\AgdaSpace{}%
\AgdaOperator{\AgdaFunction{≟}}\AgdaSpace{}%
\AgdaBound{γ'}\<%
\\
%
\>[4]\AgdaSymbol{...}\AgdaSpace{}%
\AgdaSymbol{|}\AgdaSpace{}%
\AgdaInductiveConstructor{no}\AgdaSpace{}%
\AgdaBound{¬p}\AgdaSpace{}%
\AgdaSymbol{=}\AgdaSpace{}%
\AgdaInductiveConstructor{`}\AgdaSpace{}%
\AgdaString{"Fuck it! It's impossible."}\<%
\\
%
\>[4]\AgdaSymbol{...}\AgdaSpace{}%
\AgdaSymbol{|}%
\>[627I]\AgdaInductiveConstructor{yes}\AgdaSpace{}%
\AgdaBound{p}\AgdaSpace{}%
\AgdaSymbol{=}\AgdaSpace{}%
\AgdaFunction{qt}\AgdaSpace{}%
\AgdaBound{rs}\<%
\\
\>[.][@{}l@{}]\<[627I]%
\>[10]\AgdaSymbol{(}\AgdaFunction{typeOf}\AgdaSpace{}%
\AgdaSymbol{(}\AgdaField{checkRule}\AgdaSpace{}%
\AgdaBound{r}\AgdaSymbol{)}\AgdaSpace{}%
\AgdaSymbol{(}\AgdaFunction{build}\AgdaSpace{}%
\AgdaBound{v}\AgdaSymbol{)}\AgdaSpace{}%
\AgdaBound{i}\AgdaSpace{}%
\AgdaBound{s}\AgdaSymbol{)}\<%
\\
%
\>[10]\AgdaSymbol{(}\AgdaFunction{subst}\AgdaSpace{}%
\AgdaSymbol{(λ}\AgdaSpace{}%
\AgdaBound{x}\AgdaSpace{}%
\AgdaSymbol{→}\AgdaSpace{}%
\AgdaDatatype{Const}\AgdaSpace{}%
\AgdaBound{x}\AgdaSymbol{)}\AgdaSpace{}%
\AgdaSymbol{(}\AgdaFunction{sym}\AgdaSpace{}%
\AgdaBound{p}\AgdaSymbol{)}\AgdaSpace{}%
\AgdaSymbol{((}\AgdaBound{el}\AgdaSpace{}%
\AgdaOperator{\AgdaFunction{⟨term}}\AgdaSpace{}%
\AgdaBound{θ}\AgdaSymbol{)))}\<%
\\
%
\\[\AgdaEmptyExtraSkip]%
\>[0]\AgdaFunction{normalize}\AgdaSpace{}%
\AgdaSymbol{:}%
\>[649I]\AgdaDatatype{List}\AgdaSpace{}%
\AgdaRecord{η-Rule}\AgdaSpace{}%
\AgdaSymbol{→}\<%
\\
\>[649I][@{}l@{\AgdaIndent{0}}]%
\>[13]\AgdaDatatype{List}\AgdaSpace{}%
\AgdaRecord{β-rule}\AgdaSpace{}%
\AgdaSymbol{→}\<%
\\
%
\>[13]\AgdaSymbol{(}\AgdaBound{inferer}\AgdaSpace{}%
\AgdaSymbol{:}\AgdaSpace{}%
\AgdaSymbol{∀}\AgdaSpace{}%
\AgdaSymbol{\{}\AgdaBound{γ}\AgdaSymbol{\}}\AgdaSpace{}%
\AgdaSymbol{→}\AgdaSpace{}%
\AgdaFunction{Context}\AgdaSpace{}%
\AgdaBound{γ}\AgdaSpace{}%
\AgdaSymbol{→}\AgdaSpace{}%
\AgdaSymbol{(}\AgdaBound{term}\AgdaSpace{}%
\AgdaSymbol{:}\AgdaSpace{}%
\AgdaFunction{Term}\AgdaSpace{}%
\AgdaInductiveConstructor{compu}\AgdaSpace{}%
\AgdaBound{γ}\AgdaSymbol{)}\AgdaSpace{}%
\AgdaSymbol{→}\AgdaSpace{}%
\AgdaDatatype{Failable}\AgdaSpace{}%
\AgdaSymbol{(}\AgdaFunction{Term}\AgdaSpace{}%
\AgdaInductiveConstructor{const}\AgdaSpace{}%
\AgdaBound{γ}\AgdaSymbol{))}\AgdaSpace{}%
\AgdaSymbol{→}\<%
\\
%
\>[13]\AgdaFunction{Context}\AgdaSpace{}%
\AgdaGeneralizable{γ}\AgdaSpace{}%
\AgdaSymbol{→}\<%
\\
%
\>[13]\AgdaSymbol{(}\AgdaBound{type}\AgdaSpace{}%
\AgdaSymbol{:}\AgdaSpace{}%
\AgdaDatatype{Const}\AgdaSpace{}%
\AgdaGeneralizable{γ}\AgdaSymbol{)}%
\>[32]\AgdaSymbol{→}\<%
\\
%
\>[13]\AgdaSymbol{(}\AgdaBound{term}\AgdaSpace{}%
\AgdaSymbol{:}\AgdaSpace{}%
\AgdaFunction{Term}\AgdaSpace{}%
\AgdaGeneralizable{d}\AgdaSpace{}%
\AgdaGeneralizable{γ}\AgdaSymbol{)}%
\>[32]\AgdaSymbol{→}\<%
\\
%
\>[13]\AgdaDatatype{Const}\AgdaSpace{}%
\AgdaGeneralizable{γ}\<%
\\
\>[0]\AgdaFunction{normalize}\AgdaSpace{}%
\AgdaBound{ηs}\AgdaSpace{}%
\AgdaBound{βs}\AgdaSpace{}%
\AgdaBound{inf}\AgdaSpace{}%
\AgdaBound{Γ}\AgdaSpace{}%
\AgdaBound{ty}\AgdaSpace{}%
\AgdaSymbol{=}\AgdaSpace{}%
\AgdaSymbol{(}\AgdaFunction{qt}\AgdaSpace{}%
\AgdaBound{ηs}\AgdaSpace{}%
\AgdaBound{ty}\AgdaSymbol{)}\AgdaSpace{}%
\AgdaOperator{\AgdaFunction{∘}}\AgdaSpace{}%
\AgdaSymbol{(}\AgdaFunction{reduce}\AgdaSpace{}%
\AgdaBound{βs}\AgdaSpace{}%
\AgdaOperator{\AgdaFunction{-}}\AgdaSpace{}%
\AgdaBound{inf}\AgdaSpace{}%
\AgdaOperator{\AgdaFunction{-}}\AgdaSpace{}%
\AgdaBound{Γ}\AgdaSpace{}%
\AgdaOperator{\AgdaFunction{∥\AgdaUnderscore{}∥}}\AgdaSymbol{)}\<%
\end{code}

\section{Checking Types}

\begin{code}%
\>[0]\AgdaKeyword{module}\AgdaSpace{}%
\AgdaModule{TypeChecker}\AgdaSpace{}%
\AgdaKeyword{where}\<%
\end{code}
\section{Parsing from Files}

\hide{
\begin{code}%
\>[0]\AgdaKeyword{module}\AgdaSpace{}%
\AgdaModule{Parser}\AgdaSpace{}%
\AgdaKeyword{where}\<%
\end{code}
}

\section{Parsing the DSL}

\hide{
\begin{code}%
\>[0]\AgdaKeyword{module}\AgdaSpace{}%
\AgdaModule{SpecParser}\AgdaSpace{}%
\AgdaKeyword{where}\<%
\end{code}
}
\hide{
\begin{code}%
\>[0]\AgdaKeyword{open}\AgdaSpace{}%
\AgdaKeyword{import}\AgdaSpace{}%
\AgdaModule{CoreLanguage}\<%
\\
\>[0]\AgdaKeyword{open}\AgdaSpace{}%
\AgdaKeyword{import}\AgdaSpace{}%
\AgdaModule{Pattern}\AgdaSpace{}%
\AgdaKeyword{hiding}\AgdaSpace{}%
\AgdaSymbol{(}\AgdaFunction{map}\AgdaSymbol{;}\AgdaSpace{}%
\AgdaOperator{\AgdaFunction{\AgdaUnderscore{}≟\AgdaUnderscore{}}}\AgdaSymbol{)}\<%
\\
\>[0]\AgdaKeyword{open}\AgdaSpace{}%
\AgdaKeyword{import}\AgdaSpace{}%
\AgdaModule{Thinning}\AgdaSpace{}%
\AgdaKeyword{using}\AgdaSpace{}%
\AgdaSymbol{(}\AgdaFunction{ι}\AgdaSymbol{)}\<%
\\
\>[0]\AgdaKeyword{open}\AgdaSpace{}%
\AgdaKeyword{import}\AgdaSpace{}%
\AgdaModule{Category.Monad}\AgdaSpace{}%
\AgdaKeyword{using}\AgdaSpace{}%
\AgdaSymbol{(}\AgdaFunction{RawMonad}\AgdaSymbol{)}\<%
\\
\>[0]\AgdaKeyword{open}\AgdaSpace{}%
\AgdaKeyword{import}\AgdaSpace{}%
\AgdaModule{Data.List}\AgdaSpace{}%
\AgdaKeyword{hiding}\AgdaSpace{}%
\AgdaSymbol{(}\AgdaFunction{lookup}\AgdaSymbol{;}\AgdaSpace{}%
\AgdaFunction{map}\AgdaSymbol{;}\AgdaSpace{}%
\AgdaFunction{fromMaybe}\AgdaSymbol{;}\AgdaSpace{}%
\AgdaFunction{foldr}\AgdaSymbol{;}\AgdaSpace{}%
\AgdaFunction{all}\AgdaSymbol{)}\<%
\\
\>[0]\AgdaKeyword{open}\AgdaSpace{}%
\AgdaKeyword{import}\AgdaSpace{}%
\AgdaModule{Data.Product}\AgdaSpace{}%
\AgdaKeyword{using}\AgdaSpace{}%
\AgdaSymbol{(}\AgdaOperator{\AgdaFunction{\AgdaUnderscore{}×\AgdaUnderscore{}}}\AgdaSymbol{;}\AgdaSpace{}%
\AgdaFunction{Σ-syntax}\AgdaSymbol{;}\AgdaSpace{}%
\AgdaOperator{\AgdaInductiveConstructor{\AgdaUnderscore{},\AgdaUnderscore{}}}\AgdaSymbol{;}\AgdaSpace{}%
\AgdaRecord{Σ}\AgdaSymbol{)}\<%
\\
\>[0]\AgdaKeyword{open}\AgdaSpace{}%
\AgdaKeyword{import}\AgdaSpace{}%
\AgdaModule{Data.Sum}\AgdaSpace{}%
\AgdaKeyword{using}\AgdaSpace{}%
\AgdaSymbol{(}\AgdaInductiveConstructor{inj₁}\AgdaSymbol{;}\AgdaSpace{}%
\AgdaInductiveConstructor{inj₂}\AgdaSymbol{)}\<%
\\
\>[0]\AgdaKeyword{open}\AgdaSpace{}%
\AgdaKeyword{import}\AgdaSpace{}%
\AgdaModule{Data.Char}\AgdaSpace{}%
\AgdaKeyword{hiding}\AgdaSpace{}%
\AgdaSymbol{(}\AgdaOperator{\AgdaFunction{\AgdaUnderscore{}≟\AgdaUnderscore{}}}\AgdaSymbol{;}\AgdaSpace{}%
\AgdaPrimitive{show}\AgdaSymbol{)}\<%
\\
\>[0]\AgdaKeyword{open}\AgdaSpace{}%
\AgdaKeyword{import}\AgdaSpace{}%
\AgdaModule{Data.String}\AgdaSpace{}%
\AgdaKeyword{using}\AgdaSpace{}%
\AgdaSymbol{(}\AgdaPostulate{String}\AgdaSymbol{)}\<%
\\
\>[0]\AgdaKeyword{open}\AgdaSpace{}%
\AgdaKeyword{import}\AgdaSpace{}%
\AgdaModule{Data.Maybe}\AgdaSpace{}%
\AgdaKeyword{using}\AgdaSpace{}%
\AgdaSymbol{()}\AgdaSpace{}%
\AgdaKeyword{renaming}\AgdaSpace{}%
\AgdaSymbol{(}\AgdaFunction{maybe′}\AgdaSpace{}%
\AgdaSymbol{to}\AgdaSpace{}%
\AgdaFunction{maybe}\AgdaSymbol{)}\<%
\\
\>[0]\AgdaKeyword{open}\AgdaSpace{}%
\AgdaKeyword{import}\AgdaSpace{}%
\AgdaModule{Data.String.Properties}\AgdaSpace{}%
\AgdaKeyword{using}\AgdaSpace{}%
\AgdaSymbol{(}\AgdaFunction{<-strictTotalOrder-≈}\AgdaSymbol{)}\<%
\\
\>[0]\AgdaKeyword{open}\AgdaSpace{}%
\AgdaKeyword{import}\AgdaSpace{}%
\AgdaModule{Data.Nat}\AgdaSpace{}%
\AgdaKeyword{using}\AgdaSpace{}%
\AgdaSymbol{(}\AgdaInductiveConstructor{suc}\AgdaSymbol{;}\AgdaSpace{}%
\AgdaOperator{\AgdaFunction{∣\AgdaUnderscore{}-\AgdaUnderscore{}∣}}\AgdaSymbol{;}\AgdaSpace{}%
\AgdaOperator{\AgdaFunction{\AgdaUnderscore{}<″\AgdaUnderscore{}}}\AgdaSymbol{;}\AgdaSpace{}%
\AgdaOperator{\AgdaFunction{\AgdaUnderscore{}≤″?\AgdaUnderscore{}}}\AgdaSymbol{)}\<%
\\
\>[0]\AgdaKeyword{open}\AgdaSpace{}%
\AgdaKeyword{import}\AgdaSpace{}%
\AgdaModule{Data.Bool}\AgdaSpace{}%
\AgdaKeyword{using}\AgdaSpace{}%
\AgdaSymbol{(}\AgdaDatatype{Bool}\AgdaSymbol{;}\AgdaSpace{}%
\AgdaOperator{\AgdaFunction{\AgdaUnderscore{}∧\AgdaUnderscore{}}}\AgdaSymbol{;}\AgdaSpace{}%
\AgdaOperator{\AgdaFunction{\AgdaUnderscore{}∨\AgdaUnderscore{}}}\AgdaSymbol{;}\AgdaSpace{}%
\AgdaFunction{not}\AgdaSymbol{;}\AgdaSpace{}%
\AgdaOperator{\AgdaFunction{if\AgdaUnderscore{}then\AgdaUnderscore{}else\AgdaUnderscore{}}}\AgdaSymbol{)}\<%
\\
\>[0]\AgdaKeyword{open}\AgdaSpace{}%
\AgdaKeyword{import}\AgdaSpace{}%
\AgdaModule{Function}\AgdaSpace{}%
\AgdaKeyword{using}\AgdaSpace{}%
\AgdaSymbol{(}\AgdaOperator{\AgdaFunction{\AgdaUnderscore{}∘′\AgdaUnderscore{}}}\AgdaSymbol{)}\<%
\\
\>[0]\AgdaKeyword{open}\AgdaSpace{}%
\AgdaKeyword{import}\AgdaSpace{}%
\AgdaModule{Thinning}\AgdaSpace{}%
\AgdaKeyword{using}\AgdaSpace{}%
\AgdaSymbol{(}\AgdaOperator{\AgdaDatatype{\AgdaUnderscore{}⊑\AgdaUnderscore{}}}\AgdaSymbol{)}\<%
\\
\>[0]\AgdaKeyword{import}\AgdaSpace{}%
\AgdaModule{Data.Tree.AVL.Map}\AgdaSpace{}%
\AgdaSymbol{as}\AgdaSpace{}%
\AgdaModule{MapMod}\<%
\\
\>[0]\AgdaKeyword{open}\AgdaSpace{}%
\AgdaModule{MapMod}\AgdaSpace{}%
\AgdaFunction{<-strictTotalOrder-≈}\<%
\\
\>[0]\AgdaKeyword{open}\AgdaSpace{}%
\AgdaKeyword{import}\AgdaSpace{}%
\AgdaModule{Parser}\AgdaSpace{}%
\AgdaSymbol{as}\AgdaSpace{}%
\AgdaModule{P}\<%
\\
\>[0]\AgdaKeyword{open}\AgdaSpace{}%
\AgdaModule{P.Parser}\<%
\\
\>[0]\AgdaKeyword{open}\AgdaSpace{}%
\AgdaModule{P.Parsers}\<%
\end{code}
}
\hide{
\begin{code}%
\>[0]\AgdaKeyword{private}\<%
\\
\>[0][@{}l@{\AgdaIndent{0}}]%
\>[2]\AgdaKeyword{variable}\<%
\\
\>[2][@{}l@{\AgdaIndent{0}}]%
\>[4]\AgdaGeneralizable{δ}\AgdaSpace{}%
\AgdaSymbol{:}\AgdaSpace{}%
\AgdaFunction{Scope}\<%
\\
%
\>[4]\AgdaGeneralizable{γ}\AgdaSpace{}%
\AgdaSymbol{:}\AgdaSpace{}%
\AgdaFunction{Scope}\<%
\\
%
\>[4]\AgdaGeneralizable{γ'}\AgdaSpace{}%
\AgdaSymbol{:}\AgdaSpace{}%
\AgdaFunction{Scope}\<%
\\
%
\>[4]\AgdaGeneralizable{p}\AgdaSpace{}%
\AgdaSymbol{:}\AgdaSpace{}%
\AgdaDatatype{Pattern}\AgdaSpace{}%
\AgdaGeneralizable{δ}\<%
\end{code}
}

Since we have modelled typing rules, β-rules and η-rules explicitly in our
software, parsing the DSL becomes a reasonably straightforward process. We
use our combinators to build parsers for patterns, expressions, premises and
premise chains before assembling these to parse type rules, elimination rules
and other such rules we will require before finally combining these to parse the
overall description of a type. Parsing a specification file is then a matter
of parsing one or more types. The result of parsing a specification file
successfully is a set of rules.

\hide{
\begin{code}%
\>[0]\AgdaFunction{SVar}\AgdaSpace{}%
\AgdaSymbol{:}\AgdaSpace{}%
\AgdaDatatype{Pattern}\AgdaSpace{}%
\AgdaGeneralizable{γ}\AgdaSpace{}%
\AgdaSymbol{→}\AgdaSpace{}%
\AgdaPrimitiveType{Set}\<%
\\
\>[0]\AgdaFunction{SVar}\AgdaSpace{}%
\AgdaBound{p}\AgdaSpace{}%
\AgdaSymbol{=}\AgdaSpace{}%
\AgdaFunction{Σ[}\AgdaSpace{}%
\AgdaBound{δ}\AgdaSpace{}%
\AgdaFunction{∈}\AgdaSpace{}%
\AgdaFunction{Scope}\AgdaSpace{}%
\AgdaFunction{]}\AgdaSpace{}%
\AgdaDatatype{svar}\AgdaSpace{}%
\AgdaBound{p}\AgdaSpace{}%
\AgdaBound{δ}\<%
\\
%
\\[\AgdaEmptyExtraSkip]%
\>[0]\AgdaFunction{SVarMap}\AgdaSpace{}%
\AgdaSymbol{:}\AgdaSpace{}%
\AgdaDatatype{Pattern}\AgdaSpace{}%
\AgdaGeneralizable{γ}\AgdaSpace{}%
\AgdaSymbol{→}\AgdaSpace{}%
\AgdaPrimitiveType{Set}\<%
\\
\>[0]\AgdaFunction{SVarMap}\AgdaSpace{}%
\AgdaBound{p}\AgdaSpace{}%
\AgdaSymbol{=}\AgdaSpace{}%
\AgdaFunction{Map}\AgdaSpace{}%
\AgdaSymbol{(}\AgdaFunction{SVar}\AgdaSpace{}%
\AgdaBound{p}\AgdaSymbol{)}\<%
\\
%
\\[\AgdaEmptyExtraSkip]%
\>[0]\AgdaOperator{\AgdaFunction{\AgdaUnderscore{}-svmap\AgdaUnderscore{}}}\AgdaSpace{}%
\AgdaSymbol{:}\AgdaSpace{}%
\AgdaFunction{SVarMap}\AgdaSpace{}%
\AgdaGeneralizable{p}\AgdaSpace{}%
\AgdaSymbol{→}\AgdaSpace{}%
\AgdaSymbol{(}\AgdaBound{ξ}\AgdaSpace{}%
\AgdaSymbol{:}\AgdaSpace{}%
\AgdaDatatype{svar}\AgdaSpace{}%
\AgdaGeneralizable{p}\AgdaSpace{}%
\AgdaGeneralizable{γ}\AgdaSymbol{)}\AgdaSpace{}%
\AgdaSymbol{→}\AgdaSpace{}%
\AgdaFunction{SVarMap}\AgdaSpace{}%
\AgdaSymbol{(}\AgdaGeneralizable{p}\AgdaSpace{}%
\AgdaOperator{\AgdaFunction{-}}\AgdaSpace{}%
\AgdaBound{ξ}\AgdaSymbol{)}\<%
\\
\>[0]\AgdaBound{map}\AgdaSpace{}%
\AgdaOperator{\AgdaFunction{-svmap}}\AgdaSpace{}%
\AgdaBound{ξ}%
\>[14]\AgdaSymbol{=}\AgdaSpace{}%
\AgdaFunction{foldr}\AgdaSpace{}%
\AgdaSymbol{(λ}\AgdaSpace{}%
\AgdaBound{k}\AgdaSpace{}%
\AgdaSymbol{(}\AgdaBound{δ}\AgdaSpace{}%
\AgdaOperator{\AgdaInductiveConstructor{,}}\AgdaSpace{}%
\AgdaBound{v}\AgdaSymbol{)}\AgdaSpace{}%
\AgdaBound{t}\AgdaSpace{}%
\AgdaSymbol{→}\AgdaSpace{}%
\AgdaFunction{maybe}\AgdaSpace{}%
\AgdaSymbol{(λ}\AgdaSpace{}%
\AgdaBound{v}\AgdaSpace{}%
\AgdaSymbol{→}\AgdaSpace{}%
\AgdaFunction{insert}\AgdaSpace{}%
\AgdaBound{k}\AgdaSpace{}%
\AgdaSymbol{(}\AgdaBound{δ}\AgdaSpace{}%
\AgdaOperator{\AgdaInductiveConstructor{,}}\AgdaSpace{}%
\AgdaBound{v}\AgdaSymbol{)}\AgdaSpace{}%
\AgdaBound{t}\AgdaSymbol{)}\AgdaSpace{}%
\AgdaBound{t}\AgdaSpace{}%
\AgdaSymbol{(}\AgdaBound{v}\AgdaSpace{}%
\AgdaOperator{\AgdaFunction{-svar}}\AgdaSpace{}%
\AgdaBound{ξ}\AgdaSymbol{))}\AgdaSpace{}%
\AgdaFunction{empty}\AgdaSpace{}%
\AgdaBound{map}\<%
\\
%
\\[\AgdaEmptyExtraSkip]%
\>[0]\AgdaFunction{PatternParser}\AgdaSpace{}%
\AgdaSymbol{:}\AgdaSpace{}%
\AgdaFunction{Scope}\AgdaSpace{}%
\AgdaSymbol{→}\AgdaSpace{}%
\AgdaPrimitiveType{Set}\<%
\\
\>[0]\AgdaFunction{PatternParser}\AgdaSpace{}%
\AgdaBound{γ}\AgdaSpace{}%
\AgdaSymbol{=}\AgdaSpace{}%
\AgdaFunction{Parser}\AgdaSpace{}%
\AgdaSymbol{(}\AgdaFunction{Σ[}\AgdaSpace{}%
\AgdaBound{p}\AgdaSpace{}%
\AgdaFunction{∈}\AgdaSpace{}%
\AgdaDatatype{Pattern}\AgdaSpace{}%
\AgdaBound{γ}\AgdaSpace{}%
\AgdaFunction{]}\AgdaSpace{}%
\AgdaFunction{SVarMap}\AgdaSpace{}%
\AgdaBound{p}\AgdaSymbol{)}\<%
\\
%
\\[\AgdaEmptyExtraSkip]%
\>[0]\AgdaKeyword{module}\AgdaSpace{}%
\AgdaModule{PatternParser}\AgdaSpace{}%
\AgdaKeyword{where}\<%
\\
\>[0][@{}l@{\AgdaIndent{0}}]%
\>[2]\AgdaKeyword{open}\AgdaSpace{}%
\AgdaModule{parsermonad}\<%
\\
\>[0]\<%
\\
%
\>[2]\AgdaFunction{forbidden-atom-chars}\AgdaSpace{}%
\AgdaSymbol{:}\AgdaSpace{}%
\AgdaDatatype{List}\AgdaSpace{}%
\AgdaPostulate{Char}\<%
\\
%
\>[2]\AgdaFunction{forbidden-atom-chars}\AgdaSpace{}%
\AgdaSymbol{=}\AgdaSpace{}%
\AgdaString{'('}\AgdaSpace{}%
\AgdaOperator{\AgdaInductiveConstructor{∷}}\AgdaSpace{}%
\AgdaString{')'}\AgdaSpace{}%
\AgdaOperator{\AgdaInductiveConstructor{∷}}\AgdaSpace{}%
\AgdaString{'\{'}\AgdaSpace{}%
\AgdaOperator{\AgdaInductiveConstructor{∷}}\AgdaSpace{}%
\AgdaString{'\}'}\AgdaSpace{}%
\AgdaOperator{\AgdaInductiveConstructor{∷}}\AgdaSpace{}%
\AgdaString{'['}\AgdaSpace{}%
\AgdaOperator{\AgdaInductiveConstructor{∷}}\AgdaSpace{}%
\AgdaString{']'}\AgdaSpace{}%
\AgdaOperator{\AgdaInductiveConstructor{∷}}\AgdaSpace{}%
\AgdaString{'.'}\AgdaSpace{}%
\AgdaOperator{\AgdaInductiveConstructor{∷}}\AgdaSpace{}%
\AgdaString{':'}\AgdaSpace{}%
\AgdaOperator{\AgdaInductiveConstructor{∷}}\AgdaSpace{}%
\AgdaString{','}\AgdaSpace{}%
\AgdaOperator{\AgdaInductiveConstructor{∷}}\AgdaSpace{}%
\AgdaInductiveConstructor{[]}\<%
\\
\>[0]\<%
\\
%
\>[2]\AgdaFunction{atomchar}\AgdaSpace{}%
\AgdaSymbol{:}\AgdaSpace{}%
\AgdaPostulate{Char}\AgdaSpace{}%
\AgdaSymbol{→}\AgdaSpace{}%
\AgdaDatatype{Bool}\<%
\\
%
\>[2]\AgdaFunction{atomchar}\AgdaSpace{}%
\AgdaBound{c}\AgdaSpace{}%
\AgdaSymbol{=}\AgdaSpace{}%
\AgdaPrimitive{isLower}\AgdaSpace{}%
\AgdaBound{c}\AgdaSpace{}%
\AgdaOperator{\AgdaFunction{∨}}\AgdaSpace{}%
\AgdaSymbol{(}\AgdaSpace{}%
\AgdaFunction{not}\AgdaSpace{}%
\AgdaSymbol{(}\AgdaPrimitive{isAlpha}\AgdaSpace{}%
\AgdaBound{c}\AgdaSymbol{)}\AgdaSpace{}%
\AgdaOperator{\AgdaFunction{∧}}\AgdaSpace{}%
\AgdaFunction{not}\AgdaSpace{}%
\AgdaSymbol{(}\AgdaPrimitive{isSpace}\AgdaSpace{}%
\AgdaBound{c}\AgdaSymbol{)}\AgdaSpace{}%
\AgdaOperator{\AgdaFunction{∧}}\AgdaSpace{}%
\AgdaFunction{not}\AgdaSpace{}%
\AgdaSymbol{(}\AgdaFunction{any}\AgdaSpace{}%
\AgdaSymbol{(}\AgdaBound{c}\AgdaSpace{}%
\AgdaOperator{\AgdaFunction{==\AgdaUnderscore{}}}\AgdaSymbol{)}\AgdaSpace{}%
\AgdaFunction{forbidden-atom-chars}\AgdaSymbol{))}\<%
\\
\>[0]\<%
\\
%
\>[2]\AgdaFunction{idchar}\AgdaSpace{}%
\AgdaSymbol{:}\AgdaSpace{}%
\AgdaPostulate{Char}\AgdaSpace{}%
\AgdaSymbol{→}\AgdaSpace{}%
\AgdaDatatype{Bool}\<%
\\
%
\>[2]\AgdaFunction{idchar}\AgdaSpace{}%
\AgdaBound{c}\AgdaSpace{}%
\AgdaSymbol{=}\AgdaSpace{}%
\AgdaPrimitive{isAlpha}\AgdaSpace{}%
\AgdaBound{c}\AgdaSpace{}%
\AgdaOperator{\AgdaFunction{∧}}\AgdaSpace{}%
\AgdaSymbol{(}\AgdaFunction{not}\AgdaSpace{}%
\AgdaSymbol{(}\AgdaPrimitive{isLower}\AgdaSpace{}%
\AgdaBound{c}\AgdaSymbol{))}\<%
\\
%
\\[\AgdaEmptyExtraSkip]%
\>[0]\<%
\\
%
\>[2]\AgdaFunction{identifier}\AgdaSpace{}%
\AgdaSymbol{:}\AgdaSpace{}%
\AgdaFunction{Parser}\AgdaSpace{}%
\AgdaPostulate{String}\<%
\\
%
\>[2]\AgdaFunction{identifier}\AgdaSpace{}%
\AgdaSymbol{=}\AgdaSpace{}%
\AgdaFunction{nonempty}\AgdaSpace{}%
\AgdaSymbol{(}\AgdaFunction{stringof}\AgdaSpace{}%
\AgdaFunction{idchar}\AgdaSymbol{)}\<%
\\
\>[0]\<%
\\
%
\>[2]\AgdaFunction{binding}\AgdaSpace{}%
\AgdaSymbol{:}\AgdaSpace{}%
\AgdaFunction{Parser}\AgdaSpace{}%
\AgdaPostulate{String}\<%
\\
%
\>[2]\AgdaFunction{binding}\AgdaSpace{}%
\AgdaSymbol{=}%
\>[253I]\AgdaKeyword{do}\<%
\\
\>[253I][@{}l@{\AgdaIndent{0}}]%
\>[14]\AgdaBound{name}\AgdaSpace{}%
\AgdaOperator{\AgdaFunction{←}}\AgdaSpace{}%
\AgdaFunction{identifier}\<%
\\
%
\>[14]\AgdaFunction{literal}\AgdaSpace{}%
\AgdaString{'.'}\<%
\\
%
\>[14]\AgdaFunction{return}\AgdaSpace{}%
\AgdaBound{name}\<%
\\
%
\\[\AgdaEmptyExtraSkip]%
%
\>[2]\AgdaFunction{pat}\AgdaSpace{}%
\AgdaSymbol{:}\AgdaSpace{}%
\AgdaFunction{PatternParser}\AgdaSpace{}%
\AgdaGeneralizable{γ}\<%
\\
%
\>[2]\AgdaKeyword{private}\<%
\\
\>[2][@{}l@{\AgdaIndent{0}}]%
\>[4]\AgdaFunction{atom}\AgdaSpace{}%
\AgdaSymbol{:}\AgdaSpace{}%
\AgdaFunction{PatternParser}\AgdaSpace{}%
\AgdaGeneralizable{γ}\<%
\\
%
\>[4]\AgdaFunction{atom}\AgdaSpace{}%
\AgdaSymbol{=}\AgdaSpace{}%
\AgdaSymbol{(}\AgdaOperator{\AgdaInductiveConstructor{\AgdaUnderscore{},}}\AgdaSpace{}%
\AgdaFunction{empty}\AgdaSymbol{)}\AgdaSpace{}%
\AgdaOperator{\AgdaFunction{<\$>}}\AgdaSpace{}%
\AgdaSymbol{(}\AgdaInductiveConstructor{`}\AgdaSpace{}%
\AgdaOperator{\AgdaFunction{<\$>}}\AgdaSpace{}%
\AgdaFunction{nonempty}\AgdaSpace{}%
\AgdaSymbol{(}\AgdaFunction{stringof}\AgdaSpace{}%
\AgdaFunction{atomchar}\AgdaSymbol{))}\<%
\\
\>[0]\<%
\\
%
\>[4]\AgdaFunction{plc}\AgdaSpace{}%
\AgdaSymbol{:}\AgdaSpace{}%
\AgdaFunction{PatternParser}\AgdaSpace{}%
\AgdaGeneralizable{γ}\<%
\\
%
\>[4]\AgdaFunction{plc}\AgdaSpace{}%
\AgdaSymbol{\{}\AgdaBound{γ}\AgdaSymbol{\}}\AgdaSpace{}%
\AgdaSymbol{=}%
\>[278I]\AgdaKeyword{do}\<%
\\
\>[278I][@{}l@{\AgdaIndent{0}}]%
\>[16]\AgdaBound{name}\AgdaSpace{}%
\AgdaOperator{\AgdaFunction{←}}\AgdaSpace{}%
\AgdaFunction{identifier}\<%
\\
%
\>[16]\AgdaOperator{\AgdaFunction{ifp}}\AgdaSpace{}%
\AgdaFunction{literal}\AgdaSpace{}%
\AgdaString{'.'}%
\>[283I]\AgdaOperator{\AgdaFunction{then}}\AgdaSpace{}%
\AgdaFunction{fail}\<%
\\
\>[.][@{}l@{}]\<[283I]%
\>[32]\AgdaOperator{\AgdaFunction{else}}\AgdaSpace{}%
\AgdaFunction{return}\AgdaSpace{}%
\AgdaSymbol{((}\AgdaInductiveConstructor{place}\AgdaSpace{}%
\AgdaFunction{ι}\AgdaSymbol{)}\AgdaSpace{}%
\AgdaOperator{\AgdaInductiveConstructor{,}}\AgdaSpace{}%
\AgdaFunction{singleton}\AgdaSpace{}%
\AgdaBound{name}\AgdaSpace{}%
\AgdaSymbol{(}\AgdaBound{γ}\AgdaSpace{}%
\AgdaOperator{\AgdaInductiveConstructor{,}}\AgdaSpace{}%
\AgdaInductiveConstructor{⋆}\AgdaSymbol{))}\<%
\\
%
\\[\AgdaEmptyExtraSkip]%
%
\>[4]\AgdaSymbol{\{-\#}\AgdaSpace{}%
\AgdaKeyword{TERMINATING}\AgdaSpace{}%
\AgdaSymbol{\#-\}}\<%
\\
%
\>[4]\AgdaFunction{binder}\AgdaSpace{}%
\AgdaSymbol{:}\AgdaSpace{}%
\AgdaFunction{PatternParser}\AgdaSpace{}%
\AgdaGeneralizable{γ}\<%
\\
%
\>[4]\AgdaFunction{binder}\AgdaSpace{}%
\AgdaSymbol{\{}\AgdaBound{γ}\AgdaSymbol{\}}\AgdaSpace{}%
\AgdaSymbol{=}%
\>[301I]\AgdaKeyword{do}\<%
\\
\>[301I][@{}l@{\AgdaIndent{0}}]%
\>[19]\AgdaBound{name}\AgdaSpace{}%
\AgdaOperator{\AgdaFunction{←}}\AgdaSpace{}%
\AgdaFunction{safe}\AgdaSpace{}%
\AgdaFunction{binding}\<%
\\
%
\>[19]\AgdaFunction{whitespace}\<%
\\
%
\>[19]\AgdaSymbol{(}\AgdaBound{subterm}\AgdaSpace{}%
\AgdaOperator{\AgdaInductiveConstructor{,}}\AgdaSpace{}%
\AgdaBound{vmap}\AgdaSymbol{)}\AgdaSpace{}%
\AgdaOperator{\AgdaFunction{←}}\AgdaSpace{}%
\AgdaFunction{pat}\AgdaSpace{}%
\AgdaSymbol{\{}\AgdaInductiveConstructor{suc}\AgdaSpace{}%
\AgdaBound{γ}\AgdaSymbol{\}}\<%
\\
%
\>[19]\AgdaFunction{return}\AgdaSpace{}%
\AgdaSymbol{((}\AgdaInductiveConstructor{bind}\AgdaSpace{}%
\AgdaBound{subterm}\AgdaSymbol{)}\AgdaSpace{}%
\AgdaOperator{\AgdaInductiveConstructor{,}}\AgdaSpace{}%
\AgdaFunction{map}\AgdaSpace{}%
\AgdaSymbol{(λ}\AgdaSpace{}%
\AgdaSymbol{\{(}\AgdaBound{δ}\AgdaSpace{}%
\AgdaOperator{\AgdaInductiveConstructor{,}}\AgdaSpace{}%
\AgdaBound{st}\AgdaSymbol{)}\AgdaSpace{}%
\AgdaSymbol{→}\AgdaSpace{}%
\AgdaSymbol{(}\AgdaBound{δ}\AgdaSpace{}%
\AgdaOperator{\AgdaInductiveConstructor{,}}\AgdaSpace{}%
\AgdaInductiveConstructor{bind}\AgdaSpace{}%
\AgdaBound{st}\AgdaSymbol{)}%
\>[78]\AgdaSymbol{\})}\AgdaSpace{}%
\AgdaBound{vmap}\AgdaSymbol{)}\<%
\\
\>[0]\<%
\\
%
\>[2]\AgdaFunction{pat}\AgdaSpace{}%
\AgdaSymbol{\{}\AgdaBound{γ}\AgdaSymbol{\}}\AgdaSpace{}%
\AgdaSymbol{=}%
\>[327I]\AgdaKeyword{do}\<%
\\
\>[327I][@{}l@{\AgdaIndent{0}}]%
\>[14]\AgdaSymbol{(}\AgdaBound{fst}\AgdaSpace{}%
\AgdaOperator{\AgdaInductiveConstructor{,}}\AgdaSpace{}%
\AgdaBound{vmap}\AgdaSymbol{)}\AgdaSpace{}%
\AgdaOperator{\AgdaFunction{←}}\AgdaSpace{}%
\AgdaFunction{anyof}\AgdaSymbol{(}\AgdaFunction{atom}\AgdaSpace{}%
\AgdaSymbol{\{}\AgdaBound{γ}\AgdaSymbol{\}}\AgdaSpace{}%
\AgdaOperator{\AgdaInductiveConstructor{∷}}\AgdaSpace{}%
\AgdaFunction{plc}\AgdaSpace{}%
\AgdaOperator{\AgdaInductiveConstructor{∷}}\AgdaSpace{}%
\AgdaFunction{binder}\AgdaSpace{}%
\AgdaOperator{\AgdaInductiveConstructor{∷}}\AgdaSpace{}%
\AgdaFunction{bracketed}\AgdaSpace{}%
\AgdaFunction{pat}\AgdaSpace{}%
\AgdaOperator{\AgdaInductiveConstructor{∷}}\AgdaSpace{}%
\AgdaInductiveConstructor{[]}\AgdaSymbol{)}\<%
\\
%
\>[14]\AgdaInductiveConstructor{inj₁}\AgdaSpace{}%
\AgdaSymbol{(}\AgdaBound{snd}\AgdaSpace{}%
\AgdaOperator{\AgdaInductiveConstructor{,}}\AgdaSpace{}%
\AgdaBound{vmap'}\AgdaSymbol{)}\AgdaSpace{}%
\AgdaOperator{\AgdaFunction{←}}\AgdaSpace{}%
\AgdaFunction{optional}%
\>[347I]\AgdaSymbol{(}\AgdaKeyword{do}\<%
\\
\>[347I][@{}l@{\AgdaIndent{0}}]%
\>[47]\AgdaFunction{whitespace}\<%
\\
%
\>[47]\AgdaFunction{pat}\AgdaSpace{}%
\AgdaSymbol{\{}\AgdaBound{γ}\AgdaSymbol{\})}\<%
\\
\>[14][@{}l@{\AgdaIndent{0}}]%
\>[16]\AgdaKeyword{where}\AgdaSpace{}%
\AgdaInductiveConstructor{inj₂}\AgdaSpace{}%
\AgdaSymbol{\AgdaUnderscore{}}\AgdaSpace{}%
\AgdaSymbol{→}\AgdaSpace{}%
\AgdaFunction{return}\AgdaSpace{}%
\AgdaSymbol{(}\AgdaBound{fst}\AgdaSpace{}%
\AgdaOperator{\AgdaInductiveConstructor{,}}\AgdaSpace{}%
\AgdaBound{vmap}\AgdaSymbol{)}\<%
\\
%
\>[14]\AgdaFunction{return}\AgdaSpace{}%
\AgdaSymbol{((}\AgdaBound{fst}\AgdaSpace{}%
\AgdaOperator{\AgdaInductiveConstructor{∙}}\AgdaSpace{}%
\AgdaBound{snd}\AgdaSymbol{)}\AgdaSpace{}%
\AgdaOperator{\AgdaInductiveConstructor{,}}\AgdaSpace{}%
\AgdaFunction{union}%
\>[361I]\AgdaSymbol{(}\AgdaFunction{map}\AgdaSpace{}%
\AgdaSymbol{(λ}\AgdaSpace{}%
\AgdaSymbol{\{}\AgdaSpace{}%
\AgdaSymbol{(}\AgdaBound{δ}\AgdaSpace{}%
\AgdaOperator{\AgdaInductiveConstructor{,}}\AgdaSpace{}%
\AgdaBound{st}\AgdaSymbol{)}\AgdaSpace{}%
\AgdaSymbol{→}\AgdaSpace{}%
\AgdaSymbol{(}\AgdaBound{δ}\AgdaSpace{}%
\AgdaOperator{\AgdaInductiveConstructor{,}}\AgdaSpace{}%
\AgdaSymbol{(}\AgdaBound{st}\AgdaSpace{}%
\AgdaOperator{\AgdaInductiveConstructor{∙}}\AgdaSymbol{))\})}\AgdaSpace{}%
\AgdaBound{vmap}\AgdaSymbol{)}\<%
\\
\>[.][@{}l@{}]\<[361I]%
\>[42]\AgdaSymbol{(}\AgdaFunction{map}\AgdaSpace{}%
\AgdaSymbol{(λ}\AgdaSpace{}%
\AgdaSymbol{\{}\AgdaSpace{}%
\AgdaSymbol{(}\AgdaBound{δ}\AgdaSpace{}%
\AgdaOperator{\AgdaInductiveConstructor{,}}\AgdaSpace{}%
\AgdaBound{st}\AgdaSymbol{)}\AgdaSpace{}%
\AgdaSymbol{→}\AgdaSpace{}%
\AgdaSymbol{(}\AgdaBound{δ}\AgdaSpace{}%
\AgdaOperator{\AgdaInductiveConstructor{,}}\AgdaSpace{}%
\AgdaSymbol{(}\AgdaOperator{\AgdaInductiveConstructor{∙}}\AgdaSpace{}%
\AgdaBound{st}\AgdaSymbol{))\})}\AgdaSpace{}%
\AgdaBound{vmap'}\AgdaSymbol{))}\<%
\\
%
\\[\AgdaEmptyExtraSkip]%
%
\>[2]\AgdaFunction{closed-pattern}\AgdaSpace{}%
\AgdaSymbol{=}\AgdaSpace{}%
\AgdaFunction{pat}\AgdaSpace{}%
\AgdaSymbol{\{}\AgdaNumber{0}\AgdaSymbol{\}}\<%
\\
\>[0]\AgdaKeyword{open}\AgdaSpace{}%
\AgdaModule{PatternParser}\<%
\\
\>[0]\AgdaComment{\{-
  If we want to refer to variables by name in our expressions, then we have to
  use another map

  Also, tidy up these types ya animal
-\}}\<%
\\
%
\\[\AgdaEmptyExtraSkip]%
\>[0]\AgdaKeyword{module}\AgdaSpace{}%
\AgdaModule{ExpressionParser}\AgdaSpace{}%
\AgdaKeyword{where}\<%
\\
\>[0][@{}l@{\AgdaIndent{0}}]%
\>[2]\AgdaKeyword{open}\AgdaSpace{}%
\AgdaKeyword{import}\AgdaSpace{}%
\AgdaModule{Expression}\AgdaSpace{}%
\AgdaKeyword{hiding}\AgdaSpace{}%
\AgdaSymbol{(}\AgdaFunction{map}\AgdaSymbol{)}\<%
\\
%
\>[2]\AgdaKeyword{open}\AgdaSpace{}%
\AgdaKeyword{import}\AgdaSpace{}%
\AgdaModule{Data.Maybe}\AgdaSpace{}%
\AgdaKeyword{using}\AgdaSpace{}%
\AgdaSymbol{(}\AgdaFunction{maybe′}\AgdaSymbol{)}\<%
\\
%
\>[2]\AgdaKeyword{open}\AgdaSpace{}%
\AgdaKeyword{import}\AgdaSpace{}%
\AgdaModule{Data.Nat}\AgdaSpace{}%
\AgdaKeyword{using}\AgdaSpace{}%
\AgdaSymbol{(}\AgdaInductiveConstructor{zero}\AgdaSymbol{;}\AgdaSpace{}%
\AgdaOperator{\AgdaFunction{\AgdaUnderscore{}≟\AgdaUnderscore{}}}\AgdaSymbol{)}\<%
\\
%
\>[2]\AgdaKeyword{open}\AgdaSpace{}%
\AgdaKeyword{import}\AgdaSpace{}%
\AgdaModule{Relation.Nullary}\AgdaSpace{}%
\AgdaKeyword{using}\AgdaSpace{}%
\AgdaSymbol{(}\AgdaInductiveConstructor{yes}\AgdaSymbol{;}\AgdaSpace{}%
\AgdaInductiveConstructor{no}\AgdaSymbol{)}\<%
\\
%
\>[2]\AgdaKeyword{open}\AgdaSpace{}%
\AgdaKeyword{import}\AgdaSpace{}%
\AgdaModule{Relation.Binary.PropositionalEquality}\AgdaSpace{}%
\AgdaKeyword{using}\AgdaSpace{}%
\AgdaSymbol{(}\AgdaOperator{\AgdaDatatype{\AgdaUnderscore{}≡\AgdaUnderscore{}}}\AgdaSymbol{;}\AgdaSpace{}%
\AgdaInductiveConstructor{refl}\AgdaSymbol{)}\<%
\\
%
\>[2]\AgdaKeyword{open}\AgdaSpace{}%
\AgdaKeyword{import}\AgdaSpace{}%
\AgdaModule{Substitution}\AgdaSpace{}%
\AgdaKeyword{using}\AgdaSpace{}%
\AgdaSymbol{(}\AgdaOperator{\AgdaFunction{\AgdaUnderscore{}⇒[\AgdaUnderscore{}]\AgdaUnderscore{}}}\AgdaSymbol{)}\<%
\\
%
\>[2]\AgdaKeyword{open}\AgdaSpace{}%
\AgdaKeyword{import}\AgdaSpace{}%
\AgdaModule{BwdVec}\AgdaSpace{}%
\AgdaKeyword{using}\AgdaSpace{}%
\AgdaSymbol{(}\AgdaInductiveConstructor{ε}\AgdaSymbol{;}\AgdaSpace{}%
\AgdaOperator{\AgdaInductiveConstructor{\AgdaUnderscore{}-,\AgdaUnderscore{}}}\AgdaSymbol{)}\<%
\\
%
\>[2]\AgdaKeyword{open}\AgdaSpace{}%
\AgdaModule{parsermonad}\<%
\\
%
\\[\AgdaEmptyExtraSkip]%
%
\>[2]\AgdaFunction{VarMap}\AgdaSpace{}%
\AgdaSymbol{:}\AgdaSpace{}%
\AgdaFunction{Scoped}\<%
\\
%
\>[2]\AgdaFunction{VarMap}\AgdaSpace{}%
\AgdaBound{γ}\AgdaSpace{}%
\AgdaSymbol{=}\AgdaSpace{}%
\AgdaFunction{Map}\AgdaSpace{}%
\AgdaSymbol{(}\AgdaDatatype{Var}\AgdaSpace{}%
\AgdaBound{γ}\AgdaSymbol{)}\<%
\\
%
\\[\AgdaEmptyExtraSkip]%
%
\>[2]\AgdaKeyword{private}\<%
\\
\>[2][@{}l@{\AgdaIndent{0}}]%
\>[4]\AgdaKeyword{variable}\<%
\\
\>[4][@{}l@{\AgdaIndent{0}}]%
\>[6]\AgdaGeneralizable{δ'}\AgdaSpace{}%
\AgdaSymbol{:}\AgdaSpace{}%
\AgdaFunction{Scope}\<%
\\
%
\\[\AgdaEmptyExtraSkip]%
%
\>[2]\AgdaFunction{ConstParser}\AgdaSpace{}%
\AgdaSymbol{=}\AgdaSpace{}%
\AgdaSymbol{∀}\AgdaSpace{}%
\AgdaSymbol{\{}\AgdaBound{δ}\AgdaSymbol{\}}\AgdaSpace{}%
\AgdaSymbol{→}\AgdaSpace{}%
\AgdaSymbol{(}\AgdaBound{p}\AgdaSpace{}%
\AgdaSymbol{:}\AgdaSpace{}%
\AgdaDatatype{Pattern}\AgdaSpace{}%
\AgdaBound{δ}\AgdaSymbol{)}\AgdaSpace{}%
\AgdaSymbol{→}\AgdaSpace{}%
\AgdaSymbol{(}\AgdaBound{γ}\AgdaSpace{}%
\AgdaSymbol{:}\AgdaSpace{}%
\AgdaFunction{Scope}\AgdaSymbol{)}\AgdaSpace{}%
\AgdaSymbol{→}\AgdaSpace{}%
\AgdaFunction{VarMap}\AgdaSpace{}%
\AgdaBound{γ}\AgdaSpace{}%
\AgdaSymbol{→}\AgdaSpace{}%
\AgdaFunction{SVarMap}\AgdaSpace{}%
\AgdaBound{p}\AgdaSpace{}%
\AgdaSymbol{→}\AgdaSpace{}%
\AgdaFunction{Parser}\AgdaSpace{}%
\AgdaSymbol{(}\AgdaFunction{Expr}\AgdaSpace{}%
\AgdaBound{p}\AgdaSpace{}%
\AgdaInductiveConstructor{const}\AgdaSpace{}%
\AgdaBound{γ}\AgdaSymbol{)}\<%
\\
%
\>[2]\AgdaFunction{CompuParser}\AgdaSpace{}%
\AgdaSymbol{=}\AgdaSpace{}%
\AgdaSymbol{∀}\AgdaSpace{}%
\AgdaSymbol{\{}\AgdaBound{δ}\AgdaSymbol{\}}\AgdaSpace{}%
\AgdaSymbol{→}\AgdaSpace{}%
\AgdaSymbol{(}\AgdaBound{p}\AgdaSpace{}%
\AgdaSymbol{:}\AgdaSpace{}%
\AgdaDatatype{Pattern}\AgdaSpace{}%
\AgdaBound{δ}\AgdaSymbol{)}\AgdaSpace{}%
\AgdaSymbol{→}\AgdaSpace{}%
\AgdaSymbol{(}\AgdaBound{γ}\AgdaSpace{}%
\AgdaSymbol{:}\AgdaSpace{}%
\AgdaFunction{Scope}\AgdaSymbol{)}\AgdaSpace{}%
\AgdaSymbol{→}\AgdaSpace{}%
\AgdaFunction{VarMap}\AgdaSpace{}%
\AgdaBound{γ}\AgdaSpace{}%
\AgdaSymbol{→}\AgdaSpace{}%
\AgdaFunction{SVarMap}\AgdaSpace{}%
\AgdaBound{p}\AgdaSpace{}%
\AgdaSymbol{→}\AgdaSpace{}%
\AgdaFunction{Parser}\AgdaSpace{}%
\AgdaSymbol{(}\AgdaFunction{Expr}\AgdaSpace{}%
\AgdaBound{p}\AgdaSpace{}%
\AgdaInductiveConstructor{compu}\AgdaSpace{}%
\AgdaBound{γ}\AgdaSymbol{)}\<%
\\
\>[0]\<%
\\
%
\>[2]\AgdaFunction{econst}\AgdaSpace{}%
\AgdaSymbol{:}\AgdaSpace{}%
\AgdaFunction{ConstParser}\<%
\\
%
\>[2]\AgdaFunction{ecompu}\AgdaSpace{}%
\AgdaSymbol{:}\AgdaSpace{}%
\AgdaFunction{CompuParser}\<%
\\
%
\\[\AgdaEmptyExtraSkip]%
%
\>[2]\AgdaFunction{schvar}\AgdaSpace{}%
\AgdaSymbol{:}\AgdaSpace{}%
\AgdaSymbol{(}\AgdaBound{p}\AgdaSpace{}%
\AgdaSymbol{:}\AgdaSpace{}%
\AgdaDatatype{Pattern}\AgdaSpace{}%
\AgdaGeneralizable{γ}\AgdaSymbol{)}\AgdaSpace{}%
\AgdaSymbol{→}\AgdaSpace{}%
\AgdaFunction{SVarMap}\AgdaSpace{}%
\AgdaBound{p}\AgdaSpace{}%
\AgdaSymbol{→}\AgdaSpace{}%
\AgdaFunction{Parser}\AgdaSpace{}%
\AgdaSymbol{(}\AgdaFunction{SVar}\AgdaSpace{}%
\AgdaBound{p}\AgdaSpace{}%
\AgdaOperator{\AgdaFunction{×}}\AgdaSpace{}%
\AgdaPostulate{String}\AgdaSymbol{)}\<%
\\
%
\>[2]\AgdaFunction{schvar}\AgdaSpace{}%
\AgdaBound{p}\AgdaSpace{}%
\AgdaBound{svmap}\AgdaSpace{}%
\AgdaSymbol{=}%
\>[501I]\AgdaKeyword{do}\<%
\\
\>[501I][@{}l@{\AgdaIndent{0}}]%
\>[21]\AgdaBound{name}\AgdaSpace{}%
\AgdaOperator{\AgdaFunction{←}}\AgdaSpace{}%
\AgdaFunction{identifier}\<%
\\
%
\>[21]\AgdaOperator{\AgdaFunction{ifp}}\AgdaSpace{}%
\AgdaFunction{literal}\AgdaSpace{}%
\AgdaString{'.'}%
\>[506I]\AgdaOperator{\AgdaFunction{then}}\AgdaSpace{}%
\AgdaFunction{fail}\<%
\\
\>[.][@{}l@{}]\<[506I]%
\>[37]\AgdaOperator{\AgdaFunction{else}}\AgdaSpace{}%
\AgdaFunction{maybe′}\AgdaSpace{}%
\AgdaSymbol{(}\AgdaFunction{return}\AgdaSpace{}%
\AgdaOperator{\AgdaFunction{∘′}}\AgdaSpace{}%
\AgdaSymbol{(}\AgdaOperator{\AgdaInductiveConstructor{\AgdaUnderscore{},}}\AgdaSpace{}%
\AgdaBound{name}\AgdaSymbol{))}\AgdaSpace{}%
\AgdaFunction{fail}\AgdaSpace{}%
\AgdaSymbol{(}\AgdaFunction{lookup}\AgdaSpace{}%
\AgdaBound{name}\AgdaSpace{}%
\AgdaBound{svmap}\AgdaSymbol{)}\<%
\\
%
\\[\AgdaEmptyExtraSkip]%
%
\>[2]\AgdaKeyword{private}\<%
\\
\>[2][@{}l@{\AgdaIndent{0}}]%
\>[4]\AgdaFunction{eatom}\AgdaSpace{}%
\AgdaSymbol{:}\AgdaSpace{}%
\AgdaFunction{ConstParser}\<%
\\
%
\>[4]\AgdaFunction{eatom}\AgdaSpace{}%
\AgdaSymbol{\AgdaUnderscore{}}\AgdaSpace{}%
\AgdaSymbol{\AgdaUnderscore{}}\AgdaSpace{}%
\AgdaSymbol{\AgdaUnderscore{}}\AgdaSpace{}%
\AgdaSymbol{\AgdaUnderscore{}}\AgdaSpace{}%
\AgdaSymbol{=}\AgdaSpace{}%
\AgdaInductiveConstructor{`}\AgdaSpace{}%
\AgdaOperator{\AgdaFunction{<\$>}}\AgdaSpace{}%
\AgdaFunction{nonempty}\AgdaSpace{}%
\AgdaSymbol{(}\AgdaFunction{stringof}\AgdaSpace{}%
\AgdaFunction{atomchar}\AgdaSymbol{)}\<%
\\
\>[0]\<%
\\
%
\>[4]\AgdaSymbol{\{-\#}\AgdaSpace{}%
\AgdaKeyword{TERMINATING}\AgdaSpace{}%
\AgdaSymbol{\#-\}}\<%
\\
%
\>[4]\AgdaFunction{ebinder}\AgdaSpace{}%
\AgdaSymbol{:}\AgdaSpace{}%
\AgdaFunction{ConstParser}\<%
\\
%
\>[4]\AgdaFunction{ebinder}\AgdaSpace{}%
\AgdaBound{p}\AgdaSpace{}%
\AgdaBound{γ}\AgdaSpace{}%
\AgdaBound{vmap}%
\>[536I]\AgdaBound{svmap}\AgdaSpace{}%
\AgdaSymbol{=}\AgdaSpace{}%
\AgdaKeyword{do}\<%
\\
\>[536I][@{}l@{\AgdaIndent{0}}]%
\>[25]\AgdaBound{name}\AgdaSpace{}%
\AgdaOperator{\AgdaFunction{←}}\AgdaSpace{}%
\AgdaFunction{safe}\AgdaSpace{}%
\AgdaFunction{binding}\<%
\\
%
\>[25]\AgdaFunction{whitespace}\<%
\\
%
\>[25]\AgdaBound{subexpr}\AgdaSpace{}%
\AgdaOperator{\AgdaFunction{←}}\AgdaSpace{}%
\AgdaFunction{econst}\AgdaSpace{}%
\AgdaBound{p}\AgdaSpace{}%
\AgdaSymbol{(}\AgdaInductiveConstructor{suc}\AgdaSpace{}%
\AgdaBound{γ}\AgdaSymbol{)}\AgdaSpace{}%
\AgdaSymbol{(}\AgdaFunction{insert}\AgdaSpace{}%
\AgdaBound{name}\AgdaSpace{}%
\AgdaInductiveConstructor{ze}\AgdaSpace{}%
\AgdaSymbol{(}\AgdaFunction{map}\AgdaSpace{}%
\AgdaInductiveConstructor{su}\AgdaSpace{}%
\AgdaBound{vmap}\AgdaSymbol{))}\AgdaSpace{}%
\AgdaBound{svmap}\<%
\\
%
\>[25]\AgdaFunction{return}\AgdaSpace{}%
\AgdaSymbol{(}\AgdaInductiveConstructor{bind}\AgdaSpace{}%
\AgdaBound{subexpr}\AgdaSymbol{)}\<%
\\
\>[0]\<%
\\
%
\>[4]\AgdaFunction{ethunk}\AgdaSpace{}%
\AgdaSymbol{:}\AgdaSpace{}%
\AgdaFunction{ConstParser}\<%
\\
%
\>[4]\AgdaFunction{ethunk}\AgdaSpace{}%
\AgdaBound{p}\AgdaSpace{}%
\AgdaBound{γ}\AgdaSpace{}%
\AgdaBound{vmap}%
\>[561I]\AgdaBound{svmap}\AgdaSpace{}%
\AgdaSymbol{=}\AgdaSpace{}%
\AgdaKeyword{do}\<%
\\
\>[561I][@{}l@{\AgdaIndent{0}}]%
\>[24]\AgdaBound{comp}\AgdaSpace{}%
\AgdaOperator{\AgdaFunction{←}}\AgdaSpace{}%
\AgdaFunction{curlybracketed}\AgdaSpace{}%
\AgdaSymbol{(}\AgdaFunction{ecompu}\AgdaSpace{}%
\AgdaBound{p}\AgdaSpace{}%
\AgdaBound{γ}\AgdaSpace{}%
\AgdaBound{vmap}\AgdaSpace{}%
\AgdaBound{svmap}\AgdaSymbol{)}\<%
\\
%
\>[24]\AgdaFunction{return}\AgdaSpace{}%
\AgdaSymbol{(}\AgdaInductiveConstructor{thunk}\AgdaSpace{}%
\AgdaBound{comp}\AgdaSymbol{)}\<%
\\
%
\\[\AgdaEmptyExtraSkip]%
%
\>[4]\AgdaKeyword{open}\AgdaSpace{}%
\AgdaKeyword{import}\AgdaSpace{}%
\AgdaModule{Data.Maybe}\AgdaSpace{}%
\AgdaKeyword{using}\AgdaSpace{}%
\AgdaSymbol{(}\AgdaInductiveConstructor{just}\AgdaSymbol{)}\<%
\\
%
\>[4]\AgdaFunction{eσ}\AgdaSpace{}%
\AgdaSymbol{:}\AgdaSpace{}%
\AgdaSymbol{(}\AgdaBound{δ}\AgdaSpace{}%
\AgdaBound{γ}\AgdaSpace{}%
\AgdaSymbol{:}\AgdaSpace{}%
\AgdaFunction{Scope}\AgdaSymbol{)}\AgdaSpace{}%
\AgdaSymbol{→}\AgdaSpace{}%
\AgdaSymbol{(}\AgdaBound{p}\AgdaSpace{}%
\AgdaSymbol{:}\AgdaSpace{}%
\AgdaDatatype{Pattern}\AgdaSpace{}%
\AgdaGeneralizable{δ'}\AgdaSymbol{)}\AgdaSpace{}%
\AgdaSymbol{→}\AgdaSpace{}%
\AgdaFunction{VarMap}\AgdaSpace{}%
\AgdaBound{γ}\AgdaSpace{}%
\AgdaSymbol{→}\AgdaSpace{}%
\AgdaFunction{SVarMap}\AgdaSpace{}%
\AgdaBound{p}\AgdaSpace{}%
\AgdaSymbol{→}\AgdaSpace{}%
\AgdaFunction{Parser}\AgdaSpace{}%
\AgdaSymbol{(}\AgdaBound{δ}\AgdaSpace{}%
\AgdaOperator{\AgdaFunction{⇒[}}\AgdaSpace{}%
\AgdaFunction{Expr}\AgdaSpace{}%
\AgdaBound{p}\AgdaSpace{}%
\AgdaInductiveConstructor{compu}\AgdaSpace{}%
\AgdaOperator{\AgdaFunction{]}}\AgdaSpace{}%
\AgdaBound{γ}\AgdaSymbol{)}\<%
\\
%
\>[4]\AgdaFunction{eσ}\AgdaSpace{}%
\AgdaInductiveConstructor{zero}\AgdaSpace{}%
\AgdaBound{γ}\AgdaSpace{}%
\AgdaBound{p}\AgdaSpace{}%
\AgdaBound{vmap}\AgdaSpace{}%
\AgdaBound{svmap}%
\>[30]\AgdaSymbol{=}\AgdaSpace{}%
\AgdaFunction{return}\AgdaSpace{}%
\AgdaInductiveConstructor{ε}\<%
\\
%
\>[4]\AgdaFunction{eσ}\AgdaSpace{}%
\AgdaSymbol{(}\AgdaInductiveConstructor{suc}\AgdaSpace{}%
\AgdaBound{δ}\AgdaSymbol{)}\AgdaSpace{}%
\AgdaBound{γ}\AgdaSpace{}%
\AgdaBound{p}\AgdaSpace{}%
\AgdaBound{vmap}\AgdaSpace{}%
\AgdaBound{svmap}\AgdaSpace{}%
\AgdaSymbol{=}%
\>[616I]\AgdaKeyword{do}\<%
\\
\>[616I][@{}l@{\AgdaIndent{0}}]%
\>[34]\AgdaBound{rest}\AgdaSpace{}%
\AgdaOperator{\AgdaFunction{←}}\AgdaSpace{}%
\AgdaFunction{eσ}\AgdaSpace{}%
\AgdaBound{δ}\AgdaSpace{}%
\AgdaBound{γ}\AgdaSpace{}%
\AgdaBound{p}\AgdaSpace{}%
\AgdaBound{vmap}\AgdaSpace{}%
\AgdaBound{svmap}\<%
\\
%
\>[34]\AgdaFunction{ws-tolerant}\AgdaSpace{}%
\AgdaSymbol{(}\AgdaFunction{literal}\AgdaSpace{}%
\AgdaString{','}\AgdaSymbol{)}\<%
\\
%
\>[34]\AgdaBound{this}\AgdaSpace{}%
\AgdaOperator{\AgdaFunction{←}}\AgdaSpace{}%
\AgdaFunction{ecompu}\AgdaSpace{}%
\AgdaBound{p}\AgdaSpace{}%
\AgdaBound{γ}\AgdaSpace{}%
\AgdaBound{vmap}\AgdaSpace{}%
\AgdaBound{svmap}\<%
\\
%
\>[34]\AgdaFunction{return}\AgdaSpace{}%
\AgdaSymbol{(}\AgdaBound{rest}\AgdaSpace{}%
\AgdaOperator{\AgdaInductiveConstructor{-,}}\AgdaSpace{}%
\AgdaBound{this}\AgdaSymbol{)}\<%
\\
%
\>[4]\AgdaKeyword{open}\AgdaSpace{}%
\AgdaKeyword{import}\AgdaSpace{}%
\AgdaModule{Data.Nat.Show}\AgdaSpace{}%
\AgdaKeyword{using}\AgdaSpace{}%
\AgdaSymbol{(}\AgdaFunction{show}\AgdaSymbol{)}\<%
\\
%
\>[4]\AgdaKeyword{open}\AgdaSpace{}%
\AgdaKeyword{import}\AgdaSpace{}%
\AgdaModule{Data.String}\AgdaSpace{}%
\AgdaKeyword{using}\AgdaSpace{}%
\AgdaSymbol{(}\AgdaOperator{\AgdaFunction{\AgdaUnderscore{}++\AgdaUnderscore{}}}\AgdaSymbol{)}\<%
\\
%
\>[4]\AgdaFunction{einst}\AgdaSpace{}%
\AgdaSymbol{:}\AgdaSpace{}%
\AgdaFunction{ConstParser}\<%
\\
%
\>[4]\AgdaFunction{einst}\AgdaSpace{}%
\AgdaBound{p}\AgdaSpace{}%
\AgdaBound{γ}\AgdaSpace{}%
\AgdaBound{vmap}\AgdaSpace{}%
\AgdaBound{svmap}\AgdaSpace{}%
\AgdaSymbol{=}%
\>[650I]\AgdaKeyword{do}\<%
\\
\>[650I][@{}l@{\AgdaIndent{0}}]%
\>[29]\AgdaSymbol{((}\AgdaBound{δ}\AgdaSpace{}%
\AgdaOperator{\AgdaInductiveConstructor{,}}\AgdaSpace{}%
\AgdaBound{ξ}\AgdaSymbol{)}\AgdaSpace{}%
\AgdaOperator{\AgdaInductiveConstructor{,}}\AgdaSpace{}%
\AgdaBound{name}\AgdaSymbol{)}\AgdaSpace{}%
\AgdaOperator{\AgdaFunction{←}}\AgdaSpace{}%
\AgdaFunction{schvar}\AgdaSpace{}%
\AgdaBound{p}\AgdaSpace{}%
\AgdaBound{svmap}\<%
\\
%
\>[29]\AgdaInductiveConstructor{yes}\AgdaSpace{}%
\AgdaInductiveConstructor{refl}\AgdaSpace{}%
\AgdaOperator{\AgdaFunction{←}}\AgdaSpace{}%
\AgdaFunction{return}\AgdaSpace{}%
\AgdaSymbol{(}\AgdaBound{δ}\AgdaSpace{}%
\AgdaOperator{\AgdaFunction{≟}}\AgdaSpace{}%
\AgdaNumber{0}\AgdaSymbol{)}\<%
\\
\>[29][@{}l@{\AgdaIndent{0}}]%
\>[31]\AgdaKeyword{where}\AgdaSpace{}%
\AgdaInductiveConstructor{no}\AgdaSpace{}%
\AgdaSymbol{\AgdaUnderscore{}}\AgdaSpace{}%
\AgdaSymbol{→}%
\>[668I]\AgdaKeyword{do}\<%
\\
\>[668I][@{}l@{\AgdaIndent{0}}]%
\>[46]\AgdaFunction{literal}\AgdaSpace{}%
\AgdaString{'/'}\<%
\\
%
\>[46]\AgdaBound{σ}\AgdaSpace{}%
\AgdaOperator{\AgdaFunction{←}}\AgdaSpace{}%
\AgdaFunction{squarebracketed}\AgdaSpace{}%
\AgdaSymbol{(}\AgdaFunction{eσ}\AgdaSpace{}%
\AgdaBound{δ}\AgdaSpace{}%
\AgdaBound{γ}\AgdaSpace{}%
\AgdaBound{p}\AgdaSpace{}%
\AgdaBound{vmap}\AgdaSpace{}%
\AgdaBound{svmap}\AgdaSymbol{)}\<%
\\
%
\>[46]\AgdaFunction{return}\AgdaSpace{}%
\AgdaSymbol{(}\AgdaBound{ξ}\AgdaSpace{}%
\AgdaOperator{\AgdaInductiveConstructor{/}}\AgdaSpace{}%
\AgdaBound{σ}\AgdaSymbol{)}\<%
\\
%
\>[29]\AgdaFunction{return}\AgdaSpace{}%
\AgdaSymbol{(}\AgdaBound{ξ}\AgdaSpace{}%
\AgdaOperator{\AgdaInductiveConstructor{/}}\AgdaSpace{}%
\AgdaInductiveConstructor{ε}\AgdaSymbol{)}\<%
\\
\>[0]\<%
\\
%
\>[4]\AgdaFunction{evar}\AgdaSpace{}%
\AgdaSymbol{:}\AgdaSpace{}%
\AgdaFunction{CompuParser}\<%
\\
%
\>[4]\AgdaFunction{evar}\AgdaSpace{}%
\AgdaBound{p}\AgdaSpace{}%
\AgdaBound{γ}\AgdaSpace{}%
\AgdaBound{vmap}\AgdaSpace{}%
\AgdaBound{svmap}\AgdaSpace{}%
\AgdaSymbol{=}%
\>[691I]\AgdaKeyword{do}\<%
\\
\>[691I][@{}l@{\AgdaIndent{0}}]%
\>[31]\AgdaFunction{literal}\AgdaSpace{}%
\AgdaString{'.'}\<%
\\
%
\>[31]\AgdaBound{name}\AgdaSpace{}%
\AgdaOperator{\AgdaFunction{←}}\AgdaSpace{}%
\AgdaFunction{identifier}\<%
\\
%
\>[31]\AgdaFunction{maybe′}\AgdaSpace{}%
\AgdaSymbol{(}\AgdaFunction{return}\AgdaSpace{}%
\AgdaOperator{\AgdaFunction{∘′}}\AgdaSpace{}%
\AgdaInductiveConstructor{var}\AgdaSymbol{)}\AgdaSpace{}%
\AgdaFunction{fail}\AgdaSpace{}%
\AgdaSymbol{(}\AgdaFunction{lookup}\AgdaSpace{}%
\AgdaBound{name}\AgdaSpace{}%
\AgdaBound{vmap}\AgdaSymbol{)}\<%
\\
\>[0]\<%
\\
%
\>[4]\AgdaFunction{erad}\AgdaSpace{}%
\AgdaSymbol{:}\AgdaSpace{}%
\AgdaFunction{CompuParser}\<%
\\
%
\>[4]\AgdaFunction{erad}\AgdaSpace{}%
\AgdaBound{p}\AgdaSpace{}%
\AgdaBound{γ}\AgdaSpace{}%
\AgdaBound{vmap}\AgdaSpace{}%
\AgdaBound{svmap}\AgdaSpace{}%
\AgdaSymbol{=}%
\>[709I]\AgdaKeyword{do}\<%
\\
\>[709I][@{}l@{\AgdaIndent{0}}]%
\>[28]\AgdaBound{t}\AgdaSpace{}%
\AgdaOperator{\AgdaFunction{←}}\AgdaSpace{}%
\AgdaFunction{econst}\AgdaSpace{}%
\AgdaBound{p}\AgdaSpace{}%
\AgdaBound{γ}\AgdaSpace{}%
\AgdaBound{vmap}\AgdaSpace{}%
\AgdaBound{svmap}\<%
\\
%
\>[28]\AgdaFunction{ws-tolerant}\AgdaSpace{}%
\AgdaSymbol{(}\AgdaFunction{literal}\AgdaSpace{}%
\AgdaString{':'}\AgdaSymbol{)}\<%
\\
%
\>[28]\AgdaBound{T}\AgdaSpace{}%
\AgdaOperator{\AgdaFunction{←}}\AgdaSpace{}%
\AgdaFunction{econst}\AgdaSpace{}%
\AgdaBound{p}\AgdaSpace{}%
\AgdaBound{γ}\AgdaSpace{}%
\AgdaBound{vmap}\AgdaSpace{}%
\AgdaBound{svmap}\<%
\\
%
\>[28]\AgdaFunction{return}\AgdaSpace{}%
\AgdaSymbol{(}\AgdaBound{t}\AgdaSpace{}%
\AgdaOperator{\AgdaInductiveConstructor{∷}}\AgdaSpace{}%
\AgdaBound{T}\AgdaSymbol{)}\<%
\\
\>[0]\<%
\\
%
\>[2]\AgdaFunction{econst}\AgdaSpace{}%
\AgdaBound{p}\AgdaSpace{}%
\AgdaBound{γ}%
\>[729I]\AgdaBound{vmap}\AgdaSpace{}%
\AgdaBound{svmap}\AgdaSpace{}%
\AgdaSymbol{=}\AgdaSpace{}%
\AgdaKeyword{do}\<%
\\
\>[.][@{}l@{}]\<[729I]%
\>[13]\AgdaBound{fst}\AgdaSpace{}%
\AgdaOperator{\AgdaFunction{←}}\AgdaSpace{}%
\AgdaFunction{anyof}%
\>[735I]\AgdaSymbol{(}\AgdaFunction{eatom}%
\>[34]\AgdaBound{p}\AgdaSpace{}%
\AgdaBound{γ}\AgdaSpace{}%
\AgdaBound{vmap}\AgdaSpace{}%
\AgdaBound{svmap}%
\>[50]\AgdaOperator{\AgdaInductiveConstructor{∷}}\<%
\\
\>[735I][@{}l@{\AgdaIndent{0}}]%
\>[26]\AgdaFunction{ebinder}\AgdaSpace{}%
\AgdaBound{p}\AgdaSpace{}%
\AgdaBound{γ}\AgdaSpace{}%
\AgdaBound{vmap}\AgdaSpace{}%
\AgdaBound{svmap}%
\>[50]\AgdaOperator{\AgdaInductiveConstructor{∷}}\<%
\\
%
\>[26]\AgdaFunction{ethunk}%
\>[34]\AgdaBound{p}\AgdaSpace{}%
\AgdaBound{γ}\AgdaSpace{}%
\AgdaBound{vmap}\AgdaSpace{}%
\AgdaBound{svmap}%
\>[50]\AgdaOperator{\AgdaInductiveConstructor{∷}}\<%
\\
%
\>[26]\AgdaFunction{einst}%
\>[34]\AgdaBound{p}\AgdaSpace{}%
\AgdaBound{γ}\AgdaSpace{}%
\AgdaBound{vmap}\AgdaSpace{}%
\AgdaBound{svmap}%
\>[50]\AgdaOperator{\AgdaInductiveConstructor{∷}}\<%
\\
%
\>[26]\AgdaFunction{bracketed}\AgdaSpace{}%
\AgdaSymbol{(}\AgdaFunction{econst}\AgdaSpace{}%
\AgdaBound{p}\AgdaSpace{}%
\AgdaBound{γ}\AgdaSpace{}%
\AgdaBound{vmap}\AgdaSpace{}%
\AgdaBound{svmap}\AgdaSymbol{)}\AgdaSpace{}%
\AgdaOperator{\AgdaInductiveConstructor{∷}}\AgdaSpace{}%
\AgdaInductiveConstructor{[]}\AgdaSymbol{)}\<%
\\
%
\>[13]\AgdaInductiveConstructor{inj₁}\AgdaSpace{}%
\AgdaBound{snd}\AgdaSpace{}%
\AgdaOperator{\AgdaFunction{←}}\AgdaSpace{}%
\AgdaFunction{optional}%
\>[759I]\AgdaSymbol{(}\AgdaKeyword{do}\<%
\\
\>[759I][@{}l@{\AgdaIndent{0}}]%
\>[34]\AgdaFunction{whitespace}\<%
\\
%
\>[34]\AgdaFunction{econst}\AgdaSpace{}%
\AgdaBound{p}\AgdaSpace{}%
\AgdaBound{γ}\AgdaSpace{}%
\AgdaBound{vmap}\AgdaSpace{}%
\AgdaBound{svmap}\AgdaSymbol{)}\<%
\\
\>[13][@{}l@{\AgdaIndent{0}}]%
\>[15]\AgdaKeyword{where}\AgdaSpace{}%
\AgdaInductiveConstructor{inj₂}\AgdaSpace{}%
\AgdaSymbol{\AgdaUnderscore{}}\AgdaSpace{}%
\AgdaSymbol{→}\AgdaSpace{}%
\AgdaFunction{return}\AgdaSpace{}%
\AgdaBound{fst}\<%
\\
%
\>[13]\AgdaFunction{return}\AgdaSpace{}%
\AgdaSymbol{(}\AgdaBound{fst}\AgdaSpace{}%
\AgdaOperator{\AgdaInductiveConstructor{∙}}\AgdaSpace{}%
\AgdaBound{snd}\AgdaSymbol{)}\<%
\\
%
\\[\AgdaEmptyExtraSkip]%
%
\>[2]\AgdaFunction{ecompu}\AgdaSpace{}%
\AgdaBound{p}\AgdaSpace{}%
\AgdaBound{γ}%
\>[774I]\AgdaBound{vmap}\AgdaSpace{}%
\AgdaBound{svmap}\AgdaSpace{}%
\AgdaSymbol{=}\AgdaSpace{}%
\AgdaKeyword{do}\<%
\\
\>[.][@{}l@{}]\<[774I]%
\>[13]\AgdaBound{fst}\AgdaSpace{}%
\AgdaOperator{\AgdaFunction{←}}\AgdaSpace{}%
\AgdaFunction{anyof}%
\>[780I]\AgdaSymbol{(}\AgdaFunction{evar}\AgdaSpace{}%
\AgdaBound{p}\AgdaSpace{}%
\AgdaBound{γ}\AgdaSpace{}%
\AgdaBound{vmap}\AgdaSpace{}%
\AgdaBound{svmap}\AgdaSpace{}%
\AgdaOperator{\AgdaInductiveConstructor{∷}}\<%
\\
\>[780I][@{}l@{\AgdaIndent{0}}]%
\>[26]\AgdaFunction{erad}\AgdaSpace{}%
\AgdaBound{p}\AgdaSpace{}%
\AgdaBound{γ}\AgdaSpace{}%
\AgdaBound{vmap}\AgdaSpace{}%
\AgdaBound{svmap}\AgdaSpace{}%
\AgdaOperator{\AgdaInductiveConstructor{∷}}\<%
\\
%
\>[26]\AgdaFunction{bracketed}\AgdaSpace{}%
\AgdaSymbol{(}\AgdaFunction{ecompu}\AgdaSpace{}%
\AgdaBound{p}\AgdaSpace{}%
\AgdaBound{γ}\AgdaSpace{}%
\AgdaBound{vmap}\AgdaSpace{}%
\AgdaBound{svmap}\AgdaSymbol{)}\AgdaSpace{}%
\AgdaOperator{\AgdaInductiveConstructor{∷}}\AgdaSpace{}%
\AgdaInductiveConstructor{[]}\AgdaSymbol{)}\<%
\\
%
\>[13]\AgdaInductiveConstructor{inj₁}%
\>[798I]\AgdaBound{eliminator}\AgdaSpace{}%
\AgdaOperator{\AgdaFunction{←}}\AgdaSpace{}%
\AgdaFunction{optional}%
\>[801I]\AgdaSymbol{(}\AgdaKeyword{do}\<%
\\
\>[801I][@{}l@{\AgdaIndent{0}}]%
\>[43]\AgdaFunction{whitespace}\<%
\\
%
\>[43]\AgdaFunction{econst}\AgdaSpace{}%
\AgdaBound{p}\AgdaSpace{}%
\AgdaBound{γ}\AgdaSpace{}%
\AgdaBound{vmap}\AgdaSpace{}%
\AgdaBound{svmap}\AgdaSymbol{)}\<%
\\
\>[.][@{}l@{}]\<[798I]%
\>[18]\AgdaKeyword{where}\AgdaSpace{}%
\AgdaInductiveConstructor{inj₂}\AgdaSpace{}%
\AgdaSymbol{\AgdaUnderscore{}}\AgdaSpace{}%
\AgdaSymbol{→}\AgdaSpace{}%
\AgdaFunction{return}\AgdaSpace{}%
\AgdaBound{fst}\<%
\\
%
\>[13]\AgdaFunction{return}\AgdaSpace{}%
\AgdaSymbol{(}\AgdaInductiveConstructor{elim}\AgdaSpace{}%
\AgdaBound{fst}\AgdaSpace{}%
\AgdaBound{eliminator}\AgdaSymbol{)}\<%
\\
\>[0]\AgdaKeyword{open}\AgdaSpace{}%
\AgdaModule{ExpressionParser}\<%
\\
%
\\[\AgdaEmptyExtraSkip]%
\>[0]\AgdaKeyword{module}\AgdaSpace{}%
\AgdaModule{PremiseParser}\AgdaSpace{}%
\AgdaKeyword{where}\<%
\\
%
\\[\AgdaEmptyExtraSkip]%
\>[0][@{}l@{\AgdaIndent{0}}]%
\>[2]\AgdaKeyword{open}\AgdaSpace{}%
\AgdaKeyword{import}\AgdaSpace{}%
\AgdaModule{Rules}\AgdaSpace{}%
\AgdaKeyword{using}\AgdaSpace{}%
\AgdaSymbol{(}\AgdaDatatype{Prem}\AgdaSymbol{;}\AgdaSpace{}%
\AgdaOperator{\AgdaInductiveConstructor{\AgdaUnderscore{}∋'\AgdaUnderscore{}[\AgdaUnderscore{}]}}\AgdaSymbol{;}\AgdaSpace{}%
\AgdaOperator{\AgdaInductiveConstructor{\AgdaUnderscore{}≡'\AgdaUnderscore{}}}\AgdaSymbol{;}\AgdaSpace{}%
\AgdaInductiveConstructor{univ}\AgdaSymbol{;}\AgdaSpace{}%
\AgdaOperator{\AgdaInductiveConstructor{\AgdaUnderscore{}⊢'\AgdaUnderscore{}}}\AgdaSymbol{;}\AgdaSpace{}%
\AgdaInductiveConstructor{type}\AgdaSymbol{)}\<%
\\
%
\>[2]\AgdaKeyword{open}\AgdaSpace{}%
\AgdaKeyword{import}\AgdaSpace{}%
\AgdaModule{Data.Nat}\AgdaSpace{}%
\AgdaKeyword{using}\AgdaSpace{}%
\AgdaSymbol{()}\AgdaSpace{}%
\AgdaKeyword{renaming}\AgdaSymbol{(}\AgdaOperator{\AgdaFunction{\AgdaUnderscore{}≟\AgdaUnderscore{}}}\AgdaSpace{}%
\AgdaSymbol{to}\AgdaSpace{}%
\AgdaOperator{\AgdaFunction{\AgdaUnderscore{}≟n\AgdaUnderscore{}}}\AgdaSymbol{)}\<%
\\
%
\>[2]\AgdaKeyword{open}\AgdaSpace{}%
\AgdaKeyword{import}\AgdaSpace{}%
\AgdaModule{Pattern}\AgdaSpace{}%
\AgdaKeyword{using}\AgdaSpace{}%
\AgdaSymbol{(}\AgdaOperator{\AgdaFunction{\AgdaUnderscore{}-svar\AgdaUnderscore{}}}\AgdaSymbol{)}\<%
\\
%
\>[2]\AgdaKeyword{open}\AgdaSpace{}%
\AgdaKeyword{import}\AgdaSpace{}%
\AgdaModule{Data.Maybe}\AgdaSpace{}%
\AgdaKeyword{using}\AgdaSpace{}%
\AgdaSymbol{()}\AgdaSpace{}%
\AgdaKeyword{renaming}\AgdaSpace{}%
\AgdaSymbol{(}\AgdaFunction{maybe′}\AgdaSpace{}%
\AgdaSymbol{to}\AgdaSpace{}%
\AgdaFunction{maybe}\AgdaSymbol{)}\<%
\\
%
\>[2]\AgdaKeyword{open}\AgdaSpace{}%
\AgdaKeyword{import}\AgdaSpace{}%
\AgdaModule{Relation.Nullary}\AgdaSpace{}%
\AgdaKeyword{using}\AgdaSpace{}%
\AgdaSymbol{(}\AgdaInductiveConstructor{yes}\AgdaSymbol{;}\AgdaSpace{}%
\AgdaInductiveConstructor{no}\AgdaSymbol{)}\<%
\\
%
\>[2]\AgdaKeyword{open}\AgdaSpace{}%
\AgdaKeyword{import}\AgdaSpace{}%
\AgdaModule{Relation.Binary.PropositionalEquality}\AgdaSpace{}%
\AgdaKeyword{using}\AgdaSpace{}%
\AgdaSymbol{(}\AgdaInductiveConstructor{refl}\AgdaSymbol{)}\<%
\\
%
\>[2]\AgdaKeyword{import}\AgdaSpace{}%
\AgdaModule{Data.Tree.AVL}\AgdaSpace{}%
\AgdaKeyword{using}\AgdaSpace{}%
\AgdaSymbol{(}\AgdaFunction{foldr}\AgdaSymbol{)}\<%
\\
%
\>[2]\AgdaKeyword{open}\AgdaSpace{}%
\AgdaModule{parsermonad}\<%
\\
%
\\[\AgdaEmptyExtraSkip]%
%
\>[2]\AgdaFunction{PremiseParser}\AgdaSpace{}%
\AgdaSymbol{:}\AgdaSpace{}%
\AgdaPrimitiveType{Set}\<%
\\
%
\>[2]\AgdaFunction{PremiseParser}\AgdaSpace{}%
\AgdaSymbol{=}%
\>[861I]\AgdaSymbol{∀}\AgdaSpace{}%
\AgdaSymbol{\{}\AgdaBound{δ}\AgdaSymbol{\}\{}\AgdaBound{γ}\AgdaSymbol{\}}\AgdaSpace{}%
\AgdaSymbol{→}\<%
\\
\>[.][@{}l@{}]\<[861I]%
\>[18]\AgdaSymbol{(}\AgdaBound{p}\AgdaSpace{}%
\AgdaBound{q}\AgdaSpace{}%
\AgdaSymbol{:}\AgdaSpace{}%
\AgdaDatatype{Pattern}\AgdaSpace{}%
\AgdaBound{δ}\AgdaSymbol{)}\AgdaSpace{}%
\AgdaSymbol{→}\<%
\\
%
\>[18]\AgdaSymbol{(}\AgdaFunction{SVarMap}\AgdaSpace{}%
\AgdaBound{p}\AgdaSymbol{)}\AgdaSpace{}%
\AgdaSymbol{→}\AgdaSpace{}%
\AgdaSymbol{(}\AgdaFunction{SVarMap}\AgdaSpace{}%
\AgdaBound{q}\AgdaSymbol{)}\AgdaSpace{}%
\AgdaSymbol{→}\AgdaSpace{}%
\AgdaSymbol{(}\AgdaFunction{VarMap}\AgdaSpace{}%
\AgdaBound{γ}\AgdaSymbol{)}\AgdaSpace{}%
\AgdaSymbol{→}\<%
\\
%
\>[18]\AgdaFunction{Parser}%
\>[877I]\AgdaSymbol{(}\AgdaFunction{Σ[}\AgdaSpace{}%
\AgdaBound{(p'}\AgdaSpace{}%
\AgdaBound{,}\AgdaSpace{}%
\AgdaBound{q')}\AgdaSpace{}%
\AgdaFunction{∈}\AgdaSpace{}%
\AgdaSymbol{(}\AgdaDatatype{Pattern}\AgdaSpace{}%
\AgdaBound{γ}\AgdaSpace{}%
\AgdaOperator{\AgdaFunction{×}}\AgdaSpace{}%
\AgdaDatatype{Pattern}\AgdaSpace{}%
\AgdaBound{δ}\AgdaSymbol{)}\AgdaSpace{}%
\AgdaFunction{]}\<%
\\
\>[877I][@{}l@{\AgdaIndent{0}}]%
\>[26]\AgdaFunction{SVarMap}\AgdaSpace{}%
\AgdaBound{p'}\AgdaSpace{}%
\AgdaOperator{\AgdaFunction{×}}\AgdaSpace{}%
\AgdaFunction{SVarMap}\AgdaSpace{}%
\AgdaBound{q'}\AgdaSpace{}%
\AgdaOperator{\AgdaFunction{×}}\AgdaSpace{}%
\AgdaDatatype{Prem}\AgdaSpace{}%
\AgdaBound{p}\AgdaSpace{}%
\AgdaBound{q}\AgdaSpace{}%
\AgdaBound{γ}\AgdaSpace{}%
\AgdaBound{p'}\AgdaSpace{}%
\AgdaBound{q'}\AgdaSymbol{)}\<%
\\
%
\\[\AgdaEmptyExtraSkip]%
%
\>[2]\AgdaFunction{prem}\AgdaSpace{}%
\AgdaSymbol{:}\AgdaSpace{}%
\AgdaFunction{PremiseParser}\<%
\\
%
\\[\AgdaEmptyExtraSkip]%
%
\>[2]\AgdaFunction{typeprem}\AgdaSpace{}%
\AgdaSymbol{:}\AgdaSpace{}%
\AgdaFunction{PremiseParser}\<%
\\
%
\>[2]\AgdaFunction{typeprem}\AgdaSpace{}%
\AgdaSymbol{\{}\AgdaArgument{γ}\AgdaSpace{}%
\AgdaSymbol{=}\AgdaSpace{}%
\AgdaBound{γ}\AgdaSymbol{\}}\AgdaSpace{}%
\AgdaBound{p}\AgdaSpace{}%
\AgdaBound{q}\AgdaSpace{}%
\AgdaBound{pmap}\AgdaSpace{}%
\AgdaBound{qmap}\AgdaSpace{}%
\AgdaBound{vm}\<%
\\
\>[2][@{}l@{\AgdaIndent{0}}]%
\>[4]\AgdaSymbol{=}%
\>[911I]\AgdaKeyword{do}\<%
\\
\>[911I][@{}l@{\AgdaIndent{0}}]%
\>[8]\AgdaFunction{string}\AgdaSpace{}%
\AgdaString{"type"}\<%
\\
%
\>[8]\AgdaFunction{whitespace}\<%
\\
%
\>[8]\AgdaSymbol{((}\AgdaBound{δ'}\AgdaSpace{}%
\AgdaOperator{\AgdaInductiveConstructor{,}}\AgdaSpace{}%
\AgdaBound{ξ}\AgdaSymbol{)}\AgdaSpace{}%
\AgdaOperator{\AgdaInductiveConstructor{,}}\AgdaSpace{}%
\AgdaBound{name}\AgdaSymbol{)}\AgdaSpace{}%
\AgdaOperator{\AgdaFunction{←}}\AgdaSpace{}%
\AgdaFunction{schvar}\AgdaSpace{}%
\AgdaBound{q}\AgdaSpace{}%
\AgdaBound{qmap}\<%
\\
%
\>[8]\AgdaInductiveConstructor{yes}\AgdaSpace{}%
\AgdaInductiveConstructor{refl}\AgdaSpace{}%
\AgdaOperator{\AgdaFunction{←}}\AgdaSpace{}%
\AgdaFunction{return}\AgdaSpace{}%
\AgdaSymbol{(}\AgdaBound{δ'}\AgdaSpace{}%
\AgdaOperator{\AgdaFunction{≟n}}\AgdaSpace{}%
\AgdaBound{γ}\AgdaSymbol{)}\<%
\\
\>[8][@{}l@{\AgdaIndent{0}}]%
\>[10]\AgdaKeyword{where}\AgdaSpace{}%
\AgdaInductiveConstructor{no}\AgdaSpace{}%
\AgdaSymbol{\AgdaUnderscore{}}\AgdaSpace{}%
\AgdaSymbol{→}\AgdaSpace{}%
\AgdaFunction{fail}\<%
\\
%
\>[8]\AgdaFunction{return}\AgdaSpace{}%
\AgdaSymbol{(((}\AgdaInductiveConstructor{place}\AgdaSpace{}%
\AgdaFunction{ι}\AgdaSymbol{)}\AgdaSpace{}%
\AgdaOperator{\AgdaInductiveConstructor{,}}\AgdaSpace{}%
\AgdaSymbol{(}\AgdaBound{q}\AgdaSpace{}%
\AgdaOperator{\AgdaFunction{-}}\AgdaSpace{}%
\AgdaBound{ξ}\AgdaSymbol{))}\AgdaSpace{}%
\AgdaOperator{\AgdaInductiveConstructor{,}}\AgdaSpace{}%
\AgdaSymbol{(}\AgdaFunction{singleton}\AgdaSpace{}%
\AgdaBound{name}\AgdaSpace{}%
\AgdaSymbol{(}\AgdaBound{δ'}\AgdaSpace{}%
\AgdaOperator{\AgdaInductiveConstructor{,}}\AgdaSpace{}%
\AgdaInductiveConstructor{⋆}\AgdaSymbol{)}\AgdaSpace{}%
\AgdaOperator{\AgdaInductiveConstructor{,}}\AgdaSpace{}%
\AgdaBound{qmap}\AgdaSpace{}%
\AgdaOperator{\AgdaFunction{-svmap}}\AgdaSpace{}%
\AgdaBound{ξ}\AgdaSpace{}%
\AgdaOperator{\AgdaInductiveConstructor{,}}\AgdaSpace{}%
\AgdaInductiveConstructor{type}\AgdaSpace{}%
\AgdaBound{ξ}\AgdaSpace{}%
\AgdaFunction{ι}\AgdaSymbol{))}\<%
\\
%
\\[\AgdaEmptyExtraSkip]%
%
\>[2]\AgdaFunction{∋prem}\AgdaSpace{}%
\AgdaSymbol{:}\AgdaSpace{}%
\AgdaFunction{PremiseParser}\<%
\\
%
\>[2]\AgdaFunction{∋prem}\AgdaSpace{}%
\AgdaSymbol{\{}\AgdaArgument{γ}\AgdaSpace{}%
\AgdaSymbol{=}\AgdaSpace{}%
\AgdaBound{γ}\AgdaSymbol{\}}\AgdaSpace{}%
\AgdaBound{p}\AgdaSpace{}%
\AgdaBound{q}\AgdaSpace{}%
\AgdaBound{pmap}\AgdaSpace{}%
\AgdaBound{qmap}\AgdaSpace{}%
\AgdaBound{vm}\<%
\\
\>[2][@{}l@{\AgdaIndent{0}}]%
\>[4]\AgdaSymbol{=}%
\>[961I]\AgdaKeyword{do}\<%
\\
\>[961I][@{}l@{\AgdaIndent{0}}]%
\>[8]\AgdaBound{T}\AgdaSpace{}%
\AgdaOperator{\AgdaFunction{←}}\AgdaSpace{}%
\AgdaFunction{bracketed}\AgdaSpace{}%
\AgdaSymbol{(}\AgdaFunction{econst}\AgdaSpace{}%
\AgdaBound{p}\AgdaSpace{}%
\AgdaBound{γ}\AgdaSpace{}%
\AgdaBound{vm}\AgdaSpace{}%
\AgdaBound{pmap}\AgdaSymbol{)}\<%
\\
%
\>[8]\AgdaFunction{ws-tolerant}\AgdaSpace{}%
\AgdaSymbol{(}\AgdaFunction{string}\AgdaSpace{}%
\AgdaString{"<-"}\AgdaSymbol{)}\<%
\\
%
\>[8]\AgdaSymbol{((}\AgdaBound{δ'}\AgdaSpace{}%
\AgdaOperator{\AgdaInductiveConstructor{,}}\AgdaSpace{}%
\AgdaBound{ξ}\AgdaSymbol{)}\AgdaSpace{}%
\AgdaOperator{\AgdaInductiveConstructor{,}}\AgdaSpace{}%
\AgdaBound{name}\AgdaSymbol{)}\AgdaSpace{}%
\AgdaOperator{\AgdaFunction{←}}\AgdaSpace{}%
\AgdaFunction{schvar}\AgdaSpace{}%
\AgdaBound{q}\AgdaSpace{}%
\AgdaBound{qmap}\<%
\\
%
\>[8]\AgdaInductiveConstructor{yes}\AgdaSpace{}%
\AgdaInductiveConstructor{refl}\AgdaSpace{}%
\AgdaOperator{\AgdaFunction{←}}\AgdaSpace{}%
\AgdaFunction{return}\AgdaSpace{}%
\AgdaSymbol{(}\AgdaBound{δ'}\AgdaSpace{}%
\AgdaOperator{\AgdaFunction{≟n}}\AgdaSpace{}%
\AgdaBound{γ}\AgdaSymbol{)}\<%
\\
\>[8][@{}l@{\AgdaIndent{0}}]%
\>[10]\AgdaKeyword{where}\AgdaSpace{}%
\AgdaInductiveConstructor{no}\AgdaSpace{}%
\AgdaSymbol{\AgdaUnderscore{}}\AgdaSpace{}%
\AgdaSymbol{→}\AgdaSpace{}%
\AgdaFunction{fail}\<%
\\
%
\>[8]\AgdaFunction{return}\AgdaSpace{}%
\AgdaSymbol{((}\AgdaInductiveConstructor{place}\AgdaSpace{}%
\AgdaFunction{ι}\AgdaSpace{}%
\AgdaOperator{\AgdaInductiveConstructor{,}}\AgdaSpace{}%
\AgdaSymbol{(}\AgdaBound{q}\AgdaSpace{}%
\AgdaOperator{\AgdaFunction{-}}\AgdaSpace{}%
\AgdaBound{ξ}\AgdaSymbol{))}\AgdaSpace{}%
\AgdaOperator{\AgdaInductiveConstructor{,}}\AgdaSpace{}%
\AgdaSymbol{(}\AgdaFunction{singleton}\AgdaSpace{}%
\AgdaBound{name}\AgdaSpace{}%
\AgdaSymbol{(}\AgdaBound{δ'}\AgdaSpace{}%
\AgdaOperator{\AgdaInductiveConstructor{,}}\AgdaSpace{}%
\AgdaInductiveConstructor{⋆}\AgdaSymbol{)}\AgdaSpace{}%
\AgdaOperator{\AgdaInductiveConstructor{,}}\AgdaSpace{}%
\AgdaBound{qmap}\AgdaSpace{}%
\AgdaOperator{\AgdaFunction{-svmap}}\AgdaSpace{}%
\AgdaBound{ξ}\AgdaSpace{}%
\AgdaOperator{\AgdaInductiveConstructor{,}}\AgdaSpace{}%
\AgdaSymbol{(}\AgdaBound{T}\AgdaSpace{}%
\AgdaOperator{\AgdaInductiveConstructor{∋'}}\AgdaSpace{}%
\AgdaBound{ξ}\AgdaSpace{}%
\AgdaOperator{\AgdaInductiveConstructor{[}}\AgdaSpace{}%
\AgdaFunction{ι}\AgdaSpace{}%
\AgdaOperator{\AgdaInductiveConstructor{]}}\AgdaSymbol{)))}\<%
\\
%
\\[\AgdaEmptyExtraSkip]%
%
\>[2]\AgdaFunction{≡prem}\AgdaSpace{}%
\AgdaSymbol{:}\AgdaSpace{}%
\AgdaFunction{PremiseParser}\<%
\\
%
\>[2]\AgdaFunction{≡prem}\AgdaSpace{}%
\AgdaSymbol{\{}\AgdaArgument{γ}\AgdaSpace{}%
\AgdaSymbol{=}\AgdaSpace{}%
\AgdaBound{γ}\AgdaSymbol{\}}\AgdaSpace{}%
\AgdaBound{p}\AgdaSpace{}%
\AgdaBound{q}\AgdaSpace{}%
\AgdaBound{pmap}\AgdaSpace{}%
\AgdaBound{qmap}\AgdaSpace{}%
\AgdaBound{vm}\<%
\\
\>[2][@{}l@{\AgdaIndent{0}}]%
\>[4]\AgdaSymbol{=}%
\>[1022I]\AgdaKeyword{do}\<%
\\
\>[1022I][@{}l@{\AgdaIndent{0}}]%
\>[8]\AgdaBound{S}\AgdaSpace{}%
\AgdaOperator{\AgdaFunction{←}}\AgdaSpace{}%
\AgdaFunction{bracketed}\AgdaSpace{}%
\AgdaSymbol{(}\AgdaFunction{econst}\AgdaSpace{}%
\AgdaBound{p}\AgdaSpace{}%
\AgdaBound{γ}\AgdaSpace{}%
\AgdaFunction{empty}\AgdaSpace{}%
\AgdaBound{pmap}\AgdaSymbol{)}\<%
\\
%
\>[8]\AgdaFunction{ws-tolerant}\AgdaSpace{}%
\AgdaSymbol{(}\AgdaFunction{literal}\AgdaSpace{}%
\AgdaString{'='}\AgdaSymbol{)}\<%
\\
%
\>[8]\AgdaBound{T}\AgdaSpace{}%
\AgdaOperator{\AgdaFunction{←}}\AgdaSpace{}%
\AgdaFunction{econst}\AgdaSpace{}%
\AgdaBound{p}\AgdaSpace{}%
\AgdaBound{γ}\AgdaSpace{}%
\AgdaFunction{empty}\AgdaSpace{}%
\AgdaBound{pmap}\<%
\\
%
\>[8]\AgdaFunction{return}\AgdaSpace{}%
\AgdaSymbol{(((}\AgdaInductiveConstructor{`}\AgdaSpace{}%
\AgdaString{"⊤"}\AgdaSymbol{)}\AgdaSpace{}%
\AgdaOperator{\AgdaInductiveConstructor{,}}\AgdaSpace{}%
\AgdaBound{q}\AgdaSymbol{)}\AgdaSpace{}%
\AgdaOperator{\AgdaInductiveConstructor{,}}\AgdaSpace{}%
\AgdaSymbol{(}\AgdaFunction{empty}\AgdaSpace{}%
\AgdaOperator{\AgdaInductiveConstructor{,}}\AgdaSpace{}%
\AgdaBound{qmap}\AgdaSpace{}%
\AgdaOperator{\AgdaInductiveConstructor{,}}\AgdaSpace{}%
\AgdaSymbol{(}\AgdaBound{S}\AgdaSpace{}%
\AgdaOperator{\AgdaInductiveConstructor{≡'}}\AgdaSpace{}%
\AgdaBound{T}\AgdaSymbol{)))}\<%
\\
%
\\[\AgdaEmptyExtraSkip]%
%
\>[2]\AgdaFunction{univprem}\AgdaSpace{}%
\AgdaSymbol{:}\AgdaSpace{}%
\AgdaFunction{PremiseParser}\<%
\\
%
\>[2]\AgdaFunction{univprem}\AgdaSpace{}%
\AgdaSymbol{\{}\AgdaArgument{γ}\AgdaSpace{}%
\AgdaSymbol{=}\AgdaSpace{}%
\AgdaBound{γ}\AgdaSymbol{\}}\AgdaSpace{}%
\AgdaBound{p}\AgdaSpace{}%
\AgdaBound{q}\AgdaSpace{}%
\AgdaBound{pmap}\AgdaSpace{}%
\AgdaBound{qmap}\AgdaSpace{}%
\AgdaBound{vm}\<%
\\
\>[2][@{}l@{\AgdaIndent{0}}]%
\>[4]\AgdaSymbol{=}%
\>[1060I]\AgdaKeyword{do}\<%
\\
\>[1060I][@{}l@{\AgdaIndent{0}}]%
\>[8]\AgdaFunction{string}\AgdaSpace{}%
\AgdaString{"univ"}\<%
\\
%
\>[8]\AgdaFunction{whitespace}\<%
\\
%
\>[8]\AgdaBound{U}\AgdaSpace{}%
\AgdaOperator{\AgdaFunction{←}}\AgdaSpace{}%
\AgdaFunction{econst}\AgdaSpace{}%
\AgdaBound{p}\AgdaSpace{}%
\AgdaBound{γ}\AgdaSpace{}%
\AgdaFunction{empty}\AgdaSpace{}%
\AgdaBound{pmap}\<%
\\
%
\>[8]\AgdaFunction{return}\AgdaSpace{}%
\AgdaSymbol{(((}\AgdaInductiveConstructor{`}\AgdaSpace{}%
\AgdaString{"⊤"}\AgdaSymbol{)}\AgdaSpace{}%
\AgdaOperator{\AgdaInductiveConstructor{,}}\AgdaSpace{}%
\AgdaBound{q}\AgdaSymbol{)}\AgdaSpace{}%
\AgdaOperator{\AgdaInductiveConstructor{,}}\AgdaSpace{}%
\AgdaSymbol{(}\AgdaFunction{empty}\AgdaSpace{}%
\AgdaOperator{\AgdaInductiveConstructor{,}}\AgdaSpace{}%
\AgdaBound{qmap}\AgdaSpace{}%
\AgdaOperator{\AgdaInductiveConstructor{,}}\AgdaSpace{}%
\AgdaInductiveConstructor{univ}\AgdaSpace{}%
\AgdaBound{U}\AgdaSymbol{))}\<%
\\
%
\\[\AgdaEmptyExtraSkip]%
%
\>[2]\AgdaSymbol{\{-\#}\AgdaSpace{}%
\AgdaKeyword{TERMINATING}\AgdaSpace{}%
\AgdaSymbol{\#-\}}\<%
\\
%
\>[2]\AgdaFunction{⊢prem}\AgdaSpace{}%
\AgdaSymbol{:}\AgdaSpace{}%
\AgdaFunction{PremiseParser}\<%
\\
%
\>[2]\AgdaFunction{⊢prem}\AgdaSpace{}%
\AgdaSymbol{\{}\AgdaArgument{γ}\AgdaSpace{}%
\AgdaSymbol{=}\AgdaSpace{}%
\AgdaBound{γ}\AgdaSymbol{\}}\AgdaSpace{}%
\AgdaBound{p}\AgdaSpace{}%
\AgdaBound{q}\AgdaSpace{}%
\AgdaBound{pmap}\AgdaSpace{}%
\AgdaBound{qmap}\AgdaSpace{}%
\AgdaBound{vm}\<%
\\
\>[2][@{}l@{\AgdaIndent{0}}]%
\>[4]\AgdaSymbol{=}%
\>[1091I]\AgdaKeyword{do}\<%
\\
\>[1091I][@{}l@{\AgdaIndent{0}}]%
\>[8]\AgdaBound{name}\AgdaSpace{}%
\AgdaOperator{\AgdaFunction{←}}\AgdaSpace{}%
\AgdaFunction{identifier}\<%
\\
%
\>[8]\AgdaFunction{ws-tolerant}\AgdaSpace{}%
\AgdaSymbol{(}\AgdaFunction{literal}\AgdaSpace{}%
\AgdaString{':'}\AgdaSymbol{)}\<%
\\
%
\>[8]\AgdaBound{S}\AgdaSpace{}%
\AgdaOperator{\AgdaFunction{←}}\AgdaSpace{}%
\AgdaSymbol{(}\AgdaFunction{safe}\AgdaSpace{}%
\AgdaOperator{\AgdaFunction{∘′}}\AgdaSpace{}%
\AgdaFunction{bracketed}\AgdaSymbol{)}\AgdaSpace{}%
\AgdaSymbol{(}\AgdaFunction{econst}\AgdaSpace{}%
\AgdaBound{p}\AgdaSpace{}%
\AgdaBound{γ}\AgdaSpace{}%
\AgdaFunction{empty}\AgdaSpace{}%
\AgdaBound{pmap}\AgdaSymbol{)}\<%
\\
%
\>[8]\AgdaFunction{ws-tolerant}\AgdaSpace{}%
\AgdaSymbol{(}\AgdaFunction{string}\AgdaSpace{}%
\AgdaString{"|-"}\AgdaSymbol{)}\<%
\\
%
\>[8]\AgdaSymbol{((}\AgdaBound{p'}\AgdaSpace{}%
\AgdaOperator{\AgdaInductiveConstructor{,}}\AgdaSpace{}%
\AgdaBound{q'}\AgdaSymbol{)}\AgdaOperator{\AgdaInductiveConstructor{,}}\AgdaSpace{}%
\AgdaSymbol{(}\AgdaBound{p'm}\AgdaSpace{}%
\AgdaOperator{\AgdaInductiveConstructor{,}}\AgdaSpace{}%
\AgdaBound{q'm}\AgdaSpace{}%
\AgdaOperator{\AgdaInductiveConstructor{,}}\AgdaSpace{}%
\AgdaBound{P}\AgdaSymbol{))}\AgdaSpace{}%
\AgdaOperator{\AgdaFunction{←}}\AgdaSpace{}%
\AgdaFunction{prem}\AgdaSpace{}%
\AgdaSymbol{\{}\AgdaArgument{γ}\AgdaSpace{}%
\AgdaSymbol{=}\AgdaSpace{}%
\AgdaInductiveConstructor{suc}\AgdaSpace{}%
\AgdaBound{γ}\AgdaSymbol{\}}\AgdaSpace{}%
\AgdaBound{p}\AgdaSpace{}%
\AgdaBound{q}\AgdaSpace{}%
\AgdaBound{pmap}\AgdaSpace{}%
\AgdaBound{qmap}\AgdaSpace{}%
\AgdaSymbol{(}\AgdaFunction{insert}\AgdaSpace{}%
\AgdaBound{name}\AgdaSpace{}%
\AgdaInductiveConstructor{ze}\AgdaSpace{}%
\AgdaSymbol{(}\AgdaFunction{map}\AgdaSpace{}%
\AgdaInductiveConstructor{su}\AgdaSpace{}%
\AgdaBound{vm}\AgdaSymbol{))}\<%
\\
%
\>[8]\AgdaFunction{return}\AgdaSpace{}%
\AgdaSymbol{((}\AgdaInductiveConstructor{bind}\AgdaSpace{}%
\AgdaBound{p'}\AgdaSpace{}%
\AgdaOperator{\AgdaInductiveConstructor{,}}\AgdaSpace{}%
\AgdaBound{q'}\AgdaSymbol{)}\AgdaSpace{}%
\AgdaOperator{\AgdaInductiveConstructor{,}}\AgdaSpace{}%
\AgdaSymbol{(}\AgdaFunction{map}\AgdaSpace{}%
\AgdaSymbol{(λ}\AgdaSpace{}%
\AgdaSymbol{\{(}\AgdaBound{δ}\AgdaSpace{}%
\AgdaOperator{\AgdaInductiveConstructor{,}}\AgdaSpace{}%
\AgdaBound{v}\AgdaSymbol{)}\AgdaSpace{}%
\AgdaSymbol{→}\AgdaSpace{}%
\AgdaSymbol{(}\AgdaBound{δ}\AgdaSpace{}%
\AgdaOperator{\AgdaInductiveConstructor{,}}\AgdaSpace{}%
\AgdaInductiveConstructor{bind}\AgdaSpace{}%
\AgdaBound{v}\AgdaSymbol{)\}}\AgdaSpace{}%
\AgdaSymbol{)}\AgdaSpace{}%
\AgdaBound{p'm}\AgdaSpace{}%
\AgdaOperator{\AgdaInductiveConstructor{,}}\AgdaSpace{}%
\AgdaBound{q'm}\AgdaSpace{}%
\AgdaOperator{\AgdaInductiveConstructor{,}}\AgdaSpace{}%
\AgdaSymbol{(}\AgdaBound{S}\AgdaSpace{}%
\AgdaOperator{\AgdaInductiveConstructor{⊢'}}\AgdaSpace{}%
\AgdaBound{P}\AgdaSymbol{)))}\<%
\\
%
\>[2]\AgdaKeyword{open}\AgdaSpace{}%
\AgdaKeyword{import}\AgdaSpace{}%
\AgdaModule{Expression}\AgdaSpace{}%
\AgdaKeyword{using}\AgdaSpace{}%
\AgdaSymbol{(}\AgdaInductiveConstructor{`}\AgdaSymbol{)}\<%
\\
%
\>[2]\AgdaFunction{prem}\AgdaSpace{}%
\AgdaBound{p}\AgdaSpace{}%
\AgdaBound{q}\AgdaSpace{}%
\AgdaBound{pmap}\AgdaSpace{}%
\AgdaBound{qmap}\AgdaSpace{}%
\AgdaBound{vm}\<%
\\
\>[2][@{}l@{\AgdaIndent{0}}]%
\>[4]\AgdaSymbol{=}\AgdaSpace{}%
\AgdaFunction{anyof}%
\>[1163I]\AgdaSymbol{(}\AgdaFunction{Data.List.map}\AgdaSpace{}%
\AgdaSymbol{(λ}\AgdaSpace{}%
\AgdaBound{pp}\AgdaSpace{}%
\AgdaSymbol{→}\AgdaSpace{}%
\AgdaBound{pp}\AgdaSpace{}%
\AgdaBound{p}\AgdaSpace{}%
\AgdaBound{q}\AgdaSpace{}%
\AgdaBound{pmap}\AgdaSpace{}%
\AgdaBound{qmap}\AgdaSpace{}%
\AgdaBound{vm}\AgdaSymbol{)}\<%
\\
\>[.][@{}l@{}]\<[1163I]%
\>[12]\AgdaSymbol{(}\AgdaFunction{typeprem}\AgdaSpace{}%
\AgdaOperator{\AgdaInductiveConstructor{∷}}\AgdaSpace{}%
\AgdaFunction{∋prem}\AgdaSpace{}%
\AgdaOperator{\AgdaInductiveConstructor{∷}}\AgdaSpace{}%
\AgdaFunction{≡prem}\AgdaSpace{}%
\AgdaOperator{\AgdaInductiveConstructor{∷}}\AgdaSpace{}%
\AgdaFunction{univprem}\AgdaSpace{}%
\AgdaOperator{\AgdaInductiveConstructor{∷}}\AgdaSpace{}%
\AgdaFunction{⊢prem}\AgdaSpace{}%
\AgdaOperator{\AgdaInductiveConstructor{∷}}\AgdaSpace{}%
\AgdaInductiveConstructor{[]}\AgdaSymbol{))}\<%
\\
\>[0]\AgdaKeyword{open}\AgdaSpace{}%
\AgdaModule{PremiseParser}\<%
\\
%
\\[\AgdaEmptyExtraSkip]%
\>[0]\AgdaKeyword{module}\AgdaSpace{}%
\AgdaModule{PremisechainParser}\AgdaSpace{}%
\AgdaKeyword{where}\<%
\\
%
\\[\AgdaEmptyExtraSkip]%
\>[0][@{}l@{\AgdaIndent{0}}]%
\>[2]\AgdaKeyword{open}\AgdaSpace{}%
\AgdaKeyword{import}\AgdaSpace{}%
\AgdaModule{Rules}\AgdaSpace{}%
\AgdaKeyword{using}\AgdaSpace{}%
\AgdaSymbol{(}\AgdaDatatype{Prems}\AgdaSymbol{;}\AgdaSpace{}%
\AgdaInductiveConstructor{ε}\AgdaSymbol{;}\AgdaSpace{}%
\AgdaOperator{\AgdaInductiveConstructor{\AgdaUnderscore{}⇉\AgdaUnderscore{}}}\AgdaSymbol{;}\AgdaSpace{}%
\AgdaOperator{\AgdaDatatype{\AgdaUnderscore{}Placeless}}\AgdaSymbol{;}\AgdaSpace{}%
\AgdaOperator{\AgdaFunction{\AgdaUnderscore{}-places}}\AgdaSymbol{;}\AgdaSpace{}%
\AgdaOperator{\AgdaFunction{\AgdaUnderscore{}placeless}}\AgdaSymbol{)}\<%
\\
%
\>[2]\AgdaKeyword{open}\AgdaSpace{}%
\AgdaKeyword{import}\AgdaSpace{}%
\AgdaModule{Pattern}\AgdaSpace{}%
\AgdaKeyword{using}\AgdaSpace{}%
\AgdaSymbol{(}\AgdaOperator{\AgdaFunction{\AgdaUnderscore{}≟\AgdaUnderscore{}}}\AgdaSymbol{)}\<%
\\
%
\>[2]\AgdaKeyword{open}\AgdaSpace{}%
\AgdaKeyword{import}\AgdaSpace{}%
\AgdaModule{Relation.Nullary}\AgdaSpace{}%
\AgdaKeyword{using}\AgdaSpace{}%
\AgdaSymbol{(}\AgdaInductiveConstructor{yes}\AgdaSymbol{;}\AgdaSpace{}%
\AgdaInductiveConstructor{no}\AgdaSymbol{)}\<%
\\
%
\>[2]\AgdaKeyword{open}\AgdaSpace{}%
\AgdaKeyword{import}\AgdaSpace{}%
\AgdaModule{Relation.Binary.PropositionalEquality}\AgdaSpace{}%
\AgdaKeyword{using}\AgdaSpace{}%
\AgdaSymbol{(}\AgdaInductiveConstructor{refl}\AgdaSymbol{;}\AgdaSpace{}%
\AgdaFunction{subst}\AgdaSymbol{)}\<%
\\
%
\>[2]\AgdaKeyword{open}\AgdaSpace{}%
\AgdaModule{parsermonad}\<%
\\
%
\\[\AgdaEmptyExtraSkip]%
%
\>[2]\AgdaFunction{PremisechainParser}%
\>[1210I]\AgdaSymbol{=}\<%
\\
\>[.][@{}l@{}]\<[1210I]%
\>[21]\AgdaSymbol{(}\AgdaBound{p}\AgdaSpace{}%
\AgdaBound{q}\AgdaSpace{}%
\AgdaSymbol{:}\AgdaSpace{}%
\AgdaDatatype{Pattern}\AgdaSpace{}%
\AgdaNumber{0}\AgdaSymbol{)}\AgdaSpace{}%
\AgdaSymbol{→}\<%
\\
%
\>[21]\AgdaSymbol{(}\AgdaFunction{SVarMap}\AgdaSpace{}%
\AgdaBound{p}\AgdaSymbol{)}\AgdaSpace{}%
\AgdaSymbol{→}\AgdaSpace{}%
\AgdaSymbol{(}\AgdaFunction{SVarMap}\AgdaSpace{}%
\AgdaBound{q}\AgdaSymbol{)}\AgdaSpace{}%
\AgdaSymbol{→}\<%
\\
%
\>[21]\AgdaFunction{Parser}%
\>[1221I]\AgdaSymbol{(}\AgdaFunction{Σ[}\AgdaSpace{}%
\AgdaBound{p'}\AgdaSpace{}%
\AgdaFunction{∈}\AgdaSpace{}%
\AgdaDatatype{Pattern}\AgdaSpace{}%
\AgdaNumber{0}\AgdaSpace{}%
\AgdaFunction{]}\<%
\\
\>[1221I][@{}l@{\AgdaIndent{0}}]%
\>[29]\AgdaFunction{SVarMap}\AgdaSpace{}%
\AgdaBound{p'}\AgdaSpace{}%
\AgdaOperator{\AgdaFunction{×}}\AgdaSpace{}%
\AgdaDatatype{Prems}\AgdaSpace{}%
\AgdaBound{p}\AgdaSpace{}%
\AgdaBound{q}\AgdaSpace{}%
\AgdaBound{p'}\AgdaSymbol{)}\<%
\\
\>[0]\<%
\\
%
\>[2]\AgdaFunction{chain}\AgdaSpace{}%
\AgdaSymbol{:}\AgdaSpace{}%
\AgdaFunction{PremisechainParser}\<%
\\
\>[0]\<%
\\
%
\>[2]\AgdaKeyword{private}\<%
\\
\>[2][@{}l@{\AgdaIndent{0}}]%
\>[4]\AgdaFunction{εchain}\AgdaSpace{}%
\AgdaSymbol{:}\AgdaSpace{}%
\AgdaFunction{PremisechainParser}\<%
\\
%
\>[4]\AgdaFunction{εchain}\AgdaSpace{}%
\AgdaBound{p}\AgdaSpace{}%
\AgdaBound{q}\AgdaSpace{}%
\AgdaBound{pmap}\AgdaSpace{}%
\AgdaBound{qmap}\<%
\\
\>[4][@{}l@{\AgdaIndent{0}}]%
\>[6]\AgdaSymbol{=}%
\>[1241I]\AgdaKeyword{do}\<%
\\
\>[1241I][@{}l@{\AgdaIndent{0}}]%
\>[10]\AgdaInductiveConstructor{yes}\AgdaSpace{}%
\AgdaBound{eq}\AgdaSpace{}%
\AgdaOperator{\AgdaFunction{←}}\AgdaSpace{}%
\AgdaFunction{return}\AgdaSpace{}%
\AgdaSymbol{((}\AgdaBound{q}\AgdaSpace{}%
\AgdaOperator{\AgdaFunction{-places}}\AgdaSymbol{)}\AgdaSpace{}%
\AgdaOperator{\AgdaFunction{≟}}\AgdaSpace{}%
\AgdaBound{q}\AgdaSymbol{)}\<%
\\
\>[10][@{}l@{\AgdaIndent{0}}]%
\>[12]\AgdaKeyword{where}\AgdaSpace{}%
\AgdaInductiveConstructor{no}\AgdaSpace{}%
\AgdaBound{p}\AgdaSpace{}%
\AgdaSymbol{→}\AgdaSpace{}%
\AgdaFunction{fail}\<%
\\
%
\>[10]\AgdaFunction{return}\AgdaSpace{}%
\AgdaSymbol{(}\AgdaBound{p}\AgdaSpace{}%
\AgdaOperator{\AgdaInductiveConstructor{,}}\AgdaSpace{}%
\AgdaSymbol{(}\AgdaBound{pmap}\AgdaSpace{}%
\AgdaOperator{\AgdaInductiveConstructor{,}}\AgdaSpace{}%
\AgdaInductiveConstructor{ε}\AgdaSpace{}%
\AgdaSymbol{(}\AgdaFunction{subst}\AgdaSpace{}%
\AgdaSymbol{(λ}\AgdaSpace{}%
\AgdaBound{x}\AgdaSpace{}%
\AgdaSymbol{→}\AgdaSpace{}%
\AgdaBound{x}\AgdaSpace{}%
\AgdaOperator{\AgdaDatatype{Placeless}}\AgdaSymbol{)}\AgdaSpace{}%
\AgdaBound{eq}\AgdaSpace{}%
\AgdaSymbol{(}\AgdaBound{q}\AgdaSpace{}%
\AgdaOperator{\AgdaFunction{placeless}}\AgdaSymbol{))))}\<%
\\
\>[0]\<%
\\
%
\>[4]\AgdaSymbol{\{-\#}\AgdaSpace{}%
\AgdaKeyword{TERMINATING}\AgdaSpace{}%
\AgdaSymbol{\#-\}}\<%
\\
%
\>[4]\AgdaFunction{nonε-chain}\AgdaSpace{}%
\AgdaSymbol{:}\AgdaSpace{}%
\AgdaFunction{PremisechainParser}\<%
\\
%
\>[4]\AgdaFunction{nonε-chain}\AgdaSpace{}%
\AgdaBound{p}\AgdaSpace{}%
\AgdaBound{q}\AgdaSpace{}%
\AgdaBound{pmap}\AgdaSpace{}%
\AgdaBound{qmap}\<%
\\
\>[4][@{}l@{\AgdaIndent{0}}]%
\>[6]\AgdaSymbol{=}%
\>[1275I]\AgdaKeyword{do}\<%
\\
\>[1275I][@{}l@{\AgdaIndent{0}}]%
\>[10]\AgdaSymbol{((}\AgdaBound{p'}\AgdaSpace{}%
\AgdaOperator{\AgdaInductiveConstructor{,}}\AgdaSpace{}%
\AgdaBound{q'}\AgdaSymbol{)}\AgdaSpace{}%
\AgdaOperator{\AgdaInductiveConstructor{,}}\AgdaSpace{}%
\AgdaSymbol{(}\AgdaBound{p'm}\AgdaSpace{}%
\AgdaOperator{\AgdaInductiveConstructor{,}}\AgdaSpace{}%
\AgdaBound{q'm}\AgdaSpace{}%
\AgdaOperator{\AgdaInductiveConstructor{,}}\AgdaSpace{}%
\AgdaBound{prm}\AgdaSymbol{))}\AgdaSpace{}%
\AgdaOperator{\AgdaFunction{←}}\AgdaSpace{}%
\AgdaFunction{prem}\AgdaSpace{}%
\AgdaSymbol{\{}\AgdaArgument{γ}\AgdaSpace{}%
\AgdaSymbol{=}\AgdaSpace{}%
\AgdaNumber{0}\AgdaSymbol{\}}\AgdaSpace{}%
\AgdaBound{p}\AgdaSpace{}%
\AgdaBound{q}\AgdaSpace{}%
\AgdaBound{pmap}\AgdaSpace{}%
\AgdaBound{qmap}\AgdaSpace{}%
\AgdaFunction{empty}\<%
\\
%
\>[10]\AgdaFunction{ws+nl}\<%
\\
%
\>[10]\AgdaSymbol{(}\AgdaBound{p'}\AgdaSpace{}%
\AgdaOperator{\AgdaInductiveConstructor{,}}\AgdaSpace{}%
\AgdaBound{p'm}\AgdaSpace{}%
\AgdaOperator{\AgdaInductiveConstructor{,}}\AgdaSpace{}%
\AgdaBound{rest}\AgdaSymbol{)}\AgdaSpace{}%
\AgdaOperator{\AgdaFunction{←}}\AgdaSpace{}%
\AgdaFunction{chain}\AgdaSpace{}%
\AgdaSymbol{(}\AgdaBound{p}\AgdaSpace{}%
\AgdaOperator{\AgdaInductiveConstructor{∙}}\AgdaSpace{}%
\AgdaBound{p'}\AgdaSymbol{)}%
\>[1303I]\AgdaBound{q'}%
\>[49]\AgdaSymbol{(}\AgdaFunction{union}\<%
\\
\>[49][@{}l@{\AgdaIndent{0}}]%
\>[51]\AgdaSymbol{(}\AgdaFunction{map}\AgdaSpace{}%
\AgdaSymbol{(λ}\AgdaSpace{}%
\AgdaSymbol{\{(}\AgdaBound{s}\AgdaSpace{}%
\AgdaOperator{\AgdaInductiveConstructor{,}}\AgdaSpace{}%
\AgdaBound{v}\AgdaSymbol{)}\AgdaSpace{}%
\AgdaSymbol{→}\AgdaSpace{}%
\AgdaSymbol{(}\AgdaBound{s}\AgdaSpace{}%
\AgdaOperator{\AgdaInductiveConstructor{,}}\AgdaSpace{}%
\AgdaBound{v}\AgdaSpace{}%
\AgdaOperator{\AgdaInductiveConstructor{∙}}\AgdaSymbol{)\})}\AgdaSpace{}%
\AgdaBound{pmap}\AgdaSymbol{)}\<%
\\
%
\>[51]\AgdaSymbol{(}\AgdaFunction{map}\AgdaSpace{}%
\AgdaSymbol{(λ}\AgdaSpace{}%
\AgdaSymbol{\{(}\AgdaBound{s}\AgdaSpace{}%
\AgdaOperator{\AgdaInductiveConstructor{,}}\AgdaSpace{}%
\AgdaBound{v}\AgdaSymbol{)}\AgdaSpace{}%
\AgdaSymbol{→}\AgdaSpace{}%
\AgdaSymbol{(}\AgdaBound{s}\AgdaSpace{}%
\AgdaOperator{\AgdaInductiveConstructor{,}}\AgdaSpace{}%
\AgdaOperator{\AgdaInductiveConstructor{∙}}\AgdaSpace{}%
\AgdaBound{v}\AgdaSymbol{)\})}\AgdaSpace{}%
\AgdaBound{p'm}\AgdaSymbol{))}\<%
\\
\>[.][@{}l@{}]\<[1303I]%
\>[45]\AgdaBound{q'm}\<%
\\
%
\>[10]\AgdaFunction{return}\AgdaSpace{}%
\AgdaSymbol{(}\AgdaBound{p'}\AgdaSpace{}%
\AgdaOperator{\AgdaInductiveConstructor{,}}\AgdaSpace{}%
\AgdaBound{p'm}\AgdaSpace{}%
\AgdaOperator{\AgdaInductiveConstructor{,}}\AgdaSpace{}%
\AgdaSymbol{(}\AgdaBound{prm}\AgdaSpace{}%
\AgdaOperator{\AgdaInductiveConstructor{⇉}}\AgdaSpace{}%
\AgdaBound{rest}\AgdaSymbol{))}\<%
\\
\>[0]\<%
\\
%
\>[2]\AgdaFunction{chain}\AgdaSpace{}%
\AgdaBound{p}\AgdaSpace{}%
\AgdaBound{q}\AgdaSpace{}%
\AgdaBound{pmap}\AgdaSpace{}%
\AgdaBound{qmap}\AgdaSpace{}%
\AgdaSymbol{=}%
\>[1336I]\AgdaOperator{\AgdaFunction{either}}\AgdaSpace{}%
\AgdaSymbol{(}\AgdaFunction{nonε-chain}\AgdaSpace{}%
\AgdaBound{p}\AgdaSpace{}%
\AgdaBound{q}\AgdaSpace{}%
\AgdaBound{pmap}\AgdaSpace{}%
\AgdaBound{qmap}\AgdaSymbol{)}\<%
\\
\>[1336I][@{}l@{\AgdaIndent{0}}]%
\>[28]\AgdaOperator{\AgdaFunction{or}}\AgdaSpace{}%
\AgdaSymbol{(}\AgdaFunction{εchain}\AgdaSpace{}%
\AgdaBound{p}\AgdaSpace{}%
\AgdaBound{q}\AgdaSpace{}%
\AgdaBound{pmap}\AgdaSpace{}%
\AgdaBound{qmap}\AgdaSymbol{)}\<%
\\
%
\\[\AgdaEmptyExtraSkip]%
%
\>[2]\AgdaFunction{pchain}\AgdaSpace{}%
\AgdaSymbol{:}\AgdaSpace{}%
\AgdaFunction{PremisechainParser}\<%
\\
%
\>[2]\AgdaFunction{pchain}\AgdaSpace{}%
\AgdaBound{p}\AgdaSpace{}%
\AgdaBound{q}\AgdaSpace{}%
\AgdaBound{pmap}\AgdaSpace{}%
\AgdaBound{qmap}\AgdaSpace{}%
\AgdaSymbol{=}%
\>[1354I]\AgdaOperator{\AgdaFunction{either}}\<%
\\
\>[1354I][@{}l@{\AgdaIndent{0}}]%
\>[27]\AgdaSymbol{(}\AgdaKeyword{do}\<%
\\
\>[27][@{}l@{\AgdaIndent{0}}]%
\>[30]\AgdaFunction{string}\AgdaSpace{}%
\AgdaString{"if:"}\<%
\\
%
\>[30]\AgdaFunction{ws+nl}\<%
\\
%
\>[30]\AgdaFunction{chain}\AgdaSpace{}%
\AgdaBound{p}\AgdaSpace{}%
\AgdaBound{q}\AgdaSpace{}%
\AgdaBound{pmap}\AgdaSpace{}%
\AgdaBound{qmap}\AgdaSymbol{)}\<%
\\
\>[.][@{}l@{}]\<[1354I]%
\>[25]\AgdaOperator{\AgdaFunction{or}}\<%
\\
\>[25][@{}l@{\AgdaIndent{0}}]%
\>[27]\AgdaSymbol{(}\AgdaFunction{εchain}\AgdaSpace{}%
\AgdaBound{p}\AgdaSpace{}%
\AgdaBound{q}\AgdaSpace{}%
\AgdaBound{pmap}\AgdaSpace{}%
\AgdaBound{qmap}\AgdaSymbol{)}\<%
\\
\>[0]\AgdaKeyword{open}\AgdaSpace{}%
\AgdaModule{PremisechainParser}\<%
\\
%
\\[\AgdaEmptyExtraSkip]%
\>[0]\AgdaKeyword{module}\AgdaSpace{}%
\AgdaModule{SpecfileParser}\AgdaSpace{}%
\AgdaKeyword{where}\<%
\\
%
\\[\AgdaEmptyExtraSkip]%
\>[0][@{}l@{\AgdaIndent{0}}]%
\>[2]\AgdaKeyword{open}\AgdaSpace{}%
\AgdaKeyword{import}\AgdaSpace{}%
\AgdaModule{TypeChecker}\AgdaSpace{}%
\AgdaKeyword{using}\AgdaSpace{}%
\AgdaSymbol{(}\AgdaDatatype{RuleSet}\AgdaSymbol{;}\AgdaSpace{}%
\AgdaInductiveConstructor{rs}\AgdaSymbol{)}\<%
\\
%
\>[2]\AgdaKeyword{open}\AgdaSpace{}%
\AgdaKeyword{import}\AgdaSpace{}%
\AgdaModule{Rules}\AgdaSpace{}%
\AgdaKeyword{using}\AgdaSpace{}%
\AgdaSymbol{(}\AgdaRecord{UnivRule}\AgdaSymbol{;}\AgdaSpace{}%
\AgdaRecord{TypeRule}\AgdaSymbol{;}\AgdaSpace{}%
\AgdaRecord{∋rule}\AgdaSymbol{;}\AgdaSpace{}%
\AgdaRecord{ElimRule}\AgdaSymbol{;}\AgdaSpace{}%
\AgdaInductiveConstructor{ε}\AgdaSymbol{;}\AgdaSpace{}%
\AgdaOperator{\AgdaFunction{\AgdaUnderscore{}placeless}}\AgdaSymbol{)}\<%
\\
%
\>[2]\AgdaKeyword{open}\AgdaSpace{}%
\AgdaKeyword{import}\AgdaSpace{}%
\AgdaModule{Expression}\AgdaSpace{}%
\AgdaKeyword{using}\AgdaSpace{}%
\AgdaSymbol{(}\AgdaDatatype{Con}\AgdaSymbol{;}\AgdaSpace{}%
\AgdaInductiveConstructor{`}\AgdaSymbol{)}\AgdaSpace{}%
\AgdaKeyword{renaming}\AgdaSpace{}%
\AgdaSymbol{(}\AgdaFunction{map}\AgdaSpace{}%
\AgdaSymbol{to}\AgdaSpace{}%
\AgdaFunction{emap}\AgdaSymbol{)}\<%
\\
%
\>[2]\AgdaKeyword{open}\AgdaSpace{}%
\AgdaKeyword{import}\AgdaSpace{}%
\AgdaModule{Semantics}\AgdaSpace{}%
\AgdaKeyword{using}\AgdaSpace{}%
\AgdaSymbol{(}\AgdaRecord{β-rule}\AgdaSymbol{)}\<%
\\
%
\>[2]\AgdaKeyword{open}\AgdaSpace{}%
\AgdaKeyword{import}\AgdaSpace{}%
\AgdaModule{Pattern}\AgdaSpace{}%
\AgdaKeyword{using}\AgdaSpace{}%
\AgdaSymbol{(}\AgdaOperator{\AgdaFunction{←\AgdaUnderscore{}}}\AgdaSymbol{)}\<%
\\
%
\>[2]\AgdaKeyword{open}\AgdaSpace{}%
\AgdaKeyword{import}\AgdaSpace{}%
\AgdaModule{Data.Product}\AgdaSpace{}%
\AgdaKeyword{using}\AgdaSpace{}%
\AgdaSymbol{(}\AgdaField{proj₁}\AgdaSymbol{)}\<%
\\
%
\>[2]\AgdaKeyword{open}\AgdaSpace{}%
\AgdaModule{parsermonad}\<%
\\
%
\\[\AgdaEmptyExtraSkip]%
%
\>[2]\AgdaFunction{tester}\AgdaSpace{}%
\AgdaSymbol{:}\AgdaSpace{}%
\AgdaRecord{ElimRule}\<%
\\
%
\>[2]\AgdaField{ElimRule.targetPat}\AgdaSpace{}%
\AgdaFunction{tester}\AgdaSpace{}%
\AgdaSymbol{=}\AgdaSpace{}%
\AgdaInductiveConstructor{`}\AgdaSpace{}%
\AgdaString{""}\<%
\\
%
\>[2]\AgdaField{ElimRule.eliminator}\AgdaSpace{}%
\AgdaFunction{tester}\AgdaSpace{}%
\AgdaSymbol{=}\AgdaSpace{}%
\AgdaInductiveConstructor{`}\AgdaSpace{}%
\AgdaString{""}\<%
\\
%
\>[2]\AgdaField{ElimRule.premises}\AgdaSpace{}%
\AgdaFunction{tester}\AgdaSpace{}%
\AgdaSymbol{=}\AgdaSpace{}%
\AgdaSymbol{(}\AgdaInductiveConstructor{`}\AgdaSpace{}%
\AgdaString{""}\AgdaSymbol{)}\AgdaSpace{}%
\AgdaOperator{\AgdaInductiveConstructor{,}}\AgdaSpace{}%
\AgdaSymbol{(}\AgdaInductiveConstructor{Rules.ε}\AgdaSpace{}%
\AgdaSymbol{((}\AgdaInductiveConstructor{`}\AgdaSpace{}%
\AgdaString{""}\AgdaSymbol{)}\AgdaSpace{}%
\AgdaOperator{\AgdaFunction{Rules.placeless}}\AgdaSymbol{))}\<%
\\
%
\>[2]\AgdaField{ElimRule.output}\AgdaSpace{}%
\AgdaFunction{tester}\AgdaSpace{}%
\AgdaSymbol{=}\AgdaSpace{}%
\AgdaInductiveConstructor{`}\AgdaSpace{}%
\AgdaString{""}\<%
\\
%
\\[\AgdaEmptyExtraSkip]%
%
\>[2]\AgdaFunction{β}\AgdaSpace{}%
\AgdaSymbol{:}\AgdaSpace{}%
\AgdaSymbol{(}\AgdaBound{t}\AgdaSpace{}%
\AgdaSymbol{:}\AgdaSpace{}%
\AgdaDatatype{Pattern}\AgdaSpace{}%
\AgdaNumber{0}\AgdaSymbol{)}\AgdaSpace{}%
\AgdaSymbol{→}\AgdaSpace{}%
\AgdaSymbol{(}\AgdaBound{er}\AgdaSpace{}%
\AgdaSymbol{:}\AgdaSpace{}%
\AgdaRecord{ElimRule}\AgdaSymbol{)}\AgdaSpace{}%
\AgdaSymbol{→}\AgdaSpace{}%
\AgdaFunction{SVarMap}\AgdaSpace{}%
\AgdaSymbol{(}\AgdaBound{t}\AgdaSpace{}%
\AgdaOperator{\AgdaInductiveConstructor{∙}}\AgdaSpace{}%
\AgdaField{ElimRule.targetPat}\AgdaSpace{}%
\AgdaBound{er}\AgdaSpace{}%
\AgdaOperator{\AgdaInductiveConstructor{∙}}\AgdaSpace{}%
\AgdaField{ElimRule.eliminator}\AgdaSpace{}%
\AgdaBound{er}\AgdaSymbol{)}\AgdaSpace{}%
\AgdaSymbol{→}\AgdaSpace{}%
\AgdaFunction{Parser}\AgdaSpace{}%
\AgdaRecord{β-rule}\<%
\\
%
\>[2]\AgdaFunction{β}\AgdaSpace{}%
\AgdaBound{t}\AgdaSpace{}%
\AgdaBound{er}%
\>[1449I]\AgdaBound{svmap}\AgdaSpace{}%
\AgdaSymbol{=}\AgdaSpace{}%
\AgdaKeyword{do}\<%
\\
\>[1449I][@{}l@{\AgdaIndent{0}}]%
\>[12]\AgdaFunction{string}\AgdaSpace{}%
\AgdaString{"reduces-to:"}\<%
\\
%
\>[12]\AgdaFunction{whitespace}\<%
\\
%
\>[12]\AgdaBound{redTerm}\AgdaSpace{}%
\AgdaOperator{\AgdaFunction{←}}\AgdaSpace{}%
\AgdaFunction{econst}\AgdaSpace{}%
\AgdaSymbol{(}\AgdaBound{t}\AgdaSpace{}%
\AgdaOperator{\AgdaInductiveConstructor{∙}}\AgdaSpace{}%
\AgdaField{targetPat}\AgdaSpace{}%
\AgdaBound{er}\AgdaSpace{}%
\AgdaOperator{\AgdaInductiveConstructor{∙}}\AgdaSpace{}%
\AgdaField{eliminator}\AgdaSpace{}%
\AgdaBound{er}\AgdaSymbol{)}\AgdaSpace{}%
\AgdaNumber{0}\AgdaSpace{}%
\AgdaFunction{empty}\AgdaSpace{}%
\AgdaBound{svmap}\<%
\\
%
\>[12]\AgdaFunction{return}%
\>[1465I]\AgdaSymbol{(}\AgdaKeyword{record}\<%
\\
\>[1465I][@{}l@{\AgdaIndent{0}}]%
\>[20]\AgdaSymbol{\{}\AgdaSpace{}%
\AgdaField{target}\AgdaSpace{}%
\AgdaSymbol{=}\AgdaSpace{}%
\AgdaBound{t}\<%
\\
%
\>[20]\AgdaSymbol{;}\AgdaSpace{}%
\AgdaField{erule}\AgdaSpace{}%
\AgdaSymbol{=}\AgdaSpace{}%
\AgdaBound{er}\<%
\\
%
\>[20]\AgdaSymbol{;}\AgdaSpace{}%
\AgdaField{redTerm}\AgdaSpace{}%
\AgdaSymbol{=}\AgdaSpace{}%
\AgdaBound{redTerm}\<%
\\
%
\>[20]\AgdaSymbol{\})}\<%
\\
\>[1449I][@{}l@{\AgdaIndent{0}}]%
\>[11]\AgdaKeyword{where}\AgdaSpace{}%
\AgdaKeyword{open}\AgdaSpace{}%
\AgdaModule{ElimRule}\<%
\\
%
\\[\AgdaEmptyExtraSkip]%
%
\>[2]\AgdaFunction{thingy'}\AgdaSpace{}%
\AgdaSymbol{:}\AgdaSpace{}%
\AgdaPrimitiveType{Set}\<%
\\
%
\>[2]\AgdaFunction{thingy'}\AgdaSpace{}%
\AgdaSymbol{=}\AgdaSpace{}%
\AgdaFunction{Σ[}\AgdaSpace{}%
\AgdaBound{e}\AgdaSpace{}%
\AgdaFunction{∈}\AgdaSpace{}%
\AgdaRecord{ElimRule}\AgdaSpace{}%
\AgdaFunction{]}\AgdaSpace{}%
\AgdaSymbol{(}\AgdaFunction{SVarMap}\AgdaSpace{}%
\AgdaSymbol{(}\AgdaField{ElimRule.targetPat}\AgdaSpace{}%
\AgdaBound{e}\AgdaSymbol{)}\AgdaOperator{\AgdaFunction{×}}\AgdaSpace{}%
\AgdaFunction{SVarMap}\AgdaSpace{}%
\AgdaSymbol{(}\AgdaField{ElimRule.eliminator}\AgdaSpace{}%
\AgdaBound{e}\AgdaSymbol{))}\<%
\\
%
\\[\AgdaEmptyExtraSkip]%
%
\>[2]\AgdaFunction{thingy}\AgdaSpace{}%
\AgdaSymbol{:}\AgdaSpace{}%
\AgdaPrimitiveType{Set}\<%
\\
%
\>[2]\AgdaFunction{thingy}\AgdaSpace{}%
\AgdaSymbol{=}\AgdaSpace{}%
\AgdaDatatype{List}\AgdaSpace{}%
\AgdaFunction{thingy'}\<%
\\
%
\\[\AgdaEmptyExtraSkip]%
%
\>[2]\AgdaFunction{[β]}\AgdaSpace{}%
\AgdaSymbol{:}\AgdaSpace{}%
\AgdaSymbol{(}\AgdaBound{t}\AgdaSpace{}%
\AgdaSymbol{:}\AgdaSpace{}%
\AgdaDatatype{Pattern}\AgdaSpace{}%
\AgdaNumber{0}\AgdaSymbol{)}\AgdaSpace{}%
\AgdaSymbol{→}\AgdaSpace{}%
\AgdaFunction{SVarMap}\AgdaSpace{}%
\AgdaBound{t}\AgdaSpace{}%
\AgdaSymbol{→}\AgdaSpace{}%
\AgdaFunction{thingy}\AgdaSpace{}%
\AgdaSymbol{→}\AgdaSpace{}%
\AgdaFunction{Parser}\AgdaSpace{}%
\AgdaSymbol{(}\AgdaDatatype{List}\AgdaSpace{}%
\AgdaRecord{β-rule}\AgdaSymbol{)}\<%
\\
%
\>[2]\AgdaFunction{[β]}\AgdaSpace{}%
\AgdaSymbol{\AgdaUnderscore{}}\AgdaSpace{}%
\AgdaSymbol{\AgdaUnderscore{}}\AgdaSpace{}%
\AgdaInductiveConstructor{[]}\AgdaSpace{}%
\AgdaSymbol{=}\AgdaSpace{}%
\AgdaFunction{return}\AgdaSpace{}%
\AgdaInductiveConstructor{[]}\<%
\\
%
\>[2]\AgdaFunction{[β]}\AgdaSpace{}%
\AgdaBound{t}\AgdaSpace{}%
\AgdaBound{svt}\AgdaSpace{}%
\AgdaSymbol{((}\AgdaBound{er}\AgdaSpace{}%
\AgdaOperator{\AgdaInductiveConstructor{,}}\AgdaSpace{}%
\AgdaBound{svty}\AgdaSpace{}%
\AgdaOperator{\AgdaInductiveConstructor{,}}\AgdaSpace{}%
\AgdaBound{sve}\AgdaSymbol{)}\AgdaSpace{}%
\AgdaOperator{\AgdaInductiveConstructor{∷}}\AgdaSpace{}%
\AgdaBound{xs}\AgdaSymbol{)}\<%
\\
\>[2][@{}l@{\AgdaIndent{0}}]%
\>[4]\AgdaSymbol{=}%
\>[1525I]\AgdaKeyword{do}\<%
\\
\>[1525I][@{}l@{\AgdaIndent{0}}]%
\>[8]\AgdaBound{r}\AgdaSpace{}%
\AgdaOperator{\AgdaFunction{←}}\AgdaSpace{}%
\AgdaFunction{β}\AgdaSpace{}%
\AgdaBound{t}\AgdaSpace{}%
\AgdaBound{er}%
\>[1530I]\AgdaSymbol{((}\AgdaFunction{union}\<%
\\
\>[1530I][@{}l@{\AgdaIndent{0}}]%
\>[22]\AgdaSymbol{(}\AgdaFunction{map}\AgdaSpace{}%
\AgdaSymbol{(λ}\AgdaSpace{}%
\AgdaSymbol{\{(}\AgdaBound{δ}\AgdaSpace{}%
\AgdaOperator{\AgdaInductiveConstructor{,}}\AgdaSpace{}%
\AgdaBound{v}\AgdaSymbol{)}\AgdaSpace{}%
\AgdaSymbol{→}\AgdaSpace{}%
\AgdaSymbol{(}\AgdaBound{δ}\AgdaSpace{}%
\AgdaOperator{\AgdaInductiveConstructor{,}}\AgdaSpace{}%
\AgdaSymbol{(}\AgdaBound{v}\AgdaSpace{}%
\AgdaOperator{\AgdaInductiveConstructor{∙}}\AgdaSymbol{))\})}\AgdaSpace{}%
\AgdaBound{svt}\AgdaSymbol{)}\<%
\\
%
\>[22]\AgdaSymbol{(}\AgdaFunction{union}\<%
\\
\>[22][@{}l@{\AgdaIndent{0}}]%
\>[24]\AgdaSymbol{(}\AgdaFunction{map}\AgdaSpace{}%
\AgdaSymbol{(λ}\AgdaSpace{}%
\AgdaSymbol{(}\AgdaBound{δ}\AgdaSpace{}%
\AgdaOperator{\AgdaInductiveConstructor{,}}\AgdaSpace{}%
\AgdaBound{v}\AgdaSymbol{)}\AgdaSpace{}%
\AgdaSymbol{→}\AgdaSpace{}%
\AgdaSymbol{(}\AgdaBound{δ}\AgdaSpace{}%
\AgdaOperator{\AgdaInductiveConstructor{,}}\AgdaSpace{}%
\AgdaSymbol{(}\AgdaOperator{\AgdaInductiveConstructor{∙}}\AgdaSpace{}%
\AgdaSymbol{(}\AgdaBound{v}\AgdaSpace{}%
\AgdaOperator{\AgdaInductiveConstructor{∙}}\AgdaSymbol{))))}\AgdaSpace{}%
\AgdaBound{svty}\AgdaSymbol{)}\<%
\\
%
\>[24]\AgdaSymbol{(}\AgdaFunction{map}\AgdaSpace{}%
\AgdaSymbol{(λ}\AgdaSpace{}%
\AgdaSymbol{\{(}\AgdaBound{δ}\AgdaSpace{}%
\AgdaOperator{\AgdaInductiveConstructor{,}}\AgdaSpace{}%
\AgdaBound{v}\AgdaSymbol{)}\AgdaSpace{}%
\AgdaSymbol{→}\AgdaSpace{}%
\AgdaSymbol{(}\AgdaBound{δ}\AgdaSpace{}%
\AgdaOperator{\AgdaInductiveConstructor{,}}\AgdaSpace{}%
\AgdaSymbol{(}\AgdaOperator{\AgdaInductiveConstructor{∙}}\AgdaSpace{}%
\AgdaSymbol{(}\AgdaOperator{\AgdaInductiveConstructor{∙}}\AgdaSpace{}%
\AgdaBound{v}\AgdaSymbol{)))\})}\AgdaSpace{}%
\AgdaBound{sve}\AgdaSymbol{))))}\<%
\\
%
\>[8]\AgdaFunction{ws+nl}\<%
\\
%
\>[8]\AgdaBound{rls}\AgdaSpace{}%
\AgdaOperator{\AgdaFunction{←}}\AgdaSpace{}%
\AgdaFunction{[β]}\AgdaSpace{}%
\AgdaBound{t}\AgdaSpace{}%
\AgdaBound{svt}\AgdaSpace{}%
\AgdaBound{xs}\<%
\\
%
\>[8]\AgdaFunction{return}\AgdaSpace{}%
\AgdaSymbol{(}\AgdaBound{r}\AgdaSpace{}%
\AgdaOperator{\AgdaInductiveConstructor{∷}}\AgdaSpace{}%
\AgdaBound{rls}\AgdaSymbol{)}\<%
\\
%
\\[\AgdaEmptyExtraSkip]%
%
\>[2]\AgdaFunction{∋}\AgdaSpace{}%
\AgdaSymbol{:}\AgdaSpace{}%
\AgdaSymbol{(}\AgdaBound{ty}\AgdaSpace{}%
\AgdaBound{tm}\AgdaSpace{}%
\AgdaSymbol{:}\AgdaSpace{}%
\AgdaDatatype{Pattern}\AgdaSpace{}%
\AgdaNumber{0}\AgdaSymbol{)}\AgdaSpace{}%
\AgdaSymbol{→}\AgdaSpace{}%
\AgdaFunction{SVarMap}\AgdaSpace{}%
\AgdaBound{ty}\AgdaSpace{}%
\AgdaSymbol{→}\AgdaSpace{}%
\AgdaFunction{SVarMap}\AgdaSpace{}%
\AgdaBound{tm}\AgdaSpace{}%
\AgdaSymbol{→}\AgdaSpace{}%
\AgdaFunction{Parser}\AgdaSpace{}%
\AgdaRecord{∋rule}\<%
\\
%
\>[2]\AgdaFunction{∋}\AgdaSpace{}%
\AgdaBound{ty}\AgdaSpace{}%
\AgdaBound{tm}\AgdaSpace{}%
\AgdaBound{tyvars}%
\>[1589I]\AgdaBound{tmvars}\AgdaSpace{}%
\AgdaSymbol{=}\AgdaSpace{}%
\AgdaKeyword{do}\<%
\\
\>[.][@{}l@{}]\<[1589I]%
\>[17]\AgdaSymbol{(}\AgdaBound{p'}\AgdaSpace{}%
\AgdaOperator{\AgdaInductiveConstructor{,}}\AgdaSpace{}%
\AgdaBound{p'm}\AgdaSpace{}%
\AgdaOperator{\AgdaInductiveConstructor{,}}\AgdaSpace{}%
\AgdaBound{pc}\AgdaSymbol{)}\AgdaSpace{}%
\AgdaOperator{\AgdaFunction{←}}\AgdaSpace{}%
\AgdaFunction{pchain}\AgdaSpace{}%
\AgdaBound{ty}\AgdaSpace{}%
\AgdaBound{tm}\AgdaSpace{}%
\AgdaBound{tyvars}\AgdaSpace{}%
\AgdaBound{tmvars}\<%
\\
%
\>[17]\AgdaFunction{return}\AgdaSpace{}%
\AgdaSymbol{(}\AgdaKeyword{record}\AgdaSpace{}%
\AgdaSymbol{\{}\AgdaSpace{}%
\AgdaField{subject}\AgdaSpace{}%
\AgdaSymbol{=}\AgdaSpace{}%
\AgdaBound{tm}\AgdaSpace{}%
\AgdaSymbol{;}\AgdaSpace{}%
\AgdaField{input}\AgdaSpace{}%
\AgdaSymbol{=}\AgdaSpace{}%
\AgdaBound{ty}\AgdaSpace{}%
\AgdaSymbol{;}\AgdaSpace{}%
\AgdaField{premises}\AgdaSpace{}%
\AgdaSymbol{=}\AgdaSpace{}%
\AgdaSymbol{(}\AgdaBound{p'}\AgdaSpace{}%
\AgdaOperator{\AgdaInductiveConstructor{,}}\AgdaSpace{}%
\AgdaBound{pc}\AgdaSymbol{)}\AgdaSpace{}%
\AgdaSymbol{\})}\<%
\\
%
\\[\AgdaEmptyExtraSkip]%
%
\\[\AgdaEmptyExtraSkip]%
%
\>[2]\AgdaFunction{value}\AgdaSpace{}%
\AgdaSymbol{:}\AgdaSpace{}%
\AgdaSymbol{(}\AgdaBound{tty}\AgdaSpace{}%
\AgdaSymbol{:}\AgdaSpace{}%
\AgdaDatatype{Pattern}\AgdaSpace{}%
\AgdaNumber{0}\AgdaSymbol{)}\AgdaSpace{}%
\AgdaSymbol{→}\AgdaSpace{}%
\AgdaFunction{SVarMap}\AgdaSpace{}%
\AgdaBound{tty}\AgdaSpace{}%
\AgdaSymbol{→}\AgdaSpace{}%
\AgdaFunction{thingy}\AgdaSpace{}%
\AgdaSymbol{→}\AgdaSpace{}%
\AgdaFunction{Parser}\AgdaSpace{}%
\AgdaSymbol{(}\AgdaRecord{∋rule}\AgdaSpace{}%
\AgdaOperator{\AgdaFunction{×}}\AgdaSpace{}%
\AgdaDatatype{List}\AgdaSpace{}%
\AgdaSymbol{(}\AgdaRecord{β-rule}\AgdaSymbol{))}\<%
\\
%
\>[2]\AgdaFunction{value}\AgdaSpace{}%
\AgdaBound{tty}\AgdaSpace{}%
\AgdaBound{ttymap}%
\>[1636I]\AgdaBound{els}\AgdaSpace{}%
\AgdaSymbol{=}\AgdaSpace{}%
\AgdaKeyword{do}\<%
\\
\>[1636I][@{}l@{\AgdaIndent{0}}]%
\>[20]\AgdaFunction{string}\AgdaSpace{}%
\AgdaString{"value:"}\<%
\\
%
\>[20]\AgdaSymbol{(}\AgdaBound{c}\AgdaSpace{}%
\AgdaOperator{\AgdaInductiveConstructor{,}}\AgdaSpace{}%
\AgdaBound{m}\AgdaSymbol{)}\AgdaSpace{}%
\AgdaOperator{\AgdaFunction{←}}\AgdaSpace{}%
\AgdaFunction{ws-tolerant}\AgdaSpace{}%
\AgdaFunction{closed-pattern}\<%
\\
%
\>[20]\AgdaFunction{ws+nl}\<%
\\
%
\>[20]\AgdaBound{∋r}\AgdaSpace{}%
\AgdaOperator{\AgdaFunction{←}}\AgdaSpace{}%
\AgdaFunction{∋}\AgdaSpace{}%
\AgdaBound{tty}\AgdaSpace{}%
\AgdaBound{c}\AgdaSpace{}%
\AgdaBound{ttymap}\AgdaSpace{}%
\AgdaBound{m}\<%
\\
%
\>[20]\AgdaFunction{ws+nl}\<%
\\
%
\>[20]\AgdaBound{βrls}\AgdaSpace{}%
\AgdaOperator{\AgdaFunction{←}}\AgdaSpace{}%
\AgdaFunction{[β]}\AgdaSpace{}%
\AgdaBound{c}\AgdaSpace{}%
\AgdaBound{m}\AgdaSpace{}%
\AgdaBound{els}\<%
\\
%
\>[20]\AgdaFunction{return}\AgdaSpace{}%
\AgdaSymbol{(}\AgdaBound{∋r}\AgdaSpace{}%
\AgdaOperator{\AgdaInductiveConstructor{,}}\AgdaSpace{}%
\AgdaBound{βrls}\AgdaSymbol{)}\<%
\\
%
\\[\AgdaEmptyExtraSkip]%
%
\>[2]\AgdaFunction{values}\AgdaSpace{}%
\AgdaSymbol{:}\AgdaSpace{}%
\AgdaSymbol{(}\AgdaBound{tty}\AgdaSpace{}%
\AgdaSymbol{:}\AgdaSpace{}%
\AgdaDatatype{Pattern}\AgdaSpace{}%
\AgdaNumber{0}\AgdaSymbol{)}\AgdaSpace{}%
\AgdaSymbol{→}\AgdaSpace{}%
\AgdaFunction{SVarMap}\AgdaSpace{}%
\AgdaBound{tty}\AgdaSpace{}%
\AgdaSymbol{→}\AgdaSpace{}%
\AgdaFunction{thingy}\AgdaSpace{}%
\AgdaSymbol{→}\AgdaSpace{}%
\AgdaFunction{Parser}\AgdaSpace{}%
\AgdaSymbol{(}\AgdaDatatype{List}\AgdaSpace{}%
\AgdaRecord{∋rule}\AgdaSpace{}%
\AgdaOperator{\AgdaFunction{×}}\AgdaSpace{}%
\AgdaDatatype{List}\AgdaSpace{}%
\AgdaSymbol{(}\AgdaRecord{β-rule}\AgdaSymbol{))}\<%
\\
%
\>[2]\AgdaFunction{values}\AgdaSpace{}%
\AgdaBound{tty}\AgdaSpace{}%
\AgdaBound{ttymap}\AgdaSpace{}%
\AgdaBound{els}\AgdaSpace{}%
\AgdaSymbol{=}%
\>[1680I]\AgdaKeyword{do}\<%
\\
\>[1680I][@{}l@{\AgdaIndent{0}}]%
\>[28]\AgdaBound{rst}\AgdaSpace{}%
\AgdaOperator{\AgdaFunction{←}}%
\>[1682I]\AgdaFunction{wsnl-tolerant}\AgdaSpace{}%
\AgdaSymbol{(}\AgdaFunction{value}\AgdaSpace{}%
\AgdaBound{tty}\AgdaSpace{}%
\AgdaBound{ttymap}\AgdaSpace{}%
\AgdaBound{els}\AgdaSymbol{)}\<%
\\
\>[1682I][@{}l@{\AgdaIndent{0}}]%
\>[38]\AgdaOperator{\AgdaFunction{*[}}\AgdaSpace{}%
\AgdaSymbol{(λ}\AgdaSpace{}%
\AgdaSymbol{\{(}\AgdaBound{∋r}\AgdaSpace{}%
\AgdaOperator{\AgdaInductiveConstructor{,}}\AgdaSpace{}%
\AgdaBound{newβrs}\AgdaSymbol{)}\AgdaSpace{}%
\AgdaSymbol{(}\AgdaBound{∋rs}\AgdaSpace{}%
\AgdaOperator{\AgdaInductiveConstructor{,}}\AgdaSpace{}%
\AgdaBound{βrs}\AgdaSymbol{)}\AgdaSpace{}%
\AgdaSymbol{→}\AgdaSpace{}%
\AgdaSymbol{(}\AgdaBound{∋r}\AgdaSpace{}%
\AgdaOperator{\AgdaInductiveConstructor{∷}}\AgdaSpace{}%
\AgdaBound{∋rs}\AgdaSymbol{)}\AgdaSpace{}%
\AgdaOperator{\AgdaInductiveConstructor{,}}\AgdaSpace{}%
\AgdaBound{newβrs}\AgdaSpace{}%
\AgdaOperator{\AgdaFunction{++}}\AgdaSpace{}%
\AgdaBound{βrs}\AgdaSymbol{\})}\<%
\\
%
\>[38]\AgdaOperator{\AgdaFunction{,}}\AgdaSpace{}%
\AgdaSymbol{(}\AgdaInductiveConstructor{[]}\AgdaSpace{}%
\AgdaOperator{\AgdaInductiveConstructor{,}}\AgdaSpace{}%
\AgdaInductiveConstructor{[]}\AgdaSymbol{)}\AgdaSpace{}%
\AgdaOperator{\AgdaFunction{]}}\<%
\\
%
\>[28]\AgdaFunction{return}\AgdaSpace{}%
\AgdaBound{rst}\<%
\\
%
\\[\AgdaEmptyExtraSkip]%
%
\>[2]\AgdaFunction{eliminator}\AgdaSpace{}%
\AgdaSymbol{:}\AgdaSpace{}%
\AgdaSymbol{(}\AgdaBound{tty}\AgdaSpace{}%
\AgdaSymbol{:}\AgdaSpace{}%
\AgdaDatatype{Pattern}\AgdaSpace{}%
\AgdaNumber{0}\AgdaSymbol{)}\AgdaSpace{}%
\AgdaSymbol{→}\AgdaSpace{}%
\AgdaFunction{SVarMap}\AgdaSpace{}%
\AgdaBound{tty}\AgdaSpace{}%
\AgdaSymbol{→}\AgdaSpace{}%
\AgdaFunction{Parser}\AgdaSpace{}%
\AgdaFunction{thingy'}\<%
\\
%
\>[2]\AgdaFunction{eliminator}\AgdaSpace{}%
\AgdaBound{tty}\AgdaSpace{}%
\AgdaBound{ttymap}\AgdaSpace{}%
\AgdaSymbol{=}%
\>[1721I]\AgdaKeyword{do}\<%
\\
\>[1721I][@{}l@{\AgdaIndent{0}}]%
\>[28]\AgdaFunction{string}\AgdaSpace{}%
\AgdaString{"eliminated-by:"}\<%
\\
%
\>[28]\AgdaSymbol{(}\AgdaBound{e}\AgdaSpace{}%
\AgdaOperator{\AgdaInductiveConstructor{,}}\AgdaSpace{}%
\AgdaBound{m}\AgdaSymbol{)}\AgdaSpace{}%
\AgdaOperator{\AgdaFunction{←}}\AgdaSpace{}%
\AgdaFunction{ws-tolerant}\AgdaSpace{}%
\AgdaFunction{closed-pattern}\<%
\\
%
\>[28]\AgdaFunction{ws+nl}\<%
\\
%
\>[28]\AgdaSymbol{(}\AgdaBound{p'}\AgdaSpace{}%
\AgdaOperator{\AgdaInductiveConstructor{,}}\AgdaSpace{}%
\AgdaBound{p'm}\AgdaSpace{}%
\AgdaOperator{\AgdaInductiveConstructor{,}}\AgdaSpace{}%
\AgdaBound{pc}\AgdaSymbol{)}\AgdaSpace{}%
\AgdaOperator{\AgdaFunction{←}}\AgdaSpace{}%
\AgdaFunction{pchain}\AgdaSpace{}%
\AgdaBound{tty}\AgdaSpace{}%
\AgdaBound{e}\AgdaSpace{}%
\AgdaBound{ttymap}\AgdaSpace{}%
\AgdaBound{m}\<%
\\
%
\>[28]\AgdaFunction{ws+nl}\<%
\\
%
\>[28]\AgdaFunction{string}\AgdaSpace{}%
\AgdaString{"resulting-in-type:"}\<%
\\
%
\>[28]\AgdaBound{et}\AgdaSpace{}%
\AgdaOperator{\AgdaFunction{←}}\AgdaSpace{}%
\AgdaFunction{ws-tolerant}\AgdaSpace{}%
\AgdaSymbol{(}\AgdaFunction{econst}\AgdaSpace{}%
\AgdaBound{p'}\AgdaSpace{}%
\AgdaNumber{0}\AgdaSpace{}%
\AgdaFunction{empty}\AgdaSpace{}%
\AgdaBound{p'm}\AgdaSymbol{)}\<%
\\
%
\>[28]\AgdaFunction{return}%
\>[1746I]\AgdaSymbol{((}\AgdaKeyword{record}\<%
\\
\>[1746I][@{}l@{\AgdaIndent{0}}]%
\>[39]\AgdaSymbol{\{}\AgdaSpace{}%
\AgdaField{targetPat}\AgdaSpace{}%
\AgdaSymbol{=}\AgdaSpace{}%
\AgdaBound{tty}\AgdaSpace{}%
\AgdaSymbol{;}\AgdaSpace{}%
\AgdaField{eliminator}\AgdaSpace{}%
\AgdaSymbol{=}\AgdaSpace{}%
\AgdaBound{e}\AgdaSpace{}%
\AgdaSymbol{;}\AgdaSpace{}%
\AgdaField{premises}\AgdaSpace{}%
\AgdaSymbol{=}\AgdaSpace{}%
\AgdaBound{p'}\AgdaSpace{}%
\AgdaOperator{\AgdaInductiveConstructor{,}}\AgdaSpace{}%
\AgdaBound{pc}\AgdaSpace{}%
\AgdaSymbol{;}\AgdaSpace{}%
\AgdaField{output}\AgdaSpace{}%
\AgdaSymbol{=}\AgdaSpace{}%
\AgdaBound{et}\AgdaSpace{}%
\AgdaSymbol{\})}\AgdaSpace{}%
\AgdaOperator{\AgdaInductiveConstructor{,}}\AgdaSpace{}%
\AgdaSymbol{(}\AgdaBound{ttymap}\AgdaSpace{}%
\AgdaOperator{\AgdaInductiveConstructor{,}}\AgdaSpace{}%
\AgdaBound{m}\AgdaSymbol{))}\<%
\\
%
\\[\AgdaEmptyExtraSkip]%
%
\\[\AgdaEmptyExtraSkip]%
%
\>[2]\AgdaFunction{eliminators}\AgdaSpace{}%
\AgdaSymbol{:}\AgdaSpace{}%
\AgdaSymbol{(}\AgdaBound{tty}\AgdaSpace{}%
\AgdaSymbol{:}\AgdaSpace{}%
\AgdaDatatype{Pattern}\AgdaSpace{}%
\AgdaNumber{0}\AgdaSymbol{)}\AgdaSpace{}%
\AgdaSymbol{→}\AgdaSpace{}%
\AgdaFunction{SVarMap}\AgdaSpace{}%
\AgdaBound{tty}\AgdaSpace{}%
\AgdaSymbol{→}\AgdaSpace{}%
\AgdaFunction{Parser}\AgdaSpace{}%
\AgdaSymbol{(}\AgdaFunction{thingy}\AgdaSymbol{)}\<%
\\
%
\>[2]\AgdaFunction{eliminators}\AgdaSpace{}%
\AgdaBound{tty}\AgdaSpace{}%
\AgdaBound{ttymap}\AgdaSpace{}%
\AgdaSymbol{=}%
\>[1783I]\AgdaKeyword{do}\<%
\\
\>[1783I][@{}l@{\AgdaIndent{0}}]%
\>[29]\AgdaBound{rst}\AgdaSpace{}%
\AgdaOperator{\AgdaFunction{←}}\AgdaSpace{}%
\AgdaFunction{wsnl-tolerant}\AgdaSpace{}%
\AgdaSymbol{(}\AgdaFunction{eliminator}\AgdaSpace{}%
\AgdaBound{tty}\AgdaSpace{}%
\AgdaBound{ttymap}\AgdaSymbol{)}\AgdaSpace{}%
\AgdaOperator{\AgdaFunction{*[}}\AgdaSpace{}%
\AgdaOperator{\AgdaInductiveConstructor{\AgdaUnderscore{}∷\AgdaUnderscore{}}}\AgdaSpace{}%
\AgdaOperator{\AgdaFunction{,}}\AgdaSpace{}%
\AgdaInductiveConstructor{[]}\AgdaSpace{}%
\AgdaOperator{\AgdaFunction{]}}\<%
\\
%
\>[29]\AgdaFunction{return}\AgdaSpace{}%
\AgdaBound{rst}\<%
\\
%
\\[\AgdaEmptyExtraSkip]%
%
\>[2]\AgdaKeyword{open}\AgdaSpace{}%
\AgdaKeyword{import}\AgdaSpace{}%
\AgdaModule{Data.Unit}\AgdaSpace{}%
\AgdaKeyword{using}\AgdaSpace{}%
\AgdaSymbol{(}\AgdaRecord{⊤}\AgdaSymbol{;}\AgdaSpace{}%
\AgdaInductiveConstructor{tt}\AgdaSymbol{)}\<%
\\
%
\>[2]\AgdaFunction{eta}\AgdaSpace{}%
\AgdaSymbol{:}\AgdaSpace{}%
\AgdaFunction{Parser}\AgdaSpace{}%
\AgdaRecord{⊤}\<%
\\
%
\>[2]\AgdaFunction{eta}\AgdaSpace{}%
\AgdaSymbol{=}%
\>[1804I]\AgdaKeyword{do}\<%
\\
\>[1804I][@{}l@{\AgdaIndent{0}}]%
\>[10]\AgdaFunction{string}\AgdaSpace{}%
\AgdaString{"expanded-by:"}\<%
\\
%
\>[10]\AgdaFunction{ws-tolerant}\AgdaSpace{}%
\AgdaFunction{closed-pattern}\<%
\\
%
\>[10]\AgdaFunction{return}\AgdaSpace{}%
\AgdaInductiveConstructor{tt}\<%
\\
%
\\[\AgdaEmptyExtraSkip]%
%
\>[2]\AgdaFunction{type}\AgdaSpace{}%
\AgdaSymbol{:}\AgdaSpace{}%
\AgdaFunction{Parser}\AgdaSpace{}%
\AgdaDatatype{RuleSet}\<%
\\
%
\>[2]\AgdaFunction{type}\AgdaSpace{}%
\AgdaSymbol{=}%
\>[1812I]\AgdaKeyword{do}\<%
\\
\>[1812I][@{}l@{\AgdaIndent{0}}]%
\>[11]\AgdaFunction{string}\AgdaSpace{}%
\AgdaString{"type:"}\<%
\\
%
\>[11]\AgdaSymbol{(}\AgdaBound{ty}\AgdaSpace{}%
\AgdaOperator{\AgdaInductiveConstructor{,}}\AgdaSpace{}%
\AgdaBound{ty-map}\AgdaSymbol{)}\AgdaSpace{}%
\AgdaOperator{\AgdaFunction{←}}\AgdaSpace{}%
\AgdaFunction{ws-tolerant}\AgdaSpace{}%
\AgdaFunction{closed-pattern}\<%
\\
%
\>[11]\AgdaFunction{ws+nl}\<%
\\
%
\>[11]\AgdaSymbol{(}\AgdaBound{p'}\AgdaSpace{}%
\AgdaOperator{\AgdaInductiveConstructor{,}}\AgdaSpace{}%
\AgdaBound{p'm}\AgdaSpace{}%
\AgdaOperator{\AgdaInductiveConstructor{,}}\AgdaSpace{}%
\AgdaBound{pc}\AgdaSymbol{)}\AgdaSpace{}%
\AgdaOperator{\AgdaFunction{←}}\AgdaSpace{}%
\AgdaFunction{pchain}\AgdaSpace{}%
\AgdaSymbol{(}\AgdaInductiveConstructor{`}\AgdaSpace{}%
\AgdaString{"⊤"}\AgdaSymbol{)}\AgdaSpace{}%
\AgdaBound{ty}\AgdaSpace{}%
\AgdaFunction{empty}\AgdaSpace{}%
\AgdaBound{ty-map}\<%
\\
%
\>[11]\AgdaBound{tr}\AgdaSpace{}%
\AgdaOperator{\AgdaFunction{←}}\AgdaSpace{}%
\AgdaFunction{return}\AgdaSpace{}%
\AgdaSymbol{(}\AgdaKeyword{record}\AgdaSpace{}%
\AgdaSymbol{\{}\AgdaSpace{}%
\AgdaField{subject}\AgdaSpace{}%
\AgdaSymbol{=}\AgdaSpace{}%
\AgdaBound{ty}\AgdaSpace{}%
\AgdaSymbol{;}\AgdaSpace{}%
\AgdaField{premises}\AgdaSpace{}%
\AgdaSymbol{=}\AgdaSpace{}%
\AgdaSymbol{(}\AgdaBound{p'}\AgdaSpace{}%
\AgdaOperator{\AgdaInductiveConstructor{,}}\AgdaSpace{}%
\AgdaBound{pc}\AgdaSymbol{)}\AgdaSpace{}%
\AgdaSymbol{\}}\AgdaSpace{}%
\AgdaOperator{\AgdaInductiveConstructor{∷}}\AgdaSpace{}%
\AgdaInductiveConstructor{[]}\AgdaSymbol{)}\<%
\\
%
\>[11]\AgdaFunction{ws+nl}\<%
\\
%
\>[11]\AgdaBound{elim-rules}\AgdaSpace{}%
\AgdaOperator{\AgdaFunction{←}}\AgdaSpace{}%
\AgdaFunction{eliminators}\AgdaSpace{}%
\AgdaBound{ty}\AgdaSpace{}%
\AgdaBound{ty-map}\<%
\\
%
\>[11]\AgdaFunction{ws+nl}\<%
\\
%
\>[11]\AgdaSymbol{(}\AgdaBound{∋rs}\AgdaSpace{}%
\AgdaOperator{\AgdaInductiveConstructor{,}}\AgdaSpace{}%
\AgdaBound{βrs}\AgdaSymbol{)}\AgdaSpace{}%
\AgdaOperator{\AgdaFunction{←}}\AgdaSpace{}%
\AgdaFunction{values}\AgdaSpace{}%
\AgdaBound{ty}\AgdaSpace{}%
\AgdaBound{ty-map}\AgdaSpace{}%
\AgdaBound{elim-rules}\<%
\\
%
\>[11]\AgdaFunction{ws+nl}\<%
\\
%
\>[11]\AgdaFunction{optional}\AgdaSpace{}%
\AgdaFunction{eta}\<%
\\
%
\>[11]\AgdaFunction{return}\AgdaSpace{}%
\AgdaSymbol{(}\AgdaInductiveConstructor{rs}\AgdaSpace{}%
\AgdaBound{tr}\AgdaSpace{}%
\AgdaInductiveConstructor{[]}\AgdaSpace{}%
\AgdaBound{∋rs}\AgdaSpace{}%
\AgdaSymbol{(}\AgdaFunction{Data.List.map}\AgdaSpace{}%
\AgdaField{proj₁}\AgdaSpace{}%
\AgdaBound{elim-rules}\AgdaSymbol{)}\AgdaSpace{}%
\AgdaBound{βrs}\AgdaSpace{}%
\AgdaInductiveConstructor{[]}\AgdaSymbol{)}\<%
\\
%
\\[\AgdaEmptyExtraSkip]%
%
\>[2]\AgdaFunction{setType}\AgdaSpace{}%
\AgdaSymbol{:}\AgdaSpace{}%
\AgdaRecord{TypeRule}\<%
\\
%
\>[2]\AgdaField{TypeRule.subject}\AgdaSpace{}%
\AgdaFunction{setType}\AgdaSpace{}%
\AgdaSymbol{=}\AgdaSpace{}%
\AgdaInductiveConstructor{`}\AgdaSpace{}%
\AgdaString{"set"}\<%
\\
%
\>[2]\AgdaField{TypeRule.premises}\AgdaSpace{}%
\AgdaFunction{setType}\AgdaSpace{}%
\AgdaSymbol{=}\AgdaSpace{}%
\AgdaInductiveConstructor{`}\AgdaSpace{}%
\AgdaString{"⊤"}\AgdaSpace{}%
\AgdaOperator{\AgdaInductiveConstructor{,}}\AgdaSpace{}%
\AgdaInductiveConstructor{ε}\AgdaSpace{}%
\AgdaSymbol{(}\AgdaInductiveConstructor{`}\AgdaSpace{}%
\AgdaString{"set"}\AgdaSpace{}%
\AgdaOperator{\AgdaFunction{placeless}}\AgdaSymbol{)}\<%
\\
%
\\[\AgdaEmptyExtraSkip]%
%
\>[2]\AgdaFunction{setUniv}\AgdaSpace{}%
\AgdaSymbol{:}\AgdaSpace{}%
\AgdaRecord{UnivRule}\<%
\\
%
\>[2]\AgdaField{UnivRule.input}\AgdaSpace{}%
\AgdaFunction{setUniv}\AgdaSpace{}%
\AgdaSymbol{=}\AgdaSpace{}%
\AgdaInductiveConstructor{`}\AgdaSpace{}%
\AgdaString{"set"}\<%
\\
%
\>[2]\AgdaField{UnivRule.premises}\AgdaSpace{}%
\AgdaFunction{setUniv}\AgdaSpace{}%
\AgdaSymbol{=}\AgdaSpace{}%
\AgdaInductiveConstructor{`}\AgdaSpace{}%
\AgdaString{"set"}\AgdaSpace{}%
\AgdaOperator{\AgdaInductiveConstructor{,}}\AgdaSpace{}%
\AgdaSymbol{(}\AgdaInductiveConstructor{ε}\AgdaSpace{}%
\AgdaSymbol{(}\AgdaInductiveConstructor{`}\AgdaSpace{}%
\AgdaString{"⊤"}\AgdaSpace{}%
\AgdaOperator{\AgdaFunction{placeless}}\AgdaSymbol{))}\<%
\\
%
\\[\AgdaEmptyExtraSkip]%
%
\>[2]\AgdaFunction{parse-spec}\AgdaSpace{}%
\AgdaSymbol{:}\AgdaSpace{}%
\AgdaFunction{Parser}\AgdaSpace{}%
\AgdaDatatype{RuleSet}\<%
\\
%
\>[2]\AgdaFunction{parse-spec}\AgdaSpace{}%
\AgdaSymbol{=}%
\>[1901I]\AgdaKeyword{do}\<%
\\
\>[1901I][@{}l@{\AgdaIndent{0}}]%
\>[17]\AgdaSymbol{(}\AgdaInductiveConstructor{rs}\AgdaSpace{}%
\AgdaBound{ty}\AgdaSpace{}%
\AgdaBound{u}\AgdaSpace{}%
\AgdaBound{∋}\AgdaSpace{}%
\AgdaBound{e}\AgdaSpace{}%
\AgdaBound{β}\AgdaSpace{}%
\AgdaBound{η}\AgdaSymbol{)}\AgdaSpace{}%
\AgdaOperator{\AgdaFunction{←}}%
\>[1909I]\AgdaSymbol{(}\AgdaFunction{wsnl-tolerant}\AgdaSpace{}%
\AgdaFunction{type}\AgdaSymbol{)}\AgdaSpace{}%
\AgdaOperator{\AgdaFunction{⁺[}}\AgdaSpace{}%
\AgdaSymbol{(λ}\AgdaSpace{}%
\AgdaSymbol{\{(}\AgdaInductiveConstructor{rs}\AgdaSpace{}%
\AgdaBound{a}%
\>[72]\AgdaBound{b}%
\>[75]\AgdaBound{c}%
\>[78]\AgdaBound{d}%
\>[81]\AgdaBound{e}%
\>[84]\AgdaBound{f}\AgdaSymbol{)}\<%
\\
\>[.][@{}l@{}]\<[1909I]%
\>[37]\AgdaSymbol{(}\AgdaInductiveConstructor{rs}\AgdaSpace{}%
\AgdaBound{a'}\AgdaSpace{}%
\AgdaBound{b'}\AgdaSpace{}%
\AgdaBound{c'}\AgdaSpace{}%
\AgdaBound{d'}\AgdaSpace{}%
\AgdaBound{e'}\AgdaSpace{}%
\AgdaBound{f'}\AgdaSymbol{)}\AgdaSpace{}%
\AgdaSymbol{→}\<%
\\
%
\>[37]\AgdaInductiveConstructor{rs}\AgdaSpace{}%
\AgdaSymbol{(}\AgdaBound{a}\AgdaSpace{}%
\AgdaOperator{\AgdaFunction{++}}\AgdaSpace{}%
\AgdaBound{a'}\AgdaSymbol{)}\AgdaSpace{}%
\AgdaSymbol{(}\AgdaBound{b}\AgdaSpace{}%
\AgdaOperator{\AgdaFunction{++}}\AgdaSpace{}%
\AgdaBound{b'}\AgdaSymbol{)}\AgdaSpace{}%
\AgdaSymbol{(}\AgdaBound{c}\AgdaSpace{}%
\AgdaOperator{\AgdaFunction{++}}\AgdaSpace{}%
\AgdaBound{c'}\AgdaSymbol{)}\AgdaSpace{}%
\AgdaSymbol{(}\AgdaBound{d}\AgdaSpace{}%
\AgdaOperator{\AgdaFunction{++}}\AgdaSpace{}%
\AgdaBound{d'}\AgdaSymbol{)}\AgdaSpace{}%
\AgdaSymbol{(}\AgdaBound{e}\AgdaSpace{}%
\AgdaOperator{\AgdaFunction{++}}\AgdaSpace{}%
\AgdaBound{e'}\AgdaSymbol{)}\AgdaSpace{}%
\AgdaSymbol{(}\AgdaBound{f}\AgdaSpace{}%
\AgdaOperator{\AgdaFunction{++}}\AgdaSpace{}%
\AgdaBound{f'}\AgdaSymbol{)\})}\AgdaSpace{}%
\AgdaOperator{\AgdaFunction{,}}\AgdaSpace{}%
\AgdaInductiveConstructor{rs}\AgdaSpace{}%
\AgdaInductiveConstructor{[]}\AgdaSpace{}%
\AgdaInductiveConstructor{[]}\AgdaSpace{}%
\AgdaInductiveConstructor{[]}\AgdaSpace{}%
\AgdaInductiveConstructor{[]}\AgdaSpace{}%
\AgdaInductiveConstructor{[]}\AgdaSpace{}%
\AgdaInductiveConstructor{[]}\AgdaSpace{}%
\AgdaOperator{\AgdaFunction{]}}\<%
\\
%
\>[17]\AgdaFunction{return}\AgdaSpace{}%
\AgdaSymbol{(}\AgdaInductiveConstructor{rs}\AgdaSpace{}%
\AgdaSymbol{(}\AgdaFunction{setType}\AgdaSpace{}%
\AgdaOperator{\AgdaInductiveConstructor{∷}}\AgdaSpace{}%
\AgdaBound{ty}\AgdaSymbol{)}\AgdaSpace{}%
\AgdaSymbol{(}\AgdaFunction{setUniv}\AgdaSpace{}%
\AgdaOperator{\AgdaInductiveConstructor{∷}}\AgdaSpace{}%
\AgdaBound{u}\AgdaSymbol{)}\AgdaSpace{}%
\AgdaBound{∋}\AgdaSpace{}%
\AgdaBound{e}\AgdaSpace{}%
\AgdaBound{β}\AgdaSpace{}%
\AgdaBound{η}\AgdaSymbol{)}\<%
\\
\>[0]\AgdaKeyword{open}\AgdaSpace{}%
\AgdaModule{SpecfileParser}\<%
\end{code}
}

\hide{
\begin{code}%
\>[0]\AgdaSymbol{\{-\#}\AgdaSpace{}%
\AgdaKeyword{OPTIONS}\AgdaSpace{}%
\AgdaPragma{--rewriting}\AgdaSpace{}%
\AgdaSymbol{\#-\}}\<%
\\
\>[0]\AgdaKeyword{module}\AgdaSpace{}%
\AgdaModule{LanguageParser}\AgdaSpace{}%
\AgdaKeyword{where}\<%
\end{code}
}

\hide{
\begin{code}%
\>[0]\AgdaKeyword{open}\AgdaSpace{}%
\AgdaKeyword{import}\AgdaSpace{}%
\AgdaModule{CoreLanguage}\<%
\\
\>[0]\AgdaKeyword{open}\AgdaSpace{}%
\AgdaKeyword{import}\AgdaSpace{}%
\AgdaModule{Pattern}\AgdaSpace{}%
\AgdaKeyword{using}\AgdaSpace{}%
\AgdaSymbol{(}\AgdaDatatype{Pattern}\AgdaSymbol{;}\AgdaSpace{}%
\AgdaInductiveConstructor{`}\AgdaSymbol{;}\AgdaSpace{}%
\AgdaOperator{\AgdaInductiveConstructor{\AgdaUnderscore{}∙\AgdaUnderscore{}}}\AgdaSymbol{;}\AgdaSpace{}%
\AgdaInductiveConstructor{bind}\AgdaSymbol{;}\AgdaSpace{}%
\AgdaInductiveConstructor{place}\AgdaSymbol{;}\AgdaSpace{}%
\AgdaOperator{\AgdaFunction{\AgdaUnderscore{}⊗\AgdaUnderscore{}}}\AgdaSymbol{)}\<%
\\
\>[0]\AgdaKeyword{open}\AgdaSpace{}%
\AgdaKeyword{import}\AgdaSpace{}%
\AgdaModule{Data.String}\AgdaSpace{}%
\AgdaKeyword{using}\AgdaSpace{}%
\AgdaSymbol{(}\AgdaPostulate{String}\AgdaSymbol{)}\AgdaSpace{}%
\AgdaKeyword{renaming}\AgdaSpace{}%
\AgdaSymbol{(}\AgdaOperator{\AgdaFunction{\AgdaUnderscore{}++\AgdaUnderscore{}}}\AgdaSpace{}%
\AgdaSymbol{to}\AgdaSpace{}%
\AgdaOperator{\AgdaFunction{\AgdaUnderscore{}+\AgdaUnderscore{}}}\AgdaSymbol{)}\<%
\\
\>[0]\AgdaKeyword{open}\AgdaSpace{}%
\AgdaKeyword{import}\AgdaSpace{}%
\AgdaModule{Data.Product}\AgdaSpace{}%
\AgdaKeyword{using}\AgdaSpace{}%
\AgdaSymbol{(}\AgdaOperator{\AgdaFunction{\AgdaUnderscore{}×\AgdaUnderscore{}}}\AgdaSymbol{;}\AgdaSpace{}%
\AgdaOperator{\AgdaInductiveConstructor{\AgdaUnderscore{},\AgdaUnderscore{}}}\AgdaSymbol{)}\<%
\\
\>[0]\AgdaKeyword{open}\AgdaSpace{}%
\AgdaKeyword{import}\AgdaSpace{}%
\AgdaModule{Data.Sum}\AgdaSpace{}%
\AgdaKeyword{using}\AgdaSpace{}%
\AgdaSymbol{(}\AgdaInductiveConstructor{inj₁}\AgdaSymbol{;}\AgdaSpace{}%
\AgdaInductiveConstructor{inj₂}\AgdaSymbol{)}\<%
\\
\>[0]\AgdaKeyword{open}\AgdaSpace{}%
\AgdaKeyword{import}\AgdaSpace{}%
\AgdaModule{Data.String.Properties}\AgdaSpace{}%
\AgdaKeyword{using}\AgdaSpace{}%
\AgdaSymbol{(}\AgdaFunction{<-strictTotalOrder-≈}\AgdaSymbol{)}\<%
\\
\>[0]\AgdaKeyword{open}\AgdaSpace{}%
\AgdaKeyword{import}\AgdaSpace{}%
\AgdaModule{Data.Char}\AgdaSpace{}%
\AgdaKeyword{using}\AgdaSpace{}%
\AgdaSymbol{(}\AgdaPostulate{Char}\AgdaSymbol{;}\AgdaSpace{}%
\AgdaPrimitive{isAlpha}\AgdaSymbol{;}\AgdaSpace{}%
\AgdaPrimitive{isDigit}\AgdaSymbol{;}\AgdaSpace{}%
\AgdaPrimitive{isSpace}\AgdaSymbol{;}\AgdaSpace{}%
\AgdaOperator{\AgdaFunction{\AgdaUnderscore{}==\AgdaUnderscore{}}}\AgdaSymbol{)}\<%
\\
\>[0]\AgdaKeyword{open}\AgdaSpace{}%
\AgdaKeyword{import}\AgdaSpace{}%
\AgdaModule{Data.List}\AgdaSpace{}%
\AgdaKeyword{using}\AgdaSpace{}%
\AgdaSymbol{(}\AgdaDatatype{List}\AgdaSymbol{;}\AgdaSpace{}%
\AgdaOperator{\AgdaInductiveConstructor{\AgdaUnderscore{}∷\AgdaUnderscore{}}}\AgdaSymbol{;}\AgdaSpace{}%
\AgdaInductiveConstructor{[]}\AgdaSymbol{;}\AgdaSpace{}%
\AgdaFunction{any}\AgdaSymbol{;}\AgdaSpace{}%
\AgdaOperator{\AgdaFunction{\AgdaUnderscore{}++\AgdaUnderscore{}}}\AgdaSymbol{)}\AgdaSpace{}%
\AgdaKeyword{renaming}\AgdaSpace{}%
\AgdaSymbol{(}\AgdaFunction{map}\AgdaSpace{}%
\AgdaSymbol{to}\AgdaSpace{}%
\AgdaFunction{lmap}\AgdaSymbol{)}\<%
\\
\>[0]\AgdaKeyword{open}\AgdaSpace{}%
\AgdaKeyword{import}\AgdaSpace{}%
\AgdaModule{Data.Bool}\AgdaSpace{}%
\AgdaKeyword{using}\AgdaSpace{}%
\AgdaSymbol{(}\AgdaDatatype{Bool}\AgdaSymbol{;}\AgdaSpace{}%
\AgdaOperator{\AgdaFunction{\AgdaUnderscore{}∨\AgdaUnderscore{}}}\AgdaSymbol{;}\AgdaSpace{}%
\AgdaFunction{not}\AgdaSymbol{;}\AgdaSpace{}%
\AgdaOperator{\AgdaFunction{\AgdaUnderscore{}∧\AgdaUnderscore{}}}\AgdaSymbol{)}\<%
\\
\>[0]\AgdaKeyword{open}\AgdaSpace{}%
\AgdaKeyword{import}\AgdaSpace{}%
\AgdaModule{Data.Nat}\AgdaSpace{}%
\AgdaKeyword{using}\AgdaSpace{}%
\AgdaSymbol{(}\AgdaInductiveConstructor{suc}\AgdaSymbol{)}\<%
\\
\>[0]\AgdaKeyword{open}\AgdaSpace{}%
\AgdaKeyword{import}\AgdaSpace{}%
\AgdaModule{Data.Maybe}\AgdaSpace{}%
\AgdaKeyword{using}\AgdaSpace{}%
\AgdaSymbol{(}\AgdaDatatype{Maybe}\AgdaSymbol{;}\AgdaSpace{}%
\AgdaInductiveConstructor{just}\AgdaSymbol{;}\AgdaSpace{}%
\AgdaInductiveConstructor{nothing}\AgdaSymbol{)}\<%
\\
\>[0]\AgdaKeyword{open}\AgdaSpace{}%
\AgdaKeyword{import}\AgdaSpace{}%
\AgdaModule{Function}\AgdaSpace{}%
\AgdaKeyword{using}\AgdaSpace{}%
\AgdaSymbol{(}\AgdaOperator{\AgdaFunction{\AgdaUnderscore{}∘′\AgdaUnderscore{}}}\AgdaSymbol{)}\<%
\\
\>[0]\AgdaKeyword{open}\AgdaSpace{}%
\AgdaKeyword{import}\AgdaSpace{}%
\AgdaModule{Parser}\<%
\\
\>[0]\AgdaKeyword{open}\AgdaSpace{}%
\AgdaModule{Parser.Parser}\<%
\\
\>[0]\AgdaKeyword{open}\AgdaSpace{}%
\AgdaModule{Parser.Parsers}\<%
\\
\>[0]\AgdaKeyword{open}\AgdaSpace{}%
\AgdaModule{parsermonad}\AgdaSpace{}%
\AgdaKeyword{hiding}\AgdaSpace{}%
\AgdaSymbol{(}\AgdaOperator{\AgdaFunction{\AgdaUnderscore{}⊗\AgdaUnderscore{}}}\AgdaSymbol{)}\<%
\\
\>[0]\AgdaKeyword{import}\AgdaSpace{}%
\AgdaModule{Data.Tree.AVL.Map}\AgdaSpace{}%
\AgdaSymbol{as}\AgdaSpace{}%
\AgdaModule{MapMod}\<%
\\
\>[0]\AgdaKeyword{open}\AgdaSpace{}%
\AgdaModule{MapMod}\AgdaSpace{}%
\AgdaFunction{<-strictTotalOrder-≈}\<%
\\
\>[0]\AgdaKeyword{open}\AgdaSpace{}%
\AgdaKeyword{import}\AgdaSpace{}%
\AgdaModule{Thinning}\AgdaSpace{}%
\AgdaKeyword{using}\AgdaSpace{}%
\AgdaSymbol{(}\AgdaFunction{Weakenable}\AgdaSymbol{;}\AgdaSpace{}%
\AgdaFunction{weaken}\AgdaSymbol{;}\AgdaSpace{}%
\AgdaFunction{Thinnable}\AgdaSymbol{;}\AgdaSpace{}%
\AgdaOperator{\AgdaFunction{\AgdaUnderscore{}⟨var\AgdaUnderscore{}}}\AgdaSymbol{)}\<%
\\
\>[0]\AgdaKeyword{open}\AgdaSpace{}%
\AgdaKeyword{import}\AgdaSpace{}%
\AgdaModule{Rules}\AgdaSpace{}%
\AgdaKeyword{using}\AgdaSpace{}%
\AgdaSymbol{(}\AgdaRecord{TypeRule}\AgdaSymbol{;}\AgdaSpace{}%
\AgdaRecord{∋rule}\AgdaSymbol{;}\AgdaSpace{}%
\AgdaRecord{ElimRule}\AgdaSymbol{)}\<%
\\
\>[0]\AgdaKeyword{open}\AgdaSpace{}%
\AgdaModule{TypeRule}\AgdaSpace{}%
\AgdaKeyword{renaming}\AgdaSpace{}%
\AgdaSymbol{(}\AgdaField{subject}\AgdaSpace{}%
\AgdaSymbol{to}\AgdaSpace{}%
\AgdaField{tysub}\AgdaSymbol{)}\<%
\\
\>[0]\AgdaKeyword{open}\AgdaSpace{}%
\AgdaModule{∋rule}\AgdaSpace{}%
\AgdaKeyword{renaming}\AgdaSpace{}%
\AgdaSymbol{(}\AgdaField{subject}\AgdaSpace{}%
\AgdaSymbol{to}\AgdaSpace{}%
\AgdaField{∋sub}\AgdaSymbol{)}\<%
\\
\>[0]\AgdaKeyword{open}\AgdaSpace{}%
\AgdaModule{ElimRule}\AgdaSpace{}%
\AgdaKeyword{using}\AgdaSpace{}%
\AgdaSymbol{(}\AgdaField{eliminator}\AgdaSymbol{)}\<%
\\
\>[0]\AgdaKeyword{open}\AgdaSpace{}%
\AgdaKeyword{import}\AgdaSpace{}%
\AgdaModule{Data.Unit}\AgdaSpace{}%
\AgdaKeyword{using}\AgdaSpace{}%
\AgdaSymbol{(}\AgdaRecord{⊤}\AgdaSymbol{;}\AgdaSpace{}%
\AgdaInductiveConstructor{tt}\AgdaSymbol{)}\<%
\end{code}
}

\hide{
\begin{code}%
\>[0]\AgdaKeyword{private}\<%
\\
\>[0][@{}l@{\AgdaIndent{0}}]%
\>[2]\AgdaKeyword{variable}\<%
\\
\>[2][@{}l@{\AgdaIndent{0}}]%
\>[4]\AgdaGeneralizable{δ}\AgdaSpace{}%
\AgdaSymbol{:}\AgdaSpace{}%
\AgdaFunction{Scope}\<%
\\
%
\>[4]\AgdaGeneralizable{γ}\AgdaSpace{}%
\AgdaSymbol{:}\AgdaSpace{}%
\AgdaFunction{Scope}\<%
\\
%
\>[4]\AgdaGeneralizable{A}\AgdaSpace{}%
\AgdaSymbol{:}\AgdaSpace{}%
\AgdaPrimitiveType{Set}\<%
\end{code}
}

Here we are having trouble with a non-terminating parser since any pattern with
a left-most 'place' may cause infinite recursion depending on the orders of the
rules. This is most plainly seen in the simple case of trying to parse the pattern
*place . ` "->" . place* say in the case of a function type. First it knows it must
parse some term for the place, and so it starts iterating through the rules,
potentially coming across the same rule again ad infinitum.

Clearly iterating over the rules trying to parse each one in turn is not the
best way to try to parse the language. Is there another way?

Well apparently patterns form a lattice. That is to say that every set of patterns
has a pattern we know as the "greatest lower bound" or infimum. This is a pattern
where any term that matches it also matches every term in the set of patterns, and
that there is no more 'specific' pattern which will match. I wonder if this could
help with my problem.

Try this:

\begin{enumerate}
  \item Given all the patterns, first match the infimum
  \item then what?
\end{enumerate}



\begin{code}%
\>[0]\<%
\\
\>[0]\AgdaFunction{disallowed-chars}\AgdaSpace{}%
\AgdaSymbol{:}\AgdaSpace{}%
\AgdaDatatype{List}\AgdaSpace{}%
\AgdaPostulate{Char}\<%
\\
\>[0]\AgdaFunction{disallowed-chars}\AgdaSpace{}%
\AgdaSymbol{=}\AgdaSpace{}%
\AgdaString{'('}\AgdaSpace{}%
\AgdaOperator{\AgdaInductiveConstructor{∷}}\AgdaSpace{}%
\AgdaString{')'}\AgdaSpace{}%
\AgdaOperator{\AgdaInductiveConstructor{∷}}\AgdaSpace{}%
\AgdaString{':'}\AgdaSpace{}%
\AgdaOperator{\AgdaInductiveConstructor{∷}}\AgdaSpace{}%
\AgdaInductiveConstructor{[]}\<%
\\
%
\\[\AgdaEmptyExtraSkip]%
\>[0]\AgdaFunction{isendchar}\AgdaSpace{}%
\AgdaSymbol{:}\AgdaSpace{}%
\AgdaPostulate{Char}\AgdaSpace{}%
\AgdaSymbol{→}\AgdaSpace{}%
\AgdaDatatype{Bool}\<%
\\
\>[0]\AgdaFunction{isendchar}\AgdaSpace{}%
\AgdaBound{c}\AgdaSpace{}%
\AgdaSymbol{=}\AgdaSpace{}%
\AgdaPrimitive{isSpace}\AgdaSpace{}%
\AgdaBound{c}\AgdaSpace{}%
\AgdaOperator{\AgdaFunction{∨}}\AgdaSpace{}%
\AgdaFunction{any}\AgdaSpace{}%
\AgdaSymbol{(}\AgdaBound{c}\AgdaSpace{}%
\AgdaOperator{\AgdaFunction{==\AgdaUnderscore{}}}\AgdaSymbol{)}\AgdaSpace{}%
\AgdaFunction{disallowed-chars}\<%
\\
%
\\[\AgdaEmptyExtraSkip]%
\>[0]\AgdaFunction{token}\AgdaSpace{}%
\AgdaSymbol{:}\AgdaSpace{}%
\AgdaFunction{Parser}\AgdaSpace{}%
\AgdaPostulate{String}\<%
\\
\>[0]\AgdaFunction{token}\AgdaSpace{}%
\AgdaSymbol{=}\AgdaSpace{}%
\AgdaFunction{until}\AgdaSpace{}%
\AgdaFunction{isendchar}\<%
\\
%
\\[\AgdaEmptyExtraSkip]%
\>[0]\AgdaFunction{idchar}\AgdaSpace{}%
\AgdaSymbol{:}\AgdaSpace{}%
\AgdaPostulate{Char}\AgdaSpace{}%
\AgdaSymbol{→}\AgdaSpace{}%
\AgdaDatatype{Bool}\<%
\\
\>[0]\AgdaFunction{idchar}\AgdaSpace{}%
\AgdaBound{c}\AgdaSpace{}%
\AgdaSymbol{=}\AgdaSpace{}%
\AgdaFunction{not}\AgdaSpace{}%
\AgdaSymbol{(}\AgdaFunction{any}\AgdaSpace{}%
\AgdaSymbol{(}\AgdaBound{c}\AgdaSpace{}%
\AgdaOperator{\AgdaFunction{==\AgdaUnderscore{}}}\AgdaSymbol{)}\AgdaSpace{}%
\AgdaFunction{disallowed-chars}\AgdaSymbol{)}\AgdaSpace{}%
\AgdaOperator{\AgdaFunction{∧}}\AgdaSpace{}%
\AgdaFunction{not}\AgdaSpace{}%
\AgdaSymbol{(}\AgdaPrimitive{isSpace}\AgdaSpace{}%
\AgdaBound{c}\AgdaSymbol{)}\<%
\\
%
\\[\AgdaEmptyExtraSkip]%
\>[0]\AgdaFunction{identifier}\AgdaSpace{}%
\AgdaSymbol{:}\AgdaSpace{}%
\AgdaFunction{Parser}\AgdaSpace{}%
\AgdaPostulate{String}\<%
\\
\>[0]\AgdaFunction{identifier}\AgdaSpace{}%
\AgdaSymbol{=}\AgdaSpace{}%
\AgdaFunction{stringof}\AgdaSpace{}%
\AgdaFunction{idchar}\<%
\\
\>[0]\<%
\end{code}

\begin{code}%
\>[0]\AgdaFunction{VarMap}\AgdaSpace{}%
\AgdaSymbol{:}\AgdaSpace{}%
\AgdaFunction{Scoped}\<%
\\
\>[0]\AgdaFunction{VarMap}\AgdaSpace{}%
\AgdaBound{γ}\AgdaSpace{}%
\AgdaSymbol{=}\AgdaSpace{}%
\AgdaFunction{Map}\AgdaSpace{}%
\AgdaSymbol{(}\AgdaDatatype{Var}\AgdaSpace{}%
\AgdaBound{γ}\AgdaSymbol{)}\<%
\\
%
\\[\AgdaEmptyExtraSkip]%
\>[0]\AgdaOperator{\AgdaFunction{\AgdaUnderscore{}⟨\AgdaUnderscore{}}}\AgdaSpace{}%
\AgdaSymbol{:}\AgdaSpace{}%
\AgdaFunction{Thinnable}\AgdaSpace{}%
\AgdaFunction{VarMap}\<%
\\
\>[0]\AgdaBound{vm}\AgdaSpace{}%
\AgdaOperator{\AgdaFunction{⟨}}\AgdaSpace{}%
\AgdaBound{θ}\AgdaSpace{}%
\AgdaSymbol{=}\AgdaSpace{}%
\AgdaFunction{map}\AgdaSpace{}%
\AgdaSymbol{(}\AgdaOperator{\AgdaFunction{\AgdaUnderscore{}⟨var}}\AgdaSpace{}%
\AgdaBound{θ}\AgdaSymbol{)}\AgdaSpace{}%
\AgdaBound{vm}\<%
\\
%
\\[\AgdaEmptyExtraSkip]%
\>[0]\AgdaOperator{\AgdaFunction{\AgdaUnderscore{}\textasciicircum{}}}\AgdaSpace{}%
\AgdaSymbol{:}\AgdaSpace{}%
\AgdaFunction{Weakenable}\AgdaSpace{}%
\AgdaFunction{VarMap}\<%
\\
\>[0]\AgdaOperator{\AgdaFunction{\AgdaUnderscore{}\textasciicircum{}}}\AgdaSpace{}%
\AgdaSymbol{=}\AgdaSpace{}%
\AgdaFunction{weaken}\AgdaSpace{}%
\AgdaOperator{\AgdaFunction{\AgdaUnderscore{}⟨\AgdaUnderscore{}}}\<%
\\
%
\\[\AgdaEmptyExtraSkip]%
\>[0]\AgdaFunction{fresh}\AgdaSpace{}%
\AgdaSymbol{:}\AgdaSpace{}%
\AgdaPostulate{String}\AgdaSpace{}%
\AgdaSymbol{→}\AgdaSpace{}%
\AgdaFunction{VarMap}\AgdaSpace{}%
\AgdaGeneralizable{γ}\AgdaSpace{}%
\AgdaSymbol{→}\AgdaSpace{}%
\AgdaFunction{VarMap}\AgdaSpace{}%
\AgdaSymbol{(}\AgdaInductiveConstructor{suc}\AgdaSpace{}%
\AgdaGeneralizable{γ}\AgdaSymbol{)}\<%
\\
\>[0]\AgdaFunction{fresh}\AgdaSpace{}%
\AgdaBound{name}\AgdaSpace{}%
\AgdaBound{vm}\AgdaSpace{}%
\AgdaSymbol{=}\AgdaSpace{}%
\AgdaFunction{insert}\AgdaSpace{}%
\AgdaBound{name}\AgdaSpace{}%
\AgdaInductiveConstructor{ze}\AgdaSpace{}%
\AgdaSymbol{(}\AgdaBound{vm}\AgdaSpace{}%
\AgdaOperator{\AgdaFunction{\textasciicircum{}}}\AgdaSymbol{)}\<%
\end{code}

\begin{code}%
\>[0]\<%
\\
\>[0]\AgdaFunction{Rules}\AgdaSpace{}%
\AgdaSymbol{:}\AgdaSpace{}%
\AgdaPrimitiveType{Set}\<%
\\
\>[0]\AgdaFunction{Rules}\AgdaSpace{}%
\AgdaSymbol{=}\AgdaSpace{}%
\AgdaDatatype{List}\AgdaSpace{}%
\AgdaRecord{TypeRule}\AgdaSpace{}%
\AgdaOperator{\AgdaFunction{×}}\AgdaSpace{}%
\AgdaDatatype{List}\AgdaSpace{}%
\AgdaRecord{∋rule}\AgdaSpace{}%
\AgdaOperator{\AgdaFunction{×}}\AgdaSpace{}%
\AgdaDatatype{List}\AgdaSpace{}%
\AgdaRecord{ElimRule}\<%
\\
%
\\[\AgdaEmptyExtraSkip]%
\>[0]\AgdaFunction{TermParser}\AgdaSpace{}%
\AgdaSymbol{:}\AgdaSpace{}%
\AgdaFunction{Scoped}\AgdaSpace{}%
\AgdaSymbol{→}\AgdaSpace{}%
\AgdaPrimitiveType{Set}\<%
\\
\>[0]\AgdaFunction{TermParser}\AgdaSpace{}%
\AgdaBound{A}\AgdaSpace{}%
\AgdaSymbol{=}\AgdaSpace{}%
\AgdaFunction{Rules}\AgdaSpace{}%
\AgdaSymbol{→}\AgdaSpace{}%
\AgdaSymbol{∀}\AgdaSpace{}%
\AgdaSymbol{\{}\AgdaBound{γ}\AgdaSymbol{\}}\AgdaSpace{}%
\AgdaSymbol{→}\AgdaSpace{}%
\AgdaFunction{VarMap}\AgdaSpace{}%
\AgdaBound{γ}\AgdaSpace{}%
\AgdaSymbol{→}\AgdaSpace{}%
\AgdaFunction{Parser}\AgdaSpace{}%
\AgdaSymbol{(}\AgdaBound{A}\AgdaSpace{}%
\AgdaBound{γ}\AgdaSymbol{)}\<%
\\
\>[0]\<%
\end{code}



\begin{code}%
\>[0]\<%
\\
\>[0]\AgdaFunction{construction}\AgdaSpace{}%
\AgdaSymbol{:}\AgdaSpace{}%
\AgdaFunction{TermParser}\AgdaSpace{}%
\AgdaDatatype{Const}\<%
\\
\>[0]\AgdaFunction{computation}%
\>[13]\AgdaSymbol{:}\AgdaSpace{}%
\AgdaFunction{TermParser}\AgdaSpace{}%
\AgdaDatatype{Compu}\<%
\\
%
\\[\AgdaEmptyExtraSkip]%
%
\\[\AgdaEmptyExtraSkip]%
\>[0]\AgdaFunction{ppat}\AgdaSpace{}%
\AgdaSymbol{:}\AgdaSpace{}%
\AgdaDatatype{Pattern}\AgdaSpace{}%
\AgdaGeneralizable{δ}\AgdaSpace{}%
\AgdaSymbol{→}\AgdaSpace{}%
\AgdaFunction{TermParser}\AgdaSpace{}%
\AgdaDatatype{Const}\<%
\\
\>[0]\AgdaFunction{ppat}\AgdaSpace{}%
\AgdaSymbol{(}\AgdaInductiveConstructor{`}\AgdaSpace{}%
\AgdaBound{x}\AgdaSymbol{)}%
\>[14]\AgdaSymbol{\AgdaUnderscore{}}\AgdaSpace{}%
\AgdaSymbol{\AgdaUnderscore{}}%
\>[20]\AgdaSymbol{=}%
\>[261I]\AgdaKeyword{do}\<%
\\
\>[261I][@{}l@{\AgdaIndent{0}}]%
\>[27]\AgdaBound{tok}\AgdaSpace{}%
\AgdaOperator{\AgdaFunction{←}}\AgdaSpace{}%
\AgdaFunction{token}\<%
\\
%
\>[27]\AgdaInductiveConstructor{just}\AgdaSpace{}%
\AgdaSymbol{(}\AgdaBound{at}\AgdaSpace{}%
\AgdaOperator{\AgdaInductiveConstructor{,}}\AgdaSpace{}%
\AgdaSymbol{\AgdaUnderscore{})}\AgdaSpace{}%
\AgdaOperator{\AgdaFunction{←}}\AgdaSpace{}%
\AgdaFunction{return}\AgdaSpace{}%
\AgdaSymbol{(}\AgdaFunction{complete}\AgdaSpace{}%
\AgdaSymbol{(}\AgdaFunction{string}\AgdaSpace{}%
\AgdaBound{x}\AgdaSymbol{)}\AgdaSpace{}%
\AgdaBound{tok}\AgdaSymbol{)}\<%
\\
\>[27][@{}l@{\AgdaIndent{0}}]%
\>[29]\AgdaKeyword{where}\AgdaSpace{}%
\AgdaInductiveConstructor{nothing}\AgdaSpace{}%
\AgdaSymbol{→}\AgdaSpace{}%
\AgdaFunction{fail}\<%
\\
%
\>[27]\AgdaFunction{return}\AgdaSpace{}%
\AgdaSymbol{(}\AgdaInductiveConstructor{`}\AgdaSpace{}%
\AgdaBound{at}\AgdaSymbol{)}\<%
\\
\>[0]\AgdaFunction{ppat}\AgdaSpace{}%
\AgdaSymbol{(}\AgdaBound{s}\AgdaSpace{}%
\AgdaOperator{\AgdaInductiveConstructor{∙}}\AgdaSpace{}%
\AgdaBound{t}\AgdaSymbol{)}\AgdaSpace{}%
\AgdaBound{rls}\AgdaSpace{}%
\AgdaBound{vm}%
\>[22]\AgdaSymbol{=}%
\>[283I]\AgdaKeyword{do}\<%
\\
\>[283I][@{}l@{\AgdaIndent{0}}]%
\>[27]\AgdaBound{l}\AgdaSpace{}%
\AgdaOperator{\AgdaFunction{←}}\AgdaSpace{}%
\AgdaFunction{ppat}\AgdaSpace{}%
\AgdaBound{s}\AgdaSpace{}%
\AgdaBound{rls}\AgdaSpace{}%
\AgdaBound{vm}\<%
\\
%
\>[27]\AgdaFunction{ws+nl}\<%
\\
%
\>[27]\AgdaBound{r}\AgdaSpace{}%
\AgdaOperator{\AgdaFunction{←}}\AgdaSpace{}%
\AgdaFunction{ppat}\AgdaSpace{}%
\AgdaBound{t}\AgdaSpace{}%
\AgdaBound{rls}\AgdaSpace{}%
\AgdaBound{vm}\<%
\\
%
\>[27]\AgdaFunction{return}\AgdaSpace{}%
\AgdaSymbol{(}\AgdaBound{l}\AgdaSpace{}%
\AgdaOperator{\AgdaInductiveConstructor{∙}}\AgdaSpace{}%
\AgdaBound{r}\AgdaSymbol{)}\<%
\\
\>[0]\AgdaFunction{ppat}\AgdaSpace{}%
\AgdaSymbol{(}\AgdaInductiveConstructor{bind}\AgdaSpace{}%
\AgdaBound{t}\AgdaSymbol{)}\AgdaSpace{}%
\AgdaBound{rls}\AgdaSpace{}%
\AgdaBound{vm}%
\>[22]\AgdaSymbol{=}%
\>[301I]\AgdaKeyword{do}\<%
\\
\>[301I][@{}l@{\AgdaIndent{0}}]%
\>[27]\AgdaBound{name}\AgdaSpace{}%
\AgdaOperator{\AgdaFunction{←}}\AgdaSpace{}%
\AgdaFunction{identifier}\<%
\\
%
\>[27]\AgdaFunction{ws+nl}\<%
\\
%
\>[27]\AgdaBound{under}\AgdaSpace{}%
\AgdaOperator{\AgdaFunction{←}}\AgdaSpace{}%
\AgdaFunction{ppat}\AgdaSpace{}%
\AgdaBound{t}\AgdaSpace{}%
\AgdaBound{rls}\AgdaSpace{}%
\AgdaSymbol{(}\AgdaFunction{fresh}\AgdaSpace{}%
\AgdaBound{name}\AgdaSpace{}%
\AgdaBound{vm}\AgdaSymbol{)}\<%
\\
%
\>[27]\AgdaFunction{return}\AgdaSpace{}%
\AgdaSymbol{(}\AgdaInductiveConstructor{bind}\AgdaSpace{}%
\AgdaBound{under}\AgdaSymbol{)}\<%
\\
\>[0]\AgdaFunction{ppat}\AgdaSpace{}%
\AgdaSymbol{(}\AgdaInductiveConstructor{place}\AgdaSpace{}%
\AgdaBound{x}\AgdaSymbol{)}\AgdaSpace{}%
\AgdaBound{rls}\AgdaSpace{}%
\AgdaBound{vm}\AgdaSpace{}%
\AgdaSymbol{=}\AgdaSpace{}%
\AgdaFunction{construction}\AgdaSpace{}%
\AgdaBound{rls}\AgdaSpace{}%
\AgdaBound{vm}\<%
\\
%
\\[\AgdaEmptyExtraSkip]%
%
\\[\AgdaEmptyExtraSkip]%
\>[0]\AgdaFunction{const-parsers}\AgdaSpace{}%
\AgdaSymbol{:}\AgdaSpace{}%
\AgdaFunction{Rules}\AgdaSpace{}%
\AgdaSymbol{→}\AgdaSpace{}%
\AgdaDatatype{List}\AgdaSpace{}%
\AgdaSymbol{(}\AgdaFunction{TermParser}\AgdaSpace{}%
\AgdaDatatype{Const}\AgdaSymbol{)}\<%
\\
\>[0]\AgdaFunction{const-parsers}\AgdaSpace{}%
\AgdaBound{rs}\AgdaSymbol{@(}\AgdaBound{trs}\AgdaSpace{}%
\AgdaOperator{\AgdaInductiveConstructor{,}}\AgdaSpace{}%
\AgdaBound{∋rs}\AgdaSpace{}%
\AgdaOperator{\AgdaInductiveConstructor{,}}\AgdaSpace{}%
\AgdaSymbol{\AgdaUnderscore{})}\AgdaSpace{}%
\AgdaSymbol{=}\AgdaSpace{}%
\AgdaFunction{lmap}\AgdaSpace{}%
\AgdaSymbol{(}\AgdaFunction{ppat}\AgdaSpace{}%
\AgdaOperator{\AgdaFunction{∘′}}\AgdaSpace{}%
\AgdaField{∋sub}\AgdaSymbol{)}\AgdaSpace{}%
\AgdaBound{∋rs}\AgdaSpace{}%
\AgdaOperator{\AgdaFunction{++}}\AgdaSpace{}%
\AgdaFunction{lmap}\AgdaSpace{}%
\AgdaSymbol{(}\AgdaFunction{ppat}\AgdaSpace{}%
\AgdaOperator{\AgdaFunction{∘′}}\AgdaSpace{}%
\AgdaField{tysub}\AgdaSymbol{)}\AgdaSpace{}%
\AgdaBound{trs}\<%
\\
%
\\[\AgdaEmptyExtraSkip]%
%
\\[\AgdaEmptyExtraSkip]%
\>[0]\AgdaFunction{construction}\AgdaSpace{}%
\AgdaBound{rules}\AgdaSpace{}%
\AgdaBound{vm}\AgdaSpace{}%
\AgdaSymbol{=}%
\>[347I]\AgdaKeyword{do}\<%
\\
\>[347I][@{}l@{\AgdaIndent{0}}]%
\>[27]\AgdaInductiveConstructor{inj₁}%
\>[348I]\AgdaBound{cons}\AgdaSpace{}%
\AgdaOperator{\AgdaFunction{←}}\AgdaSpace{}%
\AgdaFunction{anyof}\AgdaSpace{}%
\AgdaSymbol{(}\AgdaFunction{lmap}\AgdaSpace{}%
\AgdaSymbol{(λ}\AgdaSpace{}%
\AgdaBound{tp}\AgdaSpace{}%
\AgdaSymbol{→}\AgdaSpace{}%
\AgdaBound{tp}\AgdaSpace{}%
\AgdaBound{rules}\AgdaSpace{}%
\AgdaBound{vm}\AgdaSymbol{)}\AgdaSpace{}%
\AgdaSymbol{(}\AgdaFunction{const-parsers}\AgdaSpace{}%
\AgdaBound{rules}\AgdaSymbol{))}\<%
\\
\>[348I][@{}l@{\AgdaIndent{0}}]%
\>[36]\AgdaOperator{\AgdaFunction{or}}\AgdaSpace{}%
\AgdaFunction{computation}\AgdaSpace{}%
\AgdaBound{rules}\AgdaSpace{}%
\AgdaBound{vm}\<%
\\
%
\>[36]\AgdaKeyword{where}\AgdaSpace{}%
\AgdaInductiveConstructor{inj₂}\AgdaSpace{}%
\AgdaBound{comp}\AgdaSpace{}%
\AgdaSymbol{→}\AgdaSpace{}%
\AgdaFunction{return}\AgdaSpace{}%
\AgdaSymbol{(}\AgdaInductiveConstructor{thunk}\AgdaSpace{}%
\AgdaBound{comp}\AgdaSymbol{)}\<%
\\
%
\>[27]\AgdaFunction{return}\AgdaSpace{}%
\AgdaBound{cons}\<%
\\
%
\\[\AgdaEmptyExtraSkip]%
\>[0]\AgdaFunction{prad}\AgdaSpace{}%
\AgdaSymbol{:}\AgdaSpace{}%
\AgdaFunction{TermParser}\AgdaSpace{}%
\AgdaDatatype{Compu}\<%
\\
\>[0]\AgdaFunction{prad}\AgdaSpace{}%
\AgdaBound{rules}\AgdaSpace{}%
\AgdaBound{vm}\AgdaSpace{}%
\AgdaSymbol{=}%
\>[376I]\AgdaKeyword{do}\<%
\\
\>[376I][@{}l@{\AgdaIndent{0}}]%
\>[18]\AgdaBound{tm}\AgdaSpace{}%
\AgdaOperator{\AgdaFunction{←}}\AgdaSpace{}%
\AgdaFunction{construction}\AgdaSpace{}%
\AgdaBound{rules}\AgdaSpace{}%
\AgdaBound{vm}\<%
\\
%
\>[18]\AgdaFunction{wsnl-tolerant}\AgdaSpace{}%
\AgdaSymbol{(}\AgdaFunction{literal}\AgdaSpace{}%
\AgdaString{':'}\AgdaSymbol{)}\<%
\\
%
\>[18]\AgdaBound{ty}\AgdaSpace{}%
\AgdaOperator{\AgdaFunction{←}}\AgdaSpace{}%
\AgdaFunction{construction}\AgdaSpace{}%
\AgdaBound{rules}\AgdaSpace{}%
\AgdaBound{vm}\<%
\\
%
\>[18]\AgdaFunction{return}\AgdaSpace{}%
\AgdaSymbol{(}\AgdaBound{tm}\AgdaSpace{}%
\AgdaOperator{\AgdaInductiveConstructor{∷}}\AgdaSpace{}%
\AgdaBound{ty}\AgdaSymbol{)}\<%
\\
%
\\[\AgdaEmptyExtraSkip]%
\>[0]\AgdaFunction{pvar}\AgdaSpace{}%
\AgdaSymbol{:}\AgdaSpace{}%
\AgdaSymbol{∀}\AgdaSpace{}%
\AgdaSymbol{\{}\AgdaBound{γ}\AgdaSymbol{\}}\AgdaSpace{}%
\AgdaSymbol{→}\AgdaSpace{}%
\AgdaFunction{VarMap}\AgdaSpace{}%
\AgdaBound{γ}\AgdaSpace{}%
\AgdaSymbol{→}\AgdaSpace{}%
\AgdaFunction{Parser}\AgdaSpace{}%
\AgdaSymbol{(}\AgdaDatatype{Compu}\AgdaSpace{}%
\AgdaBound{γ}\AgdaSymbol{)}\<%
\\
\>[0]\AgdaFunction{pvar}\AgdaSpace{}%
\AgdaBound{vm}\AgdaSpace{}%
\AgdaSymbol{=}%
\>[402I]\AgdaKeyword{do}\<%
\\
\>[402I][@{}l@{\AgdaIndent{0}}]%
\>[12]\AgdaBound{name}\AgdaSpace{}%
\AgdaOperator{\AgdaFunction{←}}\AgdaSpace{}%
\AgdaFunction{identifier}\<%
\\
%
\>[12]\AgdaInductiveConstructor{just}\AgdaSpace{}%
\AgdaBound{v}\AgdaSpace{}%
\AgdaOperator{\AgdaFunction{←}}\AgdaSpace{}%
\AgdaFunction{return}\AgdaSpace{}%
\AgdaSymbol{(}\AgdaFunction{lookup}\AgdaSpace{}%
\AgdaBound{name}\AgdaSpace{}%
\AgdaBound{vm}\AgdaSymbol{)}\<%
\\
\>[12][@{}l@{\AgdaIndent{0}}]%
\>[15]\AgdaKeyword{where}\AgdaSpace{}%
\AgdaInductiveConstructor{nothing}\AgdaSpace{}%
\AgdaSymbol{→}\AgdaSpace{}%
\AgdaFunction{fail}\<%
\\
%
\>[12]\AgdaFunction{return}\AgdaSpace{}%
\AgdaSymbol{(}\AgdaInductiveConstructor{var}\AgdaSpace{}%
\AgdaBound{v}\AgdaSymbol{)}\<%
\\
%
\\[\AgdaEmptyExtraSkip]%
\>[0]\AgdaSymbol{\{-\#}\AgdaSpace{}%
\AgdaKeyword{TERMINATING}\AgdaSpace{}%
\AgdaSymbol{\#-\}}\<%
\\
\>[0]\AgdaFunction{pelim}\AgdaSpace{}%
\AgdaSymbol{:}\AgdaSpace{}%
\AgdaFunction{TermParser}\AgdaSpace{}%
\AgdaDatatype{Compu}\<%
\\
\>[0]\AgdaFunction{pelim}\AgdaSpace{}%
\AgdaBound{rules}\AgdaSpace{}%
\AgdaBound{vm}\AgdaSpace{}%
\AgdaSymbol{=}%
\>[424I]\AgdaKeyword{do}\<%
\\
\>[424I][@{}l@{\AgdaIndent{0}}]%
\>[19]\AgdaBound{tar}\AgdaSpace{}%
\AgdaOperator{\AgdaFunction{←}}\AgdaSpace{}%
\AgdaFunction{anyof}%
\>[427I]\AgdaSymbol{(}\AgdaFunction{bracketed}\AgdaSpace{}%
\AgdaSymbol{(}\AgdaFunction{pelim}\AgdaSpace{}%
\AgdaBound{rules}\AgdaSpace{}%
\AgdaBound{vm}\AgdaSymbol{)}\AgdaSpace{}%
\AgdaOperator{\AgdaInductiveConstructor{∷}}\<%
\\
\>[.][@{}l@{}]\<[427I]%
\>[31]\AgdaFunction{bracketed}\AgdaSpace{}%
\AgdaSymbol{(}\AgdaFunction{prad}\AgdaSpace{}%
\AgdaBound{rules}\AgdaSpace{}%
\AgdaBound{vm}\AgdaSymbol{)}\AgdaSpace{}%
\AgdaOperator{\AgdaInductiveConstructor{∷}}\<%
\\
%
\>[31]\AgdaFunction{pvar}\AgdaSpace{}%
\AgdaBound{vm}\AgdaSpace{}%
\AgdaOperator{\AgdaInductiveConstructor{∷}}\AgdaSpace{}%
\AgdaInductiveConstructor{[]}\AgdaSymbol{)}\<%
\\
%
\>[19]\AgdaFunction{ws+nl}\<%
\\
%
\>[19]\AgdaBound{elm}\AgdaSpace{}%
\AgdaOperator{\AgdaFunction{←}}\AgdaSpace{}%
\AgdaFunction{construction}\AgdaSpace{}%
\AgdaBound{rules}\AgdaSpace{}%
\AgdaBound{vm}\<%
\\
%
\>[19]\AgdaFunction{return}\AgdaSpace{}%
\AgdaSymbol{(}\AgdaInductiveConstructor{elim}\AgdaSpace{}%
\AgdaBound{tar}\AgdaSpace{}%
\AgdaBound{elm}\AgdaSymbol{)}\<%
\\
%
\\[\AgdaEmptyExtraSkip]%
\>[0]\AgdaFunction{computation}\AgdaSpace{}%
\AgdaBound{rules}\AgdaSpace{}%
\AgdaBound{vm}\AgdaSpace{}%
\AgdaSymbol{=}\AgdaSpace{}%
\AgdaFunction{anyof}%
\>[450I]\AgdaSymbol{(}\AgdaFunction{bracketed}\AgdaSpace{}%
\AgdaSymbol{(}\AgdaFunction{prad}\AgdaSpace{}%
\AgdaBound{rules}\AgdaSpace{}%
\AgdaBound{vm}\AgdaSymbol{)}\AgdaSpace{}%
\AgdaOperator{\AgdaInductiveConstructor{∷}}\<%
\\
\>[450I][@{}l@{\AgdaIndent{0}}]%
\>[30]\AgdaFunction{bracketed}\AgdaSpace{}%
\AgdaSymbol{(}\AgdaFunction{pelim}\AgdaSpace{}%
\AgdaBound{rules}\AgdaSpace{}%
\AgdaBound{vm}\AgdaSymbol{)}\AgdaSpace{}%
\AgdaOperator{\AgdaInductiveConstructor{∷}}\AgdaSpace{}%
\AgdaFunction{pvar}\AgdaSpace{}%
\AgdaBound{vm}\AgdaSpace{}%
\AgdaOperator{\AgdaInductiveConstructor{∷}}\AgdaSpace{}%
\AgdaInductiveConstructor{[]}\AgdaSymbol{)}\<%
\end{code}



