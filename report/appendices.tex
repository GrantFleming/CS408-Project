\begin{appendices}
  
  \section {Typing Rules}

  \subsection {System F}
  \label{appendix:sysFrules}
  \[\begin{array}{c@{\qquad}c}
      \mbox{\begin {prooftree}
        \hypo{x:\sigma \in \Gamma}
        \infer1[var]{\Gamma \vdash x:\sigma}
      \end {prooftree}}
      &
      \mbox{\begin{prooftree}
        \infer0[const]{\Gamma \vdash c:T}
      \end{prooftree}}
      \\\\
      \mbox{\begin{prooftree}
        \hypo{\Gamma, x:\sigma \vdash e : \tau}
        \infer1[abs]{\Gamma \vdash (\lambda x_\sigma \cdot e) : \sigma
          \to \tau}
      \end{prooftree}}
      &
      \mbox{\begin{prooftree}
        \hypo{\Gamma \vdash e_1 : \sigma \to \tau}
        \hypo{\Gamma \vdash e_2 : \sigma}
        \infer2[app]{\Gamma \vdash (e_1 e_2): \tau}
      \end{prooftree}}
      \\
  \end{array} \]

  Thus far the typing rules are the same as in the simply typed lambda
  calculus, to complete his system Reynolds extends it with two more
  rules to introduce parametric polymorphism:

  \[\begin{array}{c@{\qquad}c}  
      \mbox{\begin{prooftree}
        \hypo{\Gamma \vdash M : \sigma}
        \infer1[$\Delta$-abs]{\Gamma \vdash (\Lambda \alpha \cdot M) : \Delta
        \alpha \cdot \sigma}
      \end{prooftree}}
      &
      \mbox{\begin{prooftree}
        \hypo{\Gamma \vdash M : \Delta \alpha \cdot \sigma}
        \infer1[$\Delta$-app]{\Gamma \vdash (M \tau) : \sigma[\tau / \alpha]}
      \end{prooftree}}
      \\\\
    \end{array} \]

  Note that in these rules, $\alpha$ is a type variable.

  \subsection {Hindley-Milner}
  \label{appendix:HMrules}
  
  In Hindley-Milner, we have very similar rules for typing lambda
  abstraction, application, free type variables and and type constants:

  \[\begin{array}{c@{\qquad}c}  
      \mbox{\begin{prooftree}
        \hypo{x : \sigma \in \Gamma}
        \infer1[var]{\Gamma \vdash x : \sigma}
      \end{prooftree}}
      &
      \mbox{\begin{prooftree}
        \infer0[const]{\Gamma \vdash c : T}
      \end{prooftree}}
      \\\\
      \mbox{\begin{prooftree}
        \hypo{\Gamma , x : \sigma \vdash e : \tau }
        \infer1[abs]{\Gamma \vdash (\lambda x \cdot e) : \sigma \to \tau}
      \end{prooftree}}
      &
      \mbox{\begin{prooftree}
        \hypo{\Gamma \vdash e_1 : \sigma \to \tau}
        \hypo{\Gamma \vdash e_2 : \sigma}
        \infer2[app]{\Gamma \vdash (e_1 e_2) : \tau}
      \end{prooftree}}
      \\\\      
  \end{array} \]

  These rules are then extended to accomodate the 'let' language
  construct:

    \[\begin{array}{c}  
      \mbox{\begin{prooftree}
        \hypo{\Gamma \vdash e_1 : \sigma}
        \hypo{\Gamma, x : \sigma \vdash e_2 : \tau }
        \infer2[let]{\Gamma \vdash (let\; x = e_1\; in\; e_2) : \tau}
      \end{prooftree}}
      \\\\
    \end{array} \]

  Before we detail the last two rules detailing instantiation and
  generification we first outline the meaning of a judgement $\sigma
  \sqsubseteq \sigma'$. Intuitively this means that $\sigma$ is some
  subtype of $\sigma'$ - we can create it by some substitution of the
  quantified variables in $\sigma'$. More precisely it is defined as
  follows:

  \[\begin{array}{c@{\qquad}c}  
      \mbox{\begin{prooftree}
        \hypo{\tau' = \{\alpha_i \mapsto \tau_i\}\tau}
        \hypo{\beta_i \notin free(\forall \alpha_1 \ldots \forall
          \alpha_n \cdot \tau)}
        \infer2[spec]{\forall \alpha_1 \ldots \forall \alpha_n \cdot
          \tau \sqsubseteq \forall \beta_1 \ldots \forall
          \beta_m \cdot \tau' }
      \end{prooftree}}
      \\\\
  \end{array} \]  
    
  Lastly we have rules to instantiate a type scheme, or generify a
  type (i.e. to make a type more specific and narrow, or less specific
  and general):
  
  \[\begin{array}{c@{\qquad}c}  
      \mbox{\begin{prooftree}
        \hypo{\Gamma \vdash e : \sigma}
        \hypo{\sigma' \sqsubseteq \sigma}
        \infer2[inst]{\Gamma \vdash e : \sigma'}
      \end{prooftree}}
      &
      \mbox{\begin{prooftree}
        \hypo{\Gamma \vdash e : \sigma}
        \hypo{\alpha \notin free(\Gamma)}
        \infer2[gen]{\Gamma \vdash e : \forall \alpha \sigma}
      \end{prooftree}}
      \\\\
  \end{array} \]

  The latter three rules are the ones that capture the idea of
  Hindley-Milner polymorphism.
  
\end{appendices}
