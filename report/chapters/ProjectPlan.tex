\chapter{Project Plan}

** TODO - introduce the chapter

\section{Project Specification}

  \subsection{Overview}

  Writing type-checkers for compilers is a key task in the
  development and evolution of programming languages.

  Despite huge variety in the kinds of languages in existence,
  type-systems can often be described using the same kinds of
  things such as typing rules. The descriptions are becoming
  standardized enough for us to recognize common syntax for
  describing these aspects in literature.

  This project seeks to capitalize on this by abstracting
  enough common elements and leaving only what is necessary
  for describing a type-system up to a user. Given any
  description of a type-system in a specified format, a
  type-checker-generator should be able to generate a
  type-checker for the type system.
  
  \subsection{Goal}

  To design a domain specific language (DSL) capable of describing type systems and to
  write a piece of software that, given one of these
  descriptions, is able to type-check code in the language
  it describes.

  \subsection{Artifacts}
  \label{section-artifacts}

  At a minimum, the finished project will consist of:

  \begin{itemize}
  \item A type-checker generator as a standalone binary
  \item At least one example specification in the DSL
  \item At least two source code examples for each
    specification, one which type-checks successfully and one
    which does not
  \item The final report
  \end{itemize}
  
  \subsection{Scope}
  \label{sec-scope}

  \begin{center}
  \begin{tabular}{p{20em}|p{18em}}
    \large In Scope & \large Not In Scope \\ \hline \hline
    Notification on whether the type-checking succeeds
    or fails
    & Detailed feedback from the type-checking process \\ \hline
    Notification on whether or not the specification file is
    successfully parsed
    & Detailed feedback on the parsing of the specification file \\ \hline
    Example specifications and checkable code for small 'toy' languages
    & Example specifications and code for large 'real world' languages \\ \hline
    Single usable feature of 'running' the generator on a
    specification and piece of source code
    & Quality of life / usability features or other niceties \\ \hline
    High level reasoning on why the generated type-checkers
    are likely to be sound and complete
    & Formally proving the generator produces type-checkers
    that are sound and complete\\
  \end{tabular}            
  \end{center}        
  
  \subsection{Milestones}
  \label{section-milestones}

  \begin{enumerate}
  \item A DSL that we are able to use to describe existing
    type checkers, used to write \emph{specifications} of type systems.
  \item A type-checker able to check types for a generic syntax
    in a syntactic universe under a \emph{fixed} set of typing rules
    where syntaxes and checkable source code is supplied hard-coded.
  \item A type-checker able to check types when full specifications
    and checkable source code is hard-coded.
  \item A type-checker  that checks types, reading
    the specification and source code from files.
  \end{enumerate}
  
  \subsection{Acceptance Criteria}
  \label{sec-acceptance}

  \begin{itemize}
  \item The software can be ran from the command-line by
    giving a simple command such as: $$\mbox{type-check} \;
    <\emph{specification-file}> \;
    <\emph{source-code-file}>$$
  \item The software either succeeds with a simple indication
    message, or fails with one of two kinds of error:
    \begin{itemize}
    \item Failure to parse specification
    \item Failure to successfully type-check source code
    \end{itemize}
  \item The software passes a fixed set of test cases (specifications and
    source codes) - succeeding, failing and showing
    appropriate errors as detailed in the test cases.
  \item The software can generate type-checkers capable of
    checking both non-dependent and dependent types.
  \end{itemize}

\section{Project Plan}

\subsection{Objective Timeline}

The following timeline assigned deadlines to the previously
identified objectives.

\begin{center}
\begin{vtimeline}[description={text width=11cm}, 
 row sep=3em]
16/11/2020 & Complete initial readings on type-theory. \endlr
30/11/2020 & Complete specific readings relating to the design and
  implementation of various type systems and components thereof.* \endlr
07/12/2020 & Identify all elements needed to fully and
  unambiguously describe a type system. \endlr
14/12/2020 & Verify sufficiency of identified elements by fully describing
    two different type systems in the framework. \endlr
21/12/2020 & Formalize a syntax for expressing these elements. \endlr
04/01/2021 & Write a specification of a type-checker-generator,
including identifying areas for verification and/or testing. \endlr
01/03/2021 & Design and implement a type-checker-generator. \endlr
15/03/2021 & Verify the implementation according to the
  specification. \endlr
22/03/2021 & Evaluate the DSL and identify any limitations on the
  kind of type systems that it can represent. \endlr
\end{vtimeline}          
\end{center}
\hrule
* This is the target date of completion for \emph{currently planned}
readings only

\section{The Development Process}
\subsection{Development Methodology}

In this project, although the primary goal is the production
of software, much of the work is theoretical in nature. With
that in mind, it is proposed to split the project in three:
initial theoretical work (such as formalizing a language to
describe type systems), the design and implementation of the
type-checker-generator (the software development element) and
the verification of the software (again somewhat theoretical
in nature where we look at soundness and completeness).

In the development phase of the project, an agile approach to
development will be employed. 

The author will take a Scrum-style approach to the development
process, adapting the usual teamwork-focused elements to be
more suited to a single developer environment.

Of the four identified milestones, the last three of these
will make up the epics under this system, where these will be
broken down into more granular tickets for each sprint.

The development section of the project has been allocated 8
weeks, this will be split into four sprints each lasting two
weeks. 

\subsection{Design}

As development is taking place within a scrum-style agile framework, the
design will take place in small increments within each
sprint.

As we are working in a functional language, there are many
prevalent concepts that, although present in design literature
in general, are not as relevant in this context. For
instance, object-oriented programming places an emphasis on
the concepts of "design patterns", this is a concept discussed
far less with regards to functional programming - some might
argue because it is a more efficient paradigm that does not
require them, others might argue that their use is implicit.

One area of particular interest is type driven
design - where much of the type information is filled in
first, and the implementation details often follow
afterward. This style of development could aid us in making
better use of Agda's dependent type system.

It is hoped that with careful consideration of types, it might
be possible to identify areas to better verify correctness of
implementing code by introducing invariants that guarantee desirable
properties and reduce the number of possible incorrect
implementations - although this is an area of research in its
own right and is certainly not the focus of this project.

\subsection{Testing/Verification}

The testing and verification process is multi-faceted in
nature.

This project aims to utilize the type system in
order to help verify correctness at the level of component
functions (as described when discussing type-driven design
previously).

In addition, high level test cases (in other paradigms
referred to as integration tests) will be conducted by
supplying example source code to the type-checker where the
expected outcome is known in advance. Clearly these examples
will be small in nature as the the expected outcome needs to
be calculated by hand.

Finally, this project aims to explore the likelihood that generated
type-checkers are sound and complete by reasoning about our
software. 
