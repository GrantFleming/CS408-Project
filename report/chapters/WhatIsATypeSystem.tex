\chapter{What is a type system?}

From a computer science perspective, a type is piece of information
that we attach to terms or other constructs of a programming language
so that we might verify the validity of their use or otherwise reason
about the meaning of the program. The classic example of checking such
a validity is that of attaching a type to a function that identifies
the type of the term it expects as input, and thus being able to later
check that any input supplied conforms to this type.

Type systems use the types assigned to terms in order to help the
programmer avoid certain classes of errors by checking that terms of
various types are combined in ways which are valid according to a
their types and the rules of the type system. Exactly when these
checks take place can vary, sometimes occurring before compilation,
known as static type checking, and sometimes occuring after
compilation during the running of the program, known as dynamic type
checking.

We often talk about type constructors as parameterised entities that
we use to construct types. When these constructors require no
parameters, they are somewhat synonymous with types. In many languages
we have the ability to supply other types as parameters to certain
constructors, allowing us to represent data such as typed lists. When
we are allowed to supply arbitrary terms in the language as
parameters to type constructors, the language is said to be
dependently typed.

The features offered by type systems vary greatly. All will involve
some incarnation of type checking however they can often include
more advanced features such as type inference, higher-order functions,
algebraic data types, various flavours of polymorphism and termination
checking \cite{Abel_2004}. In some applications, the type system even
takes an interactive role in the production of code as is the case
with some theorem provers and dependently typed programming languages. 

Type systems are often described in literature by a collection of
inductive rules that describe aspects of the system such as when we
are able to assign some type or when we might consider types
equivalent. In the case of dependent types we also need to provide
some semantics so that we might make sense of arbitrary terms in the
type, in this case the type checking process might need to appeal to
reduction in order to perform some check.
